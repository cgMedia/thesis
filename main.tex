% This is a template for use with the MSU Thesis class
% Vesion 2.7 2017/06/22
%
% Class options:
%  [PhD] Doctor of Philosophy (default)
%  [DEd] Doctor of Education
%  [DMA] Doctor of Musical Arts
%   [MA] Master of Arts
%   [MS] Master of Science
%  [MAT] Master of Arts for Teachers
%  [MBA] Master of Business Administration
%  [MFA] Master of Fine Arts
% [MIPS] Master of International Planning Studies
% [MHRL] Master of Human Resources and Labor Relations
% [MMus] Master of Music
%  [MPH] Master of Public Health
%  [MPP] Master of Public Policy
%  [MSW] Master of Social Work
% [MURP] Master in Urban and Regional Planning
%
%
% This template has everything in the right order.
% Just add real content and you're done!

\documentclass[mixedtoc]{msu-thesis-custom}

% for a prettier, but possibly non-compliant table of contents use the [mixedtoc] option
% for a plain table of contents use the [plaintoc] option
% for a horrendous looking, but possibly required table of contents, use the [boldtoc] option
%
% If you have large tables/figures that need to be in landscape mode, add the [lscape] option

% This is standard fontenc/inputenc for pdflatex
% If you use LuaLaTeX or XeLaTeX you should replace this with the fontspec package
\usepackage[utf8]{inputenc}
\usepackage[T1]{fontenc}
%
% If the thesis office requires Times, we'll give them Times
% You can experiment with other font packages here if you like.
% If you are using XeLaTeX or LuaLaTeX load the Times or Times New Roman font with \setmainfont
%\usepackage{newtxtext,newtxmath}
\usepackage{newpxtext,newpxmath}

% Set up the drop-caps
\usepackage{GoudyIn, Rothdn}
\usepackage{lettrine}
%\renewcommand\LettrineFontHook{\Rothdnfamily}
\renewcommand\LettrineFontHook{\GoudyInfamily}

% Load any extra packages here
\usepackage{amsmath, amssymb, amsfonts}
\usepackage[scaled=0.85]{DejaVuSansMono}
\usepackage[dvipsnames]{xcolor}
\usepackage{filecontents}
\usepackage[]{graphicx}
\usepackage{import}
\usepackage{lipsum}
\usepackage{listings}
\usepackage{mathtools}
\usepackage{microtype}
\usepackage[super]{nth}
\usepackage[intoc]{nomencl}
\usepackage{paralist}
\usepackage{pgfplots}
\usepackage{physics}
\usepackage{siunitx}
\usepackage{subcaption}

\usepackage{hyperref}
\usepackage{cleveref}

\lstset{language=C++, basicstyle=\ttfamily, backgroundcolor=\color{lightgray}}

\pgfplotsset{compat=1.4}

\usepgfplotslibrary{external}
\usepgfplotslibrary{colorbrewer}
\usetikzlibrary{pgfplots.groupplots}

% --- Universal
%\newcommand{\dropcap}[2]{\lettrine[lines=3, depth=1]{\color{BrickRed}#1}{#2}} % Use with Rothenburg drop-caps
%\newcommand{\dropcap}[2]{\lettrine[lines=4, depth=0]{\color{BrickRed}#1}{#2}} % Use with Goudy Initialen drop-caps
\newcommand{\dropcap}[3][\color{BrickRed}]{%
\lettrine[lines=4, depth=0]{#1#2}{#3}%
}

\newcommand{\naive}{na\"{\i}ve}
\newcommand{\Naive}{Na\"{\i}ve}
\newcommand{\naively}{na\"{\i}vely}
\newcommand{\Naively}{Na\"{\i}vely}
\newcommand{\cafe}{caf\'e}
\newcommand{\etal}{\emph{et~al.}} % and others
\newcommand{\ie}{\emph{i.e.}}     % that is
\newcommand{\eg}{\emph{e.g.}}     % for example

% Some of the \int_{} symbols get really tight when set immediately before a fraction.
% These macros ('spaced int') give a little bit of extra space when necessary.
\newcommand{\integralspace}{\kern0.1em}
\newcommand{\sint}{\int_{} \integralspace}
\newcommand{\siint}{\iint_{} \integralspace}

% Requires mathtools
\DeclarePairedDelimiter\ceil{\lceil}{\rceil}
\DeclarePairedDelimiter\floor{\lfloor}{\rfloor}

\newcommand{\attrib}{} % dummy macro
\newenvironment{frontquote}[1]
{
  \renewcommand{\attrib}{\hspace*{\fill}---#1}
  \begin{quote}\small
}{
  \\ \attrib
  \end{quote}
}

\newcommand{\QuEST}{QuEST}

% --- Common colors
\definecolor{cbred}{HTML}{e41a1c}
\definecolor{cbblue}{HTML}{377eb8}
\definecolor{cbgreen}{HTML}{4daf4a}

\newif\ifmakeplots
\makeplotstrue
\newcommand{\conditionalFigureInput}[1]{\ifmakeplots%
  \input{#1}%
\fi%
}

% --- Chapter 02
\newcommand{\bubble}{microsphere}
\newcommand{\Bubble}{Microsphere}
\newcommand{\bubbles}{\bubble s}
\newcommand{\Bubbles}{\Bubble s}

\newcommand{\GF}{\ensuremath{g_r(\vb{r}, t; \, \vb{r}', t')}}
\DeclareRobustCommand{\redTriangle}{\textcolor{red!75!black}{\tikz{\pgfuseplotmark{triangle*}}}}
\DeclareRobustCommand{\blueCircle}{\textcolor{blue!75!black}{\tikz{\pgfuseplotmark{*}}}}

% --- Chapter 03
\newcommand{\qd}{quantum dot}
\newcommand{\qds}{quantum dots}
\newcommand{\Qd}{Quantum dot}
\newcommand{\Qds}{Quantum dots}

\newcommand{\outerprod}[2]{#1 \! \otimes \! #2}
\newcommand{\tensor}[1]{#1}


\makeplotsfalse % Used to switch off PGFplots figures with \conditionalFigureInput macro

% Zero-index the chapters and sections, just to be cheeky.
\setcounter{chapter}{-1}

% You must specify the title of your thesis, your name, the field of study (not department), and the year
\title{The \textsc{QuEST} for active media models: a self-consistent framework for simulating wave propagation in nonlinear systems}
\author{Connor A.\ Glosser}
\fieldofstudy{Physics \& Electrical Engineering} % This should be in sentence case
\date{\today}

% If you want a dedication page, specify the text of the dedication here and uncomment the next command.
%
%\dedication{This thesis is dedicated to someone.}
%
\begin{document}

% All the stuff before your actual chapters is called the front matter
\frontmatter
% First make the title page
\maketitlepage

\import{chapters/}{abstract}

% Force a newpage
\clearpage
% Make the copyright page. The Graduate School ridiculously prohibits you
% from having a copyright page unless you pay ProQuest to register the copyright.
% This should be illegal, but I didn't make up the rule.

\makecopyrightpage

% If you have a dedication page, uncomment the next command to print the dedication page
%
%\makededicationpage
%
\clearpage
% Your Acknowledgements are formatted like a chapter, but with no number
\chapter*{Acknowledgements}
\DoubleSpacing % Acknowledgements should be double spaced
Your acknowledgements here.
%
\clearpage
% We need to turn single spacing back on for the contents/figures/tables lists
\SingleSpacing
\tableofcontents* % table of contents will not be listed in the TOC
\clearpage
\listoftables % comment this out if you have no tables
\clearpage
\listoffigures % comment this out if you have no figures

% If you have a list of abbreviations/symbols it would go here preceded by a \clearpage
% See the class documentation and the Memoir manual for how to create other lists
%
\makenomenclature
\import{extra/}{abbreviations}

\mainmatter
%
% The next line removes the dots in chapter headings in the TOC
% May violate thesis office rules
%\addtocontents{toc}{\protect\renewcommand{\protect\cftchapterdotsep} {\cftnodots}}

\import{chapters/00-intro/}{ch00}
\import{chapters/01-bubbles/}{ch01}
\import{chapters/02-quantum_dots/}{ch02}
\import{chapters/03-accelerators/}{ch03}

% If you have pages that must appear in landscape mode, use the [lscape] documentclass option
% and enclose the pages in a {landscape} environment.
%\clearpage\pagestyle{lscape} % first clear the page and change the pagestyle
%\begin{landscape}
%
% your landscape table(s) or figure(s) here
%
%\end{landscape}
%\pagestyle{plain} % remember to change the pagestyle back to plain
%
%
% If you have appendices, they would go here.
% Comment these lines out if you don't
% If you have more than one appendix, you need to use
%   \begin{appendices}
%   \chapter{First appendix}
%   \chapter{Second appendix}
%   \end{appendices}
%
%\appendix
%\chapter{Your appendix}

\begin{appendices}
  \import{chapters/appendices/}{A0-QuEST_manual}
  \import{chapters/appendices/}{A2-predictor_corrector}
  \import{chapters/appendices/}{A1-units}
\end{appendices}


\backmatter
% The next lines add the dots back into the References/Bibliography heading
% of the TOC.  Only uncomment this if you need to put the dots back in having removed them for Chapter headings.
%
%\addtocontents{toc}{%
%   \protect\renewcommand{\protect\cftchapterdotsep} {\cftdotsep}}
%
\makebibliographypage % make the bibliography cover page
% Bibliography can be single spaced
%
\SingleSpacing

\bibliography{chapters/00-intro/reflist,chapters/01-bubbles/reflist,chapters/02-quantum_dots/reflist,chapters/03-accelerators/reflist,chapters/appendices/reflist}{}
\bibliographystyle{plain}

% Your bibliography command here (e.g. \bibliography{your-bib-file}) if using natbib
%
% Remember that although the bibliography is single spaced, there needs to
% be a blank line between entries. This is set by your bibliography package
% If you are using natbib it is \bibsep; if using biblatex it's \bibitemsep
\end{document}

