% This is a template for use with the MSU Thesis class
% Vesion 2.7 2017/06/22
%
% Class options:
%  [PhD] Doctor of Philosophy (default)
%  [DEd] Doctor of Education
%  [DMA] Doctor of Musical Arts
%   [MA] Master of Arts
%   [MS] Master of Science
%  [MAT] Master of Arts for Teachers
%  [MBA] Master of Business Administration
%  [MFA] Master of Fine Arts
% [MIPS] Master of International Planning Studies
% [MHRL] Master of Human Resources and Labor Relations
% [MMus] Master of Music
%  [MPH] Master of Public Health
%  [MPP] Master of Public Policy
%  [MSW] Master of Social Work
% [MURP] Master in Urban and Regional Planning
%
%
% This template has everything in the right order.
% Just add real content and you're done!

\documentclass[mixedtoc]{msu-thesis-custom}

% for a prettier, but possibly non-compliant table of contents use the [mixedtoc] option
% for a plain table of contents use the [plaintoc] option
% for a horrendous looking, but possibly required table of contents, use the [boldtoc] option
%
% If you have large tables/figures that need to be in landscape mode, add the [lscape] option

% This is standard fontenc/inputenc for pdflatex
% If you use LuaLaTeX or XeLaTeX you should replace this with the fontspec package
\usepackage[utf8]{inputenc}
\usepackage[T1]{fontenc}
%
% If the thesis office requires Times, we'll give them Times
% You can experiment with other font packages here if you like.
% If you are using XeLaTeX or LuaLaTeX load the Times or Times New Roman font with \setmainfont
%\usepackage{newtxtext,newtxmath}
\usepackage{newpxtext,newpxmath}

% Set up the drop-caps
\usepackage{GoudyIn, Rothdn}
\usepackage{lettrine}
%\renewcommand\LettrineFontHook{\Rothdnfamily}
\renewcommand\LettrineFontHook{\GoudyInfamily}

% Load any extra packages here
\usepackage{amsmath, amssymb, amsfonts}
\usepackage[scaled=0.85]{DejaVuSansMono}
\usepackage[dvipsnames]{xcolor}
\usepackage{filecontents}
\usepackage[]{graphicx}
\usepackage{import}
\usepackage{lipsum}
\usepackage{listings}
\usepackage{mathtools}
\usepackage{mathrsfs}
\usepackage{microtype}
\usepackage[super]{nth}
\usepackage[intoc]{nomencl}
\usepackage{paralist}
\usepackage{pgfplots}
\usepackage{physics}
\usepackage{siunitx}
\usepackage{subcaption}
\usepackage[normalem]{ulem}
\usepackage{wasysym}

\usepackage{hyperref}
\usepackage{cleveref}

\lstset{language=C++, basicstyle=\ttfamily, backgroundcolor=\color{lightgray}}

\pgfplotsset{compat=1.4}

\usepgfplotslibrary{external}
\usepgfplotslibrary{colorbrewer}
\usetikzlibrary{pgfplots.groupplots}

% --- Universal
%\newcommand{\dropcap}[2]{\lettrine[lines=3, depth=1]{\color{BrickRed}#1}{#2}} % Use with Rothenburg drop-caps
%\newcommand{\dropcap}[2]{\lettrine[lines=4, depth=0]{\color{BrickRed}#1}{#2}} % Use with Goudy Initialen drop-caps
\newcommand{\dropcap}[3][\color{BrickRed}]{%
\lettrine[lines=4, depth=0]{#1#2}{#3}%
}

\newcommand{\naive}{na\"{\i}ve}
\newcommand{\Naive}{Na\"{\i}ve}
\newcommand{\naively}{na\"{\i}vely}
\newcommand{\Naively}{Na\"{\i}vely}
\newcommand{\cafe}{caf\'e}
\newcommand{\etal}{\emph{et~al.}} % and others
\newcommand{\ie}{\emph{i.e.}}     % that is
\newcommand{\eg}{\emph{e.g.}}     % for example

% Some of the \int_{} symbols get really tight when set immediately before a fraction.
% These macros ('spaced int') give a little bit of extra space when necessary.
\newcommand{\integralspace}{\kern0.1em}
\newcommand{\sint}{\int_{} \integralspace}
\newcommand{\siint}{\iint_{} \integralspace}

% Requires mathtools
\DeclarePairedDelimiter\ceil{\lceil}{\rceil}
\DeclarePairedDelimiter\floor{\lfloor}{\rfloor}

\newcommand{\attrib}{} % dummy macro
\newenvironment{frontquote}[1]
{
  \renewcommand{\attrib}{\hspace*{\fill}---#1}
  \begin{quote}\small
}{
  \\ \attrib
  \end{quote}
}

\newcommand{\QuEST}{QuEST}

% --- Common colors
\definecolor{cbred}{HTML}{e41a1c}
\definecolor{cbblue}{HTML}{377eb8}
\definecolor{cbgreen}{HTML}{4daf4a}

\newif\ifmakeplots
\makeplotstrue
\newcommand{\conditionalFigureInput}[1]{\ifmakeplots%
  \input{#1}%
\fi%
}

% --- Chapter 02
\newcommand{\bubble}{microsphere}
\newcommand{\Bubble}{Microsphere}
\newcommand{\bubbles}{\bubble s}
\newcommand{\Bubbles}{\Bubble s}

\newcommand{\GF}{\ensuremath{g_r(\vb{r}, t; \, \vb{r}', t')}}
\DeclareRobustCommand{\redTriangle}{\textcolor{red!75!black}{\tikz{\pgfuseplotmark{triangle*}}}}
\DeclareRobustCommand{\blueCircle}{\textcolor{blue!75!black}{\tikz{\pgfuseplotmark{*}}}}

% --- Chapter 03
\newcommand{\qd}{quantum dot}
\newcommand{\qds}{quantum dots}
\newcommand{\Qd}{Quantum dot}
\newcommand{\Qds}{Quantum dots}

\newcommand{\outerprod}[2]{#1 \! \otimes \! #2}
\newcommand{\tensor}[1]{#1}



%\makeplotsfalse % Used to switch off PGFplots figures with \conditionalFigureInput macro

% Zero-index the chapters and sections, just to be cheeky.
%\setcounter{chapter}{-1}


% You must specify the title of your thesis, your name, the field of study (not department), and the year
\title{The \textsc{QuEST} for active media models: a self-consistent framework for simulating wave propagation in nonlinear systems}
\author{Connor Adrian Glosser}
\fieldofstudy{Physics} % This should be in sentence case
\date{2018}

% If you want a dedication page, specify the text of the dedication here and uncomment the next command.

\dedication{I dedicate this thesis to my loving mom and dad, Cindy and Ron, and their eternal, unconditional support.}

\begin{document}

% All the stuff before your actual chapters is called the front matter
\frontmatter
% First make the title page
\maketitlepage

\import{chapters/}{abstract}

% Force a newpage
\clearpage
% Make the copyright page. The Graduate School ridiculously prohibits you
% from having a copyright page unless you pay ProQuest to register the copyright.
% This should be illegal, but I didn't make up the rule.

\makecopyrightpage

% If you have a dedication page, uncomment the next command to print the dedication page

\makededicationpage

\clearpage
% Your Acknowledgements are formatted like a chapter, but with no number
\chapter*{Acknowledgments}
\DoubleSpacing % Acknowledgements should be double spaced
\dropcap{A}{s} I reflect on closing this chapter of my life, I can't help but beam at the
memories of everyone who has helped shape me into the person I am today. Thanks
in no small part to my studies, I've gotten to meet many truly incredible
people who guided me when I felt lost, encouraged my grandiose schemes, or
simply shared a portion of their story with me. Some of you remain with me and
some of you have left only marks, but I am a far better person for having spent
time with all of you and I offer my dearest thanks here.

First and foremost, I need to thank my incredible parents, Cindy Roach and Ron
Glosser. I truly don't have enough words to express my gratitude for the love
and support you've given me in everything I've decided to pursue, academic or
otherwise.  Whereas school taught me facts, you gave me what I needed to
\emph{think} and examine the world around me, no doubt setting me on a lifelong
journey of scientific and personal discovery on which this Ph.D is but a stop.

Next, I'd like to thank my two wonderful advisers-turned-friends, Carlo
Piermarocchi and Shanker Balasubramaniam.  I'm so happy my interests put me on
a course to work with both of you daily and I don't think I could have asked
for better mentors as I concurrently learned to be both a physicist and an
engineer. Together you've taught me a great deal about doing science on a
computer and a great deal more about determination and professionalism. It
saddens me that I have to leave MSU now that I've just started to come into my
own as a researcher, but it makes me much happier to think that we'll find ways
to collaborate for a long time to come.

I'd also like to thank the members of my committee---Phil Duxbury, Stuart
Tessmer, John Albrecht, and John Luginsland---for your encouragement and
insights into the work I present here. Phil, I need to thank you in particular
for giving my high-school self the chance to experience what being a scientist
is like. When we started working together in 2008 (if you could call a
high-schooler's attempts at research ``work'') I had no idea I was taking my
first steps on a road that would lead me across a couple of continents and
through a Ph.D. I realize now I've learned so much about the world and the
people that live in it for having walked that road with you, and I sincerely
hope I can pay your kindness forward to many students of my own someday.

Xander van Kooten, Liz Fujita, Kim \& Erin McGarrity, Scott O'Connor, and Bill
Martinez, thank you for being such noble and selfless companions. You all have
shown me exceptional warmth and kindness when I've needed it most, and every
day you inspire me to learn, do, or try something new and to become the best
person I can be. While we may sometimes live a great distance apart,
I find tremendous joy in knowing our shared sense of adventure will reunite us
with striking regularity. You will always have a home with me.

Jenni \& Giuseppe Zerilli, Dan \& Marilyn Olds, Shannon Nicley, Chelsea
Johnson, and Gift Pimcharoen, I cashed in all my undergrad points for this
Ph.D, but somehow I only ended up with more undergrad points. Thank you for
adopting me into Game Night and showing me the \emph{real} ropes of graduate
school. Our evenings spent betraying each other and hurling (false) accusations
of betraying each other remain some of my favorite memories of my time at MSU
and I hope we can find time for numerous reunions soon.

Jos Thijssen, Chris Verzijl, Laurens Janssen, F\'elix Carlier, Joost de Gussem,
Max Huisman, Witek Nawara, Daan Kuitenbrouwer, and all of the other ICCP-ers,
thank you for sharing so much of your culture with this loudmouth American.
Everything I ever needed to know about grad school I learned in ICCP alongside
you all, and the class simply would not have existed without the many 
international exchanges you helped facilitate.

Finally, thanks to some of the friends I've had along the way: Charlie \& JoAnn
Steele, Juan Manfredi, Tony Szedlak, Chris Sullivan, Robin de Kivit, H\aa kon
Jensen, Julia Rohrer, Louis Garcia, Mike Bennett, Chris Morse, and Steve
Hughey. You all continue to brighten my days with discussions of everything
from the latest philosophies on doing science to the most entertaining (and
possibly least effective) game strategies. 
\newpage
You all have routinely helped me find my way into truly magnificent experiences
greater than any I could have imagined, and for that I am profoundly grateful. I
consider myself beyond fortunate to count you amongst my friends and family and
I sincerely hope our bonds last long, long into the future. Thank you all for
everything you have given me.

\begin{center}
  \smiley{}
\end{center}



\clearpage
% We need to turn single spacing back on for the contents/figures/tables lists
\SingleSpacing
\tableofcontents* % table of contents will not be listed in the TOC
\clearpage
\listoftables % comment this out if you have no tables
\clearpage
\listoffigures % comment this out if you have no figures

% If you have a list of abbreviations/symbols it would go here preceded by a \clearpage
% See the class documentation and the Memoir manual for how to create other lists
%
\makenomenclature
\import{extra/}{abbreviations}

\mainmatter
%
% The next line removes the dots in chapter headings in the TOC
% May violate thesis office rules
%\addtocontents{toc}{\protect\renewcommand{\protect\cftchapterdotsep} {\cftnodots}}

\import{chapters/00-intro/}{ch00}
\import{chapters/01-bubbles/}{ch01}
\import{chapters/02-quantum_dots/}{ch02}
\import{chapters/03-accelerators/}{ch03}
\import{chapters/04-conclusions/}{ch04}

% If you have pages that must appear in landscape mode, use the [lscape] documentclass option
% and enclose the pages in a {landscape} environment.
%\clearpage\pagestyle{lscape} % first clear the page and change the pagestyle
%\begin{landscape}
%
% your landscape table(s) or figure(s) here
%
%\end{landscape}
%\pagestyle{plain} % remember to change the pagestyle back to plain

\begin{appendices}
  \crefalias{chapter}{appsec}
  \import{chapters/appendices/}{A0-QuEST_manual}
  \import{chapters/appendices/}{A2-predictor_corrector}
  \import{chapters/appendices/}{A3-chebyshev}
  \import{chapters/appendices/}{A1-units}
\end{appendices}

\backmatter
% The next lines add the dots back into the References/Bibliography heading
% of the TOC.  Only uncomment this if you need to put the dots back in having removed them for Chapter headings.
%
%\addtocontents{toc}{%
%   \protect\renewcommand{\protect\cftchapterdotsep} {\cftdotsep}}
%
\makebibliographypage % make the bibliography cover page
% Bibliography can be single spaced
%
\SingleSpacing

\bibliography{chapters/00-intro/reflist,chapters/01-bubbles/reflist,chapters/02-quantum_dots/reflist,chapters/03-accelerators/reflist,chapters/04-conclusions/reflist,chapters/appendices/reflist}{}
\bibliographystyle{plain}

% Your bibliography command here (e.g. \bibliography{your-bib-file}) if using natbib
%
% Remember that although the bibliography is single spaced, there needs to
% be a blank line between entries. This is set by your bibliography package
% If you are using natbib it is \bibsep; if using biblatex it's \bibitemsep
\end{document}

