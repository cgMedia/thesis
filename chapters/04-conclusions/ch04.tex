\chapter{\label{ch:conclusions}Conclusions and future perspective}

\begin{frontquote}{The White Queen, \emph{Through the Looking Glass}}
  ``Why, sometimes I've believed as many as six impossible things before breakfast!''
\end{frontquote}

\dropcap{T}{his} thesis serves three main purposes.
First, it introduces the concept of \emph{active media}---physical systems governed by ordinary differential equations that couple together via radiative processes---and motivates the difficulty in examining these systems at a macroscopic or continuum level.
Second, it provides a discussion of numerical solution strategies based around time-domain integral equations.
Finally, it introduces and details the \QuEST{} software package.
This chapter summarizes the main findings of each of these parts and offers some perspective on potential future research. 

\section{Conclusion}

In \cref{ch:bubbles} we presented a computational analysis of acoustically-driven microsphere systems.
By considering only small \bubbles{} we cast the problem in terms of an incompressible fluid where velocity potentials describe pressure distributions.
Acoustic radiation incident on the system scatters from rigid spherical objects, leading to pressure variations over the objects' surfaces; this variation necessarily sets the objects into motion, leading to further potential/pressure perturbations within the system.
By expressing the velocity potential on the surface of each \bubble{} in terms of spherical harmonics, we obtain a matrix equation relating the velocity of each \bubble{} to the potential on every other \bubble{}.
From this, we compute the net force on every body which we then integrate numerically, noting that directly integrating the force causes numerical instabilities due to an atypical acceleration-dependent force.
To remedy this, we note that this force serves only to impede the motion of each \bubble{}, thus we treat it as an additional mass term.
We find that an incident acoustic pulse predominantly serves to translate a collection of \bubbles{} a fixed distance along $\vu{k}$---the pulse's wavevector---though the scattering effects also produce system-scale expansions and contractions depending on the \bubbles' initial configuration.

\Cref{ch:quantum dots} departs from kinematic systems to develop a semiclassical description of coupled \qds{}.
Here, each of $N$ \qds{} evolves according to the Liouville equation in response to an ambient electric field.
From this, we compute the expectation value of a transition dipole moment.
This acts as a source term in Maxwell's equations and gives rise to secondary radiation that couples the evolution of each \qd{}.
By employing an integral equation framework to describe this radiation, we recover fields with a high degree of accuracy at both extremely short and extremely long length scales.
We find that pairs of exceptionally close \qds{} have a greatly diminished response to an incident pulse; we attribute this effect to a dynamical detuning that brings the pair out of resonance with the external driving field. 
Larger assemblages of \qds{}, however, display a much richer behavior: we observe spatially-localized regions of increased and diminished polarization as well as multiple frequency generation in the nonlinear polarization response of particular \qds{}.

The integral equation methodology of \cref{ch:quantum dots} has an implicit $\mathcal{O}(n^2)$ spatial complexity because of the pairwise interactions between \qds{}.
In practice, this limits simulations to several tens of thousands of particles.
To overcome this bottleneck, \cref{ch:accelerator} develops a FFT-based acceleration scheme that exploits the translational invariance of the radiation kernel to greatly reduce the spatial complexity of the calculation.
Moreover, our acceleration scheme retains a strong-form integral formulation and makes no assumptions about the underlying computational basis functions to effect propagation.
We give preliminary results that depict the accuracy and speed of the method and demonstrate that we can recover fields under the action of an arbitrary linear operator to the precision of the underlying machine.

\section{Discussion}

\subsection{Advanced dissipators}

Designed to provide a semblance of spontaneous emission within a semiclassical framework, the two-level dissipator in \cref{eq:dissipator} works to remove internal energy from each \qd{} system and return it to the ground state.
A semiclassical radiation model will never fully describe spontaneous emission due to the lack of a quantized electromagnetic field, but more robust operators that couple the dissipation terms of multiple \qds{} together may more accurately model co\"operative phenomena such as four-wave mixing or superradiance. 
To derive the structure of one such advanced dissipator, consider first a two-\qd{} system with particles $A$ and $B$.
Without rigorous derivation, a coupled dissipator evolves the particles according to
\begin{equation}
  \dv{\hat{\rho}}{t} = \frac{-i}{\hbar} \commutator{\hat{\mathcal{H}}_a^0 + \hat{\mathcal{H}}_b^0}{\hat{\rho}} + \sum_{\shortstack{$\scriptstyle m \in \qty{A, B}$\\$\scriptstyle n \in \qty{A, B}$}} \frac{\gamma_{mn}}{2} \qty(2 \hat{\sigma}_m^- \hat{\rho} \hat{\sigma}_n^+ - \hat{\sigma}_m^+ \hat{\sigma}_n^- \hat{\rho} - \hat{\rho} \hat{\sigma}_m^+ \hat{\sigma}_n^-).
\end{equation}
Assuming $\hat{\rho} \approx \outerprod{\hat{\rho}_A}{\hat{\rho}_B}$ and tracing over $B$ to remove its degrees of freedom,
\begin{equation}
  \begin{aligned}
    \dv{\hat{\rho}_A}{t} =& \frac{-i}{\hbar} \commutator{\hat{\mathcal{H}}_A^0}{\hat{\rho}_A} + \hat{\mathcal{D}}_{AA} + \hat{\mathcal{D}}_{AB}
  \end{aligned}
  \label{eq:advanced dissipator liouville}
\end{equation}
where
\begin{subequations}
  \begin{align}
    \hat{\mathcal{D}}_{AA} &= \gamma_{AA} \mqty(-\rho_{11}^A & -\rho_{10}^A/2 \\ -\rho_{01}^A/2 & \rho_{11}^A) \\
    \hat{\mathcal{D}}_{AB} &= \frac{\gamma_{AB}}{2} \mqty(-\rho_{10}^A\rho_{01}^B - \rho_{10}^B \rho_{01}^A & \rho_{11}^A\rho_{10}^B - \rho_{10}^B\rho_{00}^A \\ \rho_{01}^B\rho_{11}^A - \rho_{00}^A\rho_{01}^B & \rho_{01}^B \rho_{10}^A + \rho_{01}^A \rho_{10}^B) \\
    \gamma_{mn} &= \frac{4}{3}\omega^3 \abs{\vb{d}}^2 \qty\Big(J_0(r_{mn}) + P_2(\cos \theta_{mn}) J_2(r_{mn}))
  \end{align}
  \label{eq:cross matrices}
\end{subequations}
Finally, $J_\ell$ and $P_\ell$ denote the Bessel function and Legendre polynomial of order $\ell$, respectively.
Note that the first two terms on the right side of \cref{eq:advanced dissipator liouville} give the evolution equation defined in \cref{ch:quantum dots}.
The additional $\hat{\mathcal{D}}_{AB}$ dissipator, then, has terms proportional to the product of $A$ and $B$'s coherences/off-diagonal terms with a long-range inverse-$r$ scaling due to the $J_0$ Bessel function.
These terms afford a mechanism for $A$ and $B$ to influence each other's dynamics more directly than through an intermediate radiation field alone.

In a many-particle system, \cref{eq:advanced dissipator liouville} expands to include  pairwise dissipators that perturb the evolution of every particle due to the assumption of a rank-deficient $\hat{\rho}$.
Thus,
\begin{equation}
  \dv{\hat\rho_m}{t} = \frac{-i}{\hbar} \commutator{\hat{\mathcal{H}}_m^0}{\hat{\rho}_m} + \sum_{n} \gamma_{mn} \hat{\mathcal{D}}_{mn}.
  \label{eq:coupled liouville}
\end{equation}
Evaluating \cref{eq:coupled liouville} for each of $N$ then scales as $\mathcal{O}(N^2)$, making it infeasible to use in large simulations.
Fortunately, the kernel-agnostic nature of \QuEST{} makes it relatively straightforward to develop FFT-accelerated interactions of the form in \cref{eq:cross matrices} alongside those used to describe electromagnetic fields.

\subsection{Micromagnetics}

The Landau-Lifshitz-Gilbert (LLG) equation,
\begin{equation}
  \dv{\vb{M}}{t} = -\gamma' \vb{M} \cp \vb{H} - \lambda \vb{M} \cp \vb{M} \cp \vb{H}
  \label{eq:llg}
\end{equation}
where
\begin{subequations}
\begin{align}
  \gamma' = \frac{\gamma_0}{1 + \gamma^2 \eta^2 M_s^2} \\
  \lambda = \frac{\gamma_0^2 \eta}{1 + \gamma^2 \eta^2  M_s^2}
\end{align}
\end{subequations}
details the evolution of magnetization, $\vb{M}$, in response to an applied magnetic field, $\vb{H}$~\cite{Aharoni2000}.
In these equations, $\gamma_0$ denotes the electron gyromagnetic ratio
\begin{equation}
  \gamma_0 = \frac{g \abs{e}}{2 m_e c},
  \label{eq:gyro}
\end{equation}
$\eta$ represents a material-depnedent damping parametre, and $g$, $e$, $m_e$, \& $c$ stand for the Land\'e $g$-factor, elementary charge, electron mass, and speed of light, respectively.
Simulating the evolution of the Landau-Lifshitz-Gilbert equation at the level of individual particles stands ready to offer novel perspectives into spintronics or other mesoscale systems.

\subsubsection{Technical details}

Immediately, \cref{eq:llg} draws a large number of parallels with the material equation of  \cref{ch:quantum dots}.
Apart from the obvious first-order and vector nature of both equations, the dual nature of electromagnetism means that $\vb{M}$ (as a source in Maxwell's equations) relates to $\vb{H}$ in the exact same way that $\vb{P}$ relates to $\vb{E}$.
By simply exchanging some of the physical constants, then, \cref{eq:integral operator} also describes radiation emanating from a magnetization distribution~\cite{Rothwell2009}.

Solving \cref{eq:llg} within a semiclassical radiation framework requires a little more care than an equivalent Bloch problem, however.
\Cref{eq:liouville} describes the evolution of a quantum wavefunction---a vector in ``Bloch space''---that generates an electromagnetic source distribution through application of an operator (see the discussion surrounding \cref{eq:qd operators,eq:polarization definition} on \cpagerefrange{eq:qd operators}{eq:polarization definition} for an illustration of this point).
As a result, the dynamics of \cref{eq:liouville} describe the evolution of a dipole's amplitude but not its orientation and we may employ separate spatial and temporal discretizations to effect a numerical analysis.
Moreover, this allow for systems with zero initial polarization ($\vb{P} = 0$ in an eigenstate of the unperturbed Hamiltonian, e.g. $\hat{\rho} = \op{0}{0}$ or $\hat{\rho} = \op{1}{1}$), thus the external pulse induces all of the dynamical behavior.
This makes for a far easier simulation as sources/fields before the start of the simulation do not require any specification.

On the other hand, \cref{eq:llg} determines the trajectory of an electromagnetic source vector in both (real, Cartesian, directional) space and time with no ability to decouple the dimensions beyond
\begin{equation}
  \vb{M}(t) = m_x(t) \vu{x} + m_y(t) \vu{y} + m_z(t) \vu{z}.
\end{equation}
\QuEST{} accommodates histories of any numeric type (scalar or tensor, real or complex) through careful templating, and so building a propagator for \cref{eq:llg} does not present much difficulty aside from some extra index tracking.
\Cref{eq:llg} also has an implicit normalization condition, however, in that
\begin{equation}
  \dv{\abs{\vb{M}}}{t} = 0.
\end{equation}
This presents a larger computational challenge as it implies particles/magnetic domains/``spins'' always interact---even before the start of a simulation in a manner not dissimilar to molecular dynamics simulations.
The mechanism behind the predictor/corrector algorithm---updating a source distribution and then reincorporating that into the electromagnetic fields before advancing the timestep---will accommodate these interactions without issue but meaningful simulations will require an equilbiration period of order $\mathcal{O}(r_\text{max}/c)$ before collecting data.
