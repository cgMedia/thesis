\chapter{\label{ch:conclusions}Conclusions and future perspective}

\dropcap{T}{his} thesis serves three main purposes.
First, it introduces the concept of \emph{active media}---physical systems governed by ordinary differential equations that couple together via radiative processes---and motivates the difficulty in examining these systems at a macroscopic or continuum level.
Second, it provides a discussion of numerical solution strategies based around time-domain integral equations.
Finally, it introduces and details the \QuEST{} software package.
This chapter summarizes the main findings of each of these parts and offers some perspective on potential research based on these ideas.

\section{Conclusion}


\section{Discussion}

\subsection{Advanced dissipators}

\subsection{Micromagnetics}

The Landau-Lifshitz-Gilbert (LLG) equation,
\begin{equation}
  \dv{\vb{M}}{t} = -\gamma_0 \vb{M} \cp \qty(\vb{H} - \eta \dv{\vb{M}}{t}),
  \label{eq:llg}
\end{equation}
details the evolution of magnetization, $\vb{M}$, in response to an applied magnetic field, $\vb{H}$~\cite{Aharoni2000}.
Here, $\gamma_0$ denotes the gyromagnetic ratio
\begin{equation}
  \gamma_0 = \frac{g \abs{e}}{2 m_e c},
  \label{eq:gyro}
\end{equation}
$\eta$ represents a phenomenological decay constant, and $g$, $e$, $m_e$, \& $c$ stand for the Land\'e $g$-factor, elementary charge, electron mass, and speed of light, respectively.
Immediately, \cref{eq:llg} draws a large number of parallels with the optical Bloch/Liouville equation \cref{eq:liouville}.
Apart from the obvious first-order and vector nature of both equations, the dual nature of electromagnetism means that $\vb{M}$ (as a source in Maxwell's equations) relates to $\vb{H}$ in the exact same way that $\vb{P}$ relates to $\vb{E}$.
By simply exchanging some of the physical constants, then, \cref{eq:integral operator} also describes radiation eminating from a magnetization distribution~\cite{Rothwell2009}.
