\section{Numerical Results}

Here we present a series of numerically-solved systems to illustrate the utility of the method in investigating acoustic phenomena.
We perform simulations of one- and two-particle/pulse systems to determine the principal particle-field and particle-particle interactions, followed by simulations of larger assemblages of spheres to investigate group phenomena and effects in systems without symmetry.
Unless otherwise stated, \cref{table:sim parameters} gives the simulation parameters for each of the following simulations; as our interests lie in hydrodynamic applications, we use material parameters characteristic of water to define our external fluid medium.
Similarly, we consider here the case of gas-filled \bubbles~\cite{Blomley2001}, and therefore set their density much smaller than that of the exterior medium. 
The acoustic pulses lie in the ultrasonic regime, and the chosen frequency of \SI{20}{\mega\hertz} corresponds to that of typical applications in acoustic microscopy.

\begin{table}
  \centering
  %\begin{ruledtabular}
    \begin{tabular}{lll}
      Quantity                   & Symbol   & Value                                     \\ \hline
      Sound speed                & $c_s$    & \SI{1500}{\meter \per \second}            \\
      \Bubble\ radius            & $a_k$    & \SI{1}{\micro \meter}                     \\
      Density (exterior)         & $\rho_0$ & \SI{1000}{\kilogram \per \meter \cubed}   \\
      Density (interior)         & $\rho_s$ & \SI{1}{\kilogram \per \meter \cubed}      \\
      Pulse amplitude            & $P_0$    & \SI{0.05}{\meter\squared \per \second}    \\
      Center frequency           & $f_0$    & \SIrange{0.5}{20}{\mega\hertz}            \\
      Pulse duration (st.\ dev.) & $\sigma$ & \SIrange{7}{24}{\micro\second}
    \end{tabular}
  %\end{ruledtabular}
  \caption{\label{table:sim parameters}Typical simulation parameters.}
\end{table}

\subsection{Single \bubbles}

\begin{figure}
  \centering
  \begin{filecontents}{single_bubble_sigmoid.dat}
0.	-0.0001322118953794948
0.07097232079489046	-0.0001755666549163271
0.14194464158978004	-0.0002110887512222224
0.2129169623846705	-0.00023540657229538937
0.2838892831795601	-0.0002452897807620454
0.35486160397445055	-0.0002378670837477525
0.4258339247693401	-0.0002108561595007453
0.4968062455642306	-0.0001627935458901401
0.5677785663591202	-0.000093250415435408
0.6387508871540106	-3.0188645508725447e-6
0.7097232079489002	0.00010574719892384211
0.7806955287437898	0.00022945212766198978
0.8516678495386802	0.0003630537496602675
0.9226401703335698	0.0005001454507614326
0.9936124911284603	0.0006331332964887741
1.0645848119233499	0.0007535117732406103
1.1355571327182403	0.0008522362286134728
1.20652945351313	0.0009201838871555895
1.2775017743080204	0.000948688691003791
1.34847409510291	0.0009301285121371101
1.4194464158978004	0.0008585369077549379
1.49041873669269	0.0007302059847214364
1.5613910574875804	0.0005442425620972347
1.63236337828247	0.0003030371220237438
1.7033356990773605	0.000012604429280981911
1.77430801987225	-0.0003172434808559124
1.8452803406671396	-0.0006729267593749033
1.91625266146203	-0.001037357194360956
1.9872249822569197	-0.001390445214628042
2.05819730305181	-0.001709859511392674
2.1291696238466997	-0.001972029758386095
2.20014194464159	-0.002153366833561899
2.2711142654364798	-0.002231658151646778
2.3420865862313702	-0.002187579135138088
2.41305890702626	-0.002006246527739419
2.4840312278211503	-0.001678726302485964
2.55500354861604	-0.0012033994590678822
2.6259758694109303	-0.0005870841031446077
2.69694819020582	0.00015418721735722572
2.7679205110007103	0.000994829159729822
2.8388928317956	0.001899885036675227
2.9098651525904895	0.002825788740642089
2.98083747338538	0.003721741252340783
3.0518097941802704	0.0045317036139772065
3.122782114975159	0.005196971542439848
3.1937544357700496	0.005659257301729549
3.26472675656494	0.005864164063576239
3.3356990773598305	0.00576489909639805
3.406671398154719	0.005326037290918319
3.4776437189496097	0.004527118403569902
3.5486160397445	0.003365842566684232
3.6195883605393906	0.00186062144443822
3.6905606813342793	0.000052248874703424484
3.7615330021291697	-0.00199552367289211
3.83250532292406	-0.0041966843080453115
3.9034776437189507	-0.006445055411640672
3.9744499645138394	-0.008617843633220296
4.04542228530873	-0.01058062392095267
4.11639460610362	-0.012193648326633071
4.187366926898509	-0.01331927763711087
4.258339247693399	-0.013830241483676951
4.32931156848829	-0.01361834608308215
4.40028388928318	-0.01260317412166332
4.471256210078069	-0.01074026447936393
4.5422285308729595	-0.008028226128821803
4.61320085166785	-0.004514235528936606
4.6841731724627405	-0.0002973939374524956
4.755145493257629	0.004470517410441633
4.82611781405252	0.009587249863526368
4.89709013484741	0.014805503410220081
4.9680624556423005	0.01984162241601572
5.039034776437189	0.02438746383982239
5.11000709723208	0.02812513151223299
5.18097941802697	0.030744064904827607
5.251951738821859	0.03195977113639253
5.322924059616749	0.03153330720080013
5.39389638041164	0.02929046845151335
5.46486870120653	0.02513953186398261
5.535841022001419	0.01908635010310386
5.606813342796309	0.01124560467229858
5.6777856635912	0.001847110459809058
5.74875798438609	-0.00876377657480145
5.819730305180979	-0.02013236252712254
5.8907026259758695	-0.03170783938039448
5.96167494677076	-0.042863575618524136
6.03264726756565	-0.05292411097858199
6.103619588360539	-0.0611980584737149
6.1745919091554295	-0.06701568029278661
6.24556422995032	-0.06976949137064402
6.3165365507452105	-0.06895588277237451
6.387508871540099	-0.06421547383787969
6.45848119233499	-0.05536972267307955
6.52945351312988	-0.04245127074443273
6.600425833924769	-0.02572558534782906
6.671398154719659	-0.005701703049880505
6.74237047551455	0.01686973067831061
6.81334279630944	0.041012392728075435
6.884315117104329	0.06555478458119456
6.955287437899219	0.08917513419336581
7.02625975869411	0.1104596836940365
7.097232079489	0.1279724756269149
7.168204400283889	0.1403338920516748
7.239176721078779	0.1463044072988147
7.31014904187367	0.1448693381780829
7.38112136266856	0.13531986606574808
7.452093683463449	0.11732531131636609
7.5230660042583395	0.09099160333124438
7.59403832505323	0.0569011412872089
7.665010645848119	0.016129800019125243
7.735982966643009	-0.029762294200191088
7.8069552874379	-0.07876842813397283
7.87792760823279	-0.12850219316081118
7.948899929027679	-0.1762927673698919
8.01987224982257	-0.2193060064198568
8.09084457061746	-0.25468699035496484
8.16181689141235	-0.27971798491615857
8.232789212207239	-0.2919842836335468
8.30376153300213	-0.2895392128247412
8.37473385379702	-0.2710588063190602
8.44570617459191	-0.235976371907328
8.516678495386799	-0.1845874327342028
8.58765081618169	-0.1181163587471165
8.65862313697658	-0.03873739737909789
8.729595457771469	0.05045527235256946
8.800567778566359	0.14553147332423202
8.87154009936125	0.24186324423143168
8.94251242015614	0.33431467206088483
9.013484740951029	0.4174776213948578
9.084457061745919	0.48594461619754487
9.15542938254081	0.5346072938408551
9.2264017033357	0.5589663565121835
9.297374024130589	0.5554369490467947
9.36834634492548	0.5216320513212993
9.43931866572037	0.4566059381874788
9.51029098651526	0.3610401659641418
9.581263307310149	0.2373559991335742
9.65223562810504	0.08973975692197951
9.72320794889993	-0.07592876343109146
9.794180269694818	-0.2522499228937264
9.865152590489709	-0.4305795109506159
9.9361249112846	-0.601399082475551
10.00709723207949	-0.7547684360252861
10.078069552874378	-0.880841169430015
10.149041873669269	-0.9704187956908442
10.22001419446416	-1.015514590246961
10.29098651525905	-1.009895561723847
10.361958836053939	-0.9495699757549831
10.432931156848829	-0.8331888567091125
10.50390347764372	-0.6623328298825916
10.57487579843861	-0.4416603825944844
10.645848119233499	-0.17889983136368262
10.716820440028389	0.1153253793821423
10.78779276082328	0.42784034654217223
10.858765081618168	0.7434252109308819
10.929737402413059	1.0454698977312291
11.00070972320795	1.316754287061288
11.07168204400284	1.540323272564144
11.14265436479773	1.70041912192029
11.21362668559262	1.783427435936819
11.284599006387507	1.778788114625181
11.355571327182398	1.6798194775178161
11.426543647977288	1.4844025504749792
11.497515968772179	1.1954740918956401
11.56848828956707	0.8212817550564212
11.63946061036196	0.3753633235032908
11.71043293115685	-0.12377558720111309
11.78140525195174	-0.6532947907838268
11.852377572746628	-1.186953985546172
11.923349893541518	-1.6963176582747368
11.994322214336409	-2.152178662009909
12.065294535131299	-2.52612963912486
12.13626685592619	-2.792196439740088
12.20723917672108	-2.928438452080548
12.27821149751597	-2.918418459304359
12.349183818310857	-2.7524495801429243
12.420156139105748	-2.4285384582892418
12.491128459900638	-1.9529607770186501
12.562100780695529	-1.340425516391124
12.63307310149042	-0.6138061108184296
12.70404542228531	0.1965620137307118
12.7750177430802	1.053998455340377
12.84599006387509	1.916973130015245
12.916962384669977	2.740928718127764
12.987934705464868	3.480366081772031
13.058907026259758	4.091116364761116
13.129879347054649	4.532711517029687
13.20085166784954	4.770752708850405
13.27182398864443	4.779163751232062
13.34279630943932	4.542205351390873
13.413768630234207	4.05611792265061
13.484740951029098	3.330258709837686
13.555713271823988	2.3876066271616723
13.626685592618879	1.264528679669111
13.697657913413769	0.009737299811055833
13.76863023420866	-1.317581538608583
13.83960255500355	-2.650426725262024
13.91057487579844	-3.9173149563072083
13.981547196593327	-5.0461090143346325
14.052519517388218	-5.968106771813796
14.123491838183108	-6.622121712983675
14.194464158977999	-6.958278124478485
14.26543647977289	-6.941265012991001
14.33640880056778	-6.552838797545046
14.40738112136267	-5.79342810330216
14.478353442157557	-4.682764302039647
14.549325762952448	-3.259528472340598
14.620298083747338	-1.580060977299478
14.691270404542228	0.283780777646925
14.762242725337119	2.24750829970481
14.83321504613201	4.217514973442986
14.9041873669269	6.095099336089475
14.97515968772179	7.7808840284708545
15.046132008516677	9.17951343115277
15.117104329311568	10.20450089835705
15.188076650106458	10.7830714641731
15.259048970901349	10.8608077263974
15.330021291696239	10.405858946673941
15.40099361249113	9.412424286583128
15.47196593328602	7.903182990758287
15.542938254080907	5.93033302381558
15.613910574875797	3.574931632748481
15.684882895670688	0.9443193282594083
15.755855216465578	-1.8324434268879939
15.826827537260469	-4.61100015125054
15.89779985805536	-7.239635280522983
15.96877217885025	-9.56817077286382
16.03974449964514	-11.457121508740721
16.110716820440025	-12.78635408343798
16.18168914123492	-13.462547314343901
16.252661462029806	-13.42490446941057
16.3236337828247	-12.6487836091229
16.394606103619587	-11.14714750452493
16.46557842441448	-8.96994320413384
16.536550745209368	-6.201670903788356
16.607523066004255	-2.957478299012399
16.67849538679915	0.6218759478286401
16.749467707594036	4.375890747943256
16.82044002838893	8.130798161793239
16.891412349183817	11.70650055399509
16.96238466997871	14.92388847012654
17.033356990773598	17.61249191667056
17.104329311568492	19.61840861530595
17.17530163236338	20.81239415202191
17.246273953158266	21.09788965422103
17.31724627395316	20.41860994529917
17.388218594748047	18.765138256167482
17.45919091554294	16.17980696130849
17.530163236337827	12.75903732691936
17.60113555713272	8.652323936889061
17.67210787792761	4.05723425327781
17.743080198722495	-0.7898254895447888
17.81405251951739	-5.626779140352803
17.885024840312276	-10.18206398086103
17.95599716110717	-14.1923576082816
18.026969481902057	-17.42005046673298
18.09794180269695	-19.66866774702807
18.168914123491838	-20.79478512968448
18.239886444286725	-20.715570715198847
18.31085876508162	-19.411760973340808
18.381831085876506	-16.926475566008488
18.4528034066714	-13.36067737240288
18.523775727466287	-8.866248337069042
18.59474804826118	-3.637605609645326
18.665720369056068	2.097407015059179
18.736692689850962	8.08687357364987
18.80766501064585	14.06418603615685
18.878637331440736	19.757446340556232
18.94960965223563	24.89903565667112
19.020581973030517	29.23556353519278
19.09155429382541	32.53838615666727
19.162526614620297	34.614760607058926
19.23349893541519	35.319475286898275
19.30447125621008	34.56646895805251
19.375443577004965	32.339548125440416
19.44641589779986	28.70089644698476
19.517388218594746	23.7957485283379
19.58836053938964	17.851520074745192
19.659332860184527	11.169995446067821
19.73030518097942	4.111955932639115
19.801277501774308	-2.925171096589613
19.872249822569195	-9.534694708060087
19.94322214336409	-15.329977515353251
20.014194464158976	-19.97205190901473
20.08516678495387	-23.192190541945973
20.156139105748757	-24.80713004961642
20.22711142654365	-24.72610079803755
20.298083747338538	-22.95022824850205
20.369056068133425	-19.56587908893718
20.44002838892832	-14.733969767304401
20.511000709723206	-8.677185893723792
20.5819730305181	-1.6666515739528212
20.652945351312987	5.990964538191403
20.72391767210788	13.965460981951859
20.794889992902768	21.914229665923248
20.86586231369766	29.49280320127091
20.93683463449255	36.365307189055166
21.007806955287435	42.21535393822373
21.07877927608233	46.757845732755335
21.149751596877216	49.75195889497776
21.22072391767211	51.01524005407769
21.291696238466997	50.438255521053904
21.36266855926189	47.99861598882558
21.433640880056778	43.77253393292017
21.504613200851665	37.94152478699942
21.57558552164656	30.791677292182094
21.646557842441446	22.70335305377239
21.71753016323634	14.13038459561459
21.788502484031227	5.569725252986212
21.85947480482612	-2.475372463133089
21.930447125621008	-9.533892849113306
22.001419446415895	-15.2019613703213
22.07239176721079	-19.16955290975801
22.143364088005676	-21.236325680941878
22.21433640880057	-21.316210927166523
22.285308729595457	-19.43236638608831
22.35628105039035	-15.70526158381189
22.427253371185238	-10.336915548605159
22.498225691980124	-3.593872015024429
22.56919801277502	4.209287405329195
22.640170333569905	12.72492797059039
22.7111426543648	21.5855860931431
22.782114975159686	30.41567933879784
22.85308729595458	38.84225334124732
22.924059616749467	46.50545329777295
22.99503193754436	53.06940558194936
23.066004258339248	58.234068134400054
23.136976579134135	61.74835211917196
23.20794889992903	63.42441720297255
23.278921220723916	63.15249228914201
23.34989354151881	60.91490205867427
23.420865862313697	56.797284486130565
23.49183818310859	50.99445239336666
23.562810503903478	43.80824572171388
23.633782824698365	35.63530362471727
23.70475514549326	26.94408068657232
23.775727466288146	18.242472182410438
23.84669978708304	10.039582169775251
23.917672107877927	2.806734433547312
23.98864442867282	-3.056833778336577
24.059616749467708	-7.249221458517943
24.130589070262594	-9.579770572689142
24.20156139105749	-9.971698060847018
24.272533711852375	-8.455401286576533
24.34350603264727	-5.155373906265171
24.414478353442156	-0.27375747063438777
24.48545067423705	5.92698809887664
24.556422995031937	13.14063482141116
24.627395315826824	21.03211893374257
24.698367636621718	29.250677309562807
24.769339957416605	37.441458858302944
24.8403122782115	45.25613033915318
24.911284599006386	52.36308000946761
24.98225691980128	58.45777430861171
25.053229240596167	63.27366086407691
25.12420156139106	66.59374111268566
25.195173882185948	68.26256026524979
25.266146202980835	68.19789696966015
25.33711852377573	66.40092878938961
25.408090844570616	62.963201530545916
25.47906316536551	58.0684914537973
25.550035486160397	51.98779422724239
25.62100780695529	45.06633539182379
25.691980127750178	37.70267326746202
25.762952448545064	30.3214523007728
25.83392476933996	23.342766513822717
25.904897090134845	17.15194790635538
25.97586941092974	12.07358063797047
26.046841731724626	8.352640896677977
26.11781405251952	6.1441813678511785
26.188786373314407	5.511402220311683
26.259758694109294	6.43071824236196
26.330731014904188	8.801779171314967
26.401703335699075	12.46032434722327
26.47267565649397	17.19208881765526
26.543647977288856	22.74651115989122
26.61462029808375	28.84954336907647
26.685592618878637	35.21531949940845
26.756564939673524	41.556756803808895
26.827537260468418	47.595338284587726
26.898509581263305	53.070376117827315
26.9694819020582	57.748002893943216
27.040454222853086	61.43000114588562
27.111426543647973	63.962378540567634
27.182398864442867	65.24334732565934
27.253371185237754	65.23010407250284
27.324343506032648	63.94357818101509
27.395315826827535	61.47019051834607
27.46628814762243	57.959708785543434
27.537260468417315	53.618559210221626
27.608232789212202	48.698463366391
27.679205110007096	43.48094974807717
27.750177430801983	38.25900126440698
27.821149751596877	33.31765701566015
27.892122072391764	28.91562341174216
27.96309439318666	25.269786419824058
28.034066713981545	22.54399724394942
28.105039034776432	20.84277846270999
28.176011355571326	20.20986202254631
28.246983676366213	20.63089168450437
28.317955997161107	22.03929026426115
28.388928317955994	24.3242098463508
28.459900638750888	27.33959271336572
28.530872959545775	30.91358839024198
28.60184528034067	34.8578198856832
28.672817601135556	38.97621385449771
28.743789921930443	43.073274078528314
28.814762242725337	46.96177481675263
28.885734563520224	50.469882944035184
28.956706884315118	53.447695510880735
29.027679205110005	55.773116531151366
29.0986515259049	57.356910415870054
29.169623846699785	58.14667943873243
29.240596167494672	58.1294417596605
29.311568488289566	57.33245905516918
29.382540809084453	55.82200048324814
29.453513129879347	53.699845816282604
29.524485450674234	51.097523035763956
29.59545777146913	48.168522297877615
29.666430092264015	45.078987186088696
29.737402413058902	41.9976021629381
29.808374733853796	39.08552132667841
29.879347054648683	36.48718575236265
29.950319375443577	34.32275235339556
30.021291696238464	32.68263475498151
30.092264017033358	31.624385944191083
30.163236337828245	31.17188865618233
30.234208658623132	31.316607404043637
30.305180979418026	32.02052047401232
30.376153300212913	33.22029409498649
30.447125621007807	34.83227065256988
30.518097941802694	36.75789586793984
30.589070262597588	38.88928282922101
30.660042583392475	41.114684267749496
30.73101490418737	43.32370557660101
30.801987224982255	45.4121334433519
30.872959545777142	47.28627784309382
30.943931866572036	48.86673185611774
31.014904187366923	50.09145041404293
31.085876508161817	50.91804323597528
31.156848828956704	51.325176892831514
31.2278211497516	51.312993360143174
31.298793470546485	50.90248275660089
31.369765791341372	50.13379804612476
31.440738112136266	49.06356681896551
31.511710432931153	47.76133280474933
31.582682753726047	46.305336692129195
31.653655074520934	44.777909525843825
31.724627395315828	43.26079079370608
31.795599716110715	41.830689515064044
31.866572036905602	40.55537829572772
31.937544357700496	39.49055193362011
32.00851667849538	38.67760348719374
32.07948899929028	38.142384244050334
32.150461320085164	37.894932040447074
32.22143364088006	37.93008447519095
32.292405961674945	38.22884520425529
32.36337828246984	38.76034386864778
32.434350603264726	39.48422100186271
32.50532292405961	40.35327408190799
32.57629524485451	41.31621464508714
32.64726756564939	42.32040442231165
32.71823988644429	43.314457305839916
32.789212207239174	44.250611548483946
32.86018452803407	45.08679225254891
32.931156848828955	45.78829832512728
33.00212916962384	46.329061753198836
33.073101490418736	46.69244163036106
33.14407381121362	46.871532020477495
33.21504613200852	46.86898211635507
33.286018452803404	46.69634910988654
33.3569907735983	46.3730277238201
33.427963094393185	45.92482366043227
33.49893541518807	45.382258965833174
33.569907735982966	44.778713037265625
33.64088005677785	44.14851161530279
33.71185237757275	43.525076291620486
33.782824698367634	42.93923858899569
33.85379701916253	42.4178064406901
33.924769339957415	41.98244879845283
33.9957416607523	41.648938647174745
34.066713981547196	41.42676859706052
34.13768630234208	41.3191289094259
34.20865862313698	41.32321716437235
34.27963094393186	41.430832931922225
34.35060326472676	41.62920015306567
34.421575585521644	41.901954244665106
34.49254790631654	42.23022956255801
34.563520227111425	42.59378493429467
34.63449254790631	42.97210962950914
34.705464868701206	43.345458603152736
34.77643718949609	43.69577352944895
34.84740951029099	44.007454627916445
34.918381831085874	44.2679572969316
34.98935415188077	44.46819692835703
35.060326472675655	44.60275480867554
35.13129879347054	44.66988750728535
35.202271114265436	44.67135132524666
35.27324343506032	44.61206185982794
35.34421575585522	44.49961610463638
35.415188076650104	44.343710312972945
35.486160397445	44.155490714567854
35.557132718239885	43.946875817755874
35.62810503903477	43.72988834034198
35.699077359829666	43.516031876261934
35.77004968062455	43.315742497882
35.84102200141945	43.13793905447705
35.91199432221433	42.98968850506627
35.98296664300923	42.87599481180815
36.053938963804114	42.799712290157274
36.124911284599	42.76157735792735
36.195883605393895	42.760346725353344
36.26685592618878	42.793025466870276
36.337828246983676	42.855165225798984
36.40880056777856	42.94121102391847
36.47977288857346	43.04487468840867
36.550745209368344	43.1595136174883
36.62171753016324	43.2784952971463
36.692689850958125	43.39553045441237
36.76366217175301	43.50496078718157
36.834634492547906	43.60199065377915
36.90560681334279	43.68285575488995
36.97657913413769	43.744925524899045
37.047551454932574	43.78673950874586
37.11852377572747	43.80798128562173
37.189496096522355	43.80939637839515
37.26046841731724	43.79266294312756
37.331440738112136	43.76022577748195
37.40241305890702	43.715105262976586
37.47338537970192	43.66069324338589
37.5443577004968	43.60054755902407
37.6153300212917	43.538196061796754
37.686302342086584	43.47695952098763
37.75727466288147	43.41980101408663
37.828246983676365	43.3692073172246
37.89921930447125	43.32710560890753
37.970191625266146	43.294816617132874
38.04116394606103	43.27304329790949
38.11213626685593	43.26189233555593
38.183108587650814	43.260924278735715
38.2540809084457	43.26922701992541
38.325053229240595	43.28550661149707
38.39602555003548	43.30818908564777
38.466997870830376	43.33552698383241
38.53797019162526	43.36570466342401
38.60894251242016	43.39693708234605
38.679914833215044	43.42755760610339
38.75088715400994	43.45609137198548
38.821859474804825	43.48131181790956
38.89283179559971	43.50227907662469
38.963804116394606	43.51835999294959
39.03477643718949	43.52923049241799
39.10574875798439	43.534861872727575
39.17672107877927	43.53549327305533
39.24769339957417	43.531593079500254
39.318665720369054	43.52381233743713
39.38963804116394	43.51293336394025
39.460610361958835	43.49981669620441
39.53158268275372	43.485349294378686
39.602555003548616	43.470396565848084
39.6735273243435	43.455760324225935
39.7444996451384	43.44214427447374
39.815471965933284	43.43012806056298
39.88644428672817	43.420150357356654
39.957416607523065	43.412500964011365
40.02838892831795	43.407321387540485
40.099361249112846	43.40461301184503
40.17033356990773	43.404251642896945
40.24130589070263	43.406007011900996
40.312278211497514	43.40956570624141
40.38325053229241	43.41455597852852
40.454222853087295	43.42057294829746
40.52519517388218	43.4272028465491
40.596167494677076	43.43404514559986
40.66713981547196	43.44073164946175
40.73811213626686	43.44694187666234
40.80908445706174	43.45241433200438
40.88005677785664	43.45695352151122
40.951029098651524	43.46043280287446
41.02200141944641	43.46279337165537
41.092973740241305	43.46403985350755
41.16394606103619	43.4642330998065
41.234918381831086	43.4634808661007
41.30589070262597	43.461927090168
41.37686302342087	43.45974048193576
41.447835344215754	43.45710309582192
41.51880766501064	43.45419948339247
41.589779985805535	43.45120692783549
41.66075230660042	43.44828714935523
41.731724627395316	43.44557974995893
41.8026969481902	43.44319754461014
41.8736692689851	43.44122380996199
41.944641589779984	43.439711377426825
42.01561391057487	43.43868340851522
42.086586231369765	43.4381356201886
42.15755855216465	43.43803967806392
42.228530872959546	43.438347446056824
42.29950319375443	43.43899577165978
42.37047551454933	43.43991149475514
42.44144783534421	43.441016392118364
42.51242015613911	43.44223180649015
42.583392476933994	43.44348275488744
42.65436479772888	43.44470136219124
42.725337118523775	43.44582951960475
42.79630943931866	43.44682072020673
42.867281760113556	43.4476410728307
42.93825408090844	43.448269538686866
43.00922640170334	43.44869747088601
43.080198722498224	43.448927564287224
43.15117104329311	43.44897234141407
43.222143364088005	43.4488523096448
43.29311568488289	43.44859392602345
43.364088005677786	43.4482274997886
43.43506032647267	43.44778515029372
43.50603264726757	43.447298920800435
43.577004968062454	43.446799128160876
43.64797728885734	43.446313006159606
43.718949609652235	43.44586367766982
43.78992193044712	43.445469469051815
43.860894251242016	43.44514356044222
43.9318665720369	43.44489394857636
44.0028388928318	43.44472368512404
44.07381121362668	43.4446313435258
44.14478353442157	43.4446116610737
44.215755855216464	43.444656300360016
44.28672817601135	43.44475467490748
44.357700496806245	43.444894787346385
44.42867281760113	43.44506403436723
44.499645138396026	43.44524994024666
44.57061745919091	43.445440789396855
44.64158977998581	43.44562613752291
44.712562100780694	43.44579719003449
44.78353442157558	43.44594704486979
44.854506742370475	43.44607080446335
44.92547906316536	43.44616556793538
44.996451383960256	43.4462303195159
45.06742370475514	43.44626573265628
45.13839602555004	43.44627391123246
45.209368346344924	43.44625808979411
45.28034066713981	43.44622231411612
45.351312987934705	43.44617112156196
45.42228530872959	43.44610923820422
45.493257629524486	43.446041306512335
45.56422995031937	43.44597165395435
45.63520227111427	43.44590410929997
45.70617459190915	43.44584186996075
45.77714691270404	43.445787420532376
45.848119233498934	43.44574249994842
45.91909155429382	43.445708112407196
45.990063875088715	43.44568457554903
46.0610361958836	43.44567159825365
46.132008516678496	43.445668379881
46.20298083747338	43.445673722746406
46.27395315826827	43.44568615003456
46.344925479063164	43.44570402213458
46.41589779985805	43.44572564542704
46.486870120652945	43.44574936878091
46.55784244144783	43.44577366433235
46.628814762242726	43.4457971904341
46.69978708303761	43.44581883591385
46.77075940383251	43.445837745902026
46.841731724627394	43.44585333044138
46.91270404542228	43.44586525784504
46.983676366217175	43.44587343531136
47.05464868701206	43.44587797963492
47.125621007806956	43.4458791809835
47.19659332860184	43.44587746266241
47.26756564939674	43.445873339588125
47.338537970191624	43.44586737787471
47.40951029098651	43.44586015753325
47.480482611781405	43.44585223983009
47.55145493257629	43.44584414037739
47.622427253371185	43.44583630856575
47.69339957416607	43.44582911351921
47.764371894960966	43.44582283637542
47.83534421575585	43.44581766838133
47.90631653655074	43.445813714055596
47.977288857345634	43.44581099850486
48.04826117814052	43.44580947789039
48.119233498935415	43.445809052019264
48.1902058197303	43.44580957806994
48.261178140525196	43.445810884547456
48.33215046132008	43.44581278468443
48.40312278211497	43.44581508865064
48.474095102909864	43.44581761409303
48.54506742370475	43.44582019468936
48.616039744499645	43.44582268655404
48.68701206529453	43.445824972474746
48.757984386089426	43.445826964079245
48.82895670688431	43.445828602129076
48.89992902767921	43.44582985520854
48.970901348474094	43.44583071712365
49.04187366926898	43.445831203348234
49.112845990063875	43.445831346854284
49.18381831085876	43.445831193646164
49.254790631653655	43.4458307982851
49.32576295244854	43.44583021964715
49.396735273243436	43.44582951710777
49.46770759403832	43.445828747293085
49.53867991483321	43.44582796148555
49.609652235628104	43.4458272037226
49.68062455642299	43.44582650958281
49.751596877217885	43.44582590561808
49.82256919801277	43.44582540936108
49.893541518807666	43.44582502981733
\end{filecontents}

\begin{tikzpicture}
  \begin{axis}[
    xlabel={Time $(\si{\micro\second})$},
    ylabel={Displacement $(\si{\micro\meter})$},
  ]
    \addplot+[
      black, solid, smooth, thick
    ] table [
    ] {single_bubble_sigmoid.dat};
  \end{axis}
\end{tikzpicture}

  \caption{\label{fig:single_displacement}Translation of a single \bubble\ interacting with an incident pulse ($f_0 = \SI{0.5}{\mega\hertz}$, $\sigma = \SI{7}{\micro\second}$). \Bubbles\ interacting with the pulse translate a finite distance along $\vb{k}$ due to the Gaussian envelope in \cref{eq:inc_field}.}
\end{figure}

\Cref{fig:single_displacement} gives the trajectory of a single \bubble\ initially at rest under the effects of an incident Gaussian pulse.
Under the linear and ideal fluid approximations and absent the Gaussian envelope in \cref{eq:inc_field}, the \bubble\ merely oscillates about its origin in accordance with \cref{eq:landau result}.
In the pulsed case, however, the variation in pressure imposed by the finite value of $\vb{k}$ modifies the system dynamics to yield a net translation of each \bubble.
Note that the regime considered here produces no net transfer of momentum between the acoustic field and the \bubble---a consequence of the ideal fluid.

\begin{figure}
  \centering
  \begin{tikzpicture}[>=latex]
    \begin{axis}[
      point meta min=-1, point meta max=1, %needed to make colorbar ticks work
      colorbar horizontal,
      colorbar style={
        at={(0.5,1.03)}, anchor=south,
        xticklabel pos=upper,
        xtick = \empty,
        %ylabel={Pressure Gradient $\left|\nabla P\right|$},
        %ytick=\empty,
        extra x ticks={-1,1},
        %extra x tick style={yshift=-0.5cm},
        %extra x tick labels = {$-P_0$, $P_0$},
        title style={yshift=-0.85cm},
        title = {Pressure gradient, $\vu{z} \cdot \nabla p$ (relative scale)},
        ticks = none
      },
      colormap={}{rgb255=(42,72,239); rgb255=(56,94,240); rgb255=(71,115,241); rgb255=(85,137,242); rgb255=(99,155,243); rgb255=(113,171,244); rgb255=(127,186,245); rgb255=(141,201,247); rgb255=(153,211,248); rgb255=(165,222,249); rgb255=(178,232,250); rgb255=(190,239,251); rgb255=(201,245,252); rgb255=(212,250,252); rgb255=(223,254,250); rgb255=(233,254,238); rgb255=(242,254,226); rgb255=(252,254,213); rgb255=(253,251,201); rgb255=(252,247,188); rgb255=(251,244,175); rgb255=(249,238,163); rgb255=(247,229,150); rgb255=(244,219,137); rgb255=(241,210,124); rgb255=(237,196,111); rgb255=(234,181,99); rgb255=(230,167,86); rgb255=(227,150,75); rgb255=(223,132,66); rgb255=(219,113,58); rgb255=(216,94,49); },
      enlargelimits=false,
      axis on top,
      %ytick = {-112.5, -75, -37.5, 0, 37.5, 75, 112.5},
      %yticklabels = {$-\frac{3\lambda}{2}$, $-\lambda$, $-\frac{\lambda}{2}$, $0$, $\frac{\lambda}{2}$, $\lambda$, $\frac{3\lambda}{2}$},
      xlabel={Time $(\si{\micro\second})$},
      ylabel={$\vu{z}$-co\"ordinate $(\si{\micro\meter})$},
      %title style={yshift=0.6cm},
      %title={Confinement of scatterers to planes},
    ]
      \addplot graphics[
        xmin=0E-4,
        xmax=1.5984E2,
        ymin=-1.12510E2,
        ymax=1.12510E2,
        ]
      {figures/banding.png};

      \draw[thick, <->] (axis cs:110,0) -- (axis cs:110,37.5) node [midway,right] {$\frac{c_s}{2f_0} = \si{\num{37.5}\micro\meter}$};

    \end{axis}
\end{tikzpicture}

  \caption{\label{fig:planar_confinement}
  (Color online) Confinement of non-interacting spheres to planes; identical counter-propagating pulses ($f_0 = \SI{20}{\mega\hertz}$, $\sigma = \SI{23.8}{\micro\second}$) initially displaced along $\vu{z}$ tend to align objects in $\nabla P = \vb{0}$ planes at $\lambda/2$ intervals.
    Field \& trajectories sampled every 30 timesteps and smoothed with a 16-sample windowed average.
  }
\end{figure}

\Cref{fig:planar_confinement} depicts smoothed results of 128 trajectories corresponding to single \bubbles\ initially spaced along $\vu{z}$ and excited by identical counter-propagating pulses. 
By taking the width of each pulse much greater than the radius of each \bubble, the two pulses reproduce the effects of interfering standing waves. 
The confinement occurs at $\nabla P = \vb{0}$ (nodal) planes where the net force on each \bubble\ vanishes.
The half-wavelength associated with the dominant pulse frequency gives the separation between neighboring planes.

Finally, \cref{fig:dipole field} shows the relative velocity potential near a single \bubble; given a surface expansion of $\varphi$, we may compute the potential everywhere through application of \cref{eq:reduced Kirchoff-Helmholtz}.
As predicted by \cref{eq:dipole}, this field greatly resembles that of a
pointlike ``velocity dipole'' with $\vb{v}_s$ acting as a dipole moment.

The simulations described thusfar demonstrate precise acoustic control; through careful application of the incident field parameters, we may induce a (finite, given a finite pulse) translation along the principal $\vu{k}$-vector with a large degree of accuracy in the overall displacement.
In addition, the application of multiple pulses serves to confine \bubbles\ to highly localized regions in space, offering a self-consistent model of acoustic tweezing.


\begin{figure}
  \centering
  \begin{tikzpicture}[>=latex,trim axis left]
  \begin{axis}[
      %grid = major,
      %major grid style = {black, opacity=0.4},
      xlabel = $z$ (\si{\micro\meter}),
      ylabel = $x$ (\si{\micro\meter}),
      enlargelimits = false,
      axis on top,
      axis equal image
  ]
    \addplot graphics [
      xmin = -3,
      xmax = 3,
      ymin = -3,
      ymax = 3,
    ] {figures/dipolefig.pdf};

    \filldraw[ball color=white!10, draw=none] (axis cs:0,0) circle[radius = 100];
    \draw[thick, ->] (axis cs:-0.618034,0) -- (axis cs:0.618034,0) node[midway,above]{$\vb{v}_s$};
  \end{axis}
\end{tikzpicture}

  \caption{
    \label{fig:dipole field}(Color online)
    Calculated isopotential contours near a lone \bubble.
    Red and blue colorations represent regions of positive and negative potential.
    The motion of each \bubble\ through the background medium serves primarily to produce a dipolar field of velocity potential with $\vb{v}_s$ serving as the sphere's dipole moment.
  }
\end{figure}

\subsection{Many-particle simulations}

\begin{figure}
  \centering
  \begin{filecontents}{separation_scaling.dat}
4.	0.0039309379533927465
4.22621	0.0033289508931444835
4.46522	0.002819823303323973
4.71774	0.00238902363475066
4.98454	0.00202434028029143
5.26643	0.001715524331093093
5.56426	0.001453954727929501
5.87894	0.001232353849175397
6.21141	0.001044592755502801
6.56268	0.0008854819317818118
6.93382	0.000750633967395908
7.32595	0.0006363413614317283
7.74025	0.0005394661484508813
8.17799	0.00045734679715236275
8.64048	0.0003877353447200363
9.12912	0.00032872442534388524
9.6454	0.00027869729325491417
10.1909	0.00023628425646620742
10.7672	0.0002003302083772076
11.3761	0.00016984775436028568
12.0195	0.00014400196931298138
12.6992	0.00012209227015123883
13.4174	0.00010351502982230786
14.1762	0.0000877650671484348
14.9779	0.00007441187271062007
15.8249	0.00006309092253604869
16.7199	0.000053491450787474325
17.6654	0.000045353552569257357
18.6645	0.000038452933149797
19.72	0.0000326028216793368
20.8352	0.000027642717873993584
22.0135	0.000023437138276455217
23.2584	0.000019871504288451485
24.5738	0.00001684813792995451
25.9635	0.00001428492841163429
27.4318	0.000012111668082618132
28.9832	0.000010268974922168597
30.6222	8.706750744632106e-6
32.354	7.382113333174739e-6
34.1837	6.25903030683212e-6
36.1169	5.306798158836918e-6
38.1594	4.4994444507863606e-6
40.3175	3.814895678624195e-6
42.5975	3.234524676960607e-6
45.0066	2.7424252358150413e-6
47.5518	2.3252095080544028e-6
50.241	1.9714585953719584e-6
53.0823	1.671529001044099e-6
56.0843	1.4172255741565802e-6
59.256	1.2016187635946337e-6
62.6071	1.0188098810964388e-6
66.1477	8.638143204898256e-7
69.8885	7.323957291855538e-7
73.8409	6.209730149873664e-7
78.0169	5.264992815538292e-7
82.429	4.464003673803442e-7
87.0906	3.784817675994954e-7
92.0158	3.2090584009259137e-7
97.2196	2.720849488595645e-7
102.718	2.3068907745974767e-7
108.527	1.955932951561013e-7
114.664	1.658384187090558e-7
121.149	1.4060533487307196e-7
128.	1.1921985423659568e-7
\end{filecontents}

\begin{filecontents}{radius_scaling.dat}
0.5	3.96010526352381e-6
0.51861	4.411458591891459e-6
0.53792	4.915151843316253e-6
0.55794	5.477277250588658e-6
0.57871	6.104600857010282e-6
0.60025	6.804712822842489e-6
0.6226	7.586033757600877e-6
0.64578	8.457950492415727e-6
0.66981	9.431013117816673e-6
0.69475	0.000010516921552341872
0.72061	0.000011728769122494297
0.74743	0.000013081181771923167
0.77526	0.000014590394189259555
0.80412	0.000016274641171376006
0.83405	0.00001815419382051974
0.8651	0.000020251742435319166
0.8973	0.000022592483551865595
0.9307	0.000025204663321045048
0.96535	0.00002811979501337424
1.00128	0.00003137252732967781
1.03856	0.000035003551315731515
1.07722	0.00003905480963465682
1.11732	0.00004357592544458012
1.15891	0.000048620922902017265
1.20205	0.00005425111985146393
1.2468	0.000060534724037002906
1.29321	0.00006754604445468016
1.34135	0.00007537069800088828
1.39128	0.00008410226386763101
1.44307	0.00009384650080593272
1.49679	0.00010472103964658581
1.55251	0.00011685661291984169
1.6103	0.00013039839596326087
1.67024	0.0001455100975016569
1.73242	0.00016237655842455472
1.79691	0.0001811976509269865
1.8638	0.00020220122338590103
1.93318	0.00022564069042998317
2.00514	0.00025179769035423645
2.07978	0.0002809890447085276
2.1572	0.0003135671285755815
2.2375	0.00034992357742910933
2.32079	0.0003904983914548233
2.40719	0.0004357862984738502
2.49679	0.00048632057371231896
2.58974	0.0005427289473257877
2.68614	0.0006056789903473701
2.78613	0.0006759390084313763
2.88985	0.0007543648115370244
2.99742	0.0008418922587148744
3.109	0.0009395956519131531
3.22473	0.0010486533508846023
3.34477	0.0011703959520602588
3.46928	0.0013063055089837245
3.59843	0.0014580442280743926
3.73238	0.0016274512193174215
3.87132	0.0018166150081497187
4.01543	0.0020278516933801673
4.1649	0.002263759748376582
4.31994	0.002527274449927008
4.48075	0.0028216599260836965
4.64755	0.00315060563945143
4.82055	0.0035182235713139883
5.	0.003929215823027879
\end{filecontents}

\begin{tikzpicture}[trim axis left, trim axis right]
  \pgfplotsset{
    %width=8cm,
    y axis style/.style={
      yticklabel style=#1,
      ylabel style=#1,
      y axis line style=#1,
      ytick style=#1
    },
    x axis style/.style={
      xticklabel style=#1,
      xlabel style=#1,
      x axis line style=#1,
      xtick style=#1
    }
  }
  \begin{loglogaxis}[
      axis x line*=bottom,
      x axis style = red!75!black,
      axis y line*=left,
      y axis style = red!75!black,
      xlabel = {Separation $\abs{\vb{d}}$ (\tikz{\pgfuseplotmark{triangle*}}, \si{\micro\meter})},
      ylabel = {$\Delta \abs{\vb{v}_\text{max}}$ (\tikz{\pgfuseplotmark{triangle*}}, relative)},
      width=0.85\columnwidth
    ]

    \addplot[
      red!75!black,
      only marks,
      mark=triangle*,
      mark size = 2.5
    ] table {separation_scaling.dat};

    \addplot[
      red!35!black,
      opacity=0.6,
      domain=4:128
    ] {0.250754*x^-3.00077};
  \end{loglogaxis}

  \begin{loglogaxis}[
      axis x line*=top,
      x axis style = blue!75!black,
      axis y line*=right,
      y axis style = blue!75!black,
      xlabel = {Radius (\tikz{\pgfuseplotmark{*}}, $\si{\micro\meter}$)},
      ylabel = {$\Delta\abs{\vb{v}_\text{max}}$ (\tikz{\pgfuseplotmark{*}}, relative)},
      ylabel style = {rotate=-180},
      width=0.85\columnwidth
    ]
    \addplot[
      blue!75!black,
      only marks,
      mark size=0.4ex
    ] table {radius_scaling.dat};

    \addplot[
      red,
      domain=0.5:5,
      opacity=1,
    ] {0.0000313328*x^2.99814};
  \end{loglogaxis}
\end{tikzpicture}

  \caption{
    \label{fig:double scaling}(Color online)
    Scaling behavior of two \bubbles\ arranged perpendicularly to an incident pulse for various radii and initial separations.
    The (\redTriangle, \blueCircle) symbols on each axis denote data associated with that axis.
    The \redTriangle\ follow a regression of $\Delta \abs{\vb{v}}_\text{d} = \num{0.250754} d_{12}^{-3.00077}$, and the \blueCircle\ follow $\Delta\abs{\vb{v}_\text{max}}_\text{r} = \num{3.13328e-5} a_0^{2.99814}$.
    These trends strongly indicate dominant dipolar interactions between \bubbles.
  }
\end{figure}

\begin{figure}
  \centering
  \includegraphics[width=0.5\columnwidth]{figures/3d_isosurface_inset.png}
  \caption{
    \label{fig:isosurface}(Color online) 
    Isosurfaces of velocity potential (arb.~units) calculated by evaluating the $\hat{\mathcal{S}}$ and $\hat{\mathcal{D}}$ terms in \cref{eq:reduced Kirchoff-Helmholtz} for a $N = 16$ particle simulation.
    Red, blue, and yellow surfaces denote regions of positive, negative, and zero potential, with holes appearing due to intersections with the bounding box.
    The inset box shows the three dimensional arrangement of the \bubbles\ superimposed with their velocity vectors, as well as several positive and negative potential isosurfaces.
    Rendered with VisIt\cite{VisIt}.
  }
\end{figure}
We now turn our attention to collections of mutually interacting \bubbles.
To quantify the effects of scattering, we first decouple scattering forces from the incident pulse by arranging two \bubbles\ perpendicularly to the pulse's $\vb{k}$-vector.
\Cref{fig:double scaling} gives results for such a simulation where we plot the relative change in velocity as compared with the single-particle simulation,
\begin{equation}
  \Delta \abs{\vb{v}_\text{max}} = \text{max}\qty(\abs{\vb{v}_\text{double}(t) - \vb{v}_\text{single}(t)}).
\end{equation}
In principle, describing quantities found from a complete simulation as a function of initial separation could obfuscate scaling data considerably; forces arising from scattering could alter the geometry of the system.
In practice, however, the perpendicular configuration used here gives scattering forces that only influence the motion along $\vb{k}$.
Consequently, $\Delta\vb{v} \propto \vb{z}$ and the \bubbles' initial separation remains a good estimator of scaling behavior. 
We see in \cref{fig:double scaling} that the radii data scale as $a_k^3$ and the
separation data exhibit strong $\abs{\vb{d}_{12}}^{-3}$ scaling, again
indicating a dominant dipolar interaction between \bubbles\ as shown by Ilinskii \etal\ in 2007~\cite{Ilinskii2007} and predicted by \cref{eq:dipole}.

Finally, we consider the dynamics of large ($N=16$) clouds of \bubbles.
For each simulation, we generate a collection of \bubbles\ initialized with zero velocity and random positions within a $\SI{10}{\micro\meter}$ ball subject to a minimum-separation constraint to prevent collisions.
\Cref{fig:isosurface} shows a snapshot of the velocity potential isosurfaces calculated in one such simulation. Even with mutual interactions, the shape of each isosurface remains consistent with the presence of a dipolar field oriented along the \bubbles' velocity.
Again, due to the localization assumption used to justify \cref{eq:green's function}, each system predominantly translates a finite distance in accordance with the results found for a single \bubble\ in \cref{fig:single_displacement}.
To quantify small changes in the geometry of a system, we compute $V_h$, the volume of the convex hull containing each \bubble, at every timestep in the simulation~\cite{SciPy}.
\Cref{fig:hull change} shows the fractional change in the hull volume,
\begin{equation}
  \Delta V_h = \frac{V_h(t) - V_h(0)}{V_h(0)},
\end{equation}
for 20 such systems after smoothing with a weighted moving average.
Curves ending above and below zero indicate larger and smaller hull volumes (system expansion and contraction).
We note from \cref{fig:hull change} a greater tendency for random clouds to expand; the effective dipole-dipole interaction between particles with $\vb{d}_{ij} \perp \vb{k}$ gives purely repulsive forces, while the interaction between particles with $\vb{d}_{ij} \parallel \vb{k}$ gives both repulsive and attractive effects depending on $\sigma$ and the relative phase of the oscillating \bubble\ velocities.

\begin{figure}
  \centering
  \begin{filecontents}{smoothed_hull_volume.dat}
#-1.105282567835826e-9	-9.887757852970709e-10	-7.007302383200791e-10	-7.331374115716991e-10	-1.0578987881025714e-9	-7.730450695265893e-10	-7.261432561388642e-10	-9.875824009893423e-10	-1.1139350431649264e-9	-8.23346740937668e-10	-8.319992619929949e-10	-1.0451113216153655e-9	-1.0242268614588405e-9	-8.535439707997063e-10	-1.0134665858876142e-9	-6.898643316755743e-10	-1.1701787898990004e-9	-1.2076321436447525e-9	-1.0859611486831655e-9	-1.0568559462905552e-9
#-1.2048475826244038e-9	-1.0777476532984468e-9	-7.634863761715217e-10	-7.99033926799145e-10	-1.153083808097113e-9	-8.424904540687726e-10	-7.914261132738025e-10	-1.076298637508414e-9	-1.2141849979325874e-9	-8.974379059137891e-10	-9.067628768476927e-10	-1.1390682534734742e-9	-1.1165226471241546e-9	-9.30389353030075e-10	-1.104613012211232e-9	-7.516889109315632e-10	-1.275799124141443e-9	-1.316577493619235e-9	-1.1837794954081948e-9	-1.1519504845998972e-9
#-1.304380234032343e-9	-1.166676779703902e-9	-8.261184164862472e-10	-8.647788374337249e-10	-1.2482177559870487e-9	-9.11807747537509e-10	-8.565985106665983e-10	-1.1649503168218757e-9	-1.314378101529342e-9	-9.71438264465302e-10	-9.814253047941458e-10	-1.2329652428631137e-9	-1.2087669803355176e-9	-1.0071522430469713e-9	-1.1957005395838664e-9	-8.133716983900648e-10	-1.381395042760825e-9	-1.4254961148659796e-9	-1.2815677810016886e-9	-1.2469928971792915e-9
#-1.4038780824692301e-9	-1.255574619033156e-9	-8.887477671122232e-10	-9.304830036354662e-10	-1.3433085639709142e-9	-9.811064616705844e-10	-9.217713469444335e-10	-1.253568478485961e-9	-1.4145197099681324e-9	-1.0454090420479283e-9	-1.056052093565069e-9	-1.3268176486566337e-9	-1.3009610775271095e-9	-1.0838760788683347e-9	-1.286739662465907e-9	-8.750514937758309e-10	-1.4869462416726691e-9	-1.5343706086506231e-9	-1.3793231913762508e-9	-1.3419896060292563e-9
#-1.5034378393922673e-9	-1.3445354139811827e-9	-9.514712214723359e-10	-9.962210689089084e-10	-1.4384530317486393e-9	-1.0504745138258543e-9	-9.870349944817412e-10	-1.3422626762783947e-9	-1.514712762715805e-9	-1.1194330113548545e-9	-1.1307292257188833e-9	-1.4207303561893417e-9	-1.3931938894911917e-9	-1.1606392954198203e-9	-1.37782511926989e-9	-9.368253442421613e-10	-1.5925419747850594e-9	-1.6432950472925444e-9	-1.477140244973349e-9	-1.4370359859754503e-9
#-1.6031143943795803e-9	-1.4335915539257556e-9	-1.0142447771943099e-9	-1.062001023092226e-9	-1.5336910436681845e-9	-1.1199048522846262e-9	-1.052375505034513e-9	-1.431036323321826e-9	-1.615000878635222e-9	-1.1935298248243946e-9	-1.2054662292242807e-9	-1.5147264943328313e-9	-1.485521392311992e-9	-1.2374751623802152e-9	-1.4689909423295155e-9	-9.986503618952987e-10	-1.6982729471564772e-9	-1.7523566403282106e-9	-1.5750731107967925e-9	-1.5321740888833354e-9
#-1.702730824908313e-9	-1.5225647476404202e-9	-1.0768621832462667e-9	-1.127670896683983e-9	-1.6288380449926056e-9	-1.189219289287214e-9	-1.117616303559913e-9	-1.5196764662362815e-9	-1.7151912194137905e-9	-1.267541861930759e-9	-1.280091449561236e-9	-1.6086040912250158e-9	-1.5777851395727162e-9	-1.3142353197197459e-9	-1.5600550398899146e-9	-1.0603233837279923e-9	-1.8039904567184687e-9	-1.8613945297228304e-9	-1.6729484472266607e-9	-1.6272219282740738e-9
#-1.8021306019461083e-9	-1.6113419235518588e-9	-1.1393506258892613e-9	-1.1931641427236376e-9	-1.7237601398681408e-9	-1.25838077456896e-9	-1.1827431247627242e-9	-1.6081193965160594e-9	-1.8151420853027879e-9	-1.3413872175017418e-9	-1.3545335532470087e-9	-1.7022600441182019e-9	-1.6698323284812435e-9	-1.390808332000712e-9	-1.650894666129876e-9	-1.1218716098790847e-9	-1.909478950716352e-9	-1.970194832203332e-9	-1.7706116542159965e-9	-1.7220403831195996e-9
#-1.9016203262663817e-9	-1.7002429251024385e-9	-1.202115043115069e-9	-1.258750500140733e-9	-1.818779492974198e-9	-1.3277268641898998e-9	-1.2480979180290984e-9	-1.69676255874383e-9	-1.9151899571454023e-9	-1.4153514416175603e-9	-1.4291041551648016e-9	-1.7960599115208749e-9	-1.7619332849812784e-9	-1.4674542702676745e-9	-1.7418280047476475e-9	-1.1836878067343973e-9	-2.0149763081212056e-9	-2.0790183823636565e-9	-1.8683625005601985e-9	-1.816942154456852e-9
#-2.0015958076375576e-9	-1.7895794097506261e-9	-1.2652051857358834e-9	-1.324627837200553e-9	-1.914249543005985e-9	-1.3974197644850937e-9	-1.3137722309374726e-9	-1.7858211531188335e-9	-2.015703618450139e-9	-1.489665756481735e-9	-1.504025310517616e-9	-1.8903039740938315e-9	-1.8544640627532038e-9	-1.5444626629316453e-9	-1.8331823193500681e-9	-1.245791783589193e-9	-2.1209713656330977e-9	-2.188357968360005e-9	-1.9665909842974706e-9	-1.9122828022683294e-9
#-2.1015849286515342e-9	-1.8788229405695012e-9	-1.3278795587244336e-9	-1.3903478413598673e-9	-2.009670355126051e-9	-1.4668516997568253e-9	-1.3790870987366718e-9	-1.8745468107991305e-9	-2.116144181489154e-9	-1.5638212860956077e-9	-1.5787681429378381e-9	-1.9843751226833093e-9	-1.947014655187193e-9	-1.6214355064359713e-9	-1.9244539311191917e-9	-1.3074434566137487e-9	-2.2271269884723e-9	-2.2978416145216264e-9	-2.0648367025124766e-9	-2.0075712346863395e-9
#-2.2007011709647677e-9	-1.9672349349664903e-9	-1.3897982627209763e-9	-1.4555177413674977e-9	-2.1042811391218963e-9	-1.5355622162406763e-9	-1.443634356942429e-9	-1.9623323700029317e-9	-2.2156684872779754e-9	-1.6372642088188023e-9	-1.6528352892342547e-9	-2.0775812476642175e-9	-2.038761444768996e-9	-1.697758210099788e-9	-2.014925146856853e-9	-1.3683363203902295e-9	-2.3323849199935204e-9	-2.4064228361722617e-9	-2.162228948933919e-9	-2.1020289500987485e-9
#-2.2995665399812553e-9	-2.0556232584145165e-9	-1.4523412937236281e-9	-1.5209218809031301e-9	-2.198832556230183e-9	-1.6045806185079102e-9	-1.5085115429379185e-9	-2.050375564265853e-9	-2.315051003004033e-9	-1.7107699670989315e-9	-1.727117139355052e-9	-2.170868952475365e-9	-2.13026369649458e-9	-1.7740326349591478e-9	-2.1053620811789173e-9	-1.4298397909466923e-9	-2.4371120696618093e-9	-2.514568133793818e-9	-2.2593761518136966e-9	-2.196361917588662e-9
#-2.3999991796436887e-9	-2.1455869590915203e-9	-1.5165037196656055e-9	-1.5876283927045132e-9	-2.2950138117036026e-9	-1.6751417215431653e-9	-1.5748668533316988e-9	-2.1402020797950394e-9	-2.416090657565966e-9	-1.7856816982988002e-9	-1.8028850803350526e-9	-2.2658888560113833e-9	-2.2232447487955396e-9	-1.8516336825087892e-9	-2.1973961882802142e-9	-1.4929261575623905e-9	-2.543345804591917e-9	-2.6243326339630245e-9	-2.35806483744167e-9	-2.292244054996357e-9
#-2.5013422261878403e-9	-2.2361220223533187e-9	-1.5800137032333673e-9	-1.654608167194144e-9	-2.391900880882098e-9	-1.7457380879017147e-9	-1.6410643442511322e-9	-2.2301308110725697e-9	-2.5178146269017997e-9	-1.8610066406306465e-9	-1.878949955247031e-9	-2.361373025712248e-9	-2.3172136432830217e-9	-1.929853058011289e-9	-2.2900852115256137e-9	-1.555342585193159e-9	-2.6511115357316043e-9	-2.735646196632682e-9	-2.457664456053689e-9	-2.388770539284591e-9
#-2.600115734887774e-9	-2.324103614864437e-9	-1.6404251022069261e-9	-1.7195410286771151e-9	-2.48618395755713e-9	-1.8138422251402471e-9	-1.7046823599315753e-9	-2.316941508131152e-9	-2.616650668032681e-9	-1.9341171455650706e-9	-1.9526821634589645e-9	-2.454007451155267e-9	-2.409021354198449e-9	-2.006077373682203e-9	-2.380193057664108e-9	-1.6146881141453507e-9	-2.756854141795621e-9	-2.844855893745094e-9	-2.5547778965390026e-9	-2.482486269556137e-9
#-2.6966857470404264e-9	-2.4106096864677062e-9	-1.7012105333659173e-9	-1.7838315688809133e-9	-2.5787452966386285e-9	-1.8816466252962946e-9	-1.7682115350111202e-9	-2.4028689714434485e-9	-2.7133605734296204e-9	-2.0062713778407797e-9	-2.0257308168549577e-9	-2.5454178947855724e-9	-2.4989555091837947e-9	-2.081071674504191e-9	-2.468685859769983e-9	-1.6743524283924704e-9	-2.8598308971059146e-9	-2.9514942313910077e-9	-2.649774460949521e-9	-2.5739331989849557e-9
#-2.7975606264670915e-9	-2.5018020255778022e-9	-1.767694530777574e-9	-1.8523757022966881e-9	-2.6760748140226115e-9	-1.9546194254552933e-9	-1.8368754451824796e-9	-2.494422176950971e-9	-2.814454806850889e-9	-2.082787403012216e-9	-2.1036894086397034e-9	-2.6422950361246742e-9	-2.5932039067900602e-9	-2.1601667889750665e-9	-2.561794226350802e-9	-1.7395113559217495e-9	-2.9666797984146884e-9	-3.0626711899537327e-9	-2.749062122243246e-9	-2.669182185824001e-9
#-2.9040703653686675e-9	-2.5978622234962707e-9	-1.8352552909323269e-9	-1.9245994754488183e-9	-2.7788330231739522e-9	-2.0309334951767533e-9	-1.908073135553111e-9	-2.5894786319407203e-9	-2.9201655274641372e-9	-2.1634053208813633e-9	-2.1858628347978934e-9	-2.7441255484765624e-9	-2.69364507858149e-9	-2.2441703728627235e-9	-2.659793655241966e-9	-1.8054392197572582e-9	-3.0813599393339735e-9	-3.1823765793805373e-9	-2.854024158684517e-9	-2.7683799616313835e-9
#-3.0055200357019356e-9	-2.6887194738728387e-9	-1.893418740751048e-9	-1.9927816476779294e-9	-2.8765813854127713e-9	-2.101618650415522e-9	-1.9726409051123073e-9	-2.676280638061793e-9	-3.0186552338719655e-9	-2.2395941175876633e-9	-2.2634945486265127e-9	-2.8399046944502454e-9	-2.7912176828068192e-9	-2.3250629450986392e-9	-2.752354344938879e-9	-1.8615785639614166e-9	-3.194698355209473e-9	-3.3013509708628742e-9	-2.954309005283762e-9	-2.8599919684032735e-9
#-3.097436930424649e-9	-2.7726090143032685e-9	-1.947600646241051e-9	-2.0576614833127567e-9	-2.966767371819569e-9	-2.1693755083876214e-9	-2.033903502817388e-9	-2.7557642705332633e-9	-3.1056522489868003e-9	-2.311071627113541e-9	-2.3376563742464468e-9	-2.929163111832511e-9	-2.882416729913873e-9	-2.4013436489251456e-9	-2.837193937233406e-9	-1.912983241703515e-9	-3.299862646479433e-9	-3.413981417769822e-9	-3.0456706984511727e-9	-2.93905659390966e-9
#-3.198202627589016e-9	-2.868519196409417e-9	-2.0174995206357835e-9	-2.13581469751898e-9	-3.0690558131203952e-9	-2.2536198982669618e-9	-2.110625993931596e-9	-2.849089187771832e-9	-3.1991601649695653e-9	-2.395107538691256e-9	-2.4274853263031755e-9	-3.0334711777267685e-9	-2.985712629390014e-9	-2.489952142148595e-9	-2.9331156477967798e-9	-1.978495617663799e-9	-3.415165868473572e-9	-3.5412002475330995e-9	-3.1464449210699165e-9	-3.021256519269988e-9
#-3.322099933035852e-9	-2.987930838642091e-9	-2.0992772071463245e-9	-2.2354990009875115e-9	-3.196880197026574e-9	-2.36028570269769e-9	-2.205531251988496e-9	-2.9611957915571217e-9	-3.3088872410276526e-9	-2.501142065532917e-9	-2.5425195061302974e-9	-3.163851335650017e-9	-3.1183663239239736e-9	-2.603828780004371e-9	-3.0515845695624697e-9	-2.053116826019035e-9	-3.5643835293247897e-9	-3.7091182068012957e-9	-3.2713139792514556e-9	-3.114093158579183e-9
#-3.444631821534344e-9	-3.1074209118125886e-9	-2.1587964563218355e-9	-2.3394028002483077e-9	-3.326914620852538e-9	-2.4671254446123007e-9	-2.2932718609969304e-9	-3.059032529400315e-9	-3.4027478542112896e-9	-2.6097341122036316e-9	-2.663447745666035e-9	-3.294224011311114e-9	-3.2643552091841162e-9	-2.727567884389364e-9	-3.168039115099947e-9	-2.1012081170054125e-9	-3.725666820571846e-9	-3.907690488682766e-9	-3.397319911163459e-9	-3.184106345558921e-9
#-3.53998835865392e-9	-3.2109226057766256e-9	-2.193588386075699e-9	-2.4432764846689034e-9	-3.441080457678778e-9	-2.5733109454341863e-9	-2.371933414254074e-9	-3.1278376583666195e-9	-3.4487727951226788e-9	-2.7119755150340624e-9	-2.7867310528400732e-9	-3.4116867127622144e-9	-3.4083017382241336e-9	-2.8517668809898127e-9	-3.263705510356393e-9	-2.1173895150237987e-9	-3.545070040528262e-9	-4.121464851694623e-9	-3.5006724639842675e-9	-3.1947032864184516e-9
#-3.649150526836755e-9	-3.3481054318156445e-9	-2.259122065958151e-9	-2.597755088975086e-9	-3.5904255396613675e-9	-2.739908314878907e-9	-2.49801117968286e-9	-3.2216570247049076e-9	-3.479162497162236e-9	-2.8572936426417674e-9	-2.972140850435439e-9	-3.5762771520081926e-9	-3.601878058019734e-9	-3.0262456379801655e-9	-3.3853771405655854e-9	-2.1514121547640264e-9	-1.6279602530399825e-9	-4.414201384625449e-9	-3.6241422455836163e-9	-3.1658825190312295e-9
#-3.8358778845339406e-9	-3.5820793341749925e-9	-2.3720396763330168e-9	-2.8600775817230924e-9	-3.84512267613035e-9	-3.0234425382681847e-9	-2.7130652036219064e-9	-3.3828444839182487e-9	-3.532850841188838e-9	-3.104440040010482e-9	-3.287394893954597e-9	-3.857049328350229e-9	-3.9308974768802185e-9	-3.322601589315303e-9	-3.593158813346674e-9	-2.210908391585796e-9	4.5595347115956475e-9	-4.911331145159709e-9	-3.834951554693199e-9	-3.1190914759858523e-9
#-4.066037871035353e-9	-3.8858541282854814e-9	-2.4420891163228972e-9	-3.2219154998309758e-9	-4.1833174397839964e-9	-3.402044838088326e-9	-2.9733744941971276e-9	-3.5396795971739947e-9	-3.530160367476963e-9	-3.4379237460059175e-9	-3.727169408935908e-9	-4.225505134377097e-9	-4.409104748350542e-9	-3.747531733826366e-9	-3.852999132800735e-9	-2.19245988551664e-9	1.8446650432797553e-8	-5.676582014342568e-9	-4.107299661758887e-9	-2.954144600816888e-9
#-4.25202784716831e-9	-4.2093955607953e-9	-2.4100895116424548e-9	-3.682611013822081e-9	-4.5539303411106365e-9	-3.878618410008226e-9	-3.2591937085558998e-9	-3.611881776901648e-9	-3.3252794088157317e-9	-3.837455429757299e-9	-4.299285220909534e-9	-4.644037665315899e-9	-5.0221216889453045e-9	-4.29736749332256e-9	-4.100152986580155e-9	-2.0153725203454764e-9	4.448966005890726e-8	-6.7408378847310865e-9	-4.366921907959266e-9	-2.4900735392900664e-9
#-4.460673776586575e-9	-4.673255357166904e-9	-2.386353100151603e-9	-4.401785955046936e-9	-5.083527874415219e-9	-4.6447054824494145e-9	-3.725204582949264e-9	-3.6998568834849775e-9	-2.904074747388588e-9	-4.444521773118676e-9	-5.200162172722643e-9	-5.274605594744047e-9	-5.933755309643022e-9	-5.133211399943171e-9	-4.434260596431145e-9	-1.7530868852942694e-9	8.769733880309005e-8	-8.354118000890656e-9	-4.69768399158875e-9	-1.6434002468330294e-9
#-4.896606756621836e-9	-5.5244384574885475e-9	-2.456893052611535e-9	-5.647643758444022e-9	-6.045264135019914e-9	-5.9858403910081036e-9	-4.578769350406823e-9	-3.960169007771654e-9	-2.322241217790448e-9	-5.514827846800742e-9	-6.756466032107559e-9	-6.411836062526072e-9	-7.50316376564586e-9	-6.570178361867823e-9	-5.075161969636055e-9	-1.4330467997154817e-9	1.5387861533719318e-7	-1.1071029821846032e-8	-5.330273421510481e-9	-3.6003254912965466e-10
#-5.575098655354136e-9	-6.843247960132545e-9	-2.4058347532587322e-9	-7.585017459472076e-9	-7.55272793078913e-9	-8.044597042900165e-9	-5.836153816152852e-9	-4.272616160665935e-9	-1.3377638480478737e-9	-7.177387489139438e-9	-9.1883988556315e-9	-8.172335033929069e-9	-1.0018951047377662e-8	-8.848474860700777e-9	-6.058178100560967e-9	-7.514552074127864e-10	2.49930978117477e-7	-1.5479693325336913e-8	-6.32725075008474e-9	1.7533998328787825e-9
#-6.274734578063865e-9	-8.57410953457457e-9	-1.9301382266014196e-9	-1.036478637850757e-8	-9.574385678017287e-9	-1.0975707943552829e-8	-7.486144920531754e-9	-4.345954853911434e-9	6.435714665180281e-10	-9.498723771582851e-9	-1.272293328390085e-8	-1.0574300382662277e-8	-1.368960531332496e-8	-1.2168058402995015e-8	-7.248379743478046e-9	7.297465195055947e-10	3.8416100159934267e-7	-2.2160073864991615e-8	-7.543661348607652e-9	5.520180604507222e-9
#-7.036370671016645e-9	-1.1030649093611631e-8	-1.0849615307735594e-9	-1.4558025239274448e-8	-1.2465412878483054e-8	-1.544125120294033e-8	-9.95014674688412e-9	-4.241154011157006e-9	4.109588888482653e-9	-1.2938248697321733e-8	-1.8100505548882765e-8	-1.410703374605984e-8	-1.916928613640654e-8	-1.7155654924818405e-8	-8.83903159294556e-9	3.1681802340009085e-9	5.659684111348714e-7	-3.232241282654161e-8	-9.128759243968651e-9	1.1809391914233483e-8
#-8.355737598846839e-9	-1.5028949565101997e-8	-5.0626967353605e-12	-2.124255395956169e-8	-1.714779355443331e-8	-2.2620877969934707e-8	-1.4001306130405453e-8	-4.3046916564084756e-9	9.395927022537442e-9	-1.845273107456565e-8	-2.66840707407838e-8	-1.984237805908605e-8	-2.78640779320812e-8	-2.5071749903233468e-8	-1.147846427770275e-8	6.755846032558896e-9	8.053387119874296e-7	-4.8353727021520354e-8	-1.1736152792191047e-8	2.1551481966435374e-8
#-1.0468572164733583e-8	-2.1234703218494795e-8	1.922683893714784e-9	-3.153311715897982e-8	-2.4450335292706087e-8	-3.364579845844249e-8	-2.012920046310263e-8	-4.311423309403921e-9	1.7589054157173684e-8	-2.696137465137827e-8	-3.994521338473913e-8	-2.8733611477901378e-8	-4.144253166096898e-8	-3.736122190220242e-8	-1.5582109500730844e-8	1.2590051581677121e-8	1.1139174342234405e-6	-7.34523886213121e-8	-1.5851051177954555e-8	3.668644475035045e-8
#-1.2879115566502477e-8	-2.991730858695489e-8	5.968919267814073e-9	-4.657201555111556e-8	-3.482293113610415e-8	-4.9702151878048325e-8	-2.8619089728394685e-8	-3.2853191620884747e-9	3.112917824421926e-8	-3.924132387381831e-8	-5.949769222370364e-8	-4.146913936692992e-8	-6.158910999799175e-8	-5.552836804671485e-8	-2.102545835129348e-8	2.2655628009911333e-8	1.5053881254063599e-6	-1.1139145145405531e-7	-2.1362773050374612e-8	6.089100763764874e-8
#-1.5279227158341603e-8	-4.200190798436072e-8	1.3013452405545621e-8	-6.853906297951641e-8	-4.943544548462698e-8	-7.323948121638977e-8	-4.065316180217759e-8	-6.074684456630272e-10	5.307800021140015e-8	-5.693737540095404e-8	-8.830609606334848e-8	-5.972983871485479e-8	-9.112602935979603e-8	-8.2175433954219e-8	-2.8145591584761574e-8	3.889439262197772e-8	1.9941404221954727e-6	-1.6794202494332163e-7	-2.8517222408532277e-8	9.909959080038264e-8
#-1.839037620923528e-8	-5.985144114304818e-8	2.373728814487994e-8	-1.0132053923411679e-7	-7.109293767424516e-8	-1.0856039486811667e-7	-5.8624744713419146e-8	3.655631505351028e-9	8.67368896667435e-8	-8.32663343470859e-8	-1.3150980263926097e-7	-8.702616800073038e-8	-1.3525066551653962e-7	-1.2196902733723456e-7	-3.847777288917874e-8	6.354932762824201e-8	2.594637345386695e-6	-2.52842666898618e-7	-3.882198846409961e-8	1.5740589576424338e-7
#-2.2700170438576077e-8	-8.612915698504617e-8	4.064053829847447e-8	-1.4974543204487504e-7	-1.0314865337300801e-7	-1.6080517734631773e-7	-8.482045457342213e-8	1.0585546385136809e-8	1.3758817161825076e-7	-1.221175373214554e-7	-1.9564310720363357e-7	-1.2748682559138232e-7	-2.010077552903406e-7	-1.8107077261677478e-7	-5.351009392506721e-8	1.0136643448552902e-7	3.3217875813855023e-6	-3.7993626740055416e-7	-5.391528912649453e-8	2.452640914851257e-7
#-2.6757278363053285e-8	-1.226644087985786e-7	6.887512012909952e-8	-2.1916207458725707e-7	-1.4830954160510394e-7	-2.356621804626847e-7	-1.209194471890314e-7	2.3746747931159985e-8	2.1564464569272658e-7	-1.7735077951978564e-7	-2.8830950326697686e-7	-1.8491856666938677e-7	-2.9646646977030686e-7	-2.665588139882271e-7	-7.335670822388615e-8	1.6068843168501594e-7	4.1918483351572635e-6	-5.667207633953866e-7	-7.402795886811691e-8	3.78258944498265e-7
-2.827007060324444e-8	-1.721157522929459e-7	1.1447170882039354e-7	-3.1735198612015567e-7	-2.103758201469006e-7	-3.4177632644990467e-7	-1.7005918604616016e-7	4.7513619065642015e-8	3.350855198019768e-7	-2.545761093006303e-7	-4.2057973878203145e-7	-2.650105149540418e-7	-4.3270155594267814e-7	-3.883393606242885e-7	-9.821633122476212e-8	2.520704830261914e-7	5.220589026224552e-6	-8.371895465213744e-7	-9.923860467480288e-8	5.785054173467619e-7
-2.6282650315807366e-8	-2.4028491010050104e-7	1.8405008460867136e-7	-4.5672383320186294e-7	-2.968740656306066e-7	-4.930050101230844e-7	-2.3834474584150003e-7	8.572217203848859e-8	5.134095230999744e-7	-3.633604600128155e-7	-6.097421027603263e-7	-3.780170068767067e-7	-6.27230496302972e-7	-5.619759162324938e-7	-1.3049303492028075e-7	3.8849900309514854e-7	6.419079027041158e-6	-1.2272322741813547e-6	-1.3185923485105433e-7	8.747645609839599e-7
-1.9393696829085355e-8	-3.346504062045799e-7	2.8903322798588414e-7	-6.535588226341853e-7	-4.17840702267887e-7	-7.07165953540093e-7	-3.3263646709412764e-7	1.4501687283988282e-7	7.75217099520739e-7	-5.162027988501073e-7	-8.787632234630814e-7	-5.372993173627943e-7	-9.04319642344389e-7	-8.086316940810731e-7	-1.730218398910202e-7	5.902152550654738e-7	7.792248445590804e-6	-1.78720302765361e-6	-1.750231289150472e-7	1.3071944532627158e-6
-1.2534401239649388e-9	-4.6113898280076533e-7	4.4945574185459606e-7	-9.265242691114926e-7	-5.825765454097091e-7	-1.0045892752581458e-6	-4.582729376720799e-7	2.393018485864724e-7	1.1588645297894368e-6	-7.263721960349713e-7	-1.2550753389008765e-6	-7.564728918922309e-7	-1.2931475897469858e-6	-1.15357524907653e-6	-2.2532382011592302e-7	8.893061791146524e-7	9.344122260408947e-6	-2.5817907852372407e-6	-2.286775781881902e-7	1.935251250863865e-6
3.954304705570532e-8	-6.256585171153845e-7	6.923303972130332e-7	-1.2994308434908573e-6	-8.013061235169045e-7	-1.4118722166799887e-6	-6.218500266981615e-7	3.886378630879887e-7	1.718845172138763e-6	-1.010122181825508e-6	-1.774377421714311e-6	-1.0521471417668489e-6	-1.8304838260484944e-6	-1.6288995534355979e-6	-2.845311938973869e-7	1.3291409963150469e-6	0.000011073547131212427	-3.69499727884186e-6	-2.8998341147745113e-7	2.84204910695968e-6
1.1549832706341944e-7	-8.389430005542719e-7	1.050469529025169e-6	-1.8062678245474274e-6	-1.0906721440813712e-6	-1.9674706881143018e-6	-8.350613599749874e-7	6.162920795666643e-7	2.5251529155076425e-6	-1.3915630649953867e-6	-2.487146311227204e-6	-1.4499587963556095e-6	-2.5680289115420445e-6	-2.279728689177422e-6	-3.4964352108905685e-7	1.964557254090064e-6	0.000012963408391259752	-5.242016348667295e-6	-3.5785197811258134e-7	4.13629851495691e-6
2.4212410491386126e-7	-1.1154351209317192e-6	1.5701346354210033e-6	-2.4913251226437464e-6	-1.4733380932961167e-6	-2.721080178350304e-6	-1.111693417177207e-6	9.540145523198303e-7	3.6700204101147303e-6	-1.9022862470622368e-6	-3.4596916334997483e-6	-1.9838701678759037e-6	-3.5754628560346062e-6	-3.1661162448931904e-6	-4.196604301790884e-7	2.871102633661383e-6	0.00001497968504875315	-7.3781039343066325e-6	-4.319033559585419e-7	5.961722932676145e-6
4.476076941880116e-7	-1.4660536061263628e-6	2.321669450254597e-6	-3.405475876924318e-6	-1.970963451668387e-6	-3.7295989256547896e-6	-1.4613396260541934e-6	1.4529649555221564e-6	5.2830054014685215e-6	-2.576293515062255e-6	-4.771142280097411e-6	-2.6896198099078667e-6	-4.937138297781264e-6	-4.3599632635948145e-6	-4.85205590759145e-7	4.157553801530046e-6	0.000017071602914874994	-0.000010300211147544361	-5.039677710014034e-7	8.51516159285819e-6
7.776499015915369e-7	-1.897001447887924e-6	3.4010698919482844e-6	-4.6083447253335725e-6	-2.603211611996375e-6	-5.0609191159249796e-6	-1.8909551101682726e-6	2.1866294193438433e-6	7.541261149396935e-6	-3.4507641241474023e-6	-6.517519826445986e-6	-3.605759711069102e-6	-6.75390662859221e-6	-5.947169207565808e-6	-5.259456824587867e-7	5.969855860153402e-6	0.000019168812796910685	-0.00001425393776074849	-5.544700752778788e-7	0.000012061052143917824
1.2906182385988881e-6	-2.417171933011008e-6	4.9280139493063e-6	-6.176872053593755e-6	-3.396240466635088e-6	-6.804118674558039e-6	-2.41169645674035e-6	3.246576692828761e-6	0.000010671414265011041	-4.573994994212936e-6	-8.823346969170985e-6	-4.7837765879300554e-6	-9.155158192466941e-6	-8.037974453161003e-6	-5.166608392294278e-7	8.49203271304004e-6	0.000021169390992617704	-0.00001955417983410851	-5.585296346569769e-7	0.00001693872065552471
2.0587236211749567e-6	-3.0385741800486747e-6	7.057591023279237e-6	-8.207496118229707e-6	-4.3846428493270515e-6	-9.070894473959845e-6	-3.0357514105738694e-6	4.749399560392684e-6	0.000014959930646416854	-6.006833352641041e-6	-0.000011846174697935325	-6.289863571046161e-6	-0.000012307010011753717	-0.0000107726285131164	-4.262319480523302e-7	0.000011961122194394417	0.000022925509744705432	-0.000026604863966025463	-4.864082150491338e-7	0.000023578345626420398
3.1851076087935435e-6	-3.7629160140639195e-6	0.000010002538345575804	-0.000010808352896866659	-5.598730723286421e-6	-0.000011986646233679257	-3.7629761654564636e-6	6.857926212054514e-6	0.00002078051838025623	-7.812737029205129e-6	-0.000015770140596948058	-8.192473460021262e-6	-0.000016406886886417637	-0.000014315619623762416	-2.0225257205614734e-7	0.000016692894592299878	0.000024243074711357995	-0.00003590371892727204	-2.8853092349117356e-7	0.00003253398149548152
4.819117169086996e-6	-4.5725530900545734e-6	0.000014043501168645064	-0.000014096778459612806	-7.055691315662977e-6	-0.000015689967672872162	-4.576596126618529e-6	9.794134726136532e-6	0.000028618979144837207	-0.000010052649308633151	-0.00002080613009649711	-0.000010556435954769374	-0.000021680139871270866	-0.000018853201403738566	2.4040092087814554e-7	0.00002309608377040588	0.00002489178110917325	-0.00004804748347198538	1.1758256869269404e-7	0.00004451530542725821
7.162339715631992e-6	-5.438013935009306e-6	0.000019526798503663245	-0.000018209516361465875	-8.768444686807693e-6	-0.00002034610806865019	-5.45195405354198e-6	0.000013835342562529984	0.000039080617638281596	-0.000012793912005739308	-0.000027206727799729957	-0.000013455072330659283	-0.000028393915450648918	-0.000024605479964025988	1.0142178607897934e-6	0.00003167464248556782	0.000024584242135905558	-0.00006375795076884875	8.427316235769646e-7	0.000060403575307737645
0.000010512811380793537	-6.321550414907658e-6	0.000026877748018813252	-0.00002330759840322591	-0.00001075155443173662	-0.000026151610994576996	-6.35473735530625e-6	0.000019320261254556154	0.00005290142107350278	-0.00001611449692144812	-0.000035275124451720905	-0.000016976283019860424	-0.00003687132155316889	-0.00003183645804443494	2.2574179885463653e-6	0.00004304567334979787	0.000022950721672911225	-0.00008390940784561472	2.02245016108735e-6	0.00008127243692343934
0.00001530427823226164	-7.156986776279043e-6	0.00003663589179743541	-0.000029562449530853296	-0.000013002374050669575	-0.00003331792352770755	-7.218555244527432e-6	0.000026680898647243406	0.00007098541857111773	-0.00002008679501280695	-0.00004535281070423795	-0.00002120296996624257	-0.00004748279857946921	-0.00004084398653613424	4.156385956588797e-6	0.000057979029893807856	0.000019550888034593994	-0.00010949340132798251	3.838047648432495e-6	0.00010844324183211971
0.000022104658882260008	-7.82680699374576e-6	0.000049478492816659324	-0.00003714206633153578	-0.000015477970243185688	-0.000042058888943366586	-7.928557423353646e-6	0.00003647067550900207	0.00009444591202896425	-0.000024760305301778084	-0.000057807771798838	-0.000026192833334400067	-0.00006062994930319625	-0.00005194559287835274	6.971257005998404e-6	0.00007742506382304442	0.000013882488263923935	-0.0001415166135913573	6.54434393080175e-6	0.00014354291469072982
0.000031626935517202915	-8.164852932040184e-6	0.00006621672410757516	-0.00004621716794708677	-0.000018098544460878257	-0.000052601711113016094	-8.32903359550916e-6	0.00004936251647644285	0.000124617202653224	-0.00003016620313039574	-0.00007304548536690364	-0.000031986404981982975	-0.00007675236191640077	-0.00006548486389337848	0.000011035300450510518	0.00010251496472695596	5.369731554182746e-6	-0.00018100243960311589	0.000010471445435955348	0.00018850754392975587
0.0000446822505276853	-7.965580091641286e-6	0.00008780053322830945	-0.00005696850415836914	-0.000020759139776717926	-0.00006519511790223676	-8.226843548860851e-6	0.00006614610564222179	0.00016305467045234383	-0.00003632523369878419	-0.00009152060193524815	-0.00003861913497106509	-0.00009634476739147832	-0.0000818437670227123	0.000016746650187999542	0.00013456903538434201	-6.647008409294819e-6	-0.0002291638610004661	0.000016014883725716316	0.0002455903421166807
0.00006221655281041497	-6.962324771619927e-6	0.00011535696112640416	-0.00006956732066300695	-0.00002330932186535733	-0.00008008540651051842	-7.363626760300376e-6	0.00008776060497825827	0.00021156437435254822	-0.00004322755875746548	-0.00011371734050319034	-0.00004609794657770758	-0.00011994472208954736	-0.00010142781495668792	0.000024590671737617604	0.00017513691687110774	-0.000022913912613133822	-0.00028739347450378895	0.000023655491691046286	0.0003173849388827086
0.00008532696574785482	-4.789479435116001e-6	0.00015022295665595637	-0.00008414501277393258	-0.000025513973341405194	-0.0000974847857367755	-5.384699261670981e-6	0.00011533468906028274	0.00027224712675822017	-0.00005079981802080887	-0.00014011719108137684	-0.00005436236769196954	-0.00014809649857964495	-0.00012463369585118052	0.000035178465906811945	0.0002260307127973338	-0.00004423568381456512	-0.0003572929521737546	0.000034000236547531745	0.0004068708479724381
0.00011529337130107849	-9.764225724950033e-7	0.00019393315938776002	-0.00010078820002803505	-0.000027045718874602576	-0.00011757148978005168	-1.8430542489305896e-6	0.0001501835611349936	0.00034750240387739926	-0.00005889969153476458	-0.00017119485618608533	-0.00006328116314778325	-0.00018133838501567783	-0.00015183978481415794	0.000049252503759206464	0.0002893116290729428	-0.00007145437016175193	-0.00044065251386670866	0.00004779269577686445	0.0005174142804532248
0.00015353413631246241	5.032468182080834e-6	0.0002481964565178991	-0.00011954903921555489	-0.000027505716522889485	-0.00014050235596735946	3.784272059513999e-6	0.00019378225618368932	0.00043999392954985827	-0.00006732977342383773	-0.00020742970068503454	-0.00007267301954876838	-0.00022021689729557364	-0.00018341723033965375	0.00006766399061718613	0.0003672633172159201	-0.00010546183966462437	-0.0005394813090476595	0.00006589073713809664	0.0006527306022319221
0.00020160108281504945	0.000013886299737522006	0.0003149113586045	-0.00014042773114996337	-0.000026415563321496163	-0.00016638901603264529	0.00001211511954171667	0.0002477745979154954	0.000552636006152693	-0.00007581934267864447	-0.0002492713680157073	-0.00008229152989100673	-0.00026527484446747405	-0.00021971517482355097	0.00009137728206791548	0.0004624009825097281	-0.0001472003121225918	-0.0006560279979572031	0.00008926781401636435	0.0008168644069956506
0.00026122021083525404	0.0000263690719505898	0.00039619060312276276	-0.00016332992976792778	-0.000023172401362376814	-0.0001952492909680491	0.00002390000272742345	0.00031400692929966514	0.0006886082787059659	-0.00008396943048311523	-0.0002970167966052309	-0.00009177503848725719	-0.00031699766763901695	-0.0002610127822313601	0.00012150686910991869	0.0005774865201878114	-0.0001975877871072091	-0.000792644311236163	0.00011905159488136596	0.0010141870390884785
0.0003342792545817003	0.00004341216966860696	0.0004943295949243272	-0.0001880490526644593	-0.000017031168547007433	-0.00022699228564619005	0.00004002006402286664	0.00039451507473229654	0.0008513310557644786	-0.00009123597515054164	-0.0003506776160497993	-0.0001006298728513637	-0.00037577471206752403	-0.00030748931816181654	0.00015932428673434007	0.0007154848968075394	-0.00025748463053395116	-0.0009517245294034683	0.0001565378534433166	0.0012493555040307138
0.0004228065489785349	0.00006605697475247849	0.0006117341546040163	-0.00021428355957517115	-7.139199256310427e-6	-0.0002614392715474133	0.00006144814607495549	0.0004914578483885164	0.0010443742465546003	-0.00009695662613008488	-0.0004100038264795091	-0.0001082637341238324	-0.0004419069494804562	-0.000359233526354682	0.00020621251800104993	0.0008794772585750121	-0.00032770463755722734	-0.001135711174003226	0.00020314813850874028	0.0015272049883355253
0.0005288759740497764	0.00009542876858341636	0.0007508812606025067	-0.00024163831337912535	7.439292695924932e-6	-0.0002983183219881861	0.00008923656094891918	0.0006070720784574456	0.0012713661632059742	-0.00010037666217163491	-0.0004746248272544467	-0.00011399979322714417	-0.000515607983959318	-0.00041624262876607045	0.0002636296625619511	0.0010726038310868297	-0.0004090068698999324	-0.0013470467108142332	0.00026039003876035614	0.0018526262681978975
0.000654624375157082	0.00013275800068633485	0.0009143128773032028	-0.00026958849747963916	0.000027749798621885726	-0.0003372171935115448	0.00012454234024516977	0.0007436733462165132	0.0015359469063521859	-0.00010061682210218796	-0.0005440454934403722	-0.00011703805601740976	-0.0005969530442476835	-0.00047837765568313847	0.0003331198939870413	0.0012980368578787097	-0.0005020284274745413	-0.0015880588842144428	0.0003298669611163903	0.002230482977601537
0.000802223468514429	0.00017939043503785706	0.0011045830799499105	-0.0002974574633029213	0.000054966323358380045	-0.0003775594843957099	0.00016862286491249514	0.0009036267013602627	0.0018417062655818137	-0.00009666440247486138	-0.0006176321431704942	-0.00011643242273930458	-0.0006858243382275235	-0.0005453225267451871	0.0004163134030933536	0.0015589051547907923	-0.0006072161679437828	-0.00186086043716572	0.00041328418085292886	0.0026655148010311103
0.0009738169097597438	0.00023673203141486238	0.001324139452399297	-0.0003244460691281632	0.00009034279214503596	-0.0004186387316861244	0.00022277289124248648	0.0010892429471879989	0.0021920453266719136	-0.00008740983178792032	-0.0006946620653632088	-0.00011113766593761005	-0.0007819156055570801	-0.0006165964646620124	0.0005148597129258229	0.0018581524012315428	-0.0007248142425256431	-0.0021672604882776248	0.0005123869640613734	0.003162163285445274
0.001171383487103785	0.0003061764335965094	0.0015752205126078386	-0.00034967003802162063	0.00013514363533277717	-0.0004596519583085995	0.0002882666076468982	0.001302670805005588	0.0025900137847567895	-0.0000717040652252767	-0.0007743246390251602	-0.00010006865032412184	-0.0008847629130234063	-0.0006915840990807799	0.0006303423919339687	0.002198411127576249	-0.0008548899390871129	-0.002508710804580391	0.0006288708987767632	0.0037243692141111714
0.0013967149014887017	0.0003890849622351948	0.001859813991302148	-0.00037215536036925277	0.00019062850848080648	-0.0004996821146043943	0.00036635241338058705	0.0015458527255500535	0.0030382060964205954	-0.000048361170855556936	-0.0008557399798089172	-0.00008210002047615841	-0.0009937290564873142	-0.0007695235150259581	0.0007642474752850622	0.002581937731791285	-0.0009973069468004986	-0.00288627016327561	0.0007643433691936685	0.004355424101090435
0.0016514056397642207	0.0004867886250533148	0.0021796063226684186	-0.0003908245599034887	0.00025806297699457626	-0.0005376780609455084	0.0004582475353460619	0.0018204921560609887	0.0035386909286053217	-0.00001615288857925593	-0.0009379350685317497	-0.000056047546346746965	-0.001107958559018341	-0.0008494751097998444	0.000917956469715599	0.0030105406174529663	-0.0011516376014489066	-0.003300479778119739	0.0009203168261213441	0.005057856006318376
0.001936895302732433	0.0006005326156111146	0.002535863018329313	-0.0004045348443207403	0.0003386705441663063	-0.0005724983651362726	0.00056507014654416	0.00212795193075514	0.004092888108845024	0.000026144548878285784	-0.001019842541820646	-0.00002071654362708784	-0.0012263851340611484	-0.0009303407906598848	0.001092683219972218	0.0034854430139806946	-0.001317169226219953	-0.003751251844236192	0.0010981486717185655	0.005833280949652115
0.0022543400341865196	0.0007313753671422591	0.00292930141120002	-0.0004121505046285213	0.0004335338862757137	-0.0006029865484054171	0.0006877489316949709	0.0024691211346073074	0.004701404976957698	0.00007966968391205859	-0.0011003946075519573	0.000025003099861729567	-0.0013477975961781069	-0.0010109295140020335	0.0012893659380139172	0.004007142648590723	-0.0014929570376596073	-0.004237796728598718	0.00129892769230787	0.006682224789449459
0.0026044647836952605	0.0008801306024745236	0.003360041935046655	-0.00041258035331320927	0.0005435372770256856	-0.0006279993926508527	0.0008269885620707422	0.002844349780596318	0.005363933425027464	0.00014542336927591453	-0.001178533715632035	0.00008209105697753278	-0.0014708819142284775	-0.0010899961724967796	0.0015086046876646476	0.0045753546088622005	-0.0016778311138417041	-0.004758494890750358	0.0015233987632591577	0.00760398756411417
0.0029874615094362764	0.0010473690641458035	0.0038276061001565236	-0.00040477662519493155	0.000669371836039699	-0.0006463973909760317	0.0009832758925354124	0.0032534478497477665	0.006079234025715554	0.0002242746219895536	-0.0012532148405458245	0.00015140909849451802	-0.0015942091188853869	-0.0011662360722765395	0.0017506630244227862	0.005189006458299495	-0.0018704007762259328	-0.005310924618603132	0.0017719568946470638	0.00859660621092781
0.0034029252992871534	0.001233397871197053	0.004330869914646561	-0.0003877628523743541	0.0008115189498985784	-0.0006570782356141715	0.0011568455332274137	0.003695651894809351	0.006845119594085106	0.0003169275631735749	-0.0013234508618643987	0.00023367934040957085	-0.0017162436156091263	-0.0012383031261256185	0.002015450952295732	0.005846195694161725	-0.002069040535708247	-0.005891920220281353	0.0020446302695296126	0.009656831560326689
0.003849790373514875	0.001438180543484191	0.004867984662383086	-0.00036071137273381744	0.0009701688539296822	-0.0006590650286342898	0.0013476001725540887	0.0041695406032130995	0.007658393599701788	0.00042384755869396846	-0.0013883547971956164	0.0003293950995284815	-0.0018354228667633405	-0.0013048876990500386	0.002302450325963456	0.006544121700286053	-0.0022719240269900767	-0.006497586890815418	0.002341002071984296	0.010780078966451598
0.004326275297810528	0.0016612742927460743	0.005436366405518092	-0.00032300463549176316	0.0011451515621277034	-0.0006515672410446036	0.0015550706181041546	0.0046729974664074326	0.008514840160818334	0.0005451964127111209	-0.001447210568357695	0.00043875235232459196	-0.0019502452447079854	-0.0013647926368390352	0.0026106595898464395	0.007279092619641346	-0.002477161566060874	-0.007123460748142782	0.0026601427106238158	0.011960416561708546
0.004829928821070587	0.0019018488926295768	0.006032776308794698	-0.0002742362400880603	0.0013359507102098573	-0.0006339745781665548	0.0017784489282362107	0.0052032731970723615	0.009409474624820066	0.0006808373936022893	-0.0014995288841931353	0.0005616654935664783	-0.002059292204065184	-0.0014169493412263079	0.0029386239985923985	0.008046628391719695	-0.0026828252510807307	-0.007764622584504464	0.003000629213013001	0.013190659505738736
0.005357720311853007	0.0021587317117152524	0.006653397344166985	-0.00021420212362306527	0.0015417462130826874	-0.0006058517728241405	0.0020166209067657955	0.0057570664999450905	0.010336840019914174	0.0008303746392757539	-0.001545043050162164	0.0006978001925120653	-0.0021612233847767576	-0.0014604154282750472	0.0032844957190452714	0.008841565360734853	-0.002886909340460138	-0.008415656969002781	0.0033606054677007384	0.01446252752882099
0.005906066200929795	0.0024303993397833774	0.007293857172725134	-0.0001429387945426114	0.0017614003957505753	-0.0005669920626175956	0.0022681438173525557	0.006330546626535345	0.011290959715136148	0.0009931274018963408	-0.0015836759487680685	0.0008465412641492863	-0.0022548276447885676	-0.0014944220438560772	0.0036460417032485926	0.00965811171360712	-0.003087414657528588	-0.009070891522304399	0.0037377938763509533	0.015766791510669363
0.006470867058102327	0.0027149491336890493	0.007949275786367963	-0.00006077242684307104	0.001993418129756504	-0.0005174753833845911	0.002531228370284768	0.006919376026904384	0.012265285046467189	0.0011680815338658856	-0.0016154821814146718	0.0010069406512991088	-0.002339112612923709	-0.0015184462400939275	0.004020631203649927	0.010489930535904916	-0.0032823989705467317	-0.009724396576058223	0.004129479279105503	0.017093414600622046
0.00704757107094975	0.003010141228198246	0.008614409103305265	0.00003169144290848867	0.0022359768500581465	-0.0004576545226680419	0.0028037974706751428	0.007518825468090963	0.013252806438292346	0.0013539077822036396	-0.0016405455150900283	0.0011777431206614099	-0.0024133405633638867	-0.0015322294983197175	0.004405294550335943	0.011330319523389972	-0.003470105706918055	-0.010370176964449257	0.004532561190641458	0.018431782114110135
0.007631360405163719	0.0033135040683285296	0.009283825804634927	0.000133623546785802	0.0024870244869566165	-0.00038809882542882395	0.003083583252527007	0.008123946670737358	0.014246199918037486	0.001549034884787917	-0.0016590295381017335	0.0013574741850533492	-0.0024769972710101213	-0.0015357409088253243	0.004796846937479491	0.01217242336342903	-0.003648922427583411	-0.011002419770623302	0.004943680794346183	0.019771014424122088
0.008217258196138496	0.003622408285921682	0.009952020483567426	0.00024399730708178722	0.0027443468094112966	-0.0003095809525472205	0.0033681759278138884	0.00872969329548665	0.015238022148527618	0.0017516948142917552	-0.00167117563166818	0.0015444866848508686	-0.00252978033157769	-0.0015291623066638793	0.00519198002653988	0.013009381219443106	-0.003817403567340381	-0.011615673815760914	0.005359324903435295	0.02110024231252326
0.008800212525114266	0.00393408412414189	0.010613488830748625	0.0003615769228308566	0.0030055740471338576	-0.00022310312214985587	0.003655033942591847	0.00933098535015099	0.016220922214593127	0.001959931584415792	-0.0016774099695073708	0.0017369487670822051	-0.002571644906907973	-0.0015129194770694735	0.005587294139979051	0.013834433675763874	-0.003974244496811935	-0.012205014801866042	0.005775866636341478	0.02240881651378974
0.009375190962738157	0.004245665797325214	0.011262860754453536	0.00048493730609851075	0.003268212457390944	-0.00012987761607950916	0.0039415409036397215	0.009922816306896166	0.017187817183986417	0.002171631051040214	-0.001678309669461795	0.001932868697533873	-0.0026028315009870937	-0.001487689705067051	0.0059793546392602846	0.014641083051427289	-0.004118394013809099	-0.012766219743696568	0.006189622644883347	0.023686532479626474
0.009937319671283735	0.0045543081313934525	0.011895086160343578	0.0006125439118821657	0.0035297528564858594	-0.0000312411821931321	0.00422512154914894	0.010500429725939616	0.018131949202057407	0.0023846195543383306	-0.0016745679708318147	0.0021301994999046586	-0.002623815291849518	-0.0014543414978852272	0.006364819450459167	0.0154233005903358	-0.004249125616508187	-0.013295789786993346	0.006596985194920441	0.024923911206962537
0.010482075416660224	0.00485730110415371	0.012505567741025535	0.0007428282822325644	0.00378778198443183	0.00007141541210307879	0.004503330942085847	0.011059466207739625	0.01904698985010928	0.0025967568784114625	-0.0016669054874989698	0.0023269375589042626	-0.002635234913235785	-0.0014138642575969311	0.006740562915310239	0.01617567502514935	-0.004366008123842017	-0.013790898828731236	0.006994560720997115	0.02611244344148332
0.011005361360004982	0.005152109899866218	0.013090206441817954	0.0008742089173422896	0.004040018270033552	0.00017665506754183742	0.0047738744394747595	0.011596016656802209	0.019927152964132618	0.0028059685415678043	-0.0016560991741643202	0.002521142931083301	-0.0026378797507115555	-0.0013673514480765665	0.007103721330402537	0.01689346834210005	-0.004468904908092618	-0.01424947301189435	0.007379230146289471	0.02724473488312943
0.011503593441365322	0.0054363872287914805	0.013645446554400182	0.0010051022329863899	0.004284317054276148	0.0002830018787591081	0.005034626132659111	0.012106653457548475	0.020767281349969555	0.0030102605918084354	-0.0016429804625557652	0.0027109403060778916	-0.002632704631326125	-0.0013160020591094024	0.007451706397118054	0.017572665045982604	-0.004557999741233441	-0.01467026427709469	0.007748166992771068	0.028314585697767097
0.011973841120954884	0.0057080388222935565	0.01416837510364658	0.0011339831626768176	0.004518731922450043	0.0003890107616195515	0.005283704069247389	0.012588518711644195	0.02156296612785437	0.003207780760910999	-0.0016283674464533503	0.002894585718472292	-0.002620787402326422	-0.0012610703145760022	0.007782269551891058	0.018210069194550503	-0.004633790130443121	-0.015052808545296226	0.008098902355789488	0.0293170820026378
0.01241392096507164	0.005965316278125004	0.014656802035626124	0.0012594655658227805	0.00474160981851556	0.0004933525838012292	0.005519549556730413	0.013039418236613278	0.022310667468874328	0.0033969100798742543	-0.0016129916853697781	0.0030705626654110537	-0.0026032380635111043	-0.001203780784817354	0.008093592655152592	0.0188033767155995	-0.004696986046966202	-0.015397312211440688	0.008429423404122489	0.030248693454988276
0.012822351923042603	0.006206847104550649	0.015109243805526001	0.0013803321034930867	0.0049516251347709785	0.0005948367777100295	0.00574093519466733	0.013457825925241328	0.023007737128566227	0.0035762863904196425	-0.0015975156219691345	0.003237610309215319	-0.002581148299885944	-0.0011452860784779153	0.008384313575377866	0.019351147158442328	-0.004748421508780358	-0.01570459104690299	0.00873821060709015	0.031107273974795263
0.013198253372913618	0.006431599091393413	0.015524859400436977	0.0014955197670996436	0.005147745045079562	0.0006923917870534016	0.005946929739543903	0.013842820169439635	0.02365233902073025	0.0037447932154954368	-0.00158247112300906	0.0033946888306265067	-0.002555601788250058	-0.0010866756756841116	0.008653482660148924	0.019852724528184186	-0.004789081018520333	-0.015976043383745392	0.009024197366266385	0.03189197108461632
0.013541292379670887	0.006638865497785332	0.015903430379632454	0.0016041298728441556	0.005329215460367565	0.0007850835382864555	0.006136901047537216	0.014194053054362313	0.024243372739981034	0.0039015505983693646	-0.0015682444991076178	0.0035409771315733233	-0.0025276670568400365	-0.0010289653525737451	0.008900538205331595	0.020308198929283608	-0.0048201022829654945	-0.016213611133268066	0.00928673961300323	0.03260313146096586
0.01385168650663008	0.006828298922058749	0.016245366220155308	0.0017054749481440162	0.005495600745564875	0.0008721736573180889	0.006310553134507869	0.014511762609092982	0.024780464877276243	0.004045970308759391	-0.0015550426230506922	0.0036759247460062218	-0.00249833293546849	-0.0009730406237675154	0.00912533108178858	0.020718391334830354	-0.0048426719863654995	-0.016419638973229993	0.009525637417081081	0.03324224195173567
0.014130186735501298	0.0069999185984656225	0.016551644818869227	0.0017990986608655381	0.005646800125960154	0.0009531391712259789	0.00646791822525363	0.014796736630673447	0.025264029693916133	0.004177765278474843	-0.0015429385644696743	0.003799271530092485	-0.0024684534255359006	-0.0009196167725704383	0.009328123015305619	0.021084773129527543	-0.00485797409930496	-0.01659672078292796	0.0097411370549433	0.03381181445517183
0.014377981320631221	0.007154045697095627	0.016823687020533693	0.001884739562552751	0.005782989364061164	0.0010276307716524714	0.006609287918709138	0.015050198162836856	0.025695384574485974	0.004296919314206608	-0.0015319731097189605	0.003910991495004318	-0.0024387600883695416	-0.0008692589479064861	0.009509511548615318	0.021409320272327458	-0.0048671790112802295	-0.016747649651310016	0.009933856466771005	0.0343151841090106
0.014596580128761064	0.007291229814531151	0.017063257891632926	0.001962293410209974	0.005904550542212296	0.0010954396858603266	0.006735158056077183	0.01527369844721439	0.026076628797068676	0.0044036396115369695	-0.0015221915795672774	0.004011234018232983	-0.0024098936942005536	-0.0008224135954366645	0.009670345409455012	0.0216943947964646	-0.00487143932764942	-0.016875298193164	0.010104691715188225	0.034756299000294186
0.014787757885752689	0.007412230373385686	0.017272426697743825	0.002031818072110618	0.006012058896654977	0.0011565149467326822	0.006846225126994518	0.015469074936576292	0.026410424324206233	0.004498329987858712	-0.0015135912603378315	0.004100326056534675	-0.0023823865177589865	-0.000779396010928599	0.0098116962688406	0.02194268590783796	-0.004871886820350489	-0.016982528661424107	0.010254779064166965	0.0351395750910429
0.014953525031504525	0.007518012349521093	0.017453511105365367	0.0020935418559479444	0.0061062873101974235	0.001210976432261509	0.00694337564392063	0.015638412750923005	0.02669991448848538	0.0045815815520705684	-0.0015060933772744044	0.004178784039333018	-0.0023566233290666203	-0.0007403653740674359	0.009934846859745411	0.022157137317537528	-0.004869527317270899	-0.017071904852061844	0.01038548121318171	0.03546977839582158
0.01509604927993217	0.007609689369323127	0.017608962743125886	0.0021478259325714566	0.006188156473751991	0.0012590724575019833	0.007027621233032677	0.01578394533806398	0.026948561938654816	0.0046541252539201335	-0.0014995805613187748	0.00424726507192574	-0.00233285214283044	-0.0007053459809584545	0.010041226912312465	0.022340818650176158	-0.004865250916244842	-0.017145729649229363	0.010498323589091911	0.03575186406495737
0.01521754361767848	0.007688439647484299	0.01774125903733783	0.0021951080596507343	0.0062586535437530386	0.0013011230830862008	0.0071000262211904	0.015907941819295023	0.02715996272994462	0.00471676329509284	-0.0014939205813703504	0.0043064914016551995	-0.002311234175336548	-0.0006742775977952432	0.01013232181080599	0.02249680549566468	-0.004859830890566773	-0.01720613674781725	0.010594896679936558	0.03599079658630956
0.01532020054229929	0.007755466555829357	0.017852861220730418	0.0022358824144253523	0.006318794917689052	0.0013375088756913925	0.007161685722392561	0.01601265668688076	0.027337829412076166	0.004770339570972261	-0.001488957039646818	0.004357224586334436	-0.002291862514967146	-0.0006470341561570004	0.010209627332366918	0.02262812758091657	-0.004853918750709715	-0.017255120753660216	0.010676801260080828	0.03619143008361733
0.015406179426125207	0.0078119992487517935	0.017946195099750296	0.0022707048656300816	0.006369631715992164	0.0013686825312417268	0.007213726355687424	0.01610031823712539	0.027486028343278946	0.0048157413857143035	-0.001484529456499238	0.004400277569104634	-0.0022747415239822094	-0.0006234109039336719	0.010274646857571857	0.02273774196634004	-0.004847933535851874	-0.017294484349698637	0.010745642296151919	0.03635844955109849
0.015477577208848433	0.007859268259884714	0.01802359216798816	0.002300172695236945	0.006412229957700785	0.0013951451126522395	0.007257271878949319	0.016173083112081984	0.02760843545027563	0.004853876755343766	-0.0014804839329504184	0.004436493503113959	-0.0022597885082986267	-0.0006031342879060757	0.010328864735137873	0.022828469222527976	-0.004842120296934736	-0.017325809563190346	0.010803004267279665	0.03649630354011882
0.015536362461768193	0.007898447668199589	0.018087218413828154	0.0023248789019234568	0.006447613536249808	0.0014173963712411455	0.007293388013835048	0.016232964142645877	0.02770867884063226	0.004885623783180215	-0.0014766993185790565	0.004466688180369895	-0.0022468767124740876	-0.0005859046875168004	0.01037368620080387	0.022902919482539176	-0.004836641217819364	-0.017350486471184173	0.010850389665281578	0.036609113866455115
0.015584332064524059	0.007930621659734056	0.018139053290089517	0.0023453882447805685	0.006476729760879428	0.001435914808906783	0.007323062358505723	0.016281798977263142	0.02779005158036043	0.004911802622924509	-0.0014731209099706863	0.0044916205800087355	-0.0022358668471789698	-0.0005714239251014214	0.010410402402950622	0.02296347010977786	-0.004831620359023655	-0.017369743915401307	0.010889178282953176	0.03670062163707064
0.01562312023605501	0.007956795188431242	0.018180906629978157	0.002362245270907645	0.006500460802153737	0.0014511711872832462	0.007347218373082515	0.016321266509448047	0.027855520350370037	0.004933184028593404	-0.001469740218391628	0.004512008167603737	-0.002226598188725159	-0.000559387710524426	0.010440201265988339	0.02301228261111961	-0.004827137129676109	-0.017384639542304126	0.01092063552556491	0.03677419282068799
0.015654212310314913	0.007977901189916004	0.01821441452909239	0.002375974961354292	0.0065196326187687	0.0014636273099775763	0.007366713566502666	0.01635289158441741	0.0279077399022711	0.004950492607658653	-0.0014665527158056473	0.004528533161269467	-0.0022188794793500735	-0.0005494810331128188	0.010464175974387957	0.023051299989704357	-0.0048232110256466905	-0.017396047521867835	0.0109459237441784	0.03683283232849158
0.01567892652073051	0.007994777246069888	0.018241011785095528	0.0023870593532785912	0.006534991091520198	0.0014737075734838204	0.007382312586076542	0.016378018798910095	0.02794903858472095	0.0049643840561318445	-0.001463618317881069	0.004541814553550976	-0.002212512294456091	-0.0005414009640637193	0.0104833027184221	0.023082221818894494	-0.004819821890224687	-0.017404686595715615	0.010966081968516337	0.03687917259551025
0.01569839630920454	0.008008146856461154	0.018261929560619068	0.0023959208551849083	0.006547180837489711	0.0014817824986241668	0.007394675897440152	0.016397801472759398	0.027981411679754188	0.0049754281741949844	-0.0014610310037717251	0.0045523871933768854	-0.0022073179575356787	-0.0005348785011817696	0.010498422961271368	0.023106505052459578	-0.004816942987004748	-0.017411161395305415	0.010982006033317438	0.03691547027572041
0.015713588892180667	0.008018634095597334	0.01827822728972491	0.0024029318394714855	0.00655675844141188	0.0014881819644551957	0.007404378697663868	0.016413228911444817	0.028006555116862168	0.0049841206377575255	-0.0014588765940138353	0.004560716649629442	-0.002203134459320105	-0.0005296732781851456	0.010510259873358612	0.02312539930181423	-0.004814544307963948	-0.017415973605642978	0.010994463366877776	0.03694364227164315
0.015725336993451307	0.008026785842067997	0.018290816906659055	0.002408427415076482	0.00656421471118112	0.0014932075543552296	0.007411927103322186	0.016425154940314705	0.028025910607031152	0.004990899822488317	-0.0014571565457539252	0.00456721916076992	-0.002199801933188714	-0.0005255607465521632	0.0105194429708022	0.023139974772859	-0.0048125774077551085	-0.017419516614738867	0.011004119948325206	0.03696531401218345
0.015734347672526437	0.008033071554463672	0.018300462262218112	0.002412700097762952	0.006569973669207479	0.001497123046445875	0.007417752326038905	0.016434299729430356	0.028040686252562837	0.004996144176266782	-0.0014558008103672752	0.004572256171595925	-0.0021971684558761166	-0.0005223377714708582	0.010526509993767915	0.0231511256435451	-0.004810979556147216	-0.017422091511831446	0.011011545074378345	0.03698184501149977
0.0157411993701743	0.00803787562032891	0.01830778367243677	0.0024159911309008745	0.006574382677907031	0.0015001441613676754	0.0074222058063105004	0.016441248978838738	0.028051867394969704	0.005000164607884454	-0.001454715366392379	0.004576122737781128	-0.0021951066838954585	-0.000519835386203666	0.010531900852413326	0.023159578434094656	-0.004809694557584071	-0.01742393896907313	0.011017205168688652	0.036994344677351784
0.015746357321070898	0.008041508595531356	0.01831328532038366	0.002418497892072345	0.006577721868250067	0.0015024479095547515	0.007425574258243285	0.016446476181849086	0.028060247366898043	0.005003213750714502	-0.00145383213688623	0.0045790574831384556	-0.002193513878662388	-0.0005179158653758019	0.010535970568989316	0.023165922862654165	-0.004808678832771974	-0.017425254340673338	0.01102147564121522	0.03700370657838204
0.015750202201453116	0.008044228719032664	0.0183173827948115	0.0024203880957914363	0.006580225160021124	0.0015041872635771147	0.007428098346660308	0.016450371478819344	0.02806646917408297	0.005005503320308418	-0.0014530975039435338	0.004581262432807948	-0.0021922987198321107	-0.0005164598311955501	0.010539012768802578	0.02317064257783304	-0.004807889746332986	-0.017426184501208905	0.011024665567680057	0.0370106531508912
#0.01575304582627923	0.008046250278319407	0.018320411596706374	0.002421803010709215	0.0065820881612535026	0.0015054918324465146	0.007429976281263939	0.016453252574484135	0.02807105110608101	0.0050072098169445	-0.0014524713929496306	0.004582907807302353	-0.0021913779994847038	-0.0005153632204669041	0.01054126976298968	0.023174126678967254	-0.004807281285708001	-0.017426832422479263	0.01102703004624038	0.03701576499508487
#0.01575513288384591	0.008047741210025196	0.01832263075975816	0.0024228536758725593	0.006583464432433819	0.0015064622743848391	0.007431361654681982	0.016455366034791974	0.028074397837549052	0.005008471763852285	-0.0014519336822121794	0.00458412644026817	-0.002190684690581763	-0.0005145432541016793	0.010542931343226269	0.023176676019939838	-0.0048068139625443975	-0.017427275043926403	0.01102876955419918	0.03701949563133397
#0.015756648730328636	0.008048828423000831	0.01832423856283293	0.0024236244804757144	0.006584469488089622	0.0015071746637448208	0.007432371469057208	0.016456899219306903	0.028076817926670137	0.005009394236010406	-0.0014514791976759569	0.004585017904523388	-0.0021901696852109635	-0.000513937912665151	0.01054414113072822	0.023178521150528028	-0.004806460497515994	-0.017427574954471396	0.01103003560039676	0.03702219153352765
#0.01575773703188188	0.008049611741848445	0.018325391758953057	0.0024241830960973062	0.006585194288243018	0.0015076911769234024	0.007433099415847799	0.01645799937511247	0.02807854957593317	0.005010060509121839	-0.0014511096254942772	0.004585661736699791	-0.0021897933633481075	-0.000513497285138166	0.010545011638118882	0.023179843410336253	-0.004806199105755983	-0.017427778535517024	0.011030946087402266	0.03702411964417473
#0.0157585122246717	0.008050172137515235	0.018326213664711252	0.002424585299260186	0.00658571333356897	0.0015080637725016417	0.007433621106256451	0.01645878351240935	0.028079778679229686	0.005010538434646024	-0.0014508224517427208	0.00458612393422264	-0.002189520272286044	-0.0005131787105856195	0.010545633486769804	0.023180784648899534	-0.004806007538812031	-0.017427914711150586	0.011031595839887286	0.03702548730666676
#0.015759061466686847	0.008050571183592644	0.018326795340434077	0.002424873500489728	0.00658608354940112	0.0015083313932924604	0.007433992825609919	0.01645933909337057	0.028080645276088823	0.005010879586412979	-0.0014506093235692758	0.004586454555143421	-0.0021893221423264215	-0.0005129490157119968	0.010546075501442327	0.023181449787344988	-0.004805866438092588	-0.01742800247740457	0.01103205730122098	0.0370264506553362
#0.01575944672870328	0.008050852137462778	0.018327201776647657	0.002425077441076783	0.006586344640536047	0.0015085208303115124	0.0074342541627072	0.016459728067973495	0.028081250180395	0.005011120254285007	-0.001450457055839204	0.0045866880571830745	-0.0021891800930215173	-0.0005127854077141278	0.010546386240654942	0.023181914082358424	-0.004805763509575903	-0.017428058037783075	0.011032381681036216	0.037027122576548124
#0.01575971287951971	0.008051046666812162	0.01832748198842317	0.002425219291521582	0.006586525530561838	0.0015086524534477772	0.0074344350344408144	0.016459996391760177	0.028081666924455972	0.005011287229528447	-0.001450350814863856	0.0045868498801213	-0.002189080705624219	-0.000512671232880474	0.010546601198514362	0.02318223404750973	-0.0048056908317335445	-0.017428094181842137	0.011032606050409711	0.037027585396246954
#0.015759894959381807	0.008051180216331572	0.018327674203103467	0.002425317238551326	0.006586649744069076	0.0015087434725017709	0.007434559574676857	0.016460180237791944	0.028081951469136515	0.0050114021583509225	-0.0014502771348447656	0.004586961219221355	-0.002189011957418762	-0.0005125922901722008	0.010546748615883957	0.02318245331990805	-0.004805640469723519	-0.01742811778938406	0.011032759738344259	0.03702790129545012
#0.0157600194742278	0.008051272073924363	0.018327805810663234	0.002425385036373286	0.006586735314907022	0.001508806730799407	0.007434645430963842	0.016460306175328843	0.028082145121730705	0.005011481401921966	-0.001450225613568146	0.004587038223975182	-0.002188963887475123	-0.0005125373698910972	0.010546849820098113	0.023182603168893062	-0.004805604914486625	-0.017428131995137326	0.011032865097357325	0.037028116029892734
#0.015760104085716806	0.008051334759758562	0.01832789469057807	0.0024254315037211837	0.00658679385240621	0.0015088501234076055	0.007434703849828356	0.016460391519785813	0.02808227585772619	0.005011535559468866	-0.001450189978472034	0.00458709099613545	-0.0021889303685654537	-0.0005124994094444043	0.0105469187614884	0.02318270423771327	-0.004805579609994513	-0.017428139897191504	0.01103293688253971	0.037028260824819914
#0.015760160311289507	0.008051376426430421	0.018327953402182615	0.002425462475126958	0.006586832783766796	0.0015088789301350184	0.007434742550380296	0.016460447993020853	0.02808236251319455	0.0050115716064566395	-0.0014501662301413848	0.004587126028594798	-0.002188907885185516	-0.0005124740172826726	0.010546964565614807	0.023182771024535222	-0.004805562444308508	-0.017428144735311367	0.011032984623066285	0.03702835681452378
#0.01576019699896875	0.008051403647400621	0.018327991939725306	0.0024254828119847707	0.006586858180784064	0.001508897832183922	0.007434767952577208	0.01646048492744396	0.028082419146445863	0.005011595220779129	-0.0014501506705631776	0.004587148886311086	-0.002188893254627885	-0.000512457404517237	0.01054699448577192	0.02318281486081342	-0.004805551374631559	-0.017428148094167366	0.011033015768995525	0.0370284195870494
#0.015760221129008434	0.00805142168507274	0.018328017520672635	0.0024254963925171447	0.006586875022101876	0.0015089105516636818	0.007434784915304602	0.016460509381980917	0.02808245627713069	0.005011610917853006	-0.001450140167856187	0.004587164139656672	-0.0021888834645194415	-0.0005124462975668127	0.010547014272406898	0.023182843882118286	-0.004805543990969609	-0.01742815004782137	0.011033036301648638	0.03702846068805036
#0.015760237126707835	0.008051433728420923	0.0183280343413938	0.0024255054998578956	0.006586886313519582	0.0015089191177250086	0.007434796196774656	0.016460525554760408	0.028082480639237788	0.005011621408951539	-0.0014501330129179488	0.004587174409953387	-0.0021888767227438025	-0.0005124387542912594	0.010547027446625716	0.02318286291585791	-0.004805538765853897	-0.017428150813449724	0.01103304997199635	0.037028487589382326
#0.01576024737841961	0.008051441424798083	0.01832804493442618	0.0024255113185386585	0.006586893540049843	0.001508924538907527	0.007434803325843268	0.0164605358007851	0.02808249618208266	0.0050116281134603095	-0.001450128454853336	0.004587180945864742	-0.0021888723517872752	-0.0005124338976072111	0.010547035865053605	0.023182874921539344	-0.004805535296683088	-0.01742815117596465	0.011033058738890905	0.03702850476057956
#0.015760253636589613	0.008051446095545403	0.018328051458177886	0.0024255148628751296	0.0065868979019938	0.0015089278092484232	0.007434807676371366	0.016460542058410636	0.028082505765810464	0.005011632195995074	-0.0014501257318497713	0.004587184865535604	-0.0021888697697843365	-0.0005124309785700738	0.010547040985789014	0.023182882338396948	-0.004805533297687444	-0.017428151604864488	0.011033064070730464	0.037028515385428236
#0.01576025750891327	0.008051449012517002	0.01832805563886518	0.002425517105249849	0.006586900615241047	0.001508929906680533	0.007434810461968823	0.0164605460111509	0.028082511733391408	0.005011634761958646	-0.0014501240007130812	0.0045871873284797195	-0.002188868188171703	-0.0005124291586593589	0.010547044179841364	0.02318288707611567	-0.004805532130189805	-0.017428151944159608	0.011033067370248612	0.037028522000856455
#0.015760260035334022	0.0080514509474707	0.018328058364836166	0.002425518604869734	0.006586902427562989	0.0015089313330191255	0.007434812313555146	0.01646054860418212	0.028082515561765258	0.005011636466876104	-0.001450122804337149	0.004587188995850868	-0.002188867088416103	-0.0005124279196416414	0.010547046286930723	0.023182890145176625	-0.004805531291624162	-0.017428152027036216	0.01103306953993024	0.037028526222256455
#0.015760261606281703	0.008051452142751157	0.01832805998635143	0.002425519526915622	0.006586903552461908	0.0015089321935258082	0.007434813424997221	0.016460550172584458	0.02808251791338378	0.0050116375175795905	-0.0014501220708339202	0.004587190018276736	-0.0021888663840952238	-0.0005124271419412882	0.010547047588031109	0.023182891976463994	-0.004805530721623805	-0.017428152023356534	0.011033070892096207	0.037028528816212436
#0.015760262439693327	0.008051452757153025	0.018328060851931744	0.002425520001742534	0.006586904120043059	0.0015089326154991854	0.007434813996860429	0.016460550992400944	0.028082519206109293	0.005011638061472048	-0.0014501217226617843	0.004587190516539491	-0.0021888660548653777	-0.0005124267574934955	0.010547048263696094	0.02318289296894799	-0.004805530471736488	-0.017428152120748868	0.011033071598748773	0.03702853026203427
#0.015760262853458787	0.008051453060848706	0.018328061350707407	0.0024255202465939584	0.006586904389146471	0.0015089328344850166	0.007434814310410752	0.016460551429787685	0.028082519895312262	0.005011638338077649	-0.0014501215575134076	0.004587190753054182	-0.002188865929495909	-0.0005124265818324503	0.01054704860076278	0.023182893542516556	-0.004805530415289672	-0.017428152268367598	0.011033071942449627	0.037028531043830325
#0.015760263100156706	0.008051453252317676	0.018328061679158077	0.0024255204078385663	0.006586904558740413	0.0015089329892810986	0.0074348145224608555	0.01646055171031537	0.02808252030519022	0.00501163851563723	-0.0014501214397192501	0.004587190911070949	-0.002188865848929662	-0.0005124264671872936	0.010547048810495778	0.0231828939135047	-0.00480553038612668	-0.017428152352941425	0.011033072150015899	0.03702853150542838
#0.01576026323508425	0.00805145335571846	0.01832806184830701	0.0024255204960227494	0.006586904650263456	0.0015089330699064925	0.007434814632849773	0.016460551855923135	0.02808252052631924	0.005011638612365115	-0.0014501213771875615	0.004587190993407429	-0.0021888658027230325	-0.0005124264027501835	0.010547048923570102	0.023182894106133986	-0.004805530364271493	-0.017428152394647095	0.011033072264039849	0.03702853175622574
#0.015760263244085117	0.00805145335310329	0.018328061877515282	0.002425520499577357	0.006586904638914563	0.0015089330613634211	0.007434814638954296	0.016460551864660923	0.028082520573755192	0.0050116386155441605	-0.0014501213939263496	0.004587190973493053	-0.0021888658294877588	-0.0005124264136259246	0.010547048924368441	0.023182894147682685	-0.004805530403318298	-0.01742815246962203	0.01103307226531072	0.03702853182292737
#0.015760263173791814	0.008051453289768983	0.018328061849349306	0.002425520458572957	0.006586904566840849	0.0015089330128071432	0.007434814599301089	0.016460551806011255	0.028082520518744988	0.005011638567214405	-0.0014501214503697837	0.004587190899392269	-0.002188865905065388	-0.000512426469825859	0.010547048860205835	0.02318289412497773	-0.004805530494722858	-0.017428152578195365	0.011033072195804313	0.03702853177981534
#0.015760263090936248	0.008051453222479878	0.0183280618177885	0.002425520416215811	0.00658690449221901	0.001508932969120326	0.007434814560856437	0.016460551743075293	0.02808252043937361	0.00501163851574815	-0.0014501215018202101	0.004587190828095484	-0.0021888659793711154	-0.000512426526206731	0.010547048790174549	0.02318289409460152	-0.004805530586196309	-0.017428152673936676	0.011033072117514963	0.03702853170610524
#0.01576026300860883	0.008051453156329498	0.018328061773450526	0.0024255203733254123	0.00658690442090207	0.001508932924786841	0.00743481451766285	0.01646055167471279	0.02808252035083303	0.005011638464075201	-0.0014501215505340961	0.004587190759782299	-0.002188866045172164	-0.0005124265784064377	0.010547048720573611	0.023182894049070275	-0.0048055306634270595	-0.01742815274901982	0.011033072041346695	0.03702853161999056
#0.01576026290744836	0.008051453072359674	0.018328061701799354	0.0024255203152205806	0.0065869043323989055	0.001508932862471952	0.00743481445397226	0.01646055158158909	0.028082520236327707	0.005011638396236261	-0.001450121614709621	0.004587190672649065	-0.002188866122789792	-0.000512426643403277	0.010547048632466173	0.023182893975004688	-0.00480553074854858	-0.017428152829575893	0.011033071947897898	0.037028531507233994
#0.015760262787683676	0.008051452973054303	0.01832806161750564	0.002425520245712741	0.0065869042280332045	0.0015089327889322359	0.0074348143787518715	0.01646055147239216	0.02808252010054905	0.005011638315467696	-0.0014501216905214065	0.004587190570502287	-0.0021888662145625663	-0.0005124267208313593	0.010547048528347338	0.023182893887553874	-0.0048055308493038015	-0.01742815292394658	0.01103307183719322	0.03702853137277447
#0.015760262667151907	0.0080514528748815	0.018328061538029138	0.0024255201778842447	0.00658690412471644	0.0015089327190269534	0.007434814307263457	0.016460551366261646	0.02808251996361044	0.005011638236018728	-0.001450121763560499	0.004587190471012651	-0.0021888663059150245	-0.0005124267973727851	0.01054704842509443	0.023182893804403623	-0.004805530951431091	-0.017428153017844665	0.01103307172619183	0.03702853123621382
#0.01576026255221918	0.008051452781835916	0.018328061461428933	0.0024255201138174776	0.006586904027024587	0.0015089326532658895	0.007434814239367475	0.01646055126486327	0.02808251983142118	0.005011638160775388	-0.0014501218319760913	0.004587190377205438	-0.002188866391467968	-0.0005124268692457557	0.010547048327000335	0.023182893723849872	-0.004805531046778095	-0.017428153104049954	0.011033071620741752	0.03702853110378786
#0.01576026243577652	0.008051452686746088	0.018328061379619936	0.0024255200476382905	0.00658690392752914	0.0015089325844359572	0.007434814167846504	0.016460551159628353	0.028082519696513927	0.005011638083481949	-0.0014501219026838174	0.004587190280850515	-0.002188866477735722	-0.0005124269423927278	0.010547048226842368	0.023182893637988235	-0.004805531141436165	-0.017428153189543234	0.01103307151387278	0.03702853096868608
#0.015760262314357905	0.008051452587209438	0.018328061293754083	0.0024255199780259472	0.006586903823406764	0.0015089325118643978	0.007434814092580496	0.016460551049546263	0.0280825195562047	0.005011638002396939	-0.0014501219771677547	0.004587190179812382	-0.0021888665681717347	-0.0005124270192198902	0.010547048122118931	0.023182893548018373	-0.004805531240473619	-0.017428153279411267	0.011033071402272287	0.03702853082831515
#0.01576026219214408	0.008051452487447446	0.01832806120928016	0.002425519908520611	0.006586903718931622	0.001508932439919072	0.0074348140181261	0.016460550939991197	0.028082519415316844	0.0050116379212647635	-0.0014501220514022578	0.004587190078879646	-0.002188866659286634	-0.0005124270963729749	0.010547048017109302	0.023182893459381324	-0.004805531340939312	-0.017428153370436426	0.01103307128997473	0.037028530687241046
#0.015760262072089942	0.008051452389742	0.018328061126778716	0.0024255198406358948	0.006586903616612253	0.0015089323698621935	0.007434813945522588	0.016460550832727246	0.02808251927664422	0.005011637841893469	-0.00145012212375269	0.004587189980231435	-0.0021888667484346895	-0.0005124271717945065	0.010547047914183543	0.023182893372678235	-0.004805531439398628	-0.017428153459262533	0.01103307117977686	0.03702853054822358
#0.01576026195254476	0.008051452292272259	0.01832806104355013	0.00242551977277844	0.006586903514623806	0.001508932299597484	0.007434813872567302	0.016460550725266504	0.028082519138255154	0.005011637762636755	-0.0014501221960863883	0.00458718988169691	-0.0021888668370394935	-0.0005124272469125146	0.010547047811515404	0.02318289328524348	-0.004805531536877288	-0.01742815354714488	0.011033071070057453	0.037028530409498495
#0.01576026183176424	0.008051452193611185	0.018328060958996577	0.002425519703956492	0.0065869034114099684	0.0015089322281917747	0.007434813798450153	0.0164605506163919	0.0280825189985051	0.005011637682342646	-0.0014501222695148018	0.00458718978184181	-0.0021888669267016076	-0.0005124273230017744	0.010547047707636084	0.02318289319648744	-0.004805531635357705	-0.017428153636095383	0.01103307095914808	0.037028530269481164
#0.01576026171044855	0.008051452094584183	0.01832806087455782	0.002425519634926355	0.0065869033077766885	0.0015089321566804136	0.007434813724291462	0.016460550507335178	0.02808251885828998	0.005011637601777141	-0.001450122343157611	0.004587189681673977	-0.002188867016854661	-0.0005124273994428074	0.010547047603371678	0.02318289310784039	-0.004805531734550513	-0.01742815372572525	0.011033070847735952	0.03702853012899443
#0.01576026158967182	0.00805145199610626	0.018328060790802663	0.0024255195663539786	0.006586903204704081	0.0015089320857377762	0.007434813650716953	0.016460550398965754	0.02808251871867453	0.005011637521696038	-0.0014501224162700597	0.004587189582135774	-0.002188867106538349	-0.0005124274754408692	0.010547047499661765	0.023182893019869505	-0.004805531833333949	-0.017428153814893266	0.011033070736848245	0.03702852998906485
#0.015760261469238354	0.008051451897887979	0.01832806070706888	0.0024255194979474048	0.00658690310192185	0.001508932014924122	0.007434813577237174	0.016460550290775432	0.028082518579365695	0.005011637441816678	-0.0014501224892004845	0.0045871894828430565	-0.0021888671959028843	-0.0005124275511979092	0.010547047396221232	0.023182892931919573	-0.004805531931688822	-0.017428153903636057	0.01103307062628697	0.03702852984943516
#0.015760261348563984	0.00805145179941321	0.018328060622970405	0.0024255194293196884	0.00658690299888205	0.0015089319438261568	0.007434813503458496	0.01646055018224223	0.02808251843977429	0.005011637361706876	-0.00145012256238655	0.004587189383250591	-0.002188867285470554	-0.000512427627156313	0.010547047292522243	0.023182892843607562	-0.004805532030198417	-0.017428153992558284	0.01103307051549116	0.037028529709542755
#0.01576026122767072	0.008051451700761627	0.018328060538801184	0.002425519360569727	0.006586902895649574	0.0015089318726158292	0.007434813429582313	0.01646055007356	0.028082518299974536	0.005011637281454486	-0.0014501226357078287	0.004587189283482715	-0.0021888673752365226	-0.0005124277032725207	0.010547047188640803	0.023182892755224768	-0.0048055321289553745	-0.01742815408172819	0.011033070404486771	0.03702852956944774
#0.015760261106865833	0.008051451602213171	0.018328060454805603	0.0024255192919132077	0.00658690279251855	0.001508931801532889	0.0074348133558428415	0.01646054996502872	0.02808251816028492	0.005011637201297115	-0.0014501227089196217	0.0045871891838405905	-0.002188867464929182	-0.0005124277793104518	0.010547047084862532	0.023182892667013944	-0.004805532227670875	-0.017428154170843167	0.011033070293569347	0.037028529429453186
#0.015760260986183716	0.0080514515037686	0.01832806037087012	0.0024255192233316376	0.00658690268949926	0.0015089317305249138	0.007434813282171999	0.016460549856593946	0.02808251802071537	0.00501163712122515	-0.0014501227820474074	0.004587189084305055	-0.0021888675545110974	-0.0005124278552575044	0.010547046981190933	0.02318289257886283	-0.004805532326255303	-0.01742815425982453	0.011033070182769424	0.03702852928957422
#0.015760260865474492	0.008051451405286899	0.01832806028685552	0.0024255191547136717	0.0065869025864448505	0.0015089316594634962	0.007434813208442025	0.016460549748096777	0.028082517881107102	0.005011637041117476	-0.0014501228552182856	0.004587188984721458	-0.0021888676441139733	-0.0005124279312309437	0.01054704687748257	0.023182892490633288	-0.0048055324248416674	-0.017428154348814014	0.01103307007194243	0.03702852914966062
#0.01576026074469984	0.00805145130674782	0.018328060202799132	0.0024255190860528564	0.006586902483329457	0.0015089315883567538	0.00743481313466902	0.016460549639542136	0.02808251774143294	0.005011636960961837	-0.0014501229284374715	0.004587188885077556	-0.0021888677337752126	-0.0005124280072536651	0.010547046773715481	0.02318289240236224	-0.004805532523493717	-0.017428154437871244	0.011033069961052706	0.03702852900968396
#0.015760260623928882	0.008051451208218569	0.018328060118775507	0.002425519017403618	0.0065869023802223895	0.0015089315172696547	0.007434813060918779	0.01646054953100976	0.028082517601768078	0.005011636880816704	-0.0014501230016425474	0.004587188785448476	-0.0021888678234346834	-0.000512428083270638	0.010547046669957861	0.02318289231412364	-0.004805532622154411	-0.01742815452693431	0.01103306985016707	0.03702852886971493
#0.015760260503190293	0.008051451109718695	0.018328060034776928	0.002425518948776903	0.006586902277146241	0.001508931446207461	0.007434812987192771	0.01646054942250855	0.0280825174621366	0.0050116368006964465	-0.0014501230748220574	0.004587188685850913	-0.00218886791306545	-0.0005124281592627653	0.010547046566230225	0.023182892225909895	-0.004805532720784298	-0.01742815461596516	0.011033069739312576	0.037028528729777674
#0.015760260382454528	0.008051451011218269	0.01832805995076608	0.0024255188801477973	0.006586902174070489	0.0015089313751393322	0.007434812913459223	0.016460549314001042	0.02808251732250537	0.005011636720574669	-0.0014501231480047324	0.004587188586250697	-0.0021888680026929588	-0.0005124282352541089	0.010547046462502305	0.02318289213768402	-0.004805532819405562	-0.017428154704988544	0.011033069628460545	0.037028528589841216
#0.015760260261703643	0.008051450912703689	0.018328059866741495	0.0024255188115075454	0.006586902070979989	0.0015089313040584263	0.0074348128397129795	0.016460549205477706	0.028082517182858174	0.005011636640440753	-0.001450123221200107	0.004587188486635011	-0.0021888680923337544	-0.000512428311257241	0.010547046358759991	0.0231828920494445	-0.004805532918040314	-0.01742815479402627	0.011033069517593941	0.03702852844988966
#0.015760260140949247	0.008051450814187355	0.018328059782720747	0.0024255187428668177	0.0065869019678872045	0.0015089312329785106	0.0074348127659684925	0.016460549096955098	0.028082517043208553	0.005011636560305821	-0.0014501232943957964	0.004587188387018376	-0.0021888681819779974	-0.00051242838726243	0.010547046255015758	0.023182891961208776	-0.004805533016681047	-0.017428154883069577	0.011033069406724086	0.03702852830993552
#0.01576026002020144	0.00805145071567741	0.018328059698706985	0.0024255186742312435	0.0065869018648010225	0.0015089311619047342	0.0074348126922302036	0.016460548988440026	0.02808251690356608	0.005011636480176433	-0.0014501233675855694	0.004587188287408861	-0.002188868271616509	-0.0005124284632623806	0.010547046151278015	0.023182891872979998	-0.004805533115316273	-0.017428154972106917	0.011033069295860603	0.037028528169988105
#0.015760259899456175	0.00805145061716919	0.01832805961469243	0.0024255186055966173	0.006586901761716834	0.0015089310908313707	0.007434812618491972	0.016460548879925697	0.028082516763925713	0.005011636400048299	-0.0014501234407743593	0.004587188187800786	-0.002188868361252618	-0.0005124285392606386	0.010547046047542074	0.02318289178475044	-0.004805533213947926	-0.017428155061140815	0.011033069184999541	0.037028528030042814
#0.015760259778708232	0.008051450518658266	0.018328059530674547	0.0024255185369597564	0.006586901658629892	0.0015089310197552528	0.007434812544750899	0.016460548771408	0.02808251662428235	0.005011636319917794	-0.0014501235139657335	0.00458718808818962	-0.002188868450891034	-0.0005124286152610722	0.010547045943803398	0.023182891696517566	-0.0048055333125816046	-0.017428155150176965	0.011033069074135881	0.03702852789009474
#0.01576025965795874	0.008051450420146173	0.018328059446656556	0.0024255184683221564	0.0065869015555416535	0.0015089309486785738	0.007434812471009402	0.01646054866288951	0.0280825164846376	0.00501163623978639	-0.001450123587157917	0.004587187988577357	-0.0021888685405308684	-0.0005124286912625838	0.010547045840063494	0.023182891608284593	-0.004805533411217194	-0.017428155239215003	0.011033068963270769	0.03702852775014529
#0.01576025953721026	0.008051450321635128	0.01832805936264006	0.002425518399685471	0.0065869014524544784	0.0015089308776030713	0.0074348123972691315	0.016460548554372446	0.028082516344994014	0.00501163615965594	-0.0014501236603490439	0.0045871878889663185	-0.002188868630169854	-0.0005124287672632642	0.010547045736324638	0.023182891520053102	-0.0048055335098521395	-0.017428155328252287	0.01103306885240662	0.03702852761019689
#0.015760259416462582	0.008051450223124736	0.018328059278623816	0.0024255183310492525	0.006586901349368027	0.0015089308065279962	0.007434812323529219	0.01646054844585593	0.028082516205351214	0.005011636079526032	-0.0014501237335396533	0.004587187789355927	-0.0021888687198080904	-0.0005124288432633464	0.010547045632586455	0.02318289143182184	-0.004805533608486151	-0.017428155417288614	0.011033068741543259	0.037028527470249255
#0.015760259295714583	0.00805145012461396	0.01832805919460694	0.002425518262412703	0.006586901246281201	0.0015089307354524692	0.007434812249788818	0.01646054833733888	0.028082516065708007	0.00501163599939579	-0.0014501238067306463	0.004587187689745078	-0.0021888688094465997	-0.0005124289192637139	0.010547045528847878	0.023182891343589963	-0.004805533707120329	-0.017428155506325153	0.01103306863067957	0.037028527330301264
#0.015760259174966186	0.008051450026102837	0.01832805911058987	0.002425518193775916	0.006586901143193993	0.0015089306643767013	0.0074348121760481915	0.016460548228821526	0.028082515926064425	0.00501163591926527	-0.0014501238799219014	0.0045871875901338744	-0.002188868899085481	-0.0005124289952643861	0.010547045425108939	0.023182891255357895	-0.004805533805754951	-0.017428155595362125	0.011033068519815491	0.03702852719035287
#0.015760259054217875	0.008051449927591832	0.018328059026573047	0.0024255181251392533	0.006586901040106885	0.0015089305933011078	0.007434812102307753	0.016460548120304362	0.028082515786420965	0.005011635839134863	-0.0014501239531130047	0.004587187490522816	-0.0021888689887242982	-0.0005124290712649793	0.010547045321370106	0.02318289116712607	-0.004805533904389579	-0.017428155684399073	0.01103306840895149	0.03702852705040459
#0.01576025893346977	0.00805144982908098	0.018328058942556336	0.00242551805650272	0.006586900937019942	0.0015089305222256559	0.007434812028567434	0.016460548011787368	0.028082515646777682	0.005011635759004588	-0.0014501240263039563	0.004587187390911922	-0.0021888690783629515	-0.0005124291472654342	0.010547045217631426	0.02318289107889436	-0.004805534003024041	-0.01742815577343582	0.011033068298087662	0.037028526910456494
#0.015760258812721663	0.008051449730570096	0.018328058858539532	0.002425517987866167	0.006586900833932959	0.0015089304511501657	0.007434811954827051	0.016460547903270315	0.028082515507134375	0.005011635678874282	-0.0014501240994949413	0.004587187291300984	-0.0021888691680016044	-0.0005124292232659035	0.010547045113892704	0.02318289099066255	-0.004805534101658511	-0.01742815586247256	0.011033068187223827	0.03702852677050837
#0.0157602586919735	0.008051449632059131	0.01832805877452266	0.002425517919229572	0.0065869007308458774	0.0015089303800746299	0.007434811881086596	0.016460547794753186	0.028082515367490974	0.0050116355987439125	-0.0014501241726859913	0.004587187191689973	-0.0021888692576403114	-0.0005124292992664272	0.010547045010153888	0.02318289090243069	-0.004805534200293096	-0.01742815595150937	0.011033068076359903	0.03702852663056017
#0.015760258571225323	0.008051449533548166	0.018328058690505807	0.0024255178505929965	0.006586900627758769	0.0015089303089991276	0.00743481180734614	0.01646054768623608	0.02808251522784756	0.005011635518613549	-0.0014501242458770409	0.00458718709207898	-0.002188869347278992	-0.0005124293752669396	0.010547044906415054	0.023182890814198855	-0.004805534298927738	-0.01742815604054618	0.011033067965495977	0.037028526490611946
#0.01576025845047719	0.008051449435037217	0.01832805860648894	0.0024255177819564595	0.006586900524671658	0.0015089302379236718	0.007434811733605681	0.016460547577719007	0.028082515088204172	0.005011635438483216	-0.0014501243190680713	0.004587186992468036	-0.002188869436917594	-0.0005124294512674091	0.01054704480267625	0.02318289072596705	-0.00480553439756239	-0.017428156129582928	0.011033067854632093	0.03702852635066374
#0.01576025832972907	0.008051449336526257	0.018328058522472015	0.002425517713319942	0.006586900421584515	0.0015089301668482363	0.00743481165986518	0.016460547469201933	0.028082514948560785	0.005011635358352891	-0.0014501243922591066	0.004587186892857116	-0.0021888695265561246	-0.0005124295272678545	0.01054704469893745	0.023182890637735234	-0.004805534496197067	-0.01742815621861963	0.01103306774376822	0.03702852621071552
#0.015760258208980922	0.008051449238015271	0.01832805843845501	0.0024255176446834354	0.006586900318497315	0.0015089300957728123	0.007434811586124626	0.016460547360684842	0.028082514808917387	0.005011635278222571	-0.0014501244654501542	0.004587186793246222	-0.0021888696161945836	-0.0005124296032682894	0.010547044595198628	0.023182890549503416	-0.004805534594831788	-0.01742815630765631	0.011033067632904351	0.03702852607076729
#0.015760258088232764	0.008051449139504271	0.018328058354437935	0.0024255175760469518	0.0065869002154100575	0.0015089300246974107	0.007434811512384023	0.016460547252167737	0.028082514669273983	0.005011635198092262	-0.001450124538641204	0.004587186693635368	-0.0021888697058329502	-0.0005124296792687086	0.010547044491459805	0.02318289046127161	-0.004805534693466546	-0.017428156396692946	0.011033067522040488	0.03702852593081906
#0.015760257967484613	0.008051449040993253	0.01832805827042078	0.0024255175074104963	0.006586900112322746	0.0015089299536220355	0.007434811438643372	0.016460547143650642	0.028082514529630585	0.005011635117961974	-0.001450124611832251	0.004587186594024566	-0.002188869795471204	-0.0005124297552691057	0.010547044387721001	0.023182890373039836	-0.004805534792101327	-0.017428156485729512	0.011033067411176645	0.03702852579087084
#0.015760257846736438	0.00805144894248221	0.018328058186403543	0.0024255174387740695	0.006586900009235371	0.0015089298825466805	0.007434811364902663	0.016460547035133537	0.028082514389987163	0.005011635037831707	-0.001450124685023301	0.004587186494413816	-0.002188869885109341	-0.0005124298312694847	0.010547044283982206	0.023182890284808074	-0.004805534890736133	-0.017428156574766016	0.011033067300312822	0.037028525650922614
#0.015760257725988245	0.008051448843971135	0.01832805810238621	0.0024255173701376712	0.006586899906147925	0.0015089298114713417	0.007434811291161896	0.01646054692661641	0.028082514250343717	0.00501163495770146	-0.0014501247582143623	0.004587186394803118	-0.0021888699747473575	-0.0005124299072698467	0.01054704418024343	0.023182890196576343	-0.004805534989370969	-0.01742815666380245	0.011033067189449021	0.037028525510974374
#0.015760257605240035	0.008051448745460042	0.018328058018368794	0.002425517301501304	0.00658689980306041	0.001508929740396022	0.007434811217421075	0.016460546818099307	0.02808251411070024	0.005011634877571234	-0.0014501248314054429	0.004587186295192467	-0.0021888700643852444	-0.0005124299832701886	0.010547044076504676	0.02318289010834465	-0.004805535088005842	-0.017428156752838812	0.011033067078585241	0.037028525371026126
#0.0157602574844918	0.00805144864694892	0.01832805793435129	0.0024255172328649737	0.006586899699972825	0.0015089296693207269	0.0074348111436801905	0.016460546709582184	0.028082513971056735	0.005011634797441037	-0.0014501249045965538	0.004587186195581861	-0.002188870154022988	-0.0005124300592705067	0.010547043972765952	0.023182890020112975	-0.004805535186640758	-0.017428156841875087	0.011033066967721479	0.037028525231077865
\end{filecontents}

\begin{tikzpicture}
  \begin{axis}[
      xlabel = Time ($\si{\micro \second}$),
      axis x discontinuity=crunch,
      ylabel = $\Delta V_h$ (percent),
    ]

    \foreach \colindex in {0,...,19} 
      {
        \addplot+[mark=none, solid, ultra thick] table [x expr = \coordindex*(6400*4.85437e-3)/255 + 5, y index = \colindex, mark=*] {smoothed_hull_volume.dat};
        %6400: # timesteps, 255: # of downsamples, 4.85437e-3: delta t (microseconds)
      }

  \end{axis}
\end{tikzpicture}

  \caption{\label{fig:hull change}(Color online)
    Fractional change in the volume of 20 randomly-initialized \bubble\ clouds subject to the same incident pulse, smoothed with a 128-sample moving average.
    Positive and negative values denote expansion and contraction. $\sigma = \SI{1.5}{\centi\meter}$.
  }
\end{figure}
