\section{Continuum problem statement}

Consider a collection of $N$ rigid, non-intersecting spherical scatterers (\bubbles), each having radius $a_k$, position $\vb{r}_k$, and enclosing volume $V_k \subset \mathbb{R}^3$.
The \bubbles\ move in a homogeneous exterior fluid occupying $V_E$, where we denote the boundary of each \bubble\ as $\Omega_k = \partial V_k$ and thus may ascribe  to each an outward-pointing normal $\vu{n}_k\qty(\theta, \phi)$,
where $\theta$ and $\phi$ represent colatitude and azimuthal angles with respect to the local origin (\bubble\ center).
We wish to investigate the reaction of the system to an incident acoustic pulse, thus the fluid carries a prescribed (band-limited) waveform through the \bubble\ system in which it interacts with each of the $\partial \Omega_k$ according to the ``sound-hard'' regime presented in~\cite{Li2014}.
The incident acoustic pulse, in combination with the acoustic field scattered from each \bubble\ and the hydrodynamic field induced by the relative velocity of each \bubble, acts as a perturbation to the initially at-rest uniform ideal fluid~\cite{Myers1992, Landau2013}.
We consider here the linear regime, in which the perturbation induced by the acoustic and aerodynamic contribution remain sufficiently small so that the velocity field $\vb{v}(\vb{r}, t)$ satisfies the condition $\abs{\vb{v}(\vb{r}, t)} \ll c_s $, where $c_s$ represents the speed of sound in the fluid.
In this limit, the velocity potential, defined by $\vb{v}(\vb{r}, t) = \nabla \varphi\qty(\vb{r}, t)$, satisfies the scalar wave equation:
\begin{equation}
    \left(\frac{1}{c_s^2} \frac{\partial^2}{\partial t^2}-\nabla^2\right) \varphi\qty(\vb{r}, t) = 0,
  \label{eq:Helmholtz}
\end{equation}
and we may express the pressure perturbation at any point in the exterior medium as
\begin{equation} \label{eq:Bernoulli}
  p\qty(\vb{r}, t) = - \rho_0\pdv{\varphi\qty(\vb{r}, t)}{t},
\end{equation}
where $\rho_0$ denotes the equilibrium density of the fluid.
Rigidity of the $\Omega_k$ necessarily prescribes boundary conditions on the normal velocity components at each interface, namely,
\begin{equation}
  \left.\pdv{\varphi(\vb{r},t)}{\vu{n}_k}\right|_{\vb{r} \in \Omega_k} = \dv{\vb{r}_k}{t} \cdot \vu{n}_k.
  \label{eq:boundary_conditions}
\end{equation}
where $\vb{r}_k$ represents the center-of-mass coordinate for the $k^\text{th}$ \bubble.

Using these relations, we apply the Kirchoff-Helmholtz theorem to define the following system of integral equations,
\begin{equation}
  \varphi(\vb{r}, t) = \varphi_\text{inc}(\vb{r}, t) + \sum_{i = 0}^{N - 1} \siint
  \qty( \varphi(\vb{r}',t') \pdv{\GF}{\vu{n}_k} - \GF \pdv{\varphi(\vb{r}',t')}{\vu{n}_k} ) \dd{A'} \dd{t'},
  \label{eq:Kirchoff-Helmholtz}
\end{equation}
where $A' \in \Omega_k(t')$ and $\GF$ denotes the Green's function for a retarded potential,
\begin{equation}
  \GF = \frac{\delta\qty(t - t' - \abs{\vb{r}-\vb{r}'}/c_s)}{4\pi \abs{\vb{r}-\vb{r}'}}.
  \label{eq:retarded green's function}
\end{equation}
If the system remains localized to a region with small dimensions when compared to the wavelength of sound, retardation effects become negligible and we may instead use the Laplace-kernel Green's function,
\begin{equation}
  g\qty(\vb{r}, \vb{r}') = \frac{1}{4\pi\abs{\vb{r} - \vb{r}'}}.
  \label{eq:green's function}
\end{equation}
To ease notation, we define the following two integral operators,
\begin{subequations}
\begin{align}
  \hat{\mathcal{S}}_k\qty[\varphi(\vb{r} \in \Omega_k(t), t)] &= \int_{} G(\vb{r}, \vb{r}') \; \partial_{\vu{n}_k} \varphi(\vb{r}',t) \dd{A}
  \label{eq:single layer}\\
  \hat{\mathcal{D}}_k\qty[\varphi(\vb{r} \in \Omega_k(t), t)] &= \int_{} \varphi(\vb{r}',t) \; \partial_{\vu{n}_k} G(\vb{r}, \vb{r}') \dd{A}, \label{eq:double layer}
\end{align}
  \label{eq:operator_kirchoff}
\end{subequations}
reducing \cref{eq:Kirchoff-Helmholtz} to
\begin{equation}
  \varphi(\vb{r},t) = \varphi_\text{inc} + \sum_{k = 0}^{N - 1} \qty(\hat{\mathcal{D}}_k - \hat{\mathcal{S}}_k)\qty[\varphi(\vb{r} \in \Omega_k(t), t)].
  \label{eq:reduced Kirchoff-Helmholtz}
\end{equation}

In solving \cref{eq:reduced Kirchoff-Helmholtz}, we obtain the velocity potential everywhere for a given time without retarded scattered fields.
For the incident pulse, $\varphi_\text{inc}$, we consider superpositions of wave packets of the form
\begin{equation}
  \varphi_\text{inc}(\vb{r}, t) = P_0\cos(\omega_0 t - \vb{k}\cdot\vb{r})e^{-{(c_s t - \vu{k}\cdot\vb{r})}^2/(2 \sigma^2)}.
  \label{eq:inc_field}
\end{equation}
Finally, the variation in pressure (and thus $\varphi$) over each of the $\Omega_k$ necessarily propels each \bubble\ according to
\begin{equation}
  m_k \dv[2]{\vb{r}_k}{t} = \rho_0 \sint \pdv{\varphi\qty(\vb{r}, t)}{t} \dd{\vb{S}}
  \label{eq:Newton}
\end{equation}
where $\vb{S}$ denotes the outward-pointing normal of $\Omega_k(t)$.

\begin{figure}[t]
  \centering
  \conditionalFigureInput{figures/coordinate_notation.tex}
  \caption{\label{fig:diagram}Coordinate notation.}
\end{figure}
