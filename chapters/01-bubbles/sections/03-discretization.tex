\section{Discretization of the Integral Equations}

To solve the integral equation scattering problem, we begin by discretizing our field in both space and time.
As we have restricted our particles to completely spherical geometries, the spherical harmonics, defined by
\begin{equation} \label{eq:Ylms}
  Y_{\ell m}\quantity(\theta, \phi) = \sqrt{\frac{2\ell + 1}{4\pi}\frac{\quantity(\ell - m)!}{\quantity(\ell + m)!}}P_\ell^m\quantity(\cos\theta)e^{i m \phi},
\end{equation}
give simple eigenfunctions of the operators in \cref{eq:operator_kirchoff}.
As a result, they lend themselves well to an expansion of $\varphi$ on the surface of each \bubble\ with respect to the \bubble's center,
\begin{equation} \label{eq:potential_expansion}
  \varphi(\vb{r}\in\Omega_k, t) = \sum_{\ell \ge 0} \sum_{\abs{m} \le \ell} C^k_{\ell m}(t) \; Y_{\ell m}\qty(\theta, \phi).
\end{equation}
By considering \cref{eq:Bernoulli} and expressing the local velocity potential at each of the $\Omega_k$ as a linear combination of spherical harmonics, we have a complete representation of the body force acting on each \bubble,
\begin{align}
  \vb{F}_\text{body}^k(t) &= - \int_{\Omega_k(t)} p(\vb{r}, t) \dd{\vb{S}} \nonumber \\
  &= \rho_0 \sqrt{\frac{2\pi}{3}}r^2 \Big(\qty[\dot{C}^k_{11}(t) - \dot{C}^k_{1\, -1}(t)]\vu{x} + i\qty[\dot{C}^k_{11}(t) + \dot{C}^k_{1\,-1}(t)]\vu{y} - \sqrt{2}\dot{C}^k_{10}(t)\vu{z}\Big)
\end{align}
due to the orthogonality of dipole terms with the rest of the multipoles.

The problem then becomes one of solving a system of linear equations that we may compactly represent as
\begin{equation}
  \overline{\boldsymbol{\mathcal{Z}}} \cdot \boldsymbol{\varphi} = \boldsymbol{\mathcal{F}},
  \label{eq:z_matrix_system}
\end{equation}
with the overbar denoting a matrix quantity.
We define the elements of $\boldsymbol{\mathcal{F}}$ as projections of the incident field onto local spherical harmonics,
\begin{equation}
  \mathcal{F}_{\ell m}^k = \int_{\Omega_k(t)} \mkern-30mu \dd{A} Y^*_{\ell m}(\theta, \phi) \varphi_\text{inc}(\vb{r}, t),
  \label{eq:forcing projection}
\end{equation}
and detail $\mathcal{Z}^{jk}_{\ell m, \ell' m'}$ for two cases: $j = k$ and $j \not = k$.
In the instances where $j = k$, \cref{eq:operator_kirchoff} propagates effects of the interaction through to every point on a surface sharing a coordinate system with the original, thus the harmonics remain orthogonal and
\begin{equation}
  \mathcal{Z}_{\ell m, \ell' m'}^{jj} = \frac{\ell + 1}{2\ell + 1}  \delta_{\ell \ell'} \delta_{m m'}
  \label{eq:z_matrix_self}
\end{equation}
after exploiting the well-known expansion theorem for \cref{eq:green's function},
\begin{equation}
  g(\vb{r}, \vb{r}') = \sum_{\ell, m}\frac{1}{2\ell + 1}\frac{r_<^\ell}{r_>^{\ell + 1}}Y_{\ell m}\qty(\theta, \phi)Y^*_{\ell, m}\qty(\theta', \phi')
  \label{eq:expansion theorem}
\end{equation}
where $r_< = \mathrm{min}\qty(\abs{\vb{r}},\abs{\vb{r}'})$ and $r_> = \mathrm{max}\qty(\abs{\vb{r}}, \abs{\vb{r}'})$.
A description of the off-diagonal terms where $j \not = k$ proceeds much the same way, though the surface expansions no longer share a local origin, complicating the projections.
Translation operators for the spherical harmonics~\cite{Caola1978,Greengard1987} allow analytic expressions for these matrix elements, though we eschew such operators in favor of numerical integration for speed.

Thus, at every timestep of the simulation, the algorithm proceeds as follows:
\begin{inparaenum}[(i)]
  \item project the incident pulse and surface velocities onto local expansions of spherical harmonics,
  \item propagate scattering effects through space by inverting the operators in \cref{eq:reduced Kirchoff-Helmholtz},
  \item project these scattered fields onto local spherical harmonics to give a total representation of $\varphi$ on each surface, and
  \item move each \bubble\ according to \cref{eq:Newton} \& advance $t \rightarrow t + \Delta t$.
\end{inparaenum}
For rigid \bubbles\ only $\ell = 1$ terms contribute to center-of-mass motion, thus we use only the $C_{1m}$ coefficients in evolving \cref{eq:Newton}.

The inversion in step (ii) above requires some care; by simply inverting the entire propagation operator, $\hat{\mathcal{D}} - \hat{\mathcal{S}}$, to give a single surface pressure, \cref{eq:Newton} reduces to a differential equation of the form
\begin{equation}
  \label{eq:malformed}
  \dot{\vb{r}}_k = f(t, \vb{r}_k, \dot{\vb{r}}_k).
\end{equation}
This presents a number of irregularities with conventional integration schemes and will rapidly diverge towards $\pm \infty$ due to the additional $\dot{\vb{r}}_k$ on the right if implemented \naive ly.
To remedy this, we note that $\hat{\mathcal{S}}$ serves to produce \emph{only} a reaction or drag term on each \bubble\ that impedes motion.
By maintaining quantities for the inversion of $\hat{\mathcal{D}}$ and $\hat{\mathcal{S}}$ separately, we remove the explicit dependence on $\dot{\vb{r}}_k$ by introducing a linear coefficient in the form of an additional mass term---given by the $\dot{\vb{r}}_k$-dependent contribution in the single-layer $\hat{\mathcal{S}}$ operator---when solving \cref{eq:Newton}. 
