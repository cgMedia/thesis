\section{Conclusions}

This work lends a novel, fine-grained approach to the study of acoustic response via integral equation methods.
By considering a potential representation in terms of spherical harmonics on the surfaces of \bubbles\ coupled to a standard molecular dynamics scheme, we obtain a description of the \bubbles' dynamics under the effect of ultrasound pulses without resorting to time-average approximations.
Furthermore, the confined \bubble\ geometries under consideration allow us to neglect small effects arising from time-delays in scattering.
We have shown that the net effect of an ultrasound pulse on a single \bubble\ consists of a translation that we can tune through careful control of pulse parameters.
Additionally, systems with multiple incident waveforms tend to confine \bubbles\ to nodes in the pressure field governed by acoustic interference.
Finally, in the dynamics of systems with many \bubbles, we have observed the effect of weak inter-particle transient effects induced by the driving acoustic pulse.
These effects can produce both expansion and contraction of a cloud of \bubbles, in addition to the overall translation.

Prior work in this area \cite{Zeravcic2011, Tiwari2015} makes use of deformable bubble boundaries about fixed locations. 
Incorporation of these methodologies to our theoretical model naturally offers possibilities for future research, as does the addition of retardation effects. 
Additionally, we expect a straightforward approach to experimental confirmation of the results presented here. 
Optical tracking of tracer particles\cite{Toschi2009} has demonstrated its effectiveness in similar fluid-trajectory studies and would readily adapt to track physical analogues of our theoretical \bubbles.
