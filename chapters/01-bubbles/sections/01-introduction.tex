\section{Introduction}

\dropcap{C}{omputational} approaches that employ an integral equation formalism to examine acoustic scattering from particles typically assume a static environment in which scatterers remain stationary.
At present, a large body of work details such scattering problems~\cite{Waterman1969, Ding1989, Ye1997}.
While these stationary integral equation methods offer a large degree of accuracy in capturing the underlying physics, many problems of interest require a fully dynamical treatment.
For instance, in biomedical physics, gas-filled \bubbles\ exposed to ultrasonic beams have demonstrated effectiveness as a contrast imaging agent~\cite{Blomley2001} and as drug delivery method~\cite{Allen2002,Hernot2008}, and Ding \etal\ have demonstrated their manipulation using acoustic tweezers in microfluidic channels~\cite{Ding2012}.
Moreover, composite materials consisting of colloidal in-fluid suspensions have peculiar sound propagation properties that can deviate from the ones of homogeneous liquids~\cite{Ye1993}.
In each of these applications, the unconstrained motion of scatterers requires a self-consistent description of their dynamics in conjunction with a description of the acoustic field propagation.

Here, we demonstrate the applicability of coupling particle kinetics to a time-domain integral equation (TDIE) scattering framework to model rigid-sphere motion induced by a time-dependent acoustic potential.
Specifically, we consider the case of an acoustic pulse acting on \bubbles\ that move in a fluid.
Effective Langevin time-averaged radiation pressure forces~\cite{King1934, Borgnis1953}, which consider the case of a steady radiation flux incident upon a body kept in static equilibrium, do not provide an appropriate model in this case as they cannot accommodate inter-particle scattering effects.
While many theoretical and computational descriptions of higher-order acoustic interactions exist~\cite{Gumerov2002, Doinikov2004, Doinikov2005, Ilinskii2007, Azizoglu2009}, few actually make use of computed fields to predict particle trajectories.
As we consider only short-duration pulses, we refrain from time-averaging in favor of using a time-domain scattering formulation to explicitly calculate particle trajectories resulting from a prescribed pulse.
By adopting a weakly-compressible potential formulation of the fluid media, our scalar wave problem inherits a number of similarities and solution techniques from scattering problems in electromagnetic theory, a topic previous works discuss extensively~\cite{Tsang1998,Gumerov2002,Li2014}.
Moreover, our time-domain formulation readily allows the study of transient phenomena (such as acoustic tweezing); a convenience not shared with more common frequency domain approaches.

We structure the remainder of this chapter as follows: we first provide a formal mathematical description of the problem---including details on both the kinetic and field methods---followed by data obtained from various pulse and \bubble\ configurations, demonstrating both attractive and repulsive regimes suitable for subtle control of spherical systems in a homogeneous fluid.
Finally, we offer concluding remarks on the effectiveness of the simulation as well as our thoughts on possible future extensions.
