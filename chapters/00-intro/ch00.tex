\chapter{\label{ch:intro}Intro}

\begin{frontquote}{The King of Hearts~\cite{wonderland}}
``Begin at the beginning,'' the King said, very gravely, ``and go on till you come to the end: then stop.''
\end{frontquote}

\dropcap{A}{s} our ability to resolve physical systems grows, so, too, do our demands of the models we build.
Unfortunately, we rapidly approach an age where approximate analytic descriptions of physical systems simply cannot recover important features in systems of interest.
Substituting \emph{numerical}---i.e.\ \emph{in silico}---techniques into higher resolution models alleviates this problem somewhat (particularly as computer hardware evolves ``below'' the model), though the computational overhead of \naive\ algorithms limits their utility beyond nearly analytic calculations, particularly in multiscale or multiphysics applications.
The computational scientist, then, has a tricky job: they must have expertise in their own domain so as to begin formulating testable hypotheses about the (multi-)physical systems, as well as expertise in ``computation'' to understand the constraints inherent in exploring such hypotheses numerically.

\section{Active media}

Active media---dynamical systems that couple together through radiative processes---encompass a large class of problems across length and time scales and exhibit behavior fundamentally different from their passive counterparts.
These behaviors offer new avenues towards novel device designs, though engineering devices to exploit them presents a considerable challenge.
In particular, the underlying dynamical nature of active systems makes them inherently nonlinear, and accounting for these dynamics alongside the radiative coupling requires a multiphysical approach.
As such, this thesis primarily concerns itself with the development of computational techniques for use in modeling active systems both efficiently and accurately with a particular focus on acoustophoresis (acoustically-induced motion) and quantum optics.

\subsection{Acoustophoresis}

% First, explain the background/why is this interesting?
The placement of biological material or microscopic objects plays a large role in a number of technological applications related to biosensing.
For example, droplet-based microfluidics entails the careful construction, manipulation, and sensing of \si{\micro\liter} (or smaller) droplets of fluid for myriad lab-on-a-chip applications.
Similarly, DNA microarrays employ large numbers of micron-sized spots---each filled with a particular DNA sequence---to perform an experiment on thousands of genes simultaneously.
Manipulating individual droplets and fabricating a microarray both require careful placement of apparatus; though given the small length scale of these problems conventional manipulation techniques (such as pipetting) become problematic.
Instead, researchers require alternative means of precisely controlling the motion of such objects.

% This site is really great: http://bme.lth.se/research-pages/nanobiotechnology-and-lab-on-a-chip/research/acoustophoresis/
Because of their low energy and minimal invasiveness, so-called acoustic tweezers offer an ideal mechanism for precise microscopic positioning in both biological and nonbiological applications.
By varying the pressure in the surrounding environment through the precise application of ultrasonic pulses, acoustic tweezers effect acoustophoresis for a variety of purposes including object placement, fractionation~\cite{Petersson2007}, and flow shaping.
% Next, explain the specific problem
Present models of such systems often fail to account for multiple scattering events or only do so by way of time- or volume-averaged techniques.
This poses a significant problem in very small systems where such effects necessarily become important.
% Finally, tease how we solved it
To overcome this, \cref{ch:bubbles} develops the machinery to simulate the motion of an ensemble of hard spheres within a fullwave scattering framework.

\subsection{Light in disordered media}

Disordered nanostructures give rise to very striking optical phenomena.
\emph{Cyphochilus} spp.\ beetles, for instance, owe their intense white coloration to an aperiodic arrangement of scales that scatters light across wavelengths~\cite{Vukusic2007}.
Whereas manmade objects of similar whiteness and intensity require structures as thick as \SI{100}{\micro\meter}, the scales of \emph{cyphochilus} spp.\ can span as few as \SI{4}{\micro\meter} by exploiting an interior structure of disordered filaments. 
Additional effects such as Anderson localization~\cite{Anderson1985}---wherein light remains spatially confined for extended periods of time---and L\'evy flights~\cite{Barthelemy2008}---in which light experiences superdiffusive, i.e.\ superlinear (in time), behavior---all follow from similarly disordered structures.
These phenomena all fall under the broad category of light/matter interactions, and engineers seeking to develop new devices that exploit these phenomena require tools that accurately and quickly resolve such interactions. 

Quantum dots---colloquially called ``tuneable atoms'' because of the ease with which engineers can manipulate their spectra---stand as a premier element in optoelectronic device design and have enjoyed widespread areas such as light harvesting applications and quantum information storage/retrieval. 
Because of this, a number of analytic methods based on ensemble-averaged single particle and higher order Green's functions have arisen to describe light propagation in such systems.
These methods essentially cast the active quantum system as a homogeneous continuum source in the governing electromagnetic equations (Maxwell's equations for semiclassical systems)~\cite{Arecchi1965}, though such a treatment cannot account for effects arising from long-range order, partial order, or anisotropies. 
Models of these effects, then, require a detailed computational framework that captures scattering/radiation effects between discrete, disordered elements.

\begin{figure}
  \centering
  \conditionalFigureInput{figures/bloch_sphere.tex}
  \caption{\label{eq:fig}Bloch sphere representation of a two-level quantum system.}
\end{figure}


\section{Wave propagation}

\begin{figure}
  \centering
  %This figure requires both the shadows and mindmap tikzlibraries. Include them in your header!
\usetikzlibrary{shadows, mindmap}
\begin{tikzpicture}[mindmap, >=latex]
  \begin{scope}[
    every node/.style={concept, circular drop shadow,execute at begin node=\hskip0pt},
    root concept/.append style={
      concept color=black, fill=white, line width=1ex, text=black
    },
    text=white,
    IE/.style={concept color=cbred,faded/.style={concept color=cbred!50}},
    DE/.style={concept color=cbblue,faded/.style={concept color=cbblue!50}},
    measuring complexity/.style={concept color=orange,faded/.style={concept color=orange!50}},
    solving problems/.style={concept color=green!50!black,faded/.style={concept color=green!50!black!50}},
    grow cyclic,
    level 1/.append style={level distance=4.5cm, sibling angle=60},
    level 2/.append style={level distance=3cm,sibling angle=30}]

  \node[root concept]{Propagation techniques}
  child[IE]{ node (IE) {Integral equation methods}}
  child[DE]{ node (DE) {Differential equation methods}};
    \end{scope}
  \begin{scope}[every annotation/.style={fill=cbblue!10}]
    \node [annotation, xshift=3.5cm, yshift=0.0cm, text width = 6.5cm] at (DE.east) {
      Properties:
       \begin{itemize}
         \setlength\itemsep{0em}
         \item[$\pm$] Local operators
         \item[\textcolor{green!70!black}{+}] Straightforward mathematics
         \item[\textcolor{green!70!black}{+}] Incorporates material properties
         \item[\textcolor{red}{-}] ``Volumetric'' discretization
         \item[\textcolor{red}{-}] Artifical radiation boundary condition
       \end{itemize}
    };
  \end{scope}
  \begin{scope}[every annotation/.style={fill=cbred!10}]
    \node [annotation, xshift=3.5cm, yshift=-0.0cm, text width = 6.5cm] at (IE.east) {
      Properties:
       \begin{itemize}
         \setlength\itemsep{0em}
         \item[$\pm$] Global operators
         \item[\textcolor{red}{-}] Intricate mathematics
         \item[\textcolor{red}{-}] Exclusively models free/homogeneous space
         \item[\textcolor{green!70!black}{+}] ``Surface'' discretization
         \item[\textcolor{green!70!black}{+}] Automatic radiation boundary condition
       \end{itemize}
    };
  \end{scope}

  %\foreach \x in {10,10.5,...,13}
    %\draw (\x, 1.4) -- (\x, 1.8);

  %\shade [ball color=blue!50] (10, 1.6) circle [radius=0.1cm];
  %\shade [ball color=blue!50] (12.5, 1.6) circle [radius=0.1cm];

  %\draw[->] (10.0, 1.6) to[out=30, in=150] (10.5, 1.6);
  %\draw[->] (10.5, 1.6) to[out=30, in=150] (11.0, 1.6);
  %\draw[->] (11.0, 1.6) to[out=30, in=150] (11.5, 1.6);
  %\draw[->] (11.5, 1.6) to[out=30, in=150] (12.0, 1.6);
  %\draw[->] (12.0, 1.6) to[out=30, in=150] (12.5, 1.6);

  %\node[] at (11.25, 1.2) {\tiny Larger, sparse matrices};
  

  %\draw (10,-1.4) -- (10, -1.8);
  %\draw (12.5,-1.4) -- (12.5, -1.8);

  %\shade [ball color=red!50] (10, -1.6) circle [radius=0.1cm];
  %\shade [ball color=red!50] (12.5, -1.6) circle [radius=0.1cm];

  %\draw[->] (10, -1.6) to[out=30, in=150] (12.5, -1.6);

  %\node[] at (11.25, -2.0) {\tiny Smaller, dense matrices};
\end{tikzpicture}

  \caption{\label{fig:mindmap} Comparison of differential-equation and integral-equation methodologies.
    As both approaches have several advantages and disadvantages, the choice of which technique to use largely depends on the problem under consideration.
  }
\end{figure}

\subsection{Simulation techniques}

Wave phenomena influence every aspect of our lives across every scale of the universe.
Because of this, techniques to explore and analyze these phenomena lie at the heart of many scientific and engineering disciplines.
Broadly, these techniques fall into differential-equation-based and integral-equation-based categories with each having several benefits and drawbacks.
Differential-equation-based models follow directly from a discretization of the wave equation throughout the volume of interest and thus have a large number of local interactions.
As these methods discretize the entire volume, they readily accommodate material variations within the volume such as a change in the velocity of propagation (perhaps via introduction of a dielectric).
Unfortunately, these operators introduce numerical errors that manifest as dispersion and

Abruptly truncating this volume produces a ``hard'' boundary that causes potentially unwanted or even unphysical reflection effects, and thus systems formulated with a differential operators often contain artificial radiation boundary layers to suppress these effects.

