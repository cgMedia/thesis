\chapter{\label{ch:intro}Intro}

\begin{frontquote}{The King of Hearts~\cite{wonderland}}
``Begin at the beginning,'' the King said, very gravely, ``and go on till you come to the end: then stop.''
\end{frontquote}

\dropcap{A}{s} our ability to resolve physical systems grows, so, too, do our demands of the models we build.
Unfortunately, we rapidly approach an age where approximate analytic descriptions of the universe simply cannot recover important features in systems of interest.
Substituting \emph{numerical}---i.e.\ \emph{in silico}---techniques into higher resolution models alleviates this problem somewhat (particularly as the hardware evolves ``below'' the model), though the computational overhead of \naive\ algorithms almost certainly limits their utility.
The computational scientist, then, has a tricky job: they must have expertise in their own domain so as to begin formulating testable hypotheses about the universe, as well as expertise in ``computation'' to understand the constraints of exploring such hypotheses digitally.

\section{Wave propagation}

\begin{figure}
  \centering
  \input{figures/de_ie_mindmap}
  \caption{\label{fig:mindmap} Comparison of differential-equation and integral-equation methodologies.
    As both approaches have several pros and cons, the choice of which technique to use largely depends on the problem under consideration.
  }
\end{figure}

Wave phenomena influence every aspect of our lives across every scale of the universe.
From the nano-scale structure of matter, to human-scale (tele)communications, and even to a refined description of gravity, the mathematical description of waves plays a critical role in developing a coherent model of the world around us.
As a result, developing techniques to explore and analyze these phenomena lies at the heart of many scientific and engineering disciplines.

\subsection{Simulation techniques}

Broadly, these techniques fall into differential-equation-based and integral-equation-based categories with each having several benefits and drawbacks.
Differential-equation-based models follow directly from a discretization of the wave equation throughout the volume of interest and thus have a large number of local interactions.
As these methods discretize the entire volume, they readily accommodate material variations within the volume such as a change in the velocity of propagation (perhaps via introduction of a dielectric).
Unfortunately, these operators introduce numerical errors that manifest as dispersion and

Abruptly truncating this volume produces a ``hard'' boundary that causes potentially unwanted or even unphysical reflection effects, and thus systems formulated with a differential operators often contain artificial radiation boundary layers to suppress these effects.

\section{Active media}

Borrowing the title from optics where ``active media'' refers to the gain medium of a laser, we expand the term to encompass any radiative system goverened by non-radiative dynamics.
In particular, this thesis concerns itself with the investigation of microspheres---where acoustic radiation affects the kinematics of an ensemble of particles governed by Newtonian mechanics---and \qds{}---where electromagnetic radiation couples to internal quantum degrees of freedom described by the Liouville equation.


%\section{History of scattering models}



%\subsection{Frequency-domain}

%\begin{itemize}
    %\item T-matrix methods
    %\item Homogenization(?)
    %\item Bruce Draine
%\end{itemize}

%\subsection{Time-domain}

%\section{Introduction of dynamics}

%\section{Overview of computational electromagnetics}
