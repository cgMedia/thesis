\section{Results}

\subsection{Spatial analysis}

Consider two point particles located at $x_\text{src}$ and $x_\text{obs}$.
A time-independent Green's function, $g(x_\text{obs} - x_\text{src})$, describes the interaction between the two particles and we wish to construct a polynomial approximation of $g(x - x_\text{src})$ for $x$ in the vicinity of $x_\text{obs}$ as in \cref{fig:1d moments}.

To construct an interpolation polynomial over the expansion region of order $M$, we define a polynomial coordinate $x_p$ in units of $h$ such that $x_p^\text{min} \leqslant x_p \leqslant x_p^\text{min} + M$ where $x_p^\text{min} \equiv -\floor*{M/2}$.
Consequently, the expansion points about $x_\text{obs}$ correspond to $x_p \in \{-\floor*{M/2}$, $-\floor*{M/2} + 1$, $-\floor*{M/2} + 2$, $\ldots\}$ with the \nth{0} order expansion point, $x_0$, equivalent to $x_p = 0$.
Such a coordinate system defines the Vandermonde linear equation $\sum_{j}V_{ij} w_j = g_i$ for the weights of an interpolating polynomial\footnote{In principle this analysis works equally well in the original $(x_\text{src}, x_\text{obs})$ coordinate system. The polynomial coordinate has three advantages, however: it indexes the expansion points ``logically'' from left to right, $x_p^\text{min} \in \mathbb{Z}$ and thus the Vandermonde matrix has infinite precision, and it makes the interpolation error in terms of $h$ explicit.}~\cite{NumericalRecipes} where
\begin{subequations}
  \begin{align}
    V_{ij} &= (x_p^\text{min} + i)^j \\
    g_i &= g\qty((x_0 - x_\text{src}) + h(x_p^\text{min} + i))
  \end{align}
\end{subequations}
and $0 \leqslant i, j \leqslant M$.
Approximating $g(x - x_\text{src})$ at $x_\text{obs}$ then becomes a matter of evaluating this polynomial at $x_p = \qty(x_\text{obs} - x_0)/h$, i.e.
\begin{equation}
  g(x_\text{obs} - x_\text{src}) \approx \sum_{i = 0}^{M} w_i \qty(\frac{x_\text{obs} - x_0}{h})^i.
\end{equation}
Accordingly, the polynomial approximation to $g(x_\text{obs} - x_\text{src})$ contains terms of order $\mathcal{O}(h^{-M})$ and we can expect the approximation error to scale as $\mathcal{O}(h^{-(M + 1)})$ as demonstrated in \cref{fig:grid convergence}.
This also motivates using the approximation to calculate interactions involving differential operators; applying an $n^\text{th}$-order derivative reduces the polynomial order by $n$, thus the error scales like $\mathcal{O}(h^{-(M + 1) + n})$.

\begin{figure}
  \centering
  \usetikzlibrary{decorations.markings}
\tikzset{
    mark position/.style args={#1(#2)}{
        postaction={
            decorate,
            decoration={
                markings,
                mark=at position #1 with \coordinate (#2);
            }
        }
    }
}

\begin{tikzpicture}[scale=1.7]

\foreach \i in {0, ..., 4}
{
  \draw[thick] (\i, -0.6) -- (\i, 0.6);
}

\fill[green!60!black] (0,0) circle (0.1) node [anchor=north west] {$x_\text{src}$};
\fill[green!40!black] (7.2,0) circle (0.1) node [anchor=north west] {$x_\text{obs}$};

\draw[thick, blue] (7,-0.6) node[below]{$x_0$} -- (7, 0.6) node[above]{$m = 0$};
\draw[thick, blue!75!red] (8,-0.6) node[below]{$x_1$} -- (8, 0.6) node[above]{$m = 1$};
\draw[thick, blue!50!red] (6,-0.6) node[below]{$x_2$} -- (6, 0.6) node[above]{$m = 2$};
\draw[thick, blue!25!red] (9,-0.6) node[below]{$x_3$} -- (9, 0.6) node[above]{$m = 3$};
\draw[thick, blue!0!red] (5,-0.6) node[below]{$x_4$} -- (5, 0.6) node[above]{$m = 4$};

\draw[dashed, thick, orange, mark position=0.9(g)] plot [smooth] coordinates {(4.8, -0.5) (5,-0.4) (6, 0.3) (7, 0.1) (8, 0.4) (9, -0.3) (9.2, -0.4)};
%\node[orange, anchor=west] (g) at (9.2, -0.4) {$g(x-x_\text{src})$};

\end{tikzpicture}

  \caption{\label{fig:1d moments} Polynomial interpolation of $g(x - x_\text{src})$ near $x_\text{obs}$.
    Here, the green curve represents the actual $g(x - x_\text{src})$ and the dashed black line its approximation.
    Evaluating the $m^\text{th}$-order approximation requires samples of the signal at $m + 1$ grid points surrounding $x_\text{obs}$.
  }
\end{figure}

\begin{figure}
  \centering
  \begin{filecontents}{grid_spacing.dat}
Spacing             	Order0            	Order1              	Order2                	Order3                	Order4                 
0.2                 	10.311922586843853	6.547888556532848   	0.8078593676582986    	0.6900135622760024    	0.07775624373511587    
0.1861144081859398  	13.420661187167552	2.114064129983551   	0.5176983592090901    	0.3090021241170343    	0.047645445263203585   
0.17319286467201306 	16.28770034601291 	2.035108560409383   	0.4080937775243047    	0.24403206604795324   	0.10482680642842229    
0.16116843755229637 	8.519866354241426 	4.322198387023356   	0.3818170863654957    	0.2885417655222987    	0.02599109071836223    
0.14997884186649116 	8.216773910589568 	1.5357193391783812  	0.9111449652707617    	0.0870040561818314    	0.03892261096436527    
0.13956611697197327 	7.91364029423806  	1.3559066034348457  	0.2965904337336443    	0.1338093251221245    	0.03031081459048604    
0.12987632631524226 	7.475948204162177 	3.1937563384698953  	0.159787816142875     	0.1277975497486948    	0.007284023505291419   
0.12085927804762657 	6.5376413460444045	1.0567180589163159  	0.1365885118608641    	0.07980974374631232   	0.005047801515746074   
0.11246826503806982 	8.194864043013252 	2.264107617481496   	0.6967763929264409    	0.06850903937472122   	0.003057730482123484   
0.10465982293629894 	8.090874834317837 	0.609830973952806   	0.1880026044216161    	0.024048270774990978  	0.007851108795157513   
0.09739350503317262 	11.567176061843169	0.31523828541460563 	0.28331122679347015   	0.012044396686557947  	0.003518944334057881   
0.09063167275201636 	5.0412382168658825	1.0800851897321575  	0.16008918930196203   	0.01288087194156884   	0.0019306677426391755  
0.08433930068571645 	5.878811112707047 	0.5350738193040928  	0.19041019791876324   	0.005194902372171951  	0.002032009900824846   
0.07848379516969071 	4.189910886513379 	0.5756267428004771  	0.11451555388096576   	0.008271411808739722  	0.0012716498146993934  
0.07303482545096754 	4.871506522285374 	0.22727569303236173 	0.02228380547005147   	0.013159246477990934  	0.00029382849770544987 
0.06796416657885118 	4.810718548949396 	0.8587931609444456  	0.04436555600335379   	0.009426219732028694  	0.0004953588769775904  
0.06324555320336758 	5.118235369183121 	0.1330037964356479  	0.01481795934173761   	0.007348233701952196  	0.0001619123841881273  
0.058854543524185635	3.3092417760722395	0.6500461502198256  	0.08053291827668115   	0.005614857608020859  	0.00034058332297163483 
0.05476839268528723 	5.461374355999743 	0.09091063469420901 	0.04792114497884343   	0.001876640684106795  	0.00032241396681770057 
0.05096593495958693 	5.048607649019471 	0.1414572893106953  	0.02942074105071717   	0.0007256498406783059 	0.0001028175868275004  
0.04742747411323311 	4.1144028377097674	0.13156298181752749 	0.0412478856392499    	0.0021871203435882254 	0.000034511982689231246
0.0441346813816918  	2.861010434183582 	0.13898085304711633 	0.03346861431322368   	0.0004867314128905358 	0.00006423236234576763 
0.04107050052914292 	3.314999476052423 	0.07699748888464913 	0.018785609444434102  	0.0006209236944449464 	0.00005062791059409026 
0.03821905949940881 	2.2924914447262297	0.15166771570956877 	0.01586474524165783   	0.0005540311175891229 	0.000039250577097361366
0.03556558820077846 	2.2067615852765545	0.2385425561437549  	0.002545475389124544  	0.0004407641542956863 	0.0000172416666899262  
0.033096341998863625	2.421033348229719 	0.19612638327036377 	0.0024915464821930397 	0.0005497203184813668 	0.000014010853409356407
0.03079853052118984 	7.049786250490347 	0.13056354098824272 	0.002608435799812353  	0.0003125876659548715 	0.000012833891207862691
0.028660251404739254	1.6728245597148732	0.09951828089401811 	0.0011241093441040693 	0.00032585351493434974	2.4434787271313646e-6  
0.02667042864326648 	1.3856377129424915	0.13765725904678333 	0.0009285541398081519 	0.00013043391277577928	1.762441832569965e-6   
0.02481875521503439 	1.5102370425392042	0.02927583911470367 	0.007634945640631183  	0.00017060561997221736	1.1106237429215307e-6  
0.023095639693789163	1.1395802852733175	0.10409634311961662 	0.0006226158108534926 	0.00013377475499343468	7.830235804352226e-7   
0.02149215656642635 	1.3399836543074841	0.031627076327455844	0.0009069085505414092 	0.000052766871429262  	2.5426876469828946e-6  
0.02                	1.2164289973170956	0.07829042891104353 	0.00036523973292785914	0.00007791537381673744	3.654784963977491e-7   
\end{filecontents}

\begin{tikzpicture}[]
  \pgfplotsset{
    % initialize Set1-5:
    cycle list/Set1-5,
    % combine it with ’mark list*’:
    cycle multiindex* list={
      mark list*\nextlist
      Set1-5\nextlist
    }
  }
  \begin{loglogaxis}[
    legend pos=south east,
    xtick={0.02, 0.05, 0.10, 0.20},
    xticklabels={$0.02\lambda$, $0.05\lambda$, $0.10\lambda$, $0.20\lambda$},
    minor x tick num=4,
    xlabel={Grid spacing, $h$},
    ylabel={$\ell_2$ error}
  ]
    \addplot table [only marks, x=Spacing, y=Order0] {grid_spacing.dat};
    \addplot table [only marks, x=Spacing, y=Order1] {grid_spacing.dat};
    \addplot table [only marks, x=Spacing, y=Order2] {grid_spacing.dat};
    \addplot table [only marks, x=Spacing, y=Order3] {grid_spacing.dat};
    \addplot table [only marks, x=Spacing, y=Order4] {grid_spacing.dat};

    \addplot+[no marks, domain=.02:.2] {62.2728*x^0.96213};
    \addlegendentry{$62.27 h^{0.96}$}

    \addplot+[no marks, domain=.02:.2] {64.5777*x^1.87467};
    \addlegendentry{$64.57 h^{1.87}$}

    \addplot+[no marks, domain=.02:.2] {202.84*x^3.13828};
    \addlegendentry{$202.8 h^{3.13}$}

    \addplot+[no marks, domain=.02:.2] {231.444*x^3.91366};
    \addlegendentry{$231.4 h^{3.91}$}

    \addplot+[no marks, domain=.02:.2] {538.388*x^5.22565};
    \addlegendentry{$538.3 h^{5.22}$}

  \end{loglogaxis}
\end{tikzpicture}

  \caption{\label{fig:grid convergence} $\ell_2$ error calculation of $g(\vb{r} - \vb{r}') = e^{i k \abs{\vb{r} - \vb{r}'}}/{\abs{\vb{r} - \vb{r}'}}$ for expansion orders zero through four.
    For each of the points above, a source and observation box separated by $\Delta \vb{r} = 10\lambda \vu{x} + 10 \lambda \vu{y} + 10 \lambda \vu{z}$ each contain 64 randomly placed points, thus the points within each box all map to identical nodes in the auxiliary grid.
    The moment-matching expansion scheme dictates that the overall error for an expansion of order $m$ scales as $\mathcal{O}(h^{m + 1})$.
  }
\end{figure}
