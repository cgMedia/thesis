\section{Results}

Consider two point particles located at $x_\text{src}$ and $x_\text{obs}$.
A time-independent Green's function, $g(x_\text{obs} - x_\text{src})$, describes the interaction between the two particles and we wish to construct a polynomial approximation of $g(x - x_\text{src})$ for $x$ in the vicinity of $x_\text{obs}$ as in \cref{fig:1d moments}.

To construct an interpolation polynomial over the expansion region of order $M$, we define a polynomial coordinate $x_p$ in units of $h$ such that $x_p^\text{min} \leqslant x_p \leqslant x_p^\text{min} + M$ where $x_p^\text{min} \equiv -\floor*{M/2}$.
Consequently, the expansion points about $x_\text{obs}$ correspond to $x_p \in \{-\floor*{M/2}$, $-\floor*{M/2} + 1$, $-\floor*{M/2} + 2$, $\ldots\}$ with the \nth{0} order expansion point, $x_0$, equivalent to $x_p = 0$.
Such a coordinate system defines the Vandermonde linear equation $\sum_{j}V_{ij} w_j = g_i$ for the weights of an interpolating polynomial\footnote{In principle this analysis works equally well in the original $(x_\text{src}, x_\text{obs})$ coordinate system. The polynomial coordinate has three advantages, however: it indexes the expansion points ``logically'' from left to right, $x_p^\text{min} \in \mathbb{Z}$ and thus the Vandermonde matrix has infinite precision, and it makes the interpolation error in terms of $h$ explicit.}~\cite{NumericalRecipes} where
\begin{subequations}
  \begin{align}
    V_{ij} &= (x_p^\text{min} + i)^j \\
    g_i &= g\qty((x_0 - x_\text{src}) + h(x_p^\text{min} + i))
  \end{align}
\end{subequations}
and $0 \leqslant i, j \leqslant M$.
Approximating $g(x - x_\text{src})$ at $x_\text{obs}$ then becomes a matter of evaluating this polynomial at $x_p = \qty(x_\text{obs} - x_0)/h$, i.e.
\begin{equation}
  g(x_\text{obs} - x_\text{src}) \approx \sum_{i = 0}^{M} w_i \qty(\frac{x_\text{obs} - x_0}{h})^i.
\end{equation}
Accordingly, the polynomial approximation to $g(x_\text{obs} - x_\text{src})$ contains terms of order $\mathcal{O}(h^{-M})$ and we can expect the approximation error to scale as $\mathcal{O}(h^{-(M + 1)})$ as demonstrated in \cref{fig:grid convergence}.
This also motivates using the approximation to calculate interactions involving differential operators; applying an $n^\text{th}$-order derivative reduces the polynomial order by $n$, thus the error scales like $\mathcal{O}(h^{-(M + 1) + n})$.

\begin{figure}
  \centering
  \conditionalFigureInput{figures/1d_moments}
  \caption{\label{fig:1d moments} Polynomial interpolation of $g(x - x_\text{src})$ near $x_\text{obs}$.
    Here, the green curve represents the actual $g(x - x_\text{src})$ and the dashed black line its approximation.
    Evaluating the $m^\text{th}$-order approximation requires samples of the signal at $m + 1$ grid points surrounding $x_\text{obs}$.
  }
\end{figure}

\begin{figure}
  \centering
  \conditionalFigureInput{figures/grid_spacing_error}
  \caption{\label{fig:grid convergence} $\ell_2$ error calculation of $g(\vb{r} - \vb{r}') = e^{i k \abs{\vb{r} - \vb{r}'}}/{\abs{\vb{r} - \vb{r}'}}$ for expansion orders zero through four.
    For each of the points above, a source and observation box separated by $\Delta \vb{r} = 10\lambda \vu{x} + 10 \lambda \vu{y} + 10 \lambda \vu{z}$ each contain 64 randomly placed points, thus the points within each box all map to identical nodes in the auxiliary grid.
    The moment-matching expansion scheme dictates that the overall error for an expansion of order $m$ scales as $\mathcal{O}(h^{m + 1})$.
  }
\end{figure}

\begin{figure}
  \centering
  \conditionalFigureInput{figures/potential_timing.tex}
  \caption{\label{fig:timing}Simulation runtime vs.\ number of spatial basis functions for several system sizes. 
    Each simulation evaluates the retarded potentail between $n$ spatial basis functions for 1024 timesteps.
    We employ a first order expansion (in space) onto the auxiliary grid and a fifth order Lagrange basis set in time.
    The output of the \textsc{fft + nearfield} calculation reproduces that of the \textsc{direct} calculation to approximately five decimal places (the \textsc{fft only} simulation does not calculate any portion of $\mathcal{F}_\text{direct}$ so as to benchmark the inter-grid propagation).
    From top to bottom in the legend, the three simulations follow $t_1(n) = \num{2.10086e-5} n^{2.073}$, $t_2(n)=\num{1.36362e-3} n^{1.385}$, and $t_3(n) \num{2.10896e-4} n^{1.169}$ regressions, respectively.
  }
\end{figure}

\begin{figure}
  \centering
  \conditionalFigureInput{figures/fast_slow_comparison.tex}
  \caption{\label{fig:qd fast comparison}Comparison of a Maxwell-Bloch \qd{} simulation with direct (red) and FFT-accelerated (blue) propagation strategies.
    Here, we show the off-diagonal matrix elements of 25 random \qds{} (out of 1024) as a function of time.
    The \qds{} experience an incident $\pi$-pulse and their interactions induce significant secondary oscillations (see the second-to-last row).
    The FFT-accelerated algorithm developed here readily captures these effects, making it eminently suitable for large-scale \qd{} simulations.
  }
\end{figure}
