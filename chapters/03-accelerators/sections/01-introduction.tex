\section{Introduction}

The na\"ive $\mathcal{O}(n^2)$ complexity of matrix-vector products in integral equation systems stands as the singular bottleneck to large-scale simulations of related phenomena.
Since the 1990s, so-called ``fast'' solvers have played a critical role in alleviating this bottleneck; by exploiting properties of the underlying physical system (such as smoothness and translational invariance) these methods achieve space and time complexities of $\mathcal{O}(n \log n)$ or even $\mathcal{O}(n)$ with fully-controllable error\footnote{In particular this means one can choose parameters such that the error in the calculation falls below the floating point error of a given machine, thus fast solution techniques produce identical results to their na\"ive counterparts with \emph{no} discernable approximation error.}.

\emph{Kernel-type} solvers~\cite{Greengard1987,PWTD} exploit the physical similarity of sources at large distances to develop a series representation of the Green's function in terms of source and observation regions independently.
Specifically, these techniques lump adjacent sources together so as to transmit their effects in aggregate across large distances (a process not dissimilar to how a telephone network or the Internet works).
These methods scale very well for surface discretizations as they intelligently partition the simulation space into a hierarchy of boxes and only compute interactions between regions containing sources.
Unfortunately, as these methods rely on the underlying mathematical structure of the kernel to produce a suitable series representation, they have limited utility in systems with nonstandard (specifically rotating-frame) kernels.

In contrast, \emph{source-type} solvers~\cite{Bleszynski1996,Yilmaz2004,Kapur1997} apply effective transformations to the source distribution (leaving the actual interaction kernel unchanged) with the intention of producing a nearly-equivalent system with exploitable low-rank properties.
The Adaptive Integral Method (AIM), which we extend here, represents a kernel matrix as a sum of nearfield and farfield contributions, each constructed from a different set of basis functions.
For all basis functions within a prescribed distance of each other, we compute their interactions directly, thus the nearfield matrix becomes increasingly sparse in the limit of a large number of sources.
The farfield contribution arises as the interaction between a reduced number of auxiliary basis functions (commonly taken as $\delta$-functions) on a regular Cartesian grid.
Each source function maps to a finite number of overlapping auxiliary functions---thus the mapping matrices have a sparse signature and compress the distant portions of the interaction matrix---and the translationally-invariant structure imposed by the Cartesian grid allows for an efficient matrix diagonalization by way of a (multidimensional) fast Fourier transform (FFT).
Because AIM does not rely on the specific mathematical form of the interaction kernel (a so-called ``algebraic acceleration''), it works equally well for rotating and fixed-frame problems.

