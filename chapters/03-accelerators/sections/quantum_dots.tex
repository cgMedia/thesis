An incident electric field $\vb{E}_\text{inc}(\vb{r}, t)$ induces a source polarization distribution, $\vb{P}(\vb{r}, t)$, on a collection of objects (either discrete or continuous) embedded in a homogeneous background medium by way of an underlying linear~\cite{} or nonlinear~\cite{Glosser2017} process.
This current subsequently generates a secondary $\vb{E}_\text{scat}(\vb{r}, t) = \mathfrak{F}\qty{\vb{P}(\vb{r}, t)}$, thus the total field at any spacetime coordinate becomes
\begin{equation}
  \vb{E}(\vb{r}, t) = \vb{E}_\text{inc}(\vb{r}, t) + \mathfrak{F}\qty{\vb{P}(\vb{r}, t)}.
\end{equation}
Assuming $\vb{E}(\vb{r}, t)$ consists of a low-frequency envelope modulated by a high-frequency sinusoid, i.e. $\vb{E}(\vb{r}, t) = \tilde{\vb{E}}(\vb{r}, t) e^{i \omega_L t}$, we can suppress the $e^{i \omega_L t}$ term in favor of an assumed spatial phase factor.
Thus,
\begin{equation}
  \tilde{\vb{E}}(\vb{r}, t) = \tilde{\vb{E}}_\text{inc}(\vb{r}, t) + \tilde{\mathfrak{F}}\qty{\tilde{\vb{P}}(\vb{r}, t)}
\end{equation}
where tildes on field variables denote envelope functions.
By enforcing radiation boundary conditions (namely that $\vb{E}_\text{scat}(\vb{r}, t) \to 0$ as $\abs{\vb{r}} \to \infty$), Maxwell's equations require
\begin{equation}
  \mathfrak{F}\qty{\vb{P}(\vb{r}, t)} = -\frac{\mu}{4\pi} \siint \frac{\delta(t_R - t')}{\abs{\vb{r} - \vb{r}'}} \qty(\pdv[2]{{t'}} - c^2 \grad' \grad'\boldsymbol{\cdot}) \vb{P}(\vb{r}', t') \dd[1]{t'} \dd[3]{\vb{r}'}
  \label{eq:efie}
\end{equation}
where $\mu$ characterizes the permeability of the background medium and $t_R \equiv t - \abs{\vb{r} - \vb{r}'}/c$ (see \cref{appendix:vector wave equation}).
Accordingly,
\begin{equation}
  \tilde{\mathfrak{F}}\qty{\tilde{\vb{P}}(\vb{r}, t)}e^{i \omega_L t} = -\frac{\mu}{4\pi} \siint \frac{\delta(t_R - t')}{\abs{\vb{r} - \vb{r}'}} \qty(\pdv[2]{{t'}} - c^2 \grad' \grad'\boldsymbol{\cdot}) \tilde{\vb{P}}(\vb{r}', t') e^{i \omega_L t'} \dd[1]{t'} \dd[3]{\vb{r}'}
  \label{eq:efie envelope}
\end{equation}
and, after performing the temporal integration and suppressing the $e^{i \omega_L t}$ factor on both sides of the equation, we have
\begin{equation}
  \tilde{\mathfrak{F}}\qty{\tilde{\vb{P}}(\vb{r}, t)} = -\frac{\mu}{4\pi} \sint \frac{e^{- i \omega_L \abs{\vb{r} - \vb{r}'}}}{\abs{\vb{r} - \vb{r}'}} \qty(\pdv[2]{t} + 2 i \omega_L \pdv{t} - \omega_L^2 - c^2 \grad' \grad'\boldsymbol{\cdot}) \tilde{\vb{P}}(\vb{r}', t_R) \dd[3]{\vb{r}'}.
  \label{eq:efie envelope}
\end{equation}
(We note that $\partial_{t_R} = \partial_{t}$ and that \cref{eq:efie envelope} recovers \cref{eq:efie} in the limit of $\omega_L \to 0$.)

\hrulefill

Conventionally, $\Delta t^{-1} \propto \omega_\text{max}$---the highest frequency in the system.
By formulating the problem in terms of envelope functions, however, we may choose $\Delta t$ proportional to the bandwidth of the envelopes.
In narrowband applications that evade a frequency-domain analysis---such as those with nonlinear (semiclassical) trappings---this can allow a $\Delta t$ several orders of magnitude larger than in conventional simulations.
This envelope shifting approach does not work equally well in space, however; due to the presence of the $e^{i \omega_L \abs{\vb{r} - \vb{r}'}/c}$ factor in \cref{eq:efie envelope}, interference phenomena still occur on the scale of $\omega_L/c$ (i.e. at a high spatial frequency), thus we still require sub-wavelength spatial discretizations.
Finally, we require both $\vb{s}_\ell(\vb{r})$ and $T(t)$ to have finite support, and $T(t)$ must accurately interpolate derivatives of order $0 \leqslant d \leqslant 2$ to capture the full dynamics of \cref{eq:efie envelope}.

