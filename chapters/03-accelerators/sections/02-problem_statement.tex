\section{Formulation}

The underlying mechanics remain effectively unchanged from \cref{ch:quantum dots}: a collection of $N_s$ pointlike \qds{} evolve due to incident electric field via \cref{eq:liouville}, implicitly coupled by way of secondary radiated fields.
Under the rotating-wave approximation~\cref{Allen1975}, the off-diagonal elements of a given \qd's density matrix, $\tilde{\rho}$, directly characterize the (complex-valued) polarization density of the \qd ($\tilde{\vb{P}}$).
This polarization density, then, acts as a source in Maxwell's equations to produce
\begin{equation}
  \tilde{\mathfrak{F}}\qty{\tilde{\vb{P}}(\vb{r}, t)} = -\frac{\mu_0}{4\pi} \int \qty(\pdv[2]{t} - c^2 \grad' \grad'\boldsymbol{\cdot}) \frac{\tilde{\vb{P}}(\vb{r}', t_R) e^{-i \omega_L \abs{\vb{r} - \vb{r}'}/c}}{\abs{\vb{r} - \vb{r}'}} \dd[3]{\vb{r}'}
\end{equation}
(see \cref{appendix:vector wave equation}).
This form of \cref{eq:radiated envelope} has several advantages in developing accelerated calculation schemes which we shall detail later.
Representing $\tilde{\vb{P}}(\vb{r}, t)$ as a collection of spatio-temporal basis functions,
\begin{equation}
  \tilde{\vb{P}}(\vb{r}, t) = \sum_{\ell = 0}^{N_s - 1} \sum_{m = 0}^{N_t - 1} \tilde{\mathcal{A}}_\ell^{(m)} \vb{S}_\ell(\vb{r}) T(t - m \, \Delta t),
\end{equation}
we have again chosen $\vb{S}_\ell(\vb{r}) = \vb{d}_\ell \delta(\vb{r} - \vb{r}_\ell)$ and $T(t)$ as the standard Lagrange polynomial basis set.
Additionally, AIM embeds $\tilde{\vb{P}}(\vb{r}, t)$ within a cartesian grid with spacings chosen to sufficiently resolve the interference phenomena of \emph{unshifted}, fixed-frame sources.
This grid allows for the construction of auxilliary sources at each gridpoint, chosen so as to accurately reproduce fields eminating from the primary source at large distances.
As a result, interactions between gridpoints have a Toeplitz matrix structure due to the translational invariance in $g(\vb{r} - \vb{r}')$ which facilitates rapid diagonalization via $n$-dimensional fast Fourier transforms.

\subsection{Building auxilliary sources}

\begin{figure}
  \centering
  \conditionalFigureInput{figures/expansion-grid}
  \caption{\label{fig:expansion grid}Spatial expansion pattern for a two-dimensional system.
    Increasing the expansion order incorporates new grid points in such a way as to keep the original primary point as close to the center of the box as possible.
  }
\end{figure}

To facillitate FFT-based matrix-vector products, we represent the primary basis functions as a weighted sum of $\delta$-functions on the surrounding gridpoints, i.e.
\begin{equation}
  \vb{S}_\ell(\vb{r}) \approx \sum_{\vb{u} \in C_{\ell}} \Lambda_{\ell\vb{u}} \delta(\vb{r} - \vb{u}).
  \label{eq:grid linear combination}
\end{equation}
Here, $C_\ell$ denotes the collection of grid points, $\vb{u}$, enclosing $\vb{S}_\ell(\vb{r})$.
For an expansion of order\footnote{In principle, one could expand to different orders in different cartesian directions, though this involves considerably more bookkeeping for relatively little benefit. Thus, for convenience, we take ``the expansion order'' to mean the expansion order in every direction.} $M$, this sum contains $(M + 1)^3$ terms corresponding to the $(M + 1)^3$ grid points nearest to $\vb{S}_\ell(\vb{r})$.
Consequently, the $\Lambda_{\ell \vb{u}}$ matrices contain few nonzero elements and we have elected to use a moment-matching scheme to capture the $(M + 1)^3$ multipole moments of $\vb{S}_\ell(\vb{r})$.
\begin{equation}
  \int (x - x_0)^{m_x}(y - y_0)^{m_y}(z - z_0)^{m_z} \qty[\phi_\ell(\vb{r}) - \sum_{\vb{u} \in C_\ell}\Lambda_{\ell\vb{u}}\delta(\vb{r} - \vb{u})] \dd[3]{\vb{r}} = 0
  \label{eq:moment matching}
\end{equation}
In this expression, $0 \le m_x, m_y, m_z \le M$ and $\vb{r}_0 \equiv x_0 \vu{x} + y_0 \vu{y} + z_0\vu{z}$ denotes the origin about which we compute the multipoles.
With infinite precision, the choice of $\vb{r}_0$ merely sets a reference point for the multipole coordinate system.
To minimize numerical issues, however, we take $\vb{r}_0 = \vb{r}$ thus the polynomial terms in \cref{eq:moment matching} equal one (by definition) if any of the $m_{x,y,z} \not = 0$ and zero otherwise.

\Cref{eq:moment matching} defines the $\Lambda_{\ell \vb{u}}$ expansion weights thus, to calculate them, we solve the least-squares system,
\begin{equation}
  \sum_{\vb{u} \in C_\ell} W_{\vb{m}\vb{u}}\Lambda_{\ell\vb{u}} = Q_{\ell\vb{m}},
  \label{eq:expansion matrix system}
\end{equation}
where
\begin{subequations}
  \begin{align}
    Q_{\ell \vb{m}} &= \int \vb{S}_\ell(\vb{r}) (x - x_0)^{m_x} (y - y_0)^{m_y} (z - z_0)^{m_z} \dd[3]{\vb{r}} \\
    W_{\vb{m}\vb{u}} &= (u_x - x_0)^{m_x} (u_y - y_0)^{m_y} (u_z - z_0)^{m_z}.
  \end{align}
\end{subequations}
