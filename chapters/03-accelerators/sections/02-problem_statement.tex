\section{Problem statement}

We wish to calculate the field radiated from a distribution of time-dependent sources under the action of an arbitrary (linear) differential operator, $\mathfrak{F}\qty{\cdot}$.
These fields exist throughout space and time alongside/in response to an incident field, $\mathbf{E}_\text{inc}$.
Formally, we may write the total field everywhere as
\begin{equation}
  \begin{aligned}
    \vb{E}(\vb{r}, t) & = \vb{E}_\text{inc}(\vb{r}, t) + \mathfrak{F}\qty{\siint g(\vb{r} - \vb{r}', t - t') \vb{P}(\vb{r}', t') \dd[1]{t'} \dd[3]{\vb{r}'}} \\
                      & \equiv \vb{E}_\text{inc}(\vb{r}, t) + \mathfrak{F}\qty{g(\vb{r}, t) \ast \vb{P}(\vb{r}, t)}
    \end{aligned}
  \label{eq:propagator}
\end{equation}
where $g(\vb{r}, t)$ denotes a propagation potential (most commonly the retarded potential, $g(\vb{r}, t) = \delta(t - \abs{\vb{r}}/c)/\abs{\vb{r}}$).
To numerically evaluate \cref{eq:propagator}, we discretize $\vb{P}(\vb{r}, t)$ with separable space/time basis functions such that 
\begin{equation}
  \vb{P}(\vb{r}, t) \approx \sum_{\ell = 0}^{N_s - 1} \sum_{m = 0}^{N_t - 1} \mathcal{A}_\ell^{(m)} \vb{s}_\ell(\vb{r}) T(t - m \, \Delta t)
  \label{eq:discretization}
\end{equation}
where $\Delta t$ denotes a fixed time interval chosen to accurately sample the dynamics of the physical quantities involved.
Furthermore, we require both $\vb{s}_\ell(\vb{r})$ and $T(t)$ to have finite support and that $T(t)$ obey (discrete) causality (i.e. $T(t) = 0$ if $t \not \in \qty[-\Delta t, T_\text{max}]$).
Here, we consider $\delta$-functions and shifted Lagrange polynomials for the $\vb{s}_\ell(\vb{r})$ and $T(t$), respectively, though this analysis readily extends to accommodate any simliar set of functions.

Substituting \cref{eq:discretization} into \cref{eq:propagator} and projecting the resulting field onto the same set of $\vb{s}_\ell(\vb{r})$ produces an $(N_s \times N_s)$-block matrix equation of the form
\begin{equation}
  \begin{aligned}
    \mathcal{E}^{(m)} &= \mathcal{E}_\text{inc}^{(m)} + \sum_{m'= 0}^m \mathcal{F}^{(m - m')} \boldsymbol{\cdot} \mathcal{A}^{(m')}
  \end{aligned}
  \label{eq:mot}
\end{equation}
where
\begin{subequations}
  \begin{align}
    \mathcal{E}^{(m)}_\ell &= \langle \vb{s}_\ell(\vb{r}), \vb{E}(\vb{r}, m \, \Delta t) \rangle \label{eq:f elements}\\
    \mathcal{E}^{(m)}_{\text{inc},\ell} &= \langle \vb{s}_\ell(\vb{r}), \vb{E}_\text{inc}(\vb{r}, m \, \Delta t) \rangle \label{eq:l elements}\\
    \mathcal{F}^{(k)}_{\ell\ell'} &= \big\langle \vb{s}_\ell(\vb{r}), \mathfrak{F}\qty{g(\vb{r}, t) \ast \vb{s}_{\ell'}(\vb{r}) T(k \, \Delta t)} \big\rangle \label{eq:z elements}.
  \end{align}
\end{subequations}
From this, it immediately becomes apparent that \cref{eq:z elements} bottlenecks the field calculation due to the $\mathcal{O}(N_s^2)$ complexity of the inner product.

%\subsection{PWTD breakdown and motivation for kernel-independent acceleration}

%Numerous schemes exist to ameliorate this bottleneck and accelerate calculations of the form in \cref{eq:mot}.
%The Plane-Wave Time-Domain (PWTD) algorithm in particular stands as the time-domain analogue of frequency-domain fast multipole methods that reconstruct time-harmonic fields as a superposition of plane waves~\cite{PWTD}.
%Specifically, PWTD accelerates the calculation of a retarded potential,
%\begin{equation}
  %u(\vb{r}, t) = \sint \frac{\delta(t - \abs{\vb{r} - \vb{r}'}/c)}{\abs{\vb{r} - \vb{r}'}} \star q(\vb{r}', t) \dd[3]{\vb{r}'},
  %\label{eq:retarded potential}
%\end{equation}
%via the Weyl identity,
%\begin{subequations}
  %\begin{align}
    %u_W(\vb{r}, t) & = -\frac{\partial_t}{2 \pi c} \siint \delta\qty[t - \vu{k} \vdot (\vb{r} - \vb{r}')/c] \star q(\vb{r}', t )\dd[2]{\vu{k}} \dd[3]{\vb{r}'}  \label{eq:weyl-a}\\
                   %& = u(\vb{r}, t) - \sint \frac{\delta(t + \abs{\vb{r} - \vb{r}'}/c)}{\abs{\vb{r} - \vb{r}'}} \star q(\vb{r}', t) \dd[3]{\vb{r}'}. \label{eq:weyl-b}
  %\end{align}
  %\label{eq:weyl}
%\end{subequations}
%In these expressions, $\star$ indicates a convolution with respect to time, $\vu{k} = \vu{x} \sin \theta \cos \phi + \vu{y} \sin \theta \sin \phi + \vu{z} \cos \theta$, and $\sint \cdot \dd[2]{\vu{k}}$ an integration over the whole solid angle.
%A detailed explanation of how \cref{eq:weyl} accelerates retarded potentials lies beyond the scope of this thesis, however it suffices to say that \cref{eq:weyl-a} powers the method by aggregating $q(\vb{r}, t)$ within a source region as a collection of plane-waves.
%A clever\footnote{Some might even call it black magic.} time-gating procedure then removes the advanced potential contribution in \cref{eq:weyl-b} to recover $u(\vb{r}, t)$.

%Within a \emph{rotating} reference frame, however, $q(\vb{r}, t) = \tilde{q}(\vb{r}, t)e^{-i \omega_L t}$.
%Inserting this into \cref{eq:weyl-a} and removing the remnant sinusoid gives
%\begin{equation}
  %\tilde{u}_W(\vb{r}, t) = -\frac{\partial_t}{2\pi c} \siint \delta\qty[t - \vu{k} \cdot \qty(\vb{r} - \vb{r}')/c] \star \tilde{q}(\vb{r}', t)e^{i k_L \vu{k} \vdot \qty(\vb{r} - \vb{r}')} \dd[2]{\vu{k}} \dd[3]{\vb{r}'}.
%\end{equation}
%The inner integral over $\dd[2]{\vu{k}}$ effectively becomes
%\begin{equation}
  %\Xi(t) = \int_{-1}^1 \delta\qty(t - R x/c) \star \tilde{q}(\vb{r}', t)e^{i k_L R x} \dd{x}
%\end{equation}
%or, in the Fourier domain,
%\begin{equation}
  %\Xi(\omega) = \int_{-1}^1 e^{i \qty(\omega_L + \omega) R x/c} \tilde{q}(\vb{r}', \omega) \dd{x}
%\end{equation}

%\begin{figure}
  %\centering
  %\begin{filecontents}{comparison_samples.dat}
0.                  5.03840918845741e-13    5.034325783767433e-13   1.176819518478076e-14  1.135837131556083e-14  1.00817134913272e-14   9.673713300828692e-15  1.00817134913272e-14   9.673713300828692e-15  1.067364232319853e-14  1.027782868302563e-14  -3.702516063245501e-15  -4.097935594704346e-15  8.035215239050957e-15  7.639228003416348e-15  -1.414704472710738e-14  -1.454387071675579e-14  -1.326368727717548e-13  -1.330327151211613e-13  1.153002181789433e-16   -2.697145260122176e-16  -3.553078395465477e-15  -3.929004998867832e-15  -1.472158789841377e-14  -1.507906184070376e-14  -2.004017379845655e-15  -2.363538165537396e-15  2.677920696840937e-15  2.348755511170132e-15  1.556497654852352e-14  1.521787431814239e-14  -2.890338347680407e-16  -6.272874450518112e-16  -2.890338347680407e-16  -6.272874450518112e-16  4.719405984763295e-14  4.685479863865661e-14  -1.128039660069601e-14  -1.161058342737656e-14  4.60203274406879e-14   4.568507648936817e-14  -1.205336203715765e-14  -1.23793068156921e-14   -1.08822944652873e-14   -1.120109962886884e-14  8.334126824655722e-15  8.009341724480178e-15  2.590866641481971e-15  2.267804341149567e-15  1.393028093587157e-15   1.086687237424776e-15 
0.20202020202020202 9.300673553721208e-10   9.300670652482533e-10   1.781800152153207e-10  1.781794664145248e-10  1.53690924628106e-10   1.536903764750387e-10  1.53690924628106e-10   1.536903764750387e-10  1.62171711855792e-10   1.621711984495941e-10  -6.262112330532381e-11  -6.26216756680512e-11   1.2440376463049e-10    1.244032365517132e-10  -2.276304434904313e-10  -2.276310171141011e-10  -1.718374901149065e-9   -1.718374941945101e-9   -3.133251106897833e-12  -3.133767149392428e-12  -5.994222233169735e-11  -5.994272132676244e-11  -2.343937438898592e-10  -2.343942267395241e-10  -4.096248613212347e-11  -4.096292207559835e-11  3.903975162623372e-11  3.903936260998194e-11  2.431173083649522e-10  2.431169352317643e-10  -8.314046231190261e-12  -8.314448338984669e-12  -8.314046231190261e-12  -8.314448338984669e-12  7.310372218444931e-10  7.31036965653358e-10   -1.726245803590256e-10  -1.7262498567849e-10    7.109388854048437e-10  7.109386414873028e-10  -1.943391185558319e-10  -1.943395035464573e-10  -1.758562180081341e-10  -1.758565853743768e-10  1.274842371895603e-10  1.274838926872916e-10  3.697488976154174e-11  3.697452831084737e-11  1.82787584638696e-11    1.82784356469768e-11  
0.40404040404040403 3.57945713046575e-9     3.579457005127954e-9    1.114127482571177e-9   1.11412688964266e-9    1.002627911390671e-9   1.002627275192628e-9   1.002627911390671e-9   1.002627275192628e-9   1.030044702049242e-9   1.030044128965483e-9   -4.039421659173263e-10  -4.039428264904915e-10  7.971600803930266e-10  7.971594760093546e-10  -1.480964298145711e-9   -1.48096497080124e-9    -7.383211317681868e-9   -7.38321061073978e-9    -3.119140846739637e-11  -3.119201985873848e-11  -3.854330081734338e-10  -3.854336134831813e-10  -1.531797822912377e-9   -1.53179843343353e-9    -3.409154813445265e-10  -3.409159802490894e-10  2.601184294930775e-10  2.601178888182314e-10  1.58605067281745e-9    1.586050242571473e-9   -5.172160817091061e-11  -5.172215529559302e-11  -5.172160817091061e-11  -5.172215529559302e-11  4.64747726802455e-9    4.647477052041367e-9   -1.127635141729113e-9   -1.127635688728881e-9   4.497066961015541e-9   4.49706676564552e-9    -1.279002512440199e-9   -1.279003029544417e-9   -1.15663484120565e-9    -1.156635349639677e-9   8.298451245138712e-10  8.298446568523594e-10  2.430176553179021e-10  2.430171378826918e-10  1.206258224889141e-10   1.206253307235804e-10 
0.6060606060606061  1.076824591045563e-8    1.076824635630655e-8    4.25969122568025e-9    4.259690706106738e-9   4.014639724130823e-9   4.014639093856784e-9   4.014639724130823e-9   4.014639093856784e-9   3.984969899227282e-9   3.984969381576546e-9   -1.589312272715419e-9   -1.589312951785391e-9   3.120557754131412e-9   3.120557182844429e-9   -5.877781763884766e-9   -5.877782445830984e-9   -2.073411297946865e-8   -2.073411301254051e-8   -1.495698711329932e-10  -1.495704646789758e-10  -1.516674278159371e-9   -1.516674869363558e-9   -6.095557102860553e-9   -6.095557714100375e-9   -1.544525478786303e-9   -1.544525864228968e-9   1.05179213083951e-9    1.051791590364156e-9   6.185807388481568e-9   6.185807052268588e-9   -1.975401912599117e-10  -1.97540735653722e-10   -1.975401912599117e-10  -1.97540735653722e-10   1.790766316751197e-8   1.790766322560851e-8   -4.487267652551592e-9   -4.487268190037739e-9   1.724144068350551e-8   1.724144077801635e-8   -5.115909394341786e-9   -5.1159098899873e-9     -4.62347559318506e-9    -4.623476081372805e-9   3.285699264959945e-9   3.285698853311828e-9   9.779154829753893e-10  9.779149717897734e-10  4.858197554454305e-10   4.858192705736549e-10 
0.8080808080808081  3.059416731113258e-8    3.059416798759079e-8    1.381943386523291e-8   1.381943342006281e-8   1.353409667384242e-8   1.353409605117128e-8   1.353409667384242e-8   1.353409605117128e-8   1.301676281157363e-8   1.301676234569394e-8   -5.270227419253117e-9   -5.270228123214616e-9   1.031228759660638e-8   1.031228705695167e-8   -1.96196798466097e-8    -1.961968053683158e-8   -5.964869186832518e-8   -5.964869249402517e-8   -5.350485355926063e-10  -5.35049111000314e-10   -5.040132979354662e-9   -5.040133552464423e-9   -2.036198796130218e-8   -2.03619885675485e-8    -5.45387772455255e-9    -5.453877976246228e-9   3.550112609563583e-9   3.550112067183633e-9   2.010381845944647e-8   2.010381821044149e-8   -6.447171532968637e-10  -6.447176989526358e-10  -6.447171532968637e-10  -6.447176989526358e-10  5.813561465093085e-8   5.813561496944352e-8   -1.498336366518463e-8   -1.49833641878658e-8    5.574652879644946e-8   5.574652916728114e-8   -1.714164501869645e-8   -1.714164549113061e-8   -1.548451593049006e-8   -1.548451639597278e-8   1.092463120211594e-8   1.092463085251711e-8   3.303701891267566e-9   3.303701382647267e-9   1.638285301936934e-9    1.638284820084841e-9  
1.0101010101010102  8.447870250559781e-8    8.44787029078558e-8     4.142920716418322e-8   4.142920677798851e-8   4.175931530581703e-8   4.175931470017926e-8   4.175931530581703e-8   4.175931470017926e-8   3.920150679172494e-8   3.920150636477041e-8   -1.606090380759667e-8   -1.606090453614439e-8   3.134750679205135e-8   3.134750627777776e-8   -6.007617266955913e-8   -6.007617336121723e-8   -1.70907279206162e-7    -1.709072790019554e-7   -1.696185913606291e-9   -1.696186471373464e-9   -1.539237684768231e-8   -1.539237739422803e-8   -6.236100389279959e-8   -6.236100448398269e-8   -1.722676583372744e-8   -1.72267659548376e-8    1.095066176576866e-8   1.09506612233363e-8    6.0146586063266e-8     6.014658588993072e-8   -1.94651563994496e-9    -1.946516186560881e-9   -1.94651563994496e-9    -1.946516186560881e-9   1.741732485007989e-7   1.741732490106278e-7   -4.586911423752339e-8   -4.586911473637097e-8   1.664967963776166e-7   1.664967969509884e-7   -5.26003849630581e-8    -5.260038540875957e-8   -4.750175028917392e-8   -4.750175072716115e-8   3.334189893525665e-8   3.334189865315228e-8   1.020701329065776e-8   1.020701278504037e-8   5.051824113839165e-9    5.051823635049854e-9  
1.2121212121212122  2.256836087433338e-7    2.256836085799562e-7    1.179785578331133e-7   1.179785574918585e-7   1.215422495412965e-7   1.215422489754069e-7   1.215422495412965e-7   1.215422489754069e-7   1.120508287102715e-7   1.120508283067675e-7   -4.632907675234531e-8   -4.632907750429703e-8   9.02448184238177e-8    9.024481793095882e-8   -1.739239748788377e-7   -1.739239755700852e-7   -4.660393463967824e-7   -4.660393457884475e-7   -5.046878089371486e-9   -5.04687862718574e-9    -4.447158665626817e-8   -4.447158717076322e-8   -1.805639158468138e-7   -1.805639164151338e-7   -5.109697774298503e-8   -5.109697772336036e-8   3.188320252178665e-8   3.188320198057352e-8   1.711314322341031e-7   1.71131432089561e-7    -5.575430955902363e-9   -5.575431503663435e-9   -5.575430955902363e-9   -5.575431503663435e-9   4.959830321736266e-7   4.959830327635758e-7   -1.327718694833809e-7   -1.327718699502326e-7   4.7296335577569e-7     4.729633564355568e-7   -1.525208489007052e-7   -1.52520849316905e-7    -1.377124952967195e-7   -1.377124957030471e-7   9.630390380457258e-8   9.63039035929614e-8    2.974586871698274e-8   2.974586821683376e-8   1.47024206237898e-8     1.470242014846933e-8  
1.4141414141414141  5.802910780938286e-7    5.80291077498138e-7     3.221806025590096e-7   3.221806022490997e-7   3.379262797587977e-7   3.379262792630159e-7   3.379262797587977e-7   3.379262792630159e-7   3.069928420229423e-7   3.069928416264113e-7   -1.278984639874429e-7   -1.278984647607622e-7   2.48717668659099e-7    2.487176681844177e-7   -4.81586052088926e-7    -4.815860527812496e-7   -1.215876532734779e-6   -1.215876533032418e-6   -1.430903659492235e-8   -1.430903710925472e-8   -1.229281703487438e-7   -1.229281708279322e-7   -5.000351845922428e-7   -5.000351851317739e-7   -1.444440203992372e-7   -1.44444020224866e-7    8.870569154304491e-8   8.870569100372048e-8   4.672830236328174e-7   4.672830234534379e-7   -1.530033675628923e-8   -1.530033730580112e-8   -1.530033675628923e-8   -1.530033730580112e-8   1.354897104832173e-6   1.354897105374538e-6   -3.675942364438569e-7   -3.675942368719261e-7   1.289334426660707e-6   1.289334427275482e-6   -4.228692722545652e-7   -4.228692726389707e-7   -3.817655390156754e-7   -3.817655393868613e-7   2.661872394589918e-7   2.661872393195451e-7   8.280470895669286e-8   8.280470846549678e-8   4.088657186633041e-8    4.088657139497383e-8  
1.6161616161616161  1.433503557480973e-6    1.433503556904097e-6    8.461645627178779e-7   8.461645624146038e-7   9.022353457468227e-7   9.022353453444995e-7   9.022353457468227e-7   9.022353453444995e-7   8.087493106494622e-7   8.087493102401988e-7   -3.393043271419891e-7   -3.393043279317478e-7   6.588172721719782e-7   6.588172717063084e-7   -1.281122672123279e-6   -1.28112267281384e-6    -3.050517560520064e-6   -3.050517561051644e-6   -3.888462489981817e-8   -3.888462539160429e-8   -3.26503336545989e-7    -3.265033369846755e-7   -1.330346861937098e-6   -1.330346862440739e-6   -3.914959379397598e-7   -3.914959376293891e-7   2.370078825694661e-7   2.370078820331438e-7   1.227295805963107e-6   1.227295805712415e-6   -4.036923698936515e-8   -4.036923754038528e-8   -4.036923698936515e-8   -4.036923754038528e-8   3.559868751238956e-6   3.55986875161933e-6    -9.777628217360802e-7   -9.777628221194395e-7   3.3810408185705e-6     3.381040819021411e-6   -1.126230968178097e-6   -1.126230968528802e-6   -1.016652975598127e-6   -1.016652975931531e-6   7.069892192205794e-7   7.069892191545627e-7   2.21340643623513e-7    2.21340643143683e-7    1.091947835855795e-7    1.091947831185939e-7  
1.8181818181818181  3.401554787698662e-6    3.401554787569614e-6    2.138834928507535e-6   2.138834928187405e-6   2.318310188031843e-6   2.318310187737746e-6   2.318310188031843e-6   2.318310187737746e-6   2.050588189091123e-6   2.050588188652709e-6   -8.66360889735676e-7    -8.663608905361367e-7   1.679632237780254e-6   1.679632237314145e-6   -3.28009192411914e-6    -3.280091924803902e-6   -7.352629037378508e-6   -7.352629036989139e-6   -1.016133150176175e-7   -1.016133154895311e-7   -8.346668205629248e-7   -8.346668209557814e-7   -3.406462124040355e-6   -3.406462124501118e-6   -1.02059322095201e-6    -1.020593220539212e-6   6.094288695389607e-7   6.094288690066906e-7   3.102183980934336e-6   3.102183980598947e-6   -1.025191998793761e-7   -1.025192004314758e-7   -1.025191998793761e-7   -1.025192004314758e-7   9.001787569602422e-6   9.0017875697454e-6     -2.503074999100619e-6   -2.503074999434117e-6   8.532747784182932e-6   8.532747784392434e-6   -2.886788358243191e-6   -2.886788358558073e-6   -2.605651641254743e-6   -2.605651641548213e-6   1.80726191969963e-6    1.80726191970901e-6    5.693994091707626e-7   5.693994087037087e-7   2.806584070662671e-7    2.806584066039311e-7  
2.0202020202020203  7.75457013314164e-6     7.75457013355041e-6     5.203964430039113e-6   5.203964429697289e-6   5.738691900601966e-6   5.738691900427475e-6   5.738691900601966e-6   5.738691900427475e-6   5.005544787171467e-6   5.005544786698587e-6   -2.130512258414097e-6   -2.130512259222068e-6   4.123931515927658e-6   4.123931515459618e-6   -8.08925872460596e-6    -8.089258725286448e-6   -0.0000170125979870488  -0.00001701259798661517 -2.557391769920223e-7   -2.557391774432962e-7   -2.055175757303893e-6   -2.055175757649319e-6   -8.401656053147044e-6   -8.401656053560923e-6   -2.562924931145061e-6   -2.562924930643166e-6   1.509616403662675e-6   1.50961640313431e-6    7.54672158272281e-6    7.54672158228989e-6    -2.507023256139355e-7   -2.50702326167906e-7    -2.507023256139355e-7   -2.50702326167906e-7    0.00002191039803219778 0.00002191039803208218 -6.172113590050493e-6   -6.172113590330071e-6   0.00002072505074079234 0.00002072505074074323 -7.127516126550507e-6   -7.127516126826862e-6   -6.432726113742566e-6   -6.43272611399416e-6    4.449617910071652e-6   4.449617910159542e-6   1.41116842622909e-6    1.41116842577605e-6    6.949286914111112e-7    6.949286909534678e-7  
2.2222222222222223  0.00001698636406878781  0.00001698636406942544  0.0000121884100663653  0.0000121884100660062  0.00001369241455469832 0.00001369241455464486 0.00001369241455469832 0.00001369241455464486 0.00001176500016312143 0.00001176500016261658 -5.047838443895769e-6   -5.04783844470934e-6    9.754340598581824e-6   9.754340598113283e-6   -0.0000192243876790054  -0.00001922438767968396 -0.00003779308951529911 -0.00003779308951577555 -6.203298940725496e-7   -6.203298945024851e-7   -4.876148708692356e-6   -4.876148708989886e-6   -0.00001996820904205374 -0.00001996820904241792 -6.204418155568471e-6   -6.204418154991687e-6   3.604397564024251e-6   3.604397563499089e-6   0.0000176686868286712  0.00001766868682814914 -5.905186364486289e-7   -5.905186370057213e-7   -5.905186364486289e-7   -5.905186370057213e-7   0.00005133533841300994 0.00005133533841267186 -0.00001466561680467601 -0.00001466561680489845 0.00004844607030419055 0.00004844607030392724 -0.00001695888072517167 -0.00001695888072540679 -0.00001530413415646267 -0.00001530413415667104 0.00001055578673573162 0.00001055578673590137 3.371310984234385e-6   3.371310983795358e-6   1.658536757695562e-6    1.658536757242299e-6  
2.4242424242424243  0.00003575500541030865  0.00003575500541070364  0.00002748071638260661 0.00002748071638223102 0.00003150143580833526 0.00003150143580838699 0.00003150143580833526 0.00003150143580838699 0.0000266283069124471  0.00002662830691191495 -0.00001152601172733143 -0.00001152601172814772 0.00002223190251931115 0.00002223190251884022 -0.00004404080296823491 -0.00004404080296890837 -0.00008061074605229122 -0.0000806107460526291  -1.450854963756212e-6   -1.450854964168415e-6   -0.00001115138048648228 -0.00001115138048672898 -0.00004574677706306818 -0.00004574677706337851 -0.00001448637035912407 -0.0000144863703585144  8.298282137658897e-6   8.298282137136749e-6   0.00003981021079641271 0.00003981021079585157 -1.340078368877788e-6   -1.34007836943846e-6    -1.340078368877788e-6   -1.34007836943846e-6    0.0001157827721722143  0.0001157827721717377  -0.00003358958778857917 -0.00003358958778874192 0.0001089859885350083  0.000108985988534624   -0.00003889836124769144 -0.00003889836124788334 -0.00003509924086448087 -0.00003509924086464557 0.00002413523253740125 0.0000241352325376551  7.766924530919107e-6   7.766924530491947e-6   3.816770940424793e-6    3.816770939975354e-6  
2.6262626262626263  0.00007232458213104841  0.00007232458213091843  0.00005964703801932652 0.00005964703801894188 0.00006990281228293428 0.00006990281228306683 0.00006990281228293428 0.00006990281228306683 0.00005804272918521208 0.00005804272918466181 -0.00002536933129733545 -0.00002536933129815234 0.00004883614279237558 0.00004883614279190274 -0.00009728334459077721 -0.00009728334459144195 -0.0001650715060132828  -0.0001650715060127466  -3.273148225247616e-6   -3.273148225645554e-6   -0.00002458789913214331 -0.00002458789913233656 -0.0001010525285207901  -0.0001010525285210433  -0.00003263511818399914 -0.00003263511818339713 0.00001842796014929431 0.00001842796014877486 0.00008632266546269549 0.00008632266546215464 -2.93050636582402e-6    -2.930506366388726e-6   -2.93050636582402e-6    -2.930506366388726e-6   0.0002513930271504969  0.0002513930271499931  -0.00007417587780249827 -0.00007417587780260015 0.0002359551325506202  0.0002359551325502367  -0.00008603239576148667 -0.00008603239576163414 -0.00007762195841809788 -0.00007762195841821923 0.00005320104375815559 0.0000532010437584937  0.00001726135918086084 0.0000172613591804428  8.472143181606102e-6    8.472143181159924e-6  
2.8282828282828283  0.0001405940196893418   0.0001405940196887989   0.0001246350327852731  0.0001246350327848995  0.0001496578018704221  0.0001496578018706129  0.0001496578018704221  0.0001496578018706129  0.000121856230439573   0.0001218562304390187  -0.00005383947500621055 -0.00005383947500702803 0.0001034160790377539  0.0001034160790372855  -0.0002072641257773691  -0.0002072641257780254  -0.0003245014335053654  -0.0003245014335051199  -7.12524859556738e-6    -7.125248595950908e-6   -0.00005228462985518499 -0.00005228462985532494 -0.000215286649833044   -0.0002152866498332397  -0.00007096412606991759 -0.00007096412606933831 0.00003948585512446769 0.00003948585512394985 0.0001801319277983883  0.0001801319277979061  -6.176913195714453e-6   -6.176913196284418e-6   -6.176913195714453e-6   -6.176913196284418e-6   0.000525492421199159   0.0005254924211987388  -0.0001579755721548713  -0.0001579755721549131  0.0004916227186809706  0.0004916227186807112  -0.0001835311290169221  -0.0001835311290170249  -0.0001655736680199137  -0.0001655736680199926  0.0001130871138440286  0.0001130871138444482  0.0000370184172619221  0.00003701841726151118 0.00001814487535225336  0.00001814487535180999
3.0303030303030303  0.0002626661442997006   0.0002626661442991601   0.0002507194851107486  0.000250719485110401   0.0003092251692317279  0.0003092251692319524  0.0003092251692317279  0.0003092251692319524  0.0002464263379296667  0.0002464263379291188  -0.0001101961046504388  -0.000110196104651256   0.0002111635735639015  0.0002111635735634416  -0.00042603067290589    -0.000426030672906536   -0.000612358472643092   -0.0006123584726436297  -0.00001497205614977081 -0.00001497205615014094 -0.0001072537956948967  -0.0001072537956949827  -0.0004424808597532104  -0.0004424808597533486  -0.0001489990373324109  -0.0001489990373318741  0.0000816637385499599  0.00008166373854944288 0.000361728016205571   0.0003617280162051813  -0.00001255240175275285 -0.0000125524017533288  -0.00001255240175275285 -0.0000125524017533288  0.00105756042229294    0.001057560422292688   -0.0003245697500022189  -0.000324569750002203   0.0009857582979042193  0.000985758297904179   -0.0003777468020010739  -0.0003777468020011338  -0.0003407578437340036  -0.0003407578437340424  0.0002318753305687683  0.0002318753305692627  0.00007663450281213756 0.00007663450281173185 0.00003750779393291404  0.00003750779393247317
3.2323232323232323  0.0004716601278074341   0.0004716601278073023   0.0004855460941410043  0.0004855460941406839  0.0006168188168563168  0.0006168188168565489  0.0006168188168563168  0.0006168188168565489  0.0004800747588988813  0.0004800747588983419  -0.0002175829665255134  -0.0002175829665263268  0.0004158567956364295  0.0004158567956359764  -0.000845136157253348   -0.0008451361572539765  -0.001109218731941529   -0.001109218731941617   -0.00003037906533387192 -0.00003037906533423308 -0.0002123113296631186  -0.000212311329663147   -0.0008776314860444721  -0.0008776314860445506  -0.000302197751108054   -0.0003021977511075992  0.0001630793255881503  0.0001630793255876344  0.0006990050231380925  0.0006990050231378306  -0.00002459970449071662 -0.00002459970449129797 -0.00002459970449071662 -0.00002459970449129797 0.00204921786972417    0.002049217869724128   -0.0006434976372749976  -0.0006434976372749264  0.001902077478255742   0.001902077478255965   -0.0007503604344319236  -0.0007503604344319439  -0.0006768388718012042  -0.0006768388718012061  0.0004587528394456352  0.000458752839446194   0.0001531968835575725  0.0001531968835571701  0.00007485978025341384  0.00007485978025297548
3.4343434343434343  0.0008141068771039756   0.0008141068771043489   0.0009052132807858803  0.0009052132807855881  0.001188197555260975   0.001188197555261196   0.001188197555260975   0.001188197555261196   0.0009010558269648604  0.000901055826964328   -0.0004145746957035383  -0.0004145746957043442  0.0007900934432878168  0.0007900934432873703  -0.001618557476815117   -0.001618557476815722   -0.00192853611132566    -0.001928536111325058   -0.00005954619063500769 -0.00005954619063536239 -0.0004056957619330303  -0.000405695761932999   -0.001680392611728678   -0.001680392611728697   -0.0005923038367270811  -0.0005923038367267321  0.0003145712848692012  0.0003145712848686867  0.001299726947793098   0.001299726947792967   -0.0000465064298296643  -0.00004650642983025042 -0.0000465064298296643  -0.00004650642983025042 0.003823210300369277   0.00382321030036943    -0.001231527208378246   -0.001231527208378124   0.003531626264954337   0.003531626264954811   -0.001438995272876941   -0.001438995272876926   -0.001297939780394327   -0.001297939780394296   0.0008760525853302698  0.0008760525853308803  0.0002958408281805212  0.0002958408281801217  0.0001443100291634717   0.0001443100291630364 
3.6363636363636362  0.00135085707284032     0.001350857072840921    0.001624465185209266   0.001624465185209007   0.002211084097000902   0.002211084097001105   0.002211084097000902   0.002211084097001105   0.001629458972792477   0.001629458972791948   -0.000762475233112917   -0.0007624752331137123  0.001448571904168806   0.001448571904168369   -0.00299360332331792    -0.002993603323318498   -0.003218251862571988   -0.003218251862571955   -0.0001127975679089501  -0.0001127975679092973  -0.0007485885068701015  -0.0007485885068700105  -0.003106936816075023   -0.003106936816074986   -0.001122355710010222   -0.001122355710009979   0.000586357497519224   0.000586357497518711   0.002325103905494404   0.002325103905494369   -0.00008484290426368963 -0.00008484290426428023 -0.00008484290426368963 -0.00008484290426428023 0.006867951756932146   0.006867951756932431   -0.002275820483105445   -0.002275820483105275   0.006308971612014534   0.006308971612015192   -0.002665095986285753   -0.002665095986285706   -0.002403813612936176   -0.002403813612936115   0.001615320905831137   0.001615320905831784   0.0005520949270255823  0.0005520949270251872  0.000268800033424576    0.0002688000334241444 
3.8383838383838382  0.002155123850449099    0.002155123850449486    0.002805647594562035   0.002805647594561804   0.003975765096092205   0.003975765096092386   0.003975765096092205   0.003975765096092386   0.00283914436820018    0.002839144368199645   -0.001353947407233149   -0.001353947407233929   0.002563498142982557   0.002563498142982129   -0.005348844526973446   -0.005348844526973994   -0.00515453300931753    -0.005154533009318061   -0.0002065753522876732  -0.0002065753522880139  -0.001334252822386597   -0.001334252822386446   -0.005548874900706358   -0.005548874900706265   -0.002056958682183435   -0.002056958682183304   0.001056547645861619   0.001056547645861108   0.004000921278364807   0.004000921278364832   -0.000149406795113292   -0.0001494067951138862  -0.000149406795113292   -0.0001494067951138862  0.01187863687256167    0.01187863687256199    -0.004062127564482418   -0.004062127564482203   0.01084165896849906    0.01084165896849979    -0.004768256714520568   -0.004768256714520497   -0.004300882927528693   -0.004300882927528609   0.002876785610018537   0.002876785610019209   0.00099601279961798    0.0009960127996175912  0.0004839448117479609   0.0004839448117475342 
4.040404040404041   0.003306286483592332    0.003306286483592231    0.004662140314793064   0.004662140314792848   0.006908456668185509   0.006908456668185663   0.006908456668185509   0.006908456668185663   0.004765823691000697   0.004765823691000144   -0.00232157681750054    -0.002321576817501297   0.004379451722981404   0.004379451722980978   -0.009234416726360013   -0.009234416726360523   -0.0079242276008961     -0.007924227600895973   -0.0003658526276426568  -0.0003658526276429941  -0.002297613135344088   -0.002297613135343875   -0.009574292024273711   -0.009574292024273566   -0.003647222579444409   -0.00364722257944441    0.001840809181077478   0.001840809181076969   0.006619976891071798   0.006619976891071853   -0.0002540224163001442  -0.0002540224163007402  -0.0002540224163001442  -0.0002540224163007402  0.01977838868453017    0.01977838868453042    -0.007004347318461893   -0.007004347318461641   0.01791646716102828    0.01791646716102897    -0.008242901191943082   -0.00824290119194299    -0.007435513430591776   -0.007435513430591673   0.004949754747556743   0.004949754747557425   0.001737413476120907   0.001737413476120525   0.0008423482419739266   0.0008423482419735065 
4.242424242424242   0.00487859070112927     0.004878590701128775    0.007449891077182648   0.007449891077182447   0.01159803009456107    0.0115980300945612     0.01159803009456107    0.0115980300945612     0.007704771304169675   0.007704771304169094   -0.003843203037918837   -0.003843203037919565   0.007222163863791674   0.007222163863791254   -0.01540304731449941    -0.01540304731449988    -0.01169490579670991    -0.01169490579670936    -0.0006265775633590502  -0.0006265775633593836  -0.003822451604751182   -0.003822451604750907   -0.01595884005104982    -0.01595884005104962    -0.006257039978948033   -0.00625703997894816    0.003101062357887708   0.003101062357887205   0.01052675730173244    0.01052675730173249    -0.0004169788030886182  -0.0004169788030892148  -0.0004169788030886182  -0.0004169788030892148  0.03169455989550485    0.03169455989550495    -0.01166669831034918    -0.0116666983103489     0.02845916877034086    0.02845916877034141    -0.01376708845075717    -0.01376708845075706    -0.01242040447477397    -0.01242040447477385    0.008228251914574361   0.008228251914575041   0.0029301427780845     0.002930142778084129   0.00141738136186808     0.001417381361867668  
4.444444444444445   0.006924940451438295    0.00692494045143779     0.01143957333468893    0.01143957333468875    0.01879525984995857    0.01879525984995868    0.01879525984995857    0.01879525984995868    0.01198877626940267    0.01198877626940206    -0.006137736683753135   -0.006137736683753831   0.01149116077531797    0.01149116077531756    -0.0248073312346887     -0.02480733123468913    -0.01657597818580045    -0.01657597818580065    -0.001037127324823884   -0.00103712732482421    -0.006140491738104089   -0.006140491738103758   -0.02568218993195225    -0.02568218993195202    -0.01038187430486421    -0.01038187430486444    0.005048303717800447   0.00504830371779995    0.01607362323602542    0.01607362323602542    -0.0006605120680378011  -0.0006605120680383987  -0.0006605120680378011  -0.0006605120680383987  0.04885927467441818    0.04885927467441804    -0.01876093193535469    -0.01876093193535439    0.04342159900920256    0.04342159900920289    -0.0222023278764177     -0.02220232787641758    -0.02003521406824697    -0.02003521406824685    0.01321085139053457    0.01321085139053524    0.004774490341464788   0.004774490341464428   0.002304123304619084    0.00230412330461868   
4.646464646464646   0.009457367508765175    0.009457367508765045    0.01686448805249903    0.01686448805249887    0.02934451654656314    0.02934451654656323    0.02934451654656314    0.02934451654656323    0.01793534463016729    0.01793534463016666    -0.009440280809208052   -0.009440280809208714   0.01762013352597047    0.01762013352597007    -0.03851758567914103    -0.03851758567914142    -0.02258024595779361    -0.02258024595779409    -0.001656570152724944   -0.001656570152725265   -0.009512039843919953   -0.00951203984391957    -0.0398451588780683     -0.03984515887806801    -0.01664158522450004    -0.01664158522450037    0.007929802967166371   0.007929802967165878   0.02354009257624556    0.0235400925762455     -0.001008299180183082   -0.001008299180183681   -0.001008299180183082   -0.001008299180183681   0.07240650173150366    0.0724065017315033     -0.02908759439580266    -0.02908759439580234    0.0635781729582166     0.0635781729582167     -0.03452611050286669    -0.03452611050286655    -0.03116708706988217    -0.03116708706988203    0.02046589338558473    0.02046589338558537    0.007504012276090755   0.007504012276090405   0.003612930567016805    0.003612930567016409  
4.848484848484849   0.01242705719412692     0.01242705719412726     0.02385015548268704    0.02385015548268691    0.04398449321142663    0.0439844932114267     0.04398449321142663    0.0439844932114267     0.02575575052799972    0.02575575052799909    -0.01394000089784717    -0.01394000089784779    0.02598383756150343    0.02598383756150304    -0.05748859936120717    -0.05748859936120754    -0.02959856330444812    -0.02959856330444785    -0.002545810020867743   -0.002545810020868062   -0.01417264508631747    -0.01417264508631704    -0.05944015374636341    -0.05944015374636311    -0.02571608147219286    -0.02571608147219327    0.01198444211406637    0.01198444211406588    0.0330178365434513     0.03301783654345121    -0.00147942616533166    -0.001479426165332262   -0.00147942616533166    -0.001479426165332262   0.1030623091820869     0.1030623091820864     -0.04337624566093734    -0.043376245660937      0.08924392311076995    0.08924392311076984    -0.05163944516215479    -0.05163944516215464    -0.04663921271451187    -0.04663921271451171    0.03053559207374913    0.03053559207374973    0.01134049713486706    0.01134049713486671    0.005448174077290756    0.005448174077290369  
5.05050505050505    0.01570771453168448     0.01570771453168504     0.0323529242885302     0.03235292428853013    0.06295094565553999    0.06295094565554003    0.06295094565553999    0.06295094565554003    0.03543333594699637    0.03543333594699572    -0.01966507864750034    -0.01966507864750094    0.03673487950000064    0.03673487950000028    -0.0820971305611529     -0.08209713056115325    -0.03739495489478653    -0.03739495489478612    -0.003746472239570814   -0.003746472239571131   -0.02023020160227922    -0.02023020160227876    -0.08490658386434646    -0.08490658386434619    -0.03818604862785595    -0.03818604862785641    0.01734686677758232    0.01734686677758182    0.04428766909414354    0.04428766909414342    -0.002077298634766932   -0.00207729863476754    -0.002077298634766932   -0.00207729863476754    0.140781460148024      0.1407814601480233     -0.06198203649329029    -0.06198203649328997    0.1199824958774312     0.1199824958774309     -0.07399066535921031    -0.07399066535921017    -0.06687446848486546    -0.06687446848486533    0.04375491030145828    0.04375491030145882    0.01639786837892044    0.0163978683789201     0.00786347413112578     0.00786347413112539   
5.252525252525253   0.01908696717477401     0.01908696717477438     0.04215469327033043    0.04215469327033042    0.08538613631183006    0.08538613631183008    0.08538613631183006    0.08538613631183008    0.04660602977796415    0.04660602977796352    -0.02632134854257541    -0.02632134854257597    0.04958250256084998    0.04958250256084962    -0.1114451201662677     -0.111445120166268      -0.04561261562384578    -0.04561261562384614    -0.005244983462089986   -0.005244983462090301   -0.02751519785529577    -0.02751519785529528    -0.1154774032496356     -0.1154774032496353     -0.05426327116028296    -0.05426327116028344    0.02389482979424091    0.02389482979424041    0.05674527673387668    0.05674527673387655    -0.002773930866292869   -0.002773930866293485   -0.002773930866292869   -0.002773930866293485   0.1844614784594648     0.1844614784594641     -0.08444988274082584    -0.08444988274082553    0.1544345896716979     0.1544345896716976     -0.1010190281667347     -0.1010190281667345     -0.09139728515663509    -0.09139728515663498    0.05999224480201009    0.05999224480201058    0.02253044230186761    0.02253044230186727    0.01079163536253211     0.01079163536253172   
5.454545454545454   0.02227231959141507     0.02227231959141499     0.05294220567121632    0.05294220567121635    0.1087612070201459     0.108761207020146      0.1087612070201459     0.108761207020146      0.05851250203348656    0.05851250203348596    -0.03314935837888201    -0.03314935837888255    0.06358608084597366    0.06358608084597331    -0.1426454129236293     -0.1426454129236298     -0.05376398685492915    -0.05376398685492947    -0.006930621027596108   -0.006930621027596421   -0.03543671296667343    -0.03543671296667294    -0.1485243748850577     -0.1485243748850574     -0.07347049686321774    -0.07347049686321826    0.03108170815011443    0.03108170815011393    0.06944483514267559    0.06944483514267545    -0.003495800723595334   -0.003495800723595959   -0.003495800723595334   -0.003495800723595959   0.2319097365865708     0.2319097365865702     -0.1090896205443441     -0.1090896205443438     0.1904070766020653     0.190407076602065      -0.1305889144800843     -0.1305889144800841     -0.1183284328103789     -0.1183284328103787     0.07839223335158005    0.07839223335158046    0.02917085541421741    0.02917085541421705    0.01397035267262586     0.01397035267262547   
5.656565656565657   0.02492021752805512     0.02492021752805467     0.06441095078835704    0.06441095078835711    0.1287835325579497     0.1287835325579497     0.1287835325579497     0.1287835325579497     0.07004933632105786    0.07004933632105727    -0.03893890047292101    -0.03893890047292151    0.0771013018123089     0.07710130181230859    -0.1706294408287085     -0.170629440828709      -0.06118855594961236    -0.06118855594961195    -0.00857589092085124    -0.008575890920851547   -0.04296336197584186    -0.04296336197584138    -0.1793760793165585     -0.1793760793165583     -0.09444214251435408    -0.09444214251435458    0.03787052713814767    0.03787052713814717    0.08129182144528063    0.08129182144528053    -0.004125862705628519   -0.004125862705629145   -0.004125862705628519   -0.004125862705629145   0.2801740597629028     0.2801740597629023     -0.1328642383132418     -0.1328642383132416     0.2252663029693912     0.2252663029693909     -0.1588051805952889     -0.1588051805952887     -0.1442273084797841     -0.144227308479784      0.09728745188114252    0.09728745188114285    0.03527956441548816    0.0352795644154878     0.01691953824366686     0.01691953824366647   
5.858585858585859   0.02669405701234444     0.02669405701234397     0.07623216488936012    0.07623216488936022    0.1403028235438104     0.1403028235438104     0.1403028235438104     0.1403028235438104     0.07992700022770528    0.07992700022770471    -0.04234304337277928    -0.04234304337277977    0.08801328053221416    0.08801328053221383    -0.1891157533553787     -0.1891157533553791     -0.06701689516490658    -0.0670168951649063     -0.009877689763155669   -0.009877689763155968   -0.0488578365128521     -0.04885783651285161    -0.2021350994372504     -0.2021350994372502     -0.1150650777484321     -0.1150650777484326     0.0429169982898952     0.0429169982898947     0.09133033843691757    0.09133033843691747    -0.004537983535557667   -0.004537983535558288   -0.004537983535557667   -0.004537983535558288   0.3261661230662594     0.3261661230662589     -0.151902017333693      -0.1519020173336928     0.2565114308819906     0.2565114308819904     -0.1806452126865504     -0.1806452126865502     -0.164676313490423      -0.1646763134904229     0.1144522054070949     0.1144522054070951     0.03955484852861767    0.03955484852861729    0.01903814731773251     0.01903814731773212   
6.0606060606060606  0.0273431581599399      0.02734315815993978     0.0877663045923599     0.08776630459236001    0.1392342218570962     0.1392342218570963     0.1392342218570962     0.1392342218570963     0.08686063082180794    0.08686063082180744    -0.04246551425431668    -0.04246551425431717    0.09425610593418138    0.09425610593418103    -0.1928085682530319     -0.1928085682530325     -0.07023947813531428    -0.07023947813531471    -0.01056850950749084    -0.01056850950749113    -0.05215599605717379    -0.0521559960571733     -0.2115380169375949     -0.2115380169375946     -0.1330407479654763     -0.1330407479654767     0.04503325958180535    0.04503325958180488    0.09899650173478254    0.09899650173478251    -0.004660844127159999   -0.004660844127160614   -0.004660844127159999   -0.004660844127160614   0.3673279794177305     0.3673279794177304     -0.1626358348773333     -0.162635834877333      0.2823006781096635     0.2823006781096635     -0.1914586584737906     -0.1914586584737905     -0.1756454530968978     -0.1756454530968977     0.1277097371396716     0.1277097371396718     0.04091972638566681    0.04091972638566642    0.01982777787005364     0.01982777787005327   
6.262626262626263   0.02677412038666991     0.02677412038667023     0.09766329597532758    0.09766329597532769    0.1246416955733619     0.124641695573362      0.1246416955733619     0.124641695573362      0.08975547905316576    0.08975547905316526    -0.03941538939825864    -0.03941538939825912    0.09441158201977762    0.09441158201977723    -0.1797786844691566     -0.1797786844691572     -0.06995841398452918    -0.06995841398452939    -0.01054118233410241    -0.0105411823341027     -0.05264127219697637    -0.05264127219697588    -0.205034321459246      -0.2050343214592458     -0.1466533431157678     -0.1466533431157681     0.04371709786317148    0.043717097863171      0.1042168533725722     0.1042168533725722     -0.004533087213985458   -0.004533087213986067   -0.004533087213985458   -0.004533087213986067   0.4020616343917413     0.4020616343917412     -0.1630778349133362     -0.163077834913336      0.3017594450989572     0.3017594450989573     -0.1886940192046791     -0.1886940192046789     -0.1750358390203142     -0.175035839020314      0.1356374935938453     0.1356374935938456     0.03903861628024727    0.03903861628024687    0.01912991193156831     0.01912991193156794   
6.4646464646464645  0.02507863099162048     0.02507863099162101     0.1037305700936035     0.1037305700936035     0.09962788099639495    0.09962788099639502    0.09962788099639495    0.09962788099639502    0.08790390463588642    0.08790390463588593    -0.03441304663038593    -0.0344130466303864     0.08808662112669814    0.08808662112669774    -0.1524013896812023     -0.1524013896812029     -0.06576046934463944    -0.06576046934463899    -0.009895971193677997   -0.009895971193678282   -0.05096705265280568    -0.0509670526528052     -0.1838099249876608     -0.1838099249876606     -0.1553217790891147     -0.1553217790891149     0.03938847412775821    0.03938847412775776    0.1073201547226981     0.1073201547226981     -0.004300158166259365   -0.004300158166259967   -0.004300158166259365   -0.004300158166259967   0.4297810793941663     0.4297810793941663     -0.1535035102389353     -0.1535035102389352     0.3150429462293956     0.3150429462293959     -0.1728379787738495     -0.1728379787738493     -0.1634674505259889     -0.1634674505259888     0.1379804564688863     0.1379804564688864     0.03449007217501779    0.03449007217501739    0.01720560949228086     0.01720560949228048   
6.666666666666667   0.0225025457335133      0.02250254573351362     0.1033605515200656     0.1033605515200657     0.07023756267147033    0.0702375626714704     0.07023756267147033    0.0702375626714704     0.08118769759960387    0.0811876975996034     -0.02930678801267308    -0.02930678801267352    0.07590471794139317    0.07590471794139274    -0.1161311506012038     -0.1161311506012043     -0.05798450962013831    -0.05798450962013817    -0.008874196391476115   -0.008874196391476392   -0.04830993255582038    -0.04830993255581991    -0.1520903305398006     -0.1520903305398003     -0.1596274336759017     -0.1596274336759019     0.03315587622769199    0.03315587622769154    0.1088321621996167     0.1088321621996168     -0.004146150165059595   -0.004146150165060188   -0.004146150165059595   -0.004146150165060188   0.4506440886129929     0.4506440886129929     -0.1361509381200498     -0.1361509381200496     0.3231993419221779     0.3231993419221782     -0.1469982428331893     -0.1469982428331892     -0.1438164667174613     -0.1438164667174612     0.1355507647347537     0.1355507647347539     0.02845437102129153    0.02845437102129112    0.01459268611749872     0.01459268611749835   
6.8686868686868685  0.01937487146234456     0.01937487146234448     0.09441300571524093    0.09441300571524104    0.04295787121872947    0.04295787121872951    0.04295787121872947    0.04295787121872951    0.07019211363661203    0.07019211363661158    -0.02581511375746206    -0.02581511375746248    0.05918976398230026    0.05918976398229986    -0.07705282002454983    -0.07705282002455036    -0.04766130242420662    -0.04766130242420711    -0.007735237539022374   -0.007735237539022642   -0.04580678091539354    -0.04580678091539309    -0.115268057407382      -0.1152680574073817     -0.1608584661072752     -0.1608584661072754     0.02629699541859009    0.02629699541858965    0.109265775176114      0.109265775176114      -0.004211284292336138   -0.00421128429233672    -0.004211284292336138   -0.00421128429233672    0.4651541046601999     0.4651541046601999     -0.1141889997435368     -0.1141889997435366     0.3278275445805903     0.3278275445805908     -0.1154912274165486     -0.1154912274165485     -0.1198845682126284     -0.1198845682126283     0.1297326879490377     0.1297326879490378     0.02216298831671804    0.02216298831671764    0.01185336624406107     0.0118533662440607    
7.070707070707071   0.0160318312723036      0.01603183127230319     0.07609541153490554    0.07609541153490568    0.02239349249772661    0.0223934924977266     0.02239349249772661    0.0223934924977266     0.05613025271029114    0.0561302527102907     -0.02498186971970666    -0.02498186971970705    0.03955162475248718    0.0395516247524868     -0.03984691437046989    -0.0398469143704704     -0.03615884413393403    -0.0361588441339341     -0.006660206526726924   -0.006660206526727178   -0.0441528851200926     -0.04415288512009216    -0.07806472669950304    -0.07806472669950273    -0.1603801930783919     -0.160380193078392      0.01980692058379458    0.01980692058379414    0.1089935352020366     0.1089935352020367     -0.004550819700162774   -0.004550819700163342   -0.004550819700162774   -0.004550819700163342   0.4738500500015519     0.4738500500015519     -0.09060637646920877    -0.0906063764692085     0.3304778812041018     0.3304778812041025     -0.08235309655053076    -0.08235309655053065    -0.09506982833903375    -0.09506982833903367    0.1219208376102488     0.1219208376102489     0.0164849486142822     0.01648494861428183    0.009384059992347753    0.009384059992347388  
7.2727272727272725  0.01276151451380638     0.01276151451380596     0.04935846342411728    0.04935846342411745    0.01034229592058165    0.01034229592058157    0.01034229592058165    0.01034229592058157    0.04058491590644771    0.0405849159064473     -0.02707506724538574    -0.02707506724538611    0.01854490163437173    0.01854490163437134    -0.00705328550420844    -0.007053285504208941   -0.0248091646098227     -0.02480916460982219    -0.005717280423090342   -0.005717280423090583   -0.04352957673961309    -0.04352957673961266    -0.04357854150990974    -0.0435785415099094     -0.1591486569873186     -0.1591486569873188     0.01421945828080082    0.0142194582808004     0.1082384800297436     0.1082384800297437     -0.005145850564971991   -0.005145850564972547   -0.005145850564971991   -0.005145850564972547   0.4772553102885182     0.4772553102885183     -0.067543666903095      -0.06754366690309474    0.3319216760477166     0.3319216760477172     -0.05047640165901456    -0.0504764016590145     -0.0716587580705991     -0.07165875807059902    0.1131581324645602     0.1131581324645603     0.0118186390999623     0.01181863909996194    0.007364541436885541    0.007364541436885181  
7.474747474747475   0.0097755045275791      0.00977550452757899     0.01662049142378691    0.01662049142378709    0.00626776740151214    0.006267767401512001   0.00626776740151214    0.006267767401512001   0.02517844460623822    0.02517844460623784    -0.03180552983027064    -0.03180552983027098    -0.002532038096852816  -0.002532038096853209  0.02059125639611269     0.02059125639611221     -0.01469663397673349    -0.01469663397673348    -0.00487397509186225    -0.004873975091862484   -0.04376361849690366    -0.04376361849690326    -0.01326610391307132    -0.01326610391307098    -0.1575124704404476     -0.1575124704404478     0.009668197379308538   0.009668197379308123   0.1071738637988972     0.1071738637988973     -0.005938283050897252   -0.005938283050897795   -0.005938283050897252   -0.005938283050897795   0.4761402451068738     0.4761402451068738     -0.04617285263383693    -0.04617285263383666    0.331849500118888      0.3318495001188887     -0.02147613509884045    -0.02147613509884043    -0.05078610269530523    -0.05078610269530516    0.1040388380655851     0.1040388380655853     0.008205839916198101   0.008205839916197747   0.005809289781192701    0.005809289781192342  
7.6767676767676765  0.007202127257029948    0.00720212725703023     -0.01897549371550539   -0.01897549371550521   0.008350963256522521   0.00835096325652232    0.008350963256522521   0.00835096325652232    0.01129124569530593    0.01129124569530558    -0.03861860926172905    -0.03861860926172937    -0.02270871593711604   -0.02270871593711643   0.04342048706300115     0.04342048706300068     -0.006532058096175393   -0.006532058096175886   -0.004031100766796321   -0.004031100766796546   -0.04453498987286406    -0.04453498987286367    0.01257830613927921     0.01257830613927956     -0.1552716035052095     -0.1552716035052097     0.006045475325238843   0.006045475325238435   0.1060704740486056     0.1060704740486057     -0.006861144544663505   -0.006861144544664042   -0.006861144544663505   -0.006861144544664042   0.4718999403957481     0.4718999403957481     -0.02691976180260883    -0.02691976180260857    0.3297294658073945     0.3297294658073952     0.004015957418666619    0.004015957418666618    -0.03275553095190693    -0.03275553095190686    0.09479285088396129    0.09479285088396143    0.005504955030400528   0.005504955030400183   0.004647637057012747    0.004647637057012389  
7.878787878787879   0.005093808361134851    0.005093808361135318    -0.05436298875811948   -0.0543629887581193    0.01440186976885356    0.01440186976885331    0.01440186976885356    0.01440186976885331    -0.0001014464858248354 -0.000101446485825151  -0.04690702488689617    -0.04690702488689646    -0.04132848339983921   -0.04132848339983961   0.06220466486344872     0.06220466486344828     -0.0005481208429824067  -0.0005481208429823204  -0.003064348590733647   -0.003064348590733864   -0.04552185041042489    -0.04552185041042454    0.03428012598056335     0.03428012598056371     -0.1518796566103033     -0.1518796566103034     0.0031478822136112     0.003147882213610798   0.1053748130017087     0.1053748130017088     -0.00785399976009949    -0.007853999760100016   -0.00785399976009949    -0.007853999760100016   0.4665508165085474     0.4665508165085474     -0.009776080303135332   -0.009776080303135096   0.3265888704529747     0.3265888704529754     0.02596480111680822     0.02596480111680818     -0.01741106086061996    -0.01741106086061989    0.08543505198445515    0.08543505198445529    0.003520666256562946   0.003520666256562608   0.003785802665447112    0.003785802665446757  
8.080808080808081   0.003442664598118277    0.003442664598118581    -0.08708443156946324   -0.08708443156946308   0.02239408350214765    0.02239408350214735    0.02239408350214765    0.02239408350214735    -0.008460221625385837  -0.008460221625386122  -0.05611924159279557    -0.05611924159279583    -0.05799378247589502   -0.05799378247589541   0.07776996062589536     0.07776996062589493     0.003467273117867196    0.003467273117867689    -0.001866263996649689   -0.001866263996649899   -0.04646605031668369    -0.04646605031668336    0.05239145853388689     0.05239145853388725     -0.146680448117482      -0.146680448117482      0.0007672619622264695  0.0007672619622260734  0.1056158240688027     0.1056158240688029     -0.008867272990715347   -0.008867272990715864   -0.008867272990715347   -0.008867272990715864   0.4619729862110343     0.4619729862110343     0.005449834065162582    0.005449834065162792    0.3258435563998037     0.3258435563998043     0.04463181352331667     0.04463181352331662     -0.00440730669449767    -0.004407306694497609   0.07590069784166777    0.07590069784166789    0.002068646842694942   0.002068646842694615   0.003139446187309418    0.003139446187309066  
8.282828282828282   0.002199819356602826    0.00219981935660275     -0.1154652818956684    -0.1154652818956683    0.03069839094247626    0.03069839094247592    0.03069839094247626    0.03069839094247592    -0.01360943747294225   -0.0136094374729425    -0.06579513427290648    -0.06579513427290672    -0.07250277275306656   -0.07250277275306695   0.09082915393862585     0.09082915393862545     0.005945947512755543    0.005945947512755408    -0.0003799200103052078  -0.0003799200103054131  -0.04718703229364782    -0.04718703229364752    0.06747586157175901     0.06747586157175936     -0.1391087907523588     -0.1391087907523587     -0.001268078156767974  -0.001268078156768361  0.1071690421633747     0.107169042163375      -0.009861903494873533   -0.009861903494874045   -0.009861903494873533   -0.009861903494874045   0.4587890154031324     0.4587890154031324     0.01898396830475185     0.01898396830475204     0.3315857682390631     0.3315857682390636     0.06039383203135067     0.06039383203135061     0.006639682226369126    0.006639682226369186    0.06613495736766485    0.06613495736766499    0.0009959957286777333  0.0009959957286774159  0.002644249644400457    0.002644249644400112  
8.484848484848484   0.001294669558999053    0.001294669558998669    -0.1385335712368617    -0.1385335712368615    0.0381497378254659     0.03814973782546553    0.0381497378254659     0.03814973782546553    -0.01562132235835746   -0.01562132235835768   -0.075563849537168      -0.07556384953716822    -0.08479413021013191   -0.0847941302101323    0.1019353318977114      0.1019353318977111      0.007301358202495687    0.007301358202495261    0.001382823974390769    0.001382823974390568    -0.04757327702013239    -0.04757327702013212    0.08002107894636314     0.08002107894636347     -0.1288221569322209     -0.1288221569322208     -0.003083111713273687  -0.003083111713274064  0.1100312604220477     0.1100312604220481     -0.01080784660699582    -0.01080784660699632    -0.01080784660699582    -0.01080784660699632    0.4558594547425412     0.4558594547425412     0.0310213107481439      0.03102131074814406     0.3454791705560774     0.3454791705560779     0.07364893461552365     0.07364893461552358     0.01608418067181869     0.01608418067181875     0.05613550315382303    0.05613550315382315    0.0001824251053288939  0.0001824251053285864  0.002255387571903188    0.002255387571902847  
8.686868686868687   0.0006511768097654941   0.0006511768097651017   -0.1558444831893517    -0.1558444831893516    0.04403976480693892    0.04403976480693854    0.04403976480693892    0.04403976480693854    -0.01471323172804866   -0.01471323172804885   -0.08512847172654828    -0.08512847172654847    -0.09490506395163131   -0.0949050639516317    0.1114913034795799      0.1114913034795796      0.007845213776975437    0.007845213776975678    0.00333874198917475     0.003338741989174553    -0.04756810359132285    -0.0475681035913226     0.09042189454188157     0.09042189454188189     -0.1157556181578802     -0.1157556181578799     -0.004760554847816145  -0.004760554847816513  0.1137308931635471     0.1137308931635478     -0.01168266137147339    -0.01168266137147389    -0.01168266137147339    -0.01168266137147389    0.4509901460253749     0.450990146025375      0.04171100886486019     0.04171100886486033     0.3650402384485816     0.3650402384485816     0.08477010436432586     0.08477010436432578     0.02422562202586095     0.02422562202586102     0.04595962772510275    0.04595962772510288    -0.000464016219893102  -0.0004640162198933993 0.001943110489763036    0.0019431104897627    
8.88888888888889    0.0001994430010077148   0.0001994430010076118   -0.1673216566475562    -0.1673216566475562    0.04807826401360189    0.04807826401360149    0.04807826401360189    0.04807826401360149    -0.01118475414120451   -0.01118475414120466   -0.09425340373094052    -0.09425340373094068    -0.1029408751798479    -0.1029408751798482    0.1197774713020366      0.1197774713020363      0.007804120230249522    0.00780412023024993     0.00534359716615226     0.005343597166152064    -0.04715673366300202    -0.04715673366300181    0.09899238512519237     0.09899238512519269     -0.1001118069966003     -0.1001118069966        -0.006350375283057527  -0.006350375283057886  0.1173850417724072     0.1173850417724081     -0.01247081115521441    -0.01247081115521489    -0.01247081115521441    -0.01247081115521489    0.4424715630752256     0.4424715630752257     0.05116463262246676     0.05116463262246689     0.3848358660024085     0.3848358660024083     0.09408172547210875     0.09408172547210866     0.03130170637936618     0.03130170637936625     0.03570898031538874    0.03570898031538888    -0.001012703119404589  -0.001012703119404877  0.001687602545367028    0.001687602545366695  
9.090909090909092   -0.0001178872846507978  -0.0001178872846505335  -0.1731544756093602    -0.1731544756093602    0.05033199265933742    0.05033199265933701    0.05033199265933742    0.05033199265933701    -0.005388715693361133  -0.005388715693361247  -0.1027609685363191     -0.1027609685363192     -0.1090523464300679    -0.1090523464300683    0.1269818784761606      0.1269818784761603      0.007336008286321249    0.007336008286320975    0.007214504800167779    0.007214504800167584    -0.04635616128892121    -0.04635616128892101    0.1059854652467575      0.1059854652467578      -0.08231111765514242    -0.08231111765514208    -0.00787966758281192   -0.007879667582812269  0.1198519225355231     0.1198519225355242     -0.01316384598299162    -0.01316384598299209    -0.01316384598299162    -0.01316384598299209    0.4303000005081727     0.4303000005081729     0.0594707232563173      0.05947072325631741     0.3997146268954992     0.3997146268954986     0.1018483287621631      0.101848328762163       0.0374890524502843      0.03748905245028437     0.02550341118607562    0.02550341118607575    -0.001516892272458172  -0.00151689227245845   0.001474293450940266    0.001474293450939939  
9.292929292929292   -0.0003437813266412275  -0.0003437813266407973  -0.1737431963610625    -0.1737431963610626    0.05113600817344896    0.05113600817344853    0.05113600817344896    0.05113600817344853    0.002284246904983062   0.002284246904982982   -0.1105334624125508     -0.1105334624125509     -0.113418273324504     -0.1134182733245044    0.1332269013872332      0.1332269013872329      0.006562719763761885    0.006562719763761554    0.008762132582876357    0.008762132582876163    -0.04520714398101959    -0.04520714398101942    0.1116103651741903      0.1116103651741906      -0.0629289818647427     -0.0629289818647423     -0.00936093893549338   -0.009360938935493718  0.1199271748150413     0.1199271748150427     -0.01376089720061466    -0.01376089720061513    -0.01376089720061466    -0.01376089720061513    0.4160536040948267     0.4160536040948272     0.06670750647576908     0.06670750647576916     0.407498210349452      0.4074982103494512     0.1082731822585205      0.1082731822585203      0.04290880463058035     0.04290880463058043     0.01545394096435544    0.01545394096435558    -0.002019307564727602  -0.002019307564727868  0.001290399488141483    0.001290399488141161  
9.494949494949495   -0.0005081665389183652  -0.0005081665389180893  -0.1696683839509288    -0.169668383950929     0.05097979514245448    0.05097979514245406    0.05097979514245448    0.05097979514245406    0.01142352766820932    0.01142352766820928    -0.1175117053223995     -0.1175117053223996     -0.1162321885011318    -0.1162321885011322    0.138590685196021       0.1385906851960207      0.005603733537926875    0.005603733537927221    0.009825208303237761    0.009825208303237564    -0.04376761678886354    -0.04376761678886339    0.1160453251971925      0.1160453251971928      -0.04263759075038899    -0.04263759075038859    -0.01079824380729111   -0.01079824380729144   0.1165424710243238     0.1165424710243255     -0.01426862862519343    -0.01426862862519389    -0.01426862862519343    -0.01426862862519389    0.4016184343650021     0.4016184343650027     0.07295106516078591     0.07295106516078598     0.4091671548499049     0.409167154849904      0.113506167227172       0.1135061672271719      0.04763704461112798     0.04763704461112805     0.005642998443834418   0.005642998443834562   -0.002554637744071171  -0.002554637744071427  0.001123267085335585    0.001123267085335269  
9.696969696969697   -0.0006308104452664961  -0.0006308104452665742  -0.1616638254897123    -0.1616638254897125    0.05038216225533191    0.05038216225533149    0.05038216225533191    0.05038216225533149    0.02162362842947209    0.02162362842947209    -0.1236847003228531     -0.1236847003228532     -0.1176928052241697    -0.1176928052241702    0.1431227637087478      0.1431227637087476      0.004524855349328817    0.004524855349329094    0.01029747317202872     0.01029747317202852     -0.04210745264088975    -0.04210745264088961    0.1194456332114329      0.1194456332114333      -0.02215817945754512    -0.02215817945754471    -0.01219134671190692   -0.01219134671190724   0.1089280298656113     0.1089280298656133     -0.01470023689706861    -0.01470023689706905    -0.01470023689706861    -0.01470023689706905    0.3880362637923629     0.3880362637923639     0.07827904207488505     0.07827904207488508     0.4070321587682551     0.4070321587682541     0.1176587743298265      0.1176587743298263      0.05171901511523522     0.0517190151152353      -0.003883171717420366  -0.003883171717420224  -0.003148376508530356  -0.003148376508530603  0.0009604497134443545   0.0009604497134440444 
9.8989898989899     -0.0007243550534340691  -0.000724355053434434   -0.1505806916615159    -0.1505806916615161    0.04977671942717789    0.04977671942717745    0.04977671942717789    0.04977671942717745    0.03251260972780624    0.03251260972780628    -0.1290721556771345     -0.1290721556771346     -0.1179970454674078    -0.1179970454674082    0.1468556261569784      0.1468556261569782      0.003262684170667322    0.003262684170666939    0.010140923780228       0.01014092378022781     -0.04030396197442741    -0.0403039619744273     0.1219484123297845      0.1219484123297849      -0.002223329747898786   -0.002223329747898396   -0.01353828692322508   -0.0135382869232254    0.09670294985699791    0.09670294985700027    -0.01507354484455087    -0.0150735448445513     -0.01507354484455087    -0.0150735448445513     0.3753570143476048     0.3753570143476061     0.08277107776751835     0.08277107776751837     0.4026962838910996     0.4026962838910988     0.1208222059949918      0.1208222059949917      0.05518475656301394     0.05518475656301402     -0.01310920697710248   -0.01310920697710234   -0.003814289819010991  -0.003814289819011228  0.0007907770908351301   0.0007907770908348253 
10.1010101010101    -0.0007967908271166114  -0.0007967908271169792  -0.1373412621567596    -0.1373412621567598    0.04943647650788017    0.04943647650787975    0.04943647650788017    0.04943647650787975    0.04377534072494693    0.04377534072494703    -0.1337050684701447     -0.1337050684701448     -0.1173344146071293    -0.1173344146071298    0.1498157821806037      0.1498157821806035      0.001690804683283524    0.001690804683283301    0.009384663292561346    0.00938466329256115     -0.03843627646934336    -0.03843627646934326    0.1236760405303327      0.123676040530333       0.01645270702708098     0.01645270702708134     -0.01483653234506027   -0.01483653234506057   0.0798671627633332     0.07986716276333594    -0.01540821012815085    -0.01540821012815127    -0.01540821012815085    -0.01540821012815127    0.3631680248083156     0.363168024808317      0.08650743943928085     0.08650743943928085     0.3957960318184252     0.3957960318184244     0.1230841144323819      0.1230841144323818      0.05806325938626007     0.05806325938626016     -0.02203560102889869   -0.02203560102889856   -0.004554028197549133  -0.00455402819754936   0.0006055474426680554   0.000605547442667756  
10.303030303030303  -0.0008532390733732245  -0.0008532390733733199  -0.1228896927127226    -0.1228896927127228    0.04946177632469698    0.04946177632469657    0.04946177632469698    0.04946177632469657    0.05516233130635702    0.05516233130635716    -0.1376086886851472     -0.1376086886851473     -0.1158825364492076    -0.1158825364492081    0.1520357618964558      0.1520357618964556      -0.0002142259769252955  -0.000214225976924901   0.008112457383818206    0.008112457383818014    -0.03657732587608389    -0.03657732587608381    0.1247394708414941      0.1247394708414944      0.03319673143196651     0.03319673143196684     -0.01608287464562259   -0.01608287464562288   0.05870396564219936    0.05870396564220248    -0.01572216870964618    -0.0157221687096466     -0.01572216870964618    -0.0157221687096466     0.3508522492618171     0.3508522492618188     0.0895670829153382      0.08956708291533816     0.3834517301150513     0.3834517301150508     0.1245404217611668      0.1245404217611666      0.0603926922740277      0.06039269227402779     -0.03065972605909318   -0.03065972605909304   -0.005360254618676897  -0.005360254618677116  0.0003993698743943247   0.0003993698743940301 
10.505050505050505  -0.0008971471531949396  -0.000897147153194696   -0.1081498226784651    -0.1081498226784653    0.04982493766098377    0.04982493766098336    0.04982493766098377    0.04982493766098336    0.0664792920855363     0.06647929208553649    -0.1407905273036339     -0.140790527303634      -0.113804674044626     -0.1138046740446265    0.1535636672381416      0.1535636672381414      -0.002318021464224959   -0.002318021464224829   0.00644420093466534     0.006444200934665148    -0.03478519223399727    -0.03478519223399722    0.1252412785289596      0.1252412785289599      0.04739542312515682     0.04739542312515707     -0.01727260115863408   -0.01727260115863436   0.03366117939754235    0.03366117939754582    -0.01602809076472047    -0.01602809076472089    -0.01602809076472047    -0.01602809076472089    0.3370137508335092     0.3370137508335111     0.09202593909665721     0.09202593909665718     0.3602718681313458     0.3602718681313454     0.1253002756714782      0.1253002756714781      0.06222521237121166     0.06222521237121174     -0.03895956126118429   -0.03895956126118415   -0.00622063066126794   -0.00622063066126815   0.0001706856183004055   0.0001706856183001157 
10.707070707070708  -0.0009309888741456145  -0.0009309888741452191  -0.0939945854617197    -0.09399458546171995   0.05042926358330783    0.05042926358330742    0.05042926358330783    0.05042926358330742    0.07756045477936291    0.07756045477936317    -0.1432353564409043     -0.1432353564409043     -0.1112498899930428    -0.1112498899930433    0.1544655434211968      0.1544655434211966      -0.004419814158273713   -0.004419814158274153   0.004516760623907361    0.004516760623907169    -0.03309772884549711    -0.03309772884549708    0.1252774217622545      0.1252774217622548      0.05851711622390106     0.05851711622390123     -0.01839877403269195   -0.01839877403269222   0.005303847500225375   0.005303847500229169   -0.01633107029240495    -0.01633107029240536    -0.01633107029240495    -0.01633107029240536    0.3184501392758889     0.3184501392758909     0.09395574878965016     0.09395574878965013     0.3192781574044646     0.3192781574044645     0.1254839189626056      0.1254839189626055      0.06362600490858904     0.06362600490858912     -0.04688594529721651   -0.0468859452972164    -0.007118667624417306  -0.007118667624417509  -0.00007783927521397484 -0.0000778392752142603
10.909090909090908  -0.0009565652939846667  -0.0009565652939844171  -0.08122307704960383   -0.08122307704960402   0.05114390481261358    0.05114390481261317    0.05114390481261358    0.05114390481261317    0.08823666672366424    0.08823666672366456    -0.1449086694654053     -0.1449086694654052     -0.1083553085548585    -0.1083553085548589    0.1548201901680665      0.1548201901680664      -0.006333901037672359   -0.006333901037672442   0.002467519814367524    0.002467519814367334    -0.03153252558544219    -0.03153252558544217    0.1249373094974398      0.1249373094974401      0.06614081986867504     0.06614081986867512     -0.01945217365681676   -0.01945217365681703   -0.02563656913189458   -0.02563656913189048   -0.01662828216447285    -0.01662828216447325    -0.01662828216447285    -0.01662828216447325    0.2898378445617979     0.2898378445618003     0.09542344976265665     0.09542344976265661     0.2543777432164744     0.2543777432164748     0.1252148789769775      0.1252148789769774      0.06466745691762925     0.06466745691762933     -0.05436655132660267   -0.05436655132660254   -0.008031074298299055  -0.008031074298299253  -0.0003391787894869656  -0.000339178789487247 
11.11111111111111   -0.0009750798614652826  -0.000975079861465363   -0.07053650235849501   -0.07053650235849515   0.0518153754075548     0.05181537540755442    0.0518153754075548     0.05181537540755442    0.09831371269370587    0.09831371269370626    -0.1457686019621642     -0.1457686019621642     -0.1052487273137652    -0.1052487273137656    0.1547104967859766      0.1547104967859765      -0.007900319473638825   -0.007900319473638404   0.0004211314930164777   0.0004211314930162893   -0.03009098209294057    -0.03009098209294056    0.1243027808696417      0.124302780869642       0.0699882736577914      0.06998827365779138     -0.02042197254462721   -0.02042197254462747   -0.05820564357386954   -0.05820564357386517   -0.01691051692583481    -0.0169105169258352     -0.01691051692583481    -0.0169105169258352     0.2451901787763867     0.2451901787763894     0.09649093773646303     0.09649093773646292     0.1640814502662802     0.164081450266281      0.1246093316748715      0.1246093316748715      0.06542062046586272     0.06542062046586279     -0.06132046310575081   -0.06132046310575069   -0.00892569944863705   -0.008925699448637242  -0.0006025937263506502  -0.0006025937263509279
11.313131313131313  -0.0009871876159552213  -0.0009871876159555637  -0.06250720099115936   -0.06250720099115938   0.05228221094579277    0.05228221094579242    0.05228221094579277    0.05228221094579242    0.1075722339568049     0.1075722339568053     -0.1457831682338143     -0.1457831682338143     -0.1020496994330813    -0.1020496994330817    0.1542155569960117      0.1542155569960116      -0.009003535397744759   -0.00900353539774476    -0.001521007953632462   -0.00152100795363265    -0.02876388075317377    -0.02876388075317378    0.1234468731221789      0.1234468731221792      0.06994948922502561     0.06994948922502553     -0.02129691933143777   -0.02129691933143803   -0.09127520159061255   -0.09127520159060802   -0.01716497101843229    -0.01716497101843268    -0.01716497101843229    -0.01716497101843268    0.1808089857145331     0.180808985714536      0.097214953056974       0.09721495305697393     0.05425437454578238    0.05425437454578368    0.1237663044017396      0.1237663044017395      0.06594690817962598     0.06594690817962606     -0.0676776398743134    -0.06767763987431326   -0.009764078189958333  -0.009764078189958522  -0.000855390498235966   -0.0008553904982362404
11.515151515151516  -0.0009931202044850228  -0.000993120204485368   -0.0575431490296265    -0.05754314902962641   0.05240834047492456    0.05240834047492422    0.05240834047492456    0.05240834047492422    0.1157871649993665     0.1157871649993671     -0.1449467372061423     -0.1449467372061423     -0.09886857085955045   -0.09886857085955086   0.1534053971309731      0.153405397130973       -0.009612180946131981   -0.00961218094613241    -0.003287985981991023   -0.00328798598199121    -0.02753637165798407    -0.02753637165798409    0.1224329542769239      0.1224329542769241      0.06609371504309212     0.06609371504309199     -0.02206665353651104   -0.02206665353651129   -0.123684401223686     -0.1236844012236814    -0.01737830452155237    -0.01737830452155276    -0.01737830452155237    -0.01737830452155276    0.09792404730622566    0.0979240473062288     0.09764691881532549     0.09764691881532536     -0.06263630390814809   -0.06263630390814638   0.1227618818967921      0.1227618818967921      0.06629269936479246     0.06629269936479254     -0.07339542546730066   -0.07339542546730056   -0.01050811790331287   -0.01050811790331306   -0.001085296528984403   -0.001085296528984675 
11.717171717171718  -0.0009928486320580356  -0.0009928486320581265  -0.05585460814680868   -0.05585460814680845   0.05212647486638264    0.05212647486638233    0.05212647486638264    0.05212647486638233    0.1227541540406198     0.1227541540406204     -0.1432907450373715     -0.1432907450373715     -0.0958043633871556    -0.095804363387156     0.1523379908376385      0.1523379908376384      -0.00976639717677195    -0.009766397176771885   -0.004841518117486835   -0.00484151811748702    -0.02639147640600786    -0.02639147640600789    0.1213143907072478      0.121314390707248       0.05866262451187609     0.05866262451187588     -0.0227227844127559    -0.02272278441275614   -0.1543210487496287    -0.1543210487496242    -0.01753902235918854    -0.01753902235918892    -0.01753902235918854    -0.01753902235918892    0.003354064021092027   0.003354064021095226   0.0978326900102275      0.09783269001022737     -0.1716178416900092    -0.1716178416900073    0.1216488339769653      0.1216488339769652      0.06648820980086323     0.06648820980086333     -0.07846598850138152   -0.07846598850138146   -0.01112696918769211   -0.0111269691876923    -0.001282445264478978   -0.001282445264479248 
11.919191919191919  -0.0009862180405682205  -0.0009862180405679933  -0.05742752033617794   -0.0574275203361776    0.05147357439103639    0.05147357439103611    0.05147357439103639    0.05147357439103611    0.1283148597760624     0.1283148597760631     -0.1408877616895871     -0.1408877616895871     -0.09294203084016883   -0.09294203084016922   0.1510574032663243      0.1510574032663242      -0.009517620072794707   -0.00951762007279431    -0.006171668007615847   -0.006171668007616031   -0.02531258822406161    -0.02531258822406167    0.1201345691547278      0.120134569154728       0.04804703811050548     0.0480470381105052      -0.02325968078078746   -0.02325968078078771   -0.1820285113612455    -0.1820285113612412    -0.01763899074878984    -0.01763899074879022    -0.01763899074878984    -0.01763899074879022    -0.09194572736029931   -0.09194572736029635   0.09781223958259071     0.09781223958259055     -0.260172382060426     -0.2601723820604242    0.1204610472043024      0.1204610472043024      0.0665503258512198      0.0665503258512199      -0.08291271319914982   -0.08291271319914975   -0.01160127555229638   -0.01160127555229657   -0.001440453503560757   -0.001440453503561024 
12.121212121212121  -0.0009730736202238653  -0.0009730736202234956  -0.06200975156248999   -0.06200975156248952   0.05060483699296033    0.05060483699296006    0.05060483699296033    0.05060483699296006    0.1323827343328834     0.1323827343328842     -0.1378495057963757     -0.1378495057963757     -0.09034844808722045   -0.09034844808722085   0.1495925305541931      0.1495925305541929      -0.00889948844485556    -0.008899488444855705   -0.007286811882588361   -0.007286811882588546   -0.02428607290837265    -0.02428607290837272    0.1189270159393074      0.1189270159393076      0.03475068536118751     0.03475068536118718     -0.02367516761641988   -0.02367516761642013   -0.2054043606758333    -0.2054043606758293    -0.01767455994163373    -0.0176745599416341     -0.01767455994163373    -0.0176745599416341     -0.1756887946554583    -0.1756887946554559    0.09761934434542614     0.09761934434542598     -0.3219544288423776    -0.321954428842376     0.1192207230996774      0.1192207230996774      0.06648790475807448     0.06648790475807458     -0.08677765058136597   -0.08677765058136588   -0.01192502813550793   -0.01192502813550811   -0.001556832340487433   -0.001556832340487699 
12.323232323232324  -0.0009534244816009513  -0.0009534244816007165  -0.06912454074414894   -0.06912454074414834   0.0497757973005961     0.04977579730059587    0.0497757973005961     0.04977579730059587    0.1349693380914118     0.1349693380914127     -0.134317547827915      -0.134317547827915      -0.08806704242121914   -0.08806704242121953   0.1479571819856206      0.1479571819856205      -0.007938660864921569   -0.00793866086492196    -0.008200913586533026   -0.008200913586533208   -0.02330425703841699    -0.02330425703841707    0.1177156445235207      0.1177156445235208      0.01934481298323411     0.01934481298323373     -0.02397113208256481   -0.02397113208256504   -0.2226906031027551    -0.2226906031027515    -0.01764748524828964    -0.01764748524829002    -0.01764748524828964    -0.01764748524829002    -0.2375196935987166    -0.2375196935987151    0.09728141824193173     0.09728141824193155     -0.3577844964761606    -0.3577844964761597    0.1179457810337373      0.1179457810337373      0.06630757350255544     0.06630757350255556     -0.09010484278419477   -0.09010484278419473   -0.01210593538701979   -0.01210593538701998   -0.001633036018285908   -0.001633036018286172 
12.525252525252526  -0.0009276025566403615  -0.0009276025566404346  -0.07812636777231707   -0.07812636777231638   0.04928824825549522    0.04928824825549499    0.04928824825549522    0.04928824825549499    0.1362006868947133     0.1362006868947142     -0.1304457399707625     -0.1304457399707625     -0.08611319947154401   -0.08611319947154442   0.1461527439824441      0.1461527439824441      -0.006659945745293445   -0.00665994574529328    -0.00892290085358246    -0.008922900853582641   -0.02236773331825874    -0.02236773331825883    0.1165154743176264      0.1165154743176265      0.002418989177265689    0.002418989177265279    -0.0241536854232546    -0.02415368542325484   -0.2319180952567986    -0.2319180952567956    -0.01756514707335218    -0.01756514707335255    -0.01756514707335218    -0.01756514707335255    -0.2714958569798897    -0.2714958569798894    0.09681969751055018     0.09681969751055        -0.3728694806996737    -0.3728694806996733    0.1166552031684681      0.1166552031684681      0.06601822152377637     0.06601822152377647     -0.09292526542286357   -0.09292526542286353   -0.01216399013941346   -0.01216399013941365   -0.001674064553326875   -0.001674064553327138 
12.727272727272727  -0.0008963190808755671  -0.0008963190808758878  -0.08829725396151572   -0.08829725396151496   0.04941498802943316    0.04941498802943287    0.04941498802943316    0.04941498802943287    0.1363093803189347     0.1363093803189356     -0.1263777857842362     -0.1263777857842362     -0.0844735958182828    -0.08447359581828318   0.1441737227744544      0.1441737227744544      -0.005095221400155821   -0.005095221400155532   -0.009452429800040104   -0.009452429800040282   -0.02148556318803163    -0.02148556318803172    0.1153341022383938      0.1153341022383939      -0.01546278896674948    -0.01546278896674992    -0.02423252880701662   -0.02423252880701685   -0.2312945254691947    -0.2312945254691923    -0.01743951247164685    -0.01743951247164722    -0.01743951247164685    -0.01743951247164722    -0.2767219747081738    -0.2767219747081749    0.09624989755215664     0.09624989755215645     -0.3717175703334201    -0.3717175703334205    0.1153710216535709      0.1153710216535709      0.06563301960613321     0.06563301960613332     -0.09524819214251425   -0.09524819214251422   -0.01212778147499854   -0.01212778147499874   -0.001687477230265839   -0.0016874772302661   
12.929292929292929  -0.0008605957916928172  -0.0008605957916931411  -0.09895813911677982   -0.098958139116779     0.05033464403243615    0.05033464403243584    0.05033464403243615    0.05033464403243584    0.1355980116016117     0.1355980116016126     -0.1222270316257503     -0.1222270316257502     -0.08311084537080751   -0.08311084537080787   0.1420148662398828      0.1420148662398828      -0.003319125920214742   -0.003319125920215044   -0.00978278135270179    -0.009782781352701966   -0.02067290840135715    -0.02067290840135726    0.1141738160113826      0.1141738160113827      -0.0337904844621079     -0.03379048446210836    -0.02421955940938594   -0.02421955940938617   -0.2196667984260756    -0.2196667984260739    -0.01728493110494916    -0.01728493110494952    -0.01728493110494916    -0.01728493110494952    -0.2563298441756742    -0.2563298441756766    0.09558326690487028     0.09558326690487008     -0.3548154219673401    -0.3548154219673412    0.1141169433812754      0.1141169433812754      0.06516872450417213     0.06516872450417224     -0.0970612982446048    -0.0970612982446048    -0.01202945142831908   -0.01202945142831927   -0.001682000505488975   -0.001682000505489234 
13.131313131313131  -0.0008216235898271806  -0.000821623589827267   -0.1095602690318224    -0.1095602690318215    0.05209816625426653    0.05209816625426615    0.05209816625426653    0.05209816625426615    0.1343845571325252     0.1343845571325261     -0.1180648223652022     -0.1180648223652021     -0.08197212232503973   -0.08197212232504007   0.1396774888785829      0.1396774888785829      -0.001463648319930774   -0.001463648319931114   -0.009907467252049095   -0.009907467252049267   -0.01994687425966367    -0.01994687425966378    0.113033873769652       0.1130338737696521      -0.05211612869151173    -0.05211612869151219    -0.02412713215543082   -0.02412713215543105   -0.196853022496179     -0.196853022496178     -0.01711547432151765    -0.01711547432151802    -0.01711547432151765    -0.01711547432151802    -0.2158675551850587    -0.215867555185062     0.09482779856389197     0.0948277985638918      -0.3196301490732346    -0.3196301490732367    0.1129147565386786      0.1129147565386786      0.06464295978127219     0.0646429597812723      -0.09833867695283365   -0.09833867695283366   -0.01189999269265692   -0.01189999269265711   -0.001666126114204092   -0.001666126114204349 
13.333333333333334  -0.000780597408081893   -0.0007805974080816837  -0.1197317145853972    -0.1197317145853964    0.05462361149062823    0.0546236114906278     0.05462361149062823    0.0546236114906278     0.1329499250832857     0.1329499250832866     -0.1139198665664786     -0.1139198665664785     -0.08099896438217186   -0.08099896438217222   0.1371727527426722      0.1371727527426722      0.0003310852576979354   0.0003310852576981647   -0.009825851820232368   -0.009825851820232542   -0.01932208844982853    -0.01932208844982863    0.1119123701555122      0.1119123701555123      -0.07004547118707158    -0.07004547118707205    -0.02396651942583733   -0.02396651942583755   -0.1637151136109355    -0.1637151136109352    -0.01694268451308901    -0.01694268451308937    -0.01694268451308901    -0.01694268451308937    -0.1617772917038252    -0.1617772917038293    0.09398930393616164     0.09398930393616146     -0.2642356380583705    -0.2642356380583736    0.1117802978577325      0.1117802978577325      0.0640707927095015      0.0640707927095016      -0.09905316849872986   -0.09905316849872989   -0.01176605881798892   -0.01176605881798911   -0.001647034822928043   -0.001647034822928297 
13.535353535353535  -0.0007385749563488157  -0.0007385749563484719  -0.1292769742484104    -0.1292769742484096    0.05770365389340772    0.05770365389340727    0.05770365389340772    0.05770365389340727    0.131505256358666      0.1315052563586669     -0.1097861744552574     -0.1097861744552574     -0.08013558372809598   -0.08013558372809633   0.1345211996472818      0.1345211996472818      0.001976694406638744    0.001976694406638911    -0.009545286431155375   -0.009545286431155544   -0.01880755397017465    -0.01880755397017476    0.1108073018950998      0.1108073018950999      -0.08722313205335822    -0.0872231320533587     -0.02374698481760224   -0.02374698481760246   -0.1219880044947171    -0.1219880044947173    -0.01677434631847261    -0.01677434631847297    -0.01677434631847261    -0.01677434631847297    -0.1001270125869976    -0.1001270125870022    0.09307211712834605     0.09307211712834589     -0.1899577762185182    -0.189957776218522     0.1107207273055422      0.1107207273055422      0.06346204998574027     0.06346204998574037     -0.09918794358066686   -0.09918794358066693   -0.01164849767871805   -0.01164849767871823   -0.001630011952765897   -0.001630011952766149 
13.737373737373737  -0.000696422063931892   -0.0006964220639316725  -0.1381480665323081    -0.1381480665323073    0.06101809064838871    0.06101809064838828    0.06101809064838871    0.06101809064838828    0.1301837897336818     0.1301837897336827     -0.1056333314614266     -0.1056333314614265     -0.07933425157100563   -0.07933425157100596   0.1317499861742172      0.1317499861742173      0.003438776096306605    0.003438776096306244    -0.009080388174630137   -0.009080388174630302   -0.01840552686716562    -0.01840552686716573    0.1097168548986933      0.1097168548986934      -0.1033221982904579     -0.1033221982904584     -0.0234755592426086    -0.02347555924260881   -0.07402110294043505   -0.07402110294043561   -0.01661432897035254    -0.01661432897035289    -0.01661432897035254    -0.01661432897035289    -0.03579498913043927   -0.03579498913044411   0.09207935707261422     0.09207935707261403     -0.1015565859296181    -0.1015565859296224    0.1097341862491104      0.1097341862491104      0.06282038119416633     0.06282038119416643     -0.09874289604905857   -0.09874289604905867   -0.01156199012416069   -0.01156199012416087   -0.00161829360878015    -0.001618293608780399 
13.93939393939394   -0.0006548644835140348  -0.0006548644835140995  -0.1464093937020818    -0.1464093937020809    0.06415772325805769    0.06415772325805726    0.06415772325805769    0.06415772325805726    0.1290495004711053     0.1290495004711061     -0.1014128930071101     -0.10141289300711       -0.07855801495542401   -0.07855801495542432   0.1288902272446356      0.1288902272446356      0.004701305366347982    0.004701305366347784    -0.00845149098805097    -0.008451490988051135   -0.01811200095712152    -0.01811200095712164    0.1086392885698393      0.1086392885698394      -0.1180450110125886     -0.1180450110125891     -0.02315725910960509   -0.02315725910960531   -0.02260092353795409   -0.02260092353795478   -0.01646299165688875    -0.0164629916568891     -0.01646299165688875    -0.0164629916568891     0.02769867114552958    0.02769867114552477    0.09101285493602264     0.09101285493602246     -0.006176059505211891  -0.006176059505216285  0.1088118762213898      0.1088118762213898      0.06214427001872309     0.06214427001872317     -0.09773395808600663   -0.09773395808600673   -0.01151466884426276   -0.01151466884426294   -0.001613062544757607   -0.001613062544757854 
14.141414141414142  -0.0006145895712079325  -0.0006145895712082264  -0.1542066046730707    -0.1542066046730697    0.0666731773463696     0.06667317734636916    0.0666731773463696     0.06667317734636916    0.1281111322667677     0.1281111322667685     -0.09705895943976842    -0.09705895943976835    -0.07778206291055889   -0.07778206291055917   0.1259757590186244      0.1259757590186244      0.005749402886236234    0.005749402886236551    -0.007683596238051225   -0.007683596238051391   -0.01791775119179906    -0.01791775119179916    0.1075728111427641      0.1075728111427643      -0.1311348402734436     -0.1311348402734441     -0.02279537893991234   -0.02279537893991254   0.02910972381715859    0.02910972381715796    -0.01631753729959299    -0.01631753729959333    -0.01631753729959299    -0.01631753729959333    0.08779064114108832    0.08779064114108372    0.08987294689336117     0.08987294689336099     0.08761218191112334    0.08761218191111923    0.107941646038027       0.107941646038027       0.0614293909353929      0.06142939093539299     -0.09618662041282579   -0.09618662041282589   -0.01150679766015089   -0.01150679766015108   -0.001613301585464713   -0.001613301585464958 
14.343434343434344  -0.0005763175526760588  -0.0005763175526763602  -0.1617355873016697    -0.1617355873016687    0.06815729589414307    0.06815729589414268    0.06815729589414307    0.06815729589414268    0.1273345506021273     0.1273345506021281     -0.09248653127855068    -0.09248653127855062    -0.07699477535978264   -0.07699477535978291   0.1230427096378424      0.1230427096378424      0.006574016393821967    0.006574016393822053    -0.006805894997727295   -0.006805894997727466   -0.01780919073760948    -0.01780919073760958    0.1065155563627414      0.1065155563627415      -0.1423925792172585     -0.1423925792172589     -0.02239167896301855   -0.02239167896301876   0.07763600763030447    0.07763600763030405    -0.0161721103023485     -0.01617211030234884    -0.0161721103023485     -0.01617211030234884    0.1420949801782051     0.1420949801782009     0.08865830111965384     0.08865830111965366     0.1708175777866557     0.1708175777866522     0.1071116832690679      0.1071116832690679      0.06067126314206139     0.06067126314206146     -0.09412712842326315   -0.09412712842326328   -0.0115286222143801    -0.01152862221438028   -0.001615482476591181   -0.001615482476591424 
14.545454545454545  -0.0005407978042065125  -0.000540797804206596   -0.1692032625195207    -0.1692032625195196    0.06835077194096018    0.06835077194095981    0.06835077194096018    0.06835077194095981    0.1266522386505489     0.1266522386505497     -0.08759444743946927    -0.08759444743946924    -0.07619823488211283   -0.07619823488211311   0.1201282851024023      0.1201282851024023      0.007175087684179285    0.007175087684178925    -0.0058513210125454     -0.005851321012545576   -0.0177690385820017     -0.01776903858200181    0.1054654790353791      0.1054654790353792      -0.1516919212782868     -0.1516919212782873     -0.02194657021717037   -0.02194657021717057   0.1194273083791828     0.1194273083791828     -0.01601797142837044    -0.01601797142837078    -0.01601797142837044    -0.01601797142837078    0.1873442286686631     0.1873442286686595     0.08736585017519082     0.08736585017519066     0.2356070980815819     0.2356070980815793     0.1063129686714937      0.1063129686714937      0.05986717574976883     0.05986717574976888     -0.09157528134558558   -0.09157528134558572   -0.01155880233815062   -0.0115588023381508    -0.001613422765938993   -0.001613422765939235 
14.747474747474747  -0.0005087276930265782  -0.000508727693026385   -0.1767786657866564    -0.1767786657866552    0.06723337945633825    0.06723337945633792    0.06723337945633825    0.06723337945633792    0.1259721608563829     0.1259721608563838     -0.08227832417602865    -0.08227832417602868    -0.07540685375773082   -0.07540685375773108   0.1172677583136944      0.1172677583136944      0.007565173158485724    0.007565173158485675    -0.00485560303948787    -0.004855603039488053   -0.01777726200252775    -0.01777726200252785    0.1044199298760038      0.1044199298760039      -0.158989654594265      -0.1589896545942654     -0.02145953095573423   -0.02145953095573443   0.1513367976295341     0.1513367976295346     -0.01584429419558215    -0.01584429419558249    -0.01584429419558215    -0.01584429419558249    0.2183399988211288     0.2183399988211261     0.08599082150478246     0.0859908215047823      0.2768080406269403     0.2768080406269388     0.1055396522610028      0.1055396522610028      0.05901673229761287     0.05901673229761292     -0.08854169305205073   -0.08854169305205091   -0.01156542549306789   -0.01156542549306808   -0.00159877674211715    -0.00159877674211739  
14.94949494949495   -0.0004806114955878476  -0.000480611495587525   -0.1845439995005414    -0.1845439995005401    0.06504939740854777    0.06504939740854744    0.06504939740854777    0.06504939740854744    0.1251885103258536     0.1251885103258545     -0.07645333076678591    -0.07645333076678597    -0.07464277875977476   -0.07464277875977503   0.1144900525191941      0.1144900525191941      0.007775532044116321    0.007775532044116669    -0.003855660164497594   -0.003855660164497784   -0.01781265501654737    -0.01781265501654747    0.1033748549572377      0.1033748549572378      -0.1643304490515489     -0.1643304490515494     -0.02092987870873374   -0.02092987870873395   0.1712294263848715     0.1712294263848726     -0.01563984903122564    -0.01563984903122598    -0.01563984903122564    -0.01563984903122598    0.2278599185885388     0.2278599185885373     0.0845268404400253      0.08452684044002513     0.2921228432471885     0.2921228432471881     0.1047872468303554      0.1047872468303555      0.05812089378764513     0.05812089378764519     -0.08503045464173073   -0.08503045464173092   -0.01151080139873749   -0.01151080139873769   -0.001562383680322454   -0.001562383680322693 
15.151515151515152  -0.000456593580285789   -0.0004565935802855792  -0.1924615370086843    -0.192461537008683     0.06223803301405471    0.06223803301405437    0.06223803301405471    0.06223803301405437    0.1241951564198577     0.1241951564198587     -0.07008031764329421    -0.07008031764329431    -0.073928060540994     -0.07392806054099427   0.1118133863993228      0.1118133863993229      0.007848040574210338    0.00784804057421033     -0.002888011534504977   -0.002888011534505175   -0.01785495121282492    -0.01785495121282502    0.1023238134579488      0.1023238134579489      -0.1678458034359385     -0.1678458034359389     -0.02035778280091506   -0.02035778280091527   0.178446274675857      0.1784462746758588     -0.01539530243592051    -0.01539530243592084    -0.01539530243592051    -0.01539530243592084    0.2086904603013936     0.2086904603013935     0.08296609858734369     0.08296609858734352     0.281215700253666      0.2812157002536668     0.1040492981557678      0.1040492981557678      0.05717992904977821     0.05717992904977823     -0.08104608720952021   -0.08104608720952043   -0.0113594223807456    -0.0113594223807458    -0.001496239369566466   -0.001496239369566705 
15.353535353535353  -0.0004363388647917337  -0.000436338864791789   -0.2003691252657993    -0.2003691252657979    0.05929057689539568    0.0592905768953953     0.05929057689539568    0.0592905768953953     0.1229004918092184     0.1229004918092193     -0.06318558777286301    -0.06318558777286312    -0.07327542364169254   -0.07327542364169283   0.1092427023387696      0.1092427023387696      0.00781787694818068     0.007817876948180357    -0.001988704373487946   -0.001988704373488153   -0.01788693023031537    -0.01788693023031546    0.1012571632664667      0.1012571632664668      -0.1697461941963073     -0.1697461941963078     -0.01974520936224369   -0.01974520936224391   0.1738524319804653     0.1738524319804677     -0.01510542990258007    -0.01510542990258041    -0.01510542990258007    -0.01510542990258041    0.1575438793992325     0.1575438793992339     0.08129959502404621     0.08129959502404607     0.2451739999537172     0.2451739999537191     0.1033137715206095      0.1033137715206095      0.0561910813688245      0.05619108136882454     -0.07660206688683778   -0.07660206688683802   -0.01108675485182669   -0.01108675485182689   -0.001395499432108683   -0.001395499432108921 
15.555555555555555  -0.0004190456478661805  -0.0004190456478664523  -0.2080061554134779    -0.2080061554134765    0.05659769614757693    0.05659769614757649    0.05659769614757693    0.05659769614757649    0.1212413157023385     0.1212413157023395     -0.05586652569989081    -0.05586652569989096    -0.07268056776846187   -0.07268056776846213   0.1067700340709791      0.1067700340709792      0.007712097472463197    0.007712097472463294    -0.001195779584347768   -0.001195779584347984   -0.01789583314331514    -0.01789583314331522    0.1001617492174641      0.1001617492174642      -0.1703053812269665     -0.170305381226967      -0.0190964683330259    -0.01909646833302611   0.1594270456073494     0.1594270456073522     -0.01477048324508447    -0.0147704832450848     -0.01477048324508447    -0.0147704832450848     0.0783700195330315     0.07837001953303428    0.07951744240000753     0.07951744240000741     0.1869972823853248     0.1869972823853278     0.1025606058611242      0.1025606058611242      0.05514690828128811     0.05514690828128815     -0.07172746595822277   -0.071727465958223     -0.01068591841902228   -0.01068591841902249   -0.001259873149619153   -0.00125987314961939  
15.757575757575758  -0.0004035923592856047  -0.0004035923592858873  -0.2150582827861702    -0.2150582827861689    0.0543486028579646     0.05434860285796413    0.0543486028579646     0.05434860285796413    0.11919445162331       0.119194451623311      -0.04828208508387585    -0.04828208508387604    -0.07211825488375428   -0.07211825488375456   0.1043776222389518      0.1043776222389518      0.007557229178644881    0.007557229178645214    -0.0005531149824684311  -0.0005531149824686558  -0.01787369565820842    -0.01787369565820851    0.09902119740388818     0.09902119740388833     -0.1698367475610017     -0.1698367475610022     -0.01841822616275911   -0.01841822616275932   0.1376844751029736     0.1376844751029767     -0.01439635683210188    -0.01439635683210222    -0.01439635683210188    -0.01439635683210222    -0.01768556069320736   -0.01768556069320374   0.07760917943741077     0.07760917943741066     0.112167780551248      0.1121677805512518     0.101761622322856       0.101761622322856       0.05403507592762146     0.05403507592762149     -0.06646867121582924   -0.0664686712158295    -0.0101703409778474    -0.01017034097784761   -0.00109406732304724    -0.001094067323047476 
15.95959595959596   -0.000388722202364627   -0.0003887222023647053  -0.2212019126216025    -0.2212019126216012    0.05250436590322173    0.05250436590322122    0.05250436590322173    0.05250436590322122    0.1167870106423428     0.116787010642344      -0.0406340428922402     -0.04063404289224042    -0.07154244541304394   -0.07154244541304419   0.1020422531267218      0.1020422531267219      0.007372034075965975    0.007372034075965862    -0.0001125107595687035  -0.0001125107595689354  -0.0178167602347009     -0.01781676023470098    0.09781661454872803     0.09781661454872816     -0.1686638243882433     -0.1686638243882438     -0.01771911813204767   -0.01771911813204789   0.1112939819858445     0.1112939819858478     -0.01399378377501372    -0.01399378377501405    -0.01399378377501372    -0.01399378377501405    -0.1159206336334615    -0.1159206336334576    0.07556399933918878     0.07556399933918867     0.02832720299628086    0.02832720299628527    0.1008832361909489      0.1008832361909489      0.05283990834324039     0.0528399083432404      -0.06088479323860499   -0.06088479323860526   -0.009572460346306077  -0.009572460346306285  -0.0009073766521617941  -0.000907376652162029 
16.161616161616163  -0.0003731839297248938  -0.0003731839297247126  -0.2261342582650943    -0.226134258265093     0.05082953480310637    0.05082953480310585    0.05082953480310637    0.05082953480310585    0.1141062048351714     0.1141062048351726     -0.03314598528299483    -0.03314598528299505    -0.07088929462515314   -0.07088929462515338   0.0997392306198776      0.09973923061987763     0.007160255685365702    0.007160255685365438    0.00006876716302722018  0.00006876716302698216  -0.01772442447404081    -0.01772442447404089    0.09652741111696969     0.09652741111696986     -0.1670895132774838     -0.1670895132774844     -0.01700917353163328   -0.0170091735316335    0.08298181616678013    0.0829818161667834     -0.0135770266996678     -0.01357702669966813    -0.0135770266996678     -0.01357702669966813    -0.2030962499702801    -0.2030962499702761    0.07337086463859094     0.07337086463859085     -0.0558867806380642    -0.05588678063805966   0.09989139261337808     0.09989139261337808     0.05154533020960604     0.05154533020960605     -0.05503785154057353   -0.0550378515405738    -0.00893974075342141   -0.008939740753421615  -0.0007127127477381496  -0.0007127127477383824
16.363636363636363  -0.0003558212769605713  -0.0003558212769602671  -0.2295853897915084    -0.2295853897915073    0.04895474750541493    0.04895474750541441    0.04895474750541493    0.04895474750541441    0.1113060352095477     0.1113060352095489     -0.02604323115454225    -0.02604323115454246    -0.07008181737767723   -0.07008181737767746   0.09744540365381561     0.09744540365381565     0.006927022821603197    0.006927022821603401    -0.00006467149916965195 -0.00006467149916989512 -0.0175980384906191     -0.01759803849061917    0.09513213078675309     0.09513213078675327     -0.1653687933422262     -0.1653687933422268     -0.01629913210206662   -0.01629913210206684   0.05550794124334715    0.05550794124335021    -0.01316232877673337    -0.01316232877673369    -0.01316232877673337    -0.01316232877673369    -0.2712403507597192    -0.2712403507597157    0.07101859738633934     0.07101859738633927     -0.132238538440078     -0.1322385384400739    0.09875723912760402     0.09875723912760402     0.05013825763688267     0.05013825763688267     -0.04898106427483514   -0.04898106427483542   -0.008329114886585007  -0.008329114886585205  -0.0005252604480580989  -0.0005252604480583292
16.565656565656564  -0.0003356280146243241  -0.0003356280146241235  -0.2313156617915799    -0.2313156617915787    0.04644859061662717    0.04644859061662667    0.04644859061662717    0.04644859061662667    0.1086044767642629     0.1086044767642641     -0.01953450229709963    -0.01953450229709983    -0.0690355373867784    -0.06903553738677862   0.0951413551107783      0.09514135511077833     0.006695742704806273    0.00669574270480654     -0.0005564099146669282  -0.000556409914667175   -0.01743966639576532    -0.01743966639576539    0.09360928939820562     0.0936092893982058      -0.1636879845957158     -0.1636879845957163     -0.01559963212629496   -0.01559963212629517   0.03153036597037893    0.03153036597038163    -0.01276621846162045    -0.01276621846162077    -0.01276621846162045    -0.01276621846162077    -0.318619578217519     -0.3186195782175161    0.06849608928096523     0.06849608928096518     -0.1939550132401815    -0.1939550132401781    0.09746168644287218     0.09746168644287213     0.04861127332891221     0.04861127332891221     -0.04274949143297177   -0.04274949143297205   -0.007800616780906071  -0.00780061678090626   -0.0003608666358696021  -0.000360866635869829 
16.767676767676768  -0.0003117730154765003  -0.0003117730154765472  -0.2311043814307544    -0.2311043814307534    0.0428839643712069     0.04288396437120644    0.0428839643712069     0.04288396437120644    0.1062650338572684     0.1062650338572697     -0.01379689472272505    -0.01379689472272523    -0.0676646238763067    -0.06766462387630694   0.09281278905312702     0.09281278905312706     0.006498082827573363    0.006498082827573162    -0.001429311542936374   -0.001429311542936622   -0.01725093930152331    -0.01725093930152337    0.09193818997500967     0.09193818997500985     -0.1621516711018098     -0.1621516711018102     -0.01492030208477004   -0.01492030208477025   0.01338792286637335    0.0133879228663755     -0.01240383643558017    -0.01240383643558048    -0.01240383643558017    -0.01240383643558048    -0.3477271867001228    -0.3477271867001204    0.06579273220790663     0.0657927322079066      -0.2361369972528966    -0.2361369972528943    0.09599731016498145     0.09599731016498145     0.04696365987277515     0.04696365987277515     -0.0363565270864494    -0.03635652708644969   -0.007411132435834324  -0.007411132435834506  -0.0002343561419984881  -0.000234356141998711 
16.96969696969697   -0.0002836081269128371  -0.0002836081269130895  -0.2287376177421193    -0.2287376177421184    0.03789116618636629    0.03789116618636588    0.03789116618636629    0.03789116618636588    0.1045608840040011     0.1045608840040025     -0.008966004970888636   -0.008966004970888779   -0.06588821729996426   -0.06588821729996451   0.09045113090410033     0.0904511309041004      0.006359194201736497    0.006359194201736316    -0.002680761507312838   -0.002680761507313088   -0.01703213740849508    -0.01703213740849513    0.09009966322952591     0.09009966322952609     -0.1607769706047219     -0.1607769706047223     -0.01426884033043191   -0.01426884033043212   0.002932089588709537   0.002932089588710989   -0.01208750185875146    -0.01208750185875176    -0.01208750185875146    -0.01208750185875176    -0.3616758291037531    -0.3616758291037515    0.0628990949802481      0.06289909498024808     -0.2556835415037617    -0.2556835415037608    0.09436681619390055     0.09436681619390055     0.0452003969688301      0.0452003969688301      -0.02979777098079385   -0.02979777098079413   -0.007209159976036885  -0.00720915997603705   -0.0001580270429093637  -0.000158027042909582 
17.171717171717173  -0.0002506908288631536  -0.0002506908288634168  -0.2240051367177126    -0.2240051367177118    0.03119889527868282    0.03119889527868247    0.03119889527868282    0.03119889527868247    0.1037257524373908     0.1037257524373921     -0.005131064920276806   -0.005131064920276914   -0.06363714272872743   -0.06363714272872771   0.08805371464727701     0.08805371464727711     0.006301105872899404    0.006301105872899667    -0.004283421376562832   -0.004283421376563082   -0.01678147638621979    -0.01678147638621983    0.08807677151275503     0.08807677151275525     -0.1594940203145276     -0.159494020314528      -0.01365012260014233   -0.01365012260014254   0.001429733568877593   0.001429733568878255   -0.01182557006835943    -0.01182557006835973    -0.01182557006835943    -0.01182557006835973    -0.3611295044511366    -0.3611295044511359    0.05980781272080232     0.05980781272080233     -0.251247145329593     -0.2512471453295937    0.09257827037539598     0.09257827037539598     0.04332933258277657     0.04332933258277657     -0.0230614057558122    -0.02306140575581248   -0.007230748905589611  -0.007230748905589766  -0.0001404529435172254  -0.0001404529435174387
17.373737373737374  -0.0002128460669429495  -0.0002128460669430241  -0.2167157455280777    -0.2167157455280769    0.02267088528209896    0.02267088528209868    0.02267088528209896    0.02267088528209868    0.1039009726090819     0.1039009726090832     -0.002332895686598989   -0.00233289568659906    -0.06086148293553156   -0.06086148293553182   0.08562412619950298     0.0856241261995031      0.006342968885973997    0.006342968885974146    -0.006190909543412289   -0.006190909543412537   -0.01649442306620034    -0.01649442306620038    0.08585557066922697     0.08585557066922717     -0.1581515325119511     -0.1581515325119515     -0.01306533319275687   -0.01306533319275707   0.009430705721438141   0.009430705721437957   -0.01162145318201497    -0.01162145318201526    -0.01162145318201497    -0.01162145318201526    -0.3433587748918227    -0.343358774891823     0.05651459001842404     0.05651459001842406     -0.223515776357528     -0.2235157763575303    0.09063826557308906     0.09063826557308903     0.04135727702037605     0.04135727702037605     -0.01614209344100286   -0.01614209344100315   -0.007496041279447067  -0.007496041279447206  -0.0001855540037434589  -0.0001855540037436671
17.575757575757574  -0.0001702615092113102  -0.0001702615092111419  -0.206735496122361     -0.2067354961223603    0.01234497980487623    0.012344979804876      0.01234497980487623    0.012344979804876      0.1050909220109626     0.1050909220109639     -0.0005624435749731172  -0.0005624435749731534  -0.05753904133417532   -0.0575390413341756    0.08317301364098202     0.08317301364098212     0.006485318923853129    0.006485318923852855    -0.00834666667227086    -0.008346666672271104   -0.01616290483882186    -0.01616290483882188    0.08342595943560098     0.0834259594356012      -0.1565269265775968     -0.1565269265775972     -0.01251116037443589   -0.0125111603744361    0.02652349861855217    0.02652349861855118    -0.01147268327510467    -0.01147268327510496    -0.01147268327510467    -0.01147268327510496    -0.3037781217856774    -0.303778121785679     0.0530191415345332      0.05301914153453324     -0.1755674751969707    -0.1755674751969742    0.08854494110599687     0.08854494110599687     0.03928614427031687     0.03928614427031687     -0.009054166208806549  -0.009054166208806828  -0.008005802888136793  -0.00800580288813692   -0.0002918624097648949  -0.0002918624097650984
17.77777777777778   -0.0001235836867544462  -0.0001235836867541619  -0.1940452128925342    -0.1940452128925336    0.0004736231617313793  0.0004736231617311791  0.0004736231617313793  0.0004736231617311791  0.1071375563828085     0.1071375563828097     0.0002408127444768351   0.0002408127444768315   -0.05368371639548978   -0.05368371639549008   0.0807192336919824      0.08071923369198251     0.006704897081062192    0.006704897081062094    -0.01069395481790877    -0.01069395481790901    -0.01577445346881711    -0.01577445346881712    0.08078252219061641     0.08078252219061664     -0.1543413227306559     -0.1543413227306563     -0.0119791926200611    -0.01197919262006131   0.05105965346297114    0.05105965346296948    -0.01137006085704295    -0.01137006085704323    -0.01137006085704295    -0.01137006085704323    -0.2392503073975551    -0.2392503073975578    0.04932584357320965     0.04932584357320972     -0.1128093998720096    -0.1128093998720142    0.08628303358218206     0.086283033582182       0.03711035474769197     0.03711035474769197     -0.001839874401115422  -0.001839874401115687  -0.00873797474120919   -0.008737974741209308  -0.0004519957573873668  -0.0004519957573875663
17.97979797979798   -0.00007396702037366619 -0.00007396702037347598 -0.1788040764894439    -0.1788040764894433    -0.01244684886654236   -0.01244684886654255   -0.01244684886654236   -0.01244684886654255   0.1097200555778001     0.1097200555778013     0.0001891013353086569   0.0001891013353086791   -0.04935164998947595   -0.04935164998947623   0.07829071200057496     0.07829071200057507     0.006967318048554222    0.006967318048554495    -0.01318480906705286    -0.01318480906705309    -0.01531157378797021    -0.01531157378797022    0.07792513189614642     0.07792513189614668     -0.1512800993961344     -0.1512800993961348     -0.0114557507283418    -0.01145575072834201   0.08005654845708239    0.08005654845708028    -0.01129716951192794    -0.01129716951192821    -0.01129716951192794    -0.01129716951192821    -0.1517252760358211    -0.1517252760358247    0.04544387883061827     0.04544387883061836     -0.04226608254881306   -0.04226608254881827   0.08382275355337217     0.08382275355337214     0.03481643239304017     0.03481643239304018     0.005430260740213627   0.005430260740213376   -0.009645263714911003  -0.009645263714911118  -0.0006525029316022192  -0.0006525029316024156
18.181818181818183  -0.00002302216143664568 -0.0000230221614366855  -0.1613967132751591    -0.1613967132751585    -0.02568160588886581   -0.02568160588886605   -0.02568160588886581   -0.02568160588886605   0.1123806132834816     0.1123806132834828     -0.0005584916550827087  -0.0005584916550826693  -0.04464219181597439   -0.0446421918159747    0.07592387319038672     0.07592387319038685     0.00722811877880253     0.007228118778802547    -0.01578543214233381    -0.01578543214233404    -0.01475191315802516    -0.01475191315802517    0.0748589461698926      0.07485894616989286     -0.1470192352386334     -0.1470192352386339     -0.01092246248207405   -0.01092246248207425   0.1094626812030903     0.109462681203088      -0.01123079354497656    -0.01123079354497684    -0.01123079354497656    -0.01123079354497684    -0.05030995418297651   -0.05030995418298027   0.04138673349048948     0.04138673349048957     0.02858012306225474    0.02858012306224939    0.0811232544352866      0.08112325443528658     0.03238511162553778     0.03238511162553779     0.01266560886731927    0.01266560886731903    -0.01065582163103164   -0.01065582163103175   -0.0008744207734853237  -0.0008744207734855185
18.383838383838384  0.00002737272717149773  0.00002737272717126522  -0.142437455408591     -0.1424374554085905    -0.03832266879153615   -0.03832266879153649   -0.03832266879153615   -0.03832266879153649   0.1145734986734535     0.1145734986734547     -0.001810381577868458   -0.00181038157786841    -0.039691001219179     -0.03969100121917931   0.07366040996201101     0.07366040996201109     0.007420136983913282    0.007420136983912979    -0.01847557486226632    -0.01847557486226654    -0.01406989878642117    -0.01406989878642118    0.07159344014567004     0.07159344014567032     -0.1412560103372981     -0.1412560103372986     -0.01035783147824322   -0.01035783147824343   0.134792592890207      0.1347925928902049     -0.01114284012532122    -0.0111428401253215     -0.01114284012532122    -0.0111428401253215     0.05001039019303856    0.05001039019303537    0.03717106316850387     0.03717106316850399     0.0928221750553566     0.09282217505535154    0.0781400476605387      0.07814004766053866     0.02979548781368194     0.02979548781368197     0.01977001040478682    0.01977001040478659    -0.01167940104733671   -0.01167940104733682   -0.001094941067460237   -0.001094941067460431 
18.585858585858585  0.00007537795693351682  0.0000753779569332727   -0.1227135149488457    -0.1227135149488451    -0.04942230311349352   -0.04942230311349401   -0.04942230311349352   -0.04942230311349401   0.1157310059657006     0.1157310059657018     -0.003367846394098277   -0.003367846394098229   -0.03465459572664814   -0.03465459572664844   0.07154104408391145     0.07154104408391154     0.00745987644344868     0.007459876443448672    -0.02124105086827339    -0.0212410508682736     -0.01324010769574748    -0.01324010769574749    0.06814042650579764     0.06814042650579792     -0.1337406093952145     -0.133740609395215      -0.009739760344706716  -0.009739760344706918  0.1519465912684338     0.1519465912684321     -0.01100396368638817    -0.01100396368638844    -0.01100396368638817    -0.01100396368638844    0.1332641280405231     0.1332641280405213     0.03281516531190742     0.03281516531190756     0.1444080995743098     0.1444080995743054     0.07483440927037431     0.07483440927037428     0.02703006535227345     0.02703006535227348     0.0266585659515339     0.02665856595153368    -0.01261996532317741   -0.01261996532317754   -0.001290335935743995   -0.001290335935744189 
18.78787878787879   0.0001195214016774137   0.0001195214016773423   -0.1030699547318675    -0.103069954731867     -0.05816753888913998   -0.05816753888914068   -0.05816753888913998   -0.05816753888914068   0.115337102243668      0.1153371022436693     -0.005056758864250686   -0.005056758864250646   -0.02968886817482623   -0.02968886817482652   0.06959747036588575     0.06959747036588583     0.007273442990439345    0.007273442990439594    -0.02406204154481014    -0.02406204154481034    -0.01224181647758323    -0.01224181647758324    0.06451150875338922     0.06451150875338951     -0.1243040915578035     -0.124304091557804      -0.00904854059073815   -0.00904854059073836   0.1579628671754902     0.1579628671754892     -0.0107882728071364     -0.01078827280713667    -0.0107882728071364     -0.01078827280713667    0.1886142533165955     0.1886142533165955     0.02833748499463707     0.02833748499463723     0.1776710135504646     0.1776710135504615     0.07118210956044109     0.07118210956044106     0.02407923878906255     0.02407923878906258     0.03327171139930186    0.03327171139930166    -0.01339254339219434   -0.01339254339219449   -0.001439729691467211   -0.001439729691467407 
18.98989898989899   0.0001589254324316447   0.0001589254324318012   -0.0842650358011695    -0.084265035801169     -0.06403694325516829   -0.06403694325516918   -0.06403694325516829   -0.06403694325516918   0.1129988585577146     0.1129988585577159     -0.006755755033137478   -0.006755755033137446   -0.02492686244294449   -0.02492686244294477   0.06784477156153362     0.06784477156153368     0.006809891714215088    0.006809891714214997    -0.02690331451820761    -0.0269033145182078     -0.01106348377870712    -0.01106348377870714    0.06071575505229667     0.06071575505229697     -0.1128787553270218     -0.1128787553270224     -0.008269502976981544  -0.008269502976981754  0.1514974362881057     0.1514974362881054     -0.0104777861615402     -0.01047778616154048    -0.0104777861615402     -0.01047778616154048    0.2138454189514903     0.213845418951492      0.02375561292662564     0.02375561292662582     0.1871239443088563     0.1871239443088549     0.06717891308622828     0.06717891308622824     0.02094390318951285     0.02094390318951287     0.03958249285494769    0.03958249285494751    -0.01393910507001626   -0.01393910507001642   -0.001528768504343248   -0.001528768504343447 
19.19191919191919   0.0001933559336697561   0.0001933559336700247   -0.06684043566109706   -0.06684043566109658   -0.06687959889598998   -0.06687959889599106   -0.06687959889598998   -0.06687959889599106   0.1085064938771926     0.1085064938771939     -0.008410874029213459   -0.008410874029213436   -0.02046172472958497   -0.02046172472958524   0.06627640998608457     0.06627640998608461     0.006040408642391843    0.006040408642391562    -0.02971244749794822    -0.02971244749794841    -0.009705879232856304   -0.009705879232856318   0.05675830648430177     0.05675830648430206     -0.0995088787892068     -0.09950887878920747    -0.007394584594266546  -0.007394584594266757  0.1329080365562678     0.1329080365562682     -0.01006525653707574    -0.01006525653707602    -0.01006525653707574    -0.01006525653707602    0.2141079205047919     0.214107920504795      0.01908606617340911     0.01908606617340932     0.1686302318648419     0.1686302318648423     0.06284118060286659     0.06284118060286654     0.01763545664750369     0.01763545664750372     0.04559496197615239    0.04559496197615222    -0.01423794250285318   -0.01423794250285336   -0.001552123358546368   -0.00155212335854657  
19.393939393939394  0.000223112274591222    0.0002231122745914066   -0.05104621337993537   -0.05104621337993499   -0.06689275380789544   -0.0668927538078967    -0.06689275380789544   -0.0668927538078967    0.1018731538228901     0.1018731538228915     -0.01003275433523919    -0.01003275433523916    -0.01633871112993618   -0.01633871112993643   0.06486254531724221     0.06486254531724224     0.004966541530539269    0.004966541530539351    -0.03242733402939246    -0.03242733402939264    -0.00818325575143277    -0.008183255751432784   0.05264017292413158     0.05264017292413186     -0.08435202457266956    -0.08435202457267028    -0.006422516560688012  -0.006422516560688224  0.1039440507594785     0.1039440507594795     -0.009554708773544595   -0.009554708773544876   -0.009554708773544595   -0.009554708773544876   0.1974224975713601     0.1974224975713642     0.01434484731993332     0.01434484731993354     0.1216521436829243     0.1216521436829264     0.05820131927628017     0.05820131927628013     0.01417323784764265     0.01417323784764268     0.05133418749338394    0.05133418749338379    -0.01430409798968196   -0.01430409798968217   -0.001514193221326519   -0.001514193221326723 
19.595959595959595  0.0002488500858199784   0.0002488500858199492   -0.03683626639653715   -0.03683626639653691   -0.06452484483448406   -0.06452484483448545   -0.06452484483448406   -0.06452484483448545   0.09334645001702359    0.09334645001702488    -0.01168089629722108    -0.01168089629722104    -0.01255643841691982   -0.01255643841692007   0.0635508200927444      0.0635508200927444      0.003631007185514555    0.003631007185514747    -0.03498805207878408    -0.03498805207878425    -0.006522921898266154   -0.006522921898266171   0.04835889836801407     0.04835889836801438     -0.0676720519938504     -0.06767205199385115    -0.005357928201599156  -0.005357928201599371  0.06722787608288094    0.06722787608288251    -0.008960105030230094   -0.008960105030230373   -0.008960105030230094   -0.008960105030230373   0.1704104049696447     0.1704104049696495     0.009548488853762891    0.009548488853763132    0.05088200298761803    0.05088200298762163    0.05329940562746507     0.05329940562746503     0.01058021569871492     0.01058021569871496     0.0568308266834085     0.05683082668340837    -0.01418277160604165   -0.01418277160604187   -0.001428132357719337   -0.001428132357719542 
19.7979797979798    0.0002714382022385714   0.0002714382022383605   -0.02392120291953035   -0.02392120291953029   -0.06035988288973905   -0.06035988288974051   -0.06035988288973905   -0.06035988288974051   0.08338829046586284    0.08338829046586403    -0.01344452885737404    -0.01344452885737398    -0.009074394084040203  -0.009074394084040463  0.06226813502262659     0.06226813502262658     0.002110763353242774    0.00211076335324259     -0.03734576267547167    -0.03734576267547182    -0.004764093753116984   -0.004764093753117002   0.04390949467996692     0.04390949467996721     -0.04982462068048484    -0.0498246206804856     -0.004210030588983624  -0.00421003058898384   0.02581448173671682    0.02581448173671878    -0.0083032371649856     -0.008303237164985879   -0.0083032371649856     -0.008303237164985879   0.1366720926575626     0.1366720926575677     0.004715142488942699    0.004715142488942958    -0.0340868282771015    -0.03408682827709714   0.04817355754302732     0.04817355754302729     0.006878296622561567    0.006878296622561614    0.06210479160181104    0.06210479160181093    -0.01394037193577004   -0.01394037193577027   -0.001313937479910584   -0.001313937479910788 
20.                 0.000291891699309844    0.0002918916993096182   -0.01185121306753325   -0.01185121306753343   -0.05503041841450904   -0.05503041841451047   -0.05503041841450904   -0.05503041841451047   0.07262610912241396    0.072626109122415      -0.0154274738558879     -0.0154274738558878     -0.005823190253302255  -0.005823190253302525  0.06092293385274559     0.06092293385274555     0.0005038338977606525   0.0005038338977604148   -0.03946417151449484    -0.039464171514495      -0.002956519866966393   -0.002956519866966412   0.03928530450683378     0.03928530450683407     -0.03123617383704854    -0.03123617383704932    -0.002991411018483902  -0.002991411018484117  -0.01699728396384837   -0.01699728396384624   -0.007611715534224568   -0.007611715534224842   -0.007611715534224568   -0.007611715534224842   0.09796789894532001    0.0979678989453254     -0.0001346521290893043  -0.0001346521290890282  -0.120904596083939     -0.1209045960839345    0.04285208405160838     0.04285208405160836     0.003084740561068351    0.003084740561068405    0.06715275772008536    0.06715275772008529    -0.0136570972448768    -0.01365709724487703   -0.001196330514528139   -0.00119633051452834  
\end{filecontents}

\pgfplotstableread{comparison_samples.dat}{\sampletable}
\begin{tikzpicture}
  \pgfplotsset{cycle list/Set1}
  \begin{groupplot}[group style={
        group name=matrix,
        group size= 5 by 5, 
        vertical sep = 2pt,
        horizontal sep = 2pt,
        xlabels at = edge bottom, 
        xticklabels at = edge bottom,
        ylabels at = edge left, 
        yticklabels at = edge left
      },
      ymin = -0.5, ymax = 0.5,
      %xlabel={Time (\si{\pico\second})},
      %ylabel={Polarization ($\rho_{01}$)},
      width=0.25\textwidth,
      height=0.25\textwidth
    ]

    \nextgroupplot[]
    \addplot+[smooth, ultra thick] table [x index=0, y index=1] {\sampletable};
    \addplot+[smooth, ultra thick] table [x index=0, y index=2] {\sampletable};

    \nextgroupplot[]
    \addplot+[smooth, ultra thick] table [x index=0, y index=3] {\sampletable};
    \addplot+[smooth, ultra thick] table [x index=0, y index=4] {\sampletable};

    \nextgroupplot[]
    \addplot+[smooth, ultra thick] table [x index=0, y index=5] {\sampletable};
    \addplot+[smooth, ultra thick] table [x index=0, y index=6] {\sampletable};

    \nextgroupplot[]
    \addplot+[smooth, ultra thick] table [x index=0, y index=7] {\sampletable};
    \addplot+[smooth, ultra thick] table [x index=0, y index=8] {\sampletable};

    \nextgroupplot[]
    \addplot+[smooth, ultra thick] table [x index=0, y index=9] {\sampletable};
    \addplot+[smooth, ultra thick] table [x index=0, y index=10] {\sampletable};

    \nextgroupplot[]
    \addplot+[smooth, ultra thick] table [x index=0, y index=11] {\sampletable};
    \addplot+[smooth, ultra thick] table [x index=0, y index=12] {\sampletable};

    \nextgroupplot[]
    \addplot+[smooth, ultra thick] table [x index=0, y index=13] {\sampletable};
    \addplot+[smooth, ultra thick] table [x index=0, y index=14] {\sampletable};

    \nextgroupplot[]
    \addplot+[smooth, ultra thick] table [x index=0, y index=15] {\sampletable};
    \addplot+[smooth, ultra thick] table [x index=0, y index=16] {\sampletable};

    \nextgroupplot[]
    \addplot+[smooth, ultra thick] table [x index=0, y index=17] {\sampletable};
    \addplot+[smooth, ultra thick] table [x index=0, y index=18] {\sampletable};

    \nextgroupplot[]
    \addplot+[smooth, ultra thick] table [x index=0, y index=19] {\sampletable};
    \addplot+[smooth, ultra thick] table [x index=0, y index=20] {\sampletable};

    \nextgroupplot[]
    \addplot+[smooth, ultra thick] table [x index=0, y index=21] {\sampletable};
    \addplot+[smooth, ultra thick] table [x index=0, y index=22] {\sampletable};

    \nextgroupplot[]
    \addplot+[smooth, ultra thick] table [x index=0, y index=23] {\sampletable};
    \addplot+[smooth, ultra thick] table [x index=0, y index=24] {\sampletable};

    \nextgroupplot[]
    \addplot+[smooth, ultra thick] table [x index=0, y index=25] {\sampletable};
    \addplot+[smooth, ultra thick] table [x index=0, y index=26] {\sampletable};

    \nextgroupplot[]
    \addplot+[smooth, ultra thick] table [x index=0, y index=27] {\sampletable};
    \addplot+[smooth, ultra thick] table [x index=0, y index=28] {\sampletable};

    \nextgroupplot[]
    \addplot+[smooth, ultra thick] table [x index=0, y index=29] {\sampletable};
    \addplot+[smooth, ultra thick] table [x index=0, y index=30] {\sampletable};

    \nextgroupplot[]
    \addplot+[smooth, ultra thick] table [x index=0, y index=31] {\sampletable};
    \addplot+[smooth, ultra thick] table [x index=0, y index=32] {\sampletable};

    \nextgroupplot[]
    \addplot+[smooth, ultra thick] table [x index=0, y index=33] {\sampletable};
    \addplot+[smooth, ultra thick] table [x index=0, y index=34] {\sampletable};

    \nextgroupplot[]
    \addplot+[smooth, ultra thick] table [x index=0, y index=35] {\sampletable};
    \addplot+[smooth, ultra thick] table [x index=0, y index=36] {\sampletable};

    \nextgroupplot[]
    \addplot+[smooth, ultra thick] table [x index=0, y index=37] {\sampletable};
    \addplot+[smooth, ultra thick] table [x index=0, y index=38] {\sampletable};

    \nextgroupplot[]
    \addplot+[smooth, ultra thick] table [x index=0, y index=39] {\sampletable};
    \addplot+[smooth, ultra thick] table [x index=0, y index=40] {\sampletable};

    \nextgroupplot[]
    \addplot+[smooth, ultra thick] table [x index=0, y index=41] {\sampletable};
    \addplot+[smooth, ultra thick] table [x index=0, y index=42] {\sampletable};

    \nextgroupplot[]
    \addplot+[smooth, ultra thick] table [x index=0, y index=43] {\sampletable};
    \addplot+[smooth, ultra thick] table [x index=0, y index=44] {\sampletable};

    \nextgroupplot[]
    \addplot+[smooth, ultra thick] table [x index=0, y index=45] {\sampletable};
    \addplot+[smooth, ultra thick] table [x index=0, y index=46] {\sampletable};

    \nextgroupplot[]
    \addplot+[smooth, ultra thick] table [x index=0, y index=47] {\sampletable};
    \addplot+[smooth, ultra thick] table [x index=0, y index=48] {\sampletable};

    \nextgroupplot[]
    \addplot+[smooth, ultra thick] table [x index=0, y index=49] {\sampletable};
    \addplot+[smooth, ultra thick] table [x index=0, y index=50] {\sampletable};
  \end{groupplot}
  %\node[anchor=south] at ($(matrix c1r1.north east)!0.5!(matrix c2r1.north west)$){Some long title};
  %\node[anchor=south] at ($(matrix c1r2.north east)!0.5!(matrix c2r2.north west)$){Another long title};

  \node[anchor=north, yshift=-0.7cm] at ($(matrix c3r5.south)$){Time (\si{\pico\second})};
  \node[rotate=90, yshift=2.8cm] at ($(matrix c1r3)$){Polarization ($\rho_{01}$)};
\end{tikzpicture}

%\end{figure}
