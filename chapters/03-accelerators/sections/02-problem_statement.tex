\section{Formulation}

The underlying mechanics remain effectively unchanged from \cref{ch:quantum dots}: a collection of $N_s$ pointlike \qds{} evolve due to incident electric field via \cref{eq:liouville}, implicitly coupled by way of secondary radiated fields.
Under the rotating-wave approximation~\cref{Allen1975}, the off-diagonal elements of a given \qd's density matrix, $\tilde{\rho}$, directly characterize the (complex-valued) polarization density of the \qd ($\tilde{\vb{P}}$).
This polarization density, then, acts as a source in Maxwell's equations to produce
\begin{equation}
  \tilde{\mathfrak{F}}\qty{\tilde{\vb{P}}(\vb{r}, t)} = -\frac{\mu_0}{4\pi} \int \qty(\pdv[2]{t} - c^2 \grad' \grad'\boldsymbol{\cdot}) \frac{\tilde{\vb{P}}(\vb{r}', t_R) e^{-i \omega_L \abs{\vb{r} - \vb{r}'}/c}}{\abs{\vb{r} - \vb{r}'}} \dd[3]{\vb{r}'}
\end{equation}
(see \cref{appendix:vector wave equation}).
This form of \cref{eq:radiated envelope} has several advantages in developing accelerated calculation schemes which we shall detail later.
Representing $\tilde{\vb{P}}(\vb{r}, t)$ as a collection of spatio-temporal basis functions,
\begin{equation}
  \tilde{\vb{P}}(\vb{r}, t) = \sum_{\ell = 0}^{N_s - 1} \sum_{m = 0}^{N_t - 1} \tilde{\mathcal{A}}_\ell^{(m)} \vb{S}_\ell(\vb{r}) T(t - m \, \Delta t),
\end{equation}
we have again chosen $\vb{S}_\ell(\vb{r}) = \vb{d}_\ell \delta(\vb{r} - \vb{r}_\ell)$ and $T(t)$ as the standard Lagrange polynomial basis set.
Additionally, AIM embeds $\tilde{\vb{P}}(\vb{r}, t)$ within an auxilliary cartesian grid with spacings chosen to sufficiently resolve the interference phenomena of \emph{unshifted}, fixed-frame sources.
