\section{Formulation}

We wish to calculate the the field radiated from a distribution of time-dependent sources under the action of an arbitrary (linear) differential operator, $\mathfrak{F}\qty{\cdot}$, in both space and time, presumably alongside/in response to an incident field, $\mathbf{E}_\text{inc}$.
Formally, we may write the total field everywhere as
\begin{equation}
  \begin{aligned}
    \vb{E}(\vb{r}, t) & = \vb{E}_\text{inc}(\vb{r}, t) + \mathfrak{F}\qty{\siint g(\vb{r} - \vb{r}', t - t') \vb{P}(\vb{r}', t') \dd[1]{t'} \dd[3]{\vb{r}'}} \\
                      & \equiv \vb{E}_\text{inc}(\vb{r}, t) + \mathfrak{F}\qty{g(\vb{r}, t) \ast \vb{P}(\vb{r}, t)}
    \end{aligned}
  \label{eq:propagator}
\end{equation}
where $g(\vb{r}, t)$ denotes a propagation potential (most commonly the retarded potential, $g(\vb{r}, t) = \delta(t - \abs{\vb{r}}/c)/\abs{\vb{r}}$).
To numerically evaluate \cref{eq:propagator}, we discretize $\vb{P}(\vb{r}, t)$ with separable space/time basis functions such that 
\begin{equation}
  \vb{P}(\vb{r}, t) \approx \sum_{\ell = 0}^{N_s - 1} \sum_{m = 0}^{N_t - 1} \mathcal{A}_\ell^{(m)} \vb{s}_\ell(\vb{r}) T(t - m \, \Delta t)
  \label{eq:discretization}
\end{equation}
where $\Delta t$ denotes a fixed time interval chosen to accurately oversample $\omega_\text{max}$---the highest frequency in $\vb{P}$.
Substituting \cref{eq:discretization} into \cref{eq:propagator} and projecting the resulting field onto the same set of $\vb{s}_\ell(\vb{r})$ produces an $(N_s \times N_s)$-block matrix equation of the form
\begin{equation}
  \mathcal{E}^{(m)} = \mathcal{E}_\text{inc}^{(m)} + \sum_{m'= 0}^m \mathcal{F}^{(m - m')} \boldsymbol{\cdot} \mathcal{A}^{(m')}
  \label{eq:mot}
\end{equation}
where
\begin{subequations}
  \begin{align}
    \mathcal{E}^{(m)}_\ell &= \langle \vb{s}_\ell(\vb{r}), \vb{E}(\vb{r}, m \, \Delta t) \rangle \label{eq:f elements}\\
    \mathcal{E}^{(m)}_{\text{inc},\ell} &= \langle \vb{s}_\ell(\vb{r}), \vb{E}_\text{inc}(\vb{r}, m \, \Delta t) \rangle \label{eq:l elements}\\
    \mathcal{F}^{(k)}_{\ell\ell'} &= \big\langle \vb{s}_\ell(\vb{r}), \mathfrak{F}\qty{g(\vb{r}, t) \ast \vb{s}_{\ell'}(\vb{r}) T(k \, \Delta t)} \big\rangle \label{eq:z elements}.
  \end{align}
\end{subequations}
From this, it immediately becomes apparent that \cref{eq:z elements} bottlenecks the field calculation due to the $\mathcal{O}(N_s^2)$ complexity of the inner product.

\subsection{PWTD breakdown and motivation for FFT-based methods}

Numerous schemes exist to ameliorate this bottleneck and accelerate calculations of the form in \cref{eq:mot}.
The Plane-Wave Time-Domain (PWTD) algorithm~\cite{PWTD} in particular stands as the time-domain analogue of frequency-domain fast multipole methods that reconstruct time-harmonic fields as a superposition of plane waves.
In particular, PWTD accelerates the calculation of a retarded potential,
\begin{equation}
  u(\vb{r}, t) = \sint \frac{\delta(t - \abs{\vb{r} - \vb{r}'}/c)}{\abs{\vb{r} - \vb{r}'}} \star q(\vb{r}', t) \dd[3]{\vb{r}'},
  \label{eq:retarded potential}
\end{equation}
via the Weyl identity,
\begin{subequations}
  \begin{align}
    u_W(\vb{r}, t) & = -\frac{\partial_t}{2c} \siint \delta\qty[t - \vu{k} \vdot (\vb{r} - \vb{r}')/c] \star q(\vb{r}', t') \dd[3]{\vb{r}'} \dd[2]{\vu{k}} \label{eq:weyl-a}\\
                   & = u(\vb{r}, t) - \sint \frac{\delta(t + \abs{\vb{r} - \vb{r}'}/c)}{\abs{\vb{r} - \vb{r}'}} \star q(\vb{r}', t) \dd[3]{\vb{r}'}. \label{eq:weyl-b}
  \end{align}
  \label{eq:weyl}
\end{subequations}
In these expressions, $\star$ indicates a convolution with respect to time, $\vu{k} = \vu{x} \sin \theta \cos \phi + \vu{y} \sin \theta \sin \phi \vu{y} + \vu{z} \cos \theta$, and $\sint \cdot \dd[2]{\vu{k}}$ an integration over the whole solid angle.
A detailed explanation of how \cref{eq:weyl} accelerates potential evaluations lies beyond the scope of this thesis, however it suffices to say that \cref{eq:weyl-a} aggregates $q(\vb{r}, t)$ within a source region onto a spectrum of plane-waves and that a clever\footnote{Some might even call it black magic.} time-gating procedure removes the advanced potential contribution in \cref{eq:weyl-b} to recover $u(\vb{r}, t)$.

To conjure an integration rule for \cref{eq:weyl-a}, consider
\begin{equation}
  \Xi \equiv \int_{0}^\pi \delta(t - R \cos \theta/c) \star f(t) \dd{(\cos\theta)}.
\end{equation}
A well-known expansion gives the $\delta$-function in terms of Legendre polynomials, thus
\begin{equation}
  \Xi = \int_{0}^\pi \qty(\sum_{n=0}^{\infty} P_n(t) P_n(R\cos\theta/c)) \star f(t) \dd{(\cos\theta)}
\end{equation}
