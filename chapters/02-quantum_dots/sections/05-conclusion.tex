\section{\label{section:conclusion}Conclusions \& future work}
Here we developed a robust, fine-grained algorithm to solve for the dynamics of an ensemble of \qds{} that couple in response to external light fields.
By making use of an integral equation kernel to propagate radiated fields, our model facilitates simulations of thousands of \qds{} in three dimensions with accurate bookkeeping of both near and far radiation fields.
Our simulations predict a suppression effect between adjacent \qds{} that screens out the incident laser pulse and we interpret this effect as a dynamical detuning that shifts the effective $\omega_0$ of the affected \qds{}.
Moreover, we observe additional oscillatory behavior and localization effects in larger clusters of particles.
These effects could prove useful in optically identifying quantum dot ``molecules'' in an extended sample by detecting residual localized polarization following integral $\pi$ pulse(s)---we expect that an experimental $\pi$-pulse calibrated to a single \qd{} with a scanning-type polarization measurement~\cite{Asakura2013} would reveal signatures of these effects in dense samples.
Finally, in larger systems of densely-packed \qds{}, we see significant localization/enhanced polarization over length scales comparable to that of the incident wavelength.
These effects persist in simulations with other extended geometries and in simulations with inhomogeneously-broadened transition frequencies, though the effects quickly disappear with only a few $\si{\milli\eV}$ detuning.

Semi-classical approaches can describe some superradiant effects within a continuum formulation~\cite{Gross1982,PhysRevA.4.302,PhysRevA.4.854}.
First predicted in 2005~\cite{Temnov2005} and observed in 2007~\cite{Scheibner2007}, superradiant effects in \qd{} ensembles have since spurred theoretical analyses into cooperative radiation mechanisms~\cite{Temnov2009,Chen2008}.
While our semiclassical approach accounts for collective effects due to the secondary field emission from \qds{}, we do so in the Hamiltonian term on the right hand side of \cref{eq:liouville} and not in the $\hat{\mathcal{D}}\qty[\hat{\rho}]$ dissipator.
In future work, we plan to extend our microscopic approach to include collective dissipation effects so as to better model superradiant phenomena.
We expect that our approach---when extended to systems containing a larger number of \qds{}---will aid in investigating the role of many-dot interactions in systems such as nanolasers~\cite{jahnke2016giant} that exploit these phenomena.
Unfortunately, the na\"ive $\mathcal{O}\qty(N_s^2)$ interaction calculation presented here hampers attempts to extend these calculations to systems more than $\sim 10^4$ spatial unknowns.
Our ongoing research includes the development of accelerated computational techniques that exploit the structure of $\mathcal{Z}$ to reduce the big-$\mathcal{O}$ complexity of \cref{eq:zmatrix}.
Additionally, the technique presented here readily extends to model atomic, molecular, and semiconductor systems with richer structure (e.g.\ systems with energy degeneracies or biexcitonic transitions).

%\acknowledgments
%We gratefully acknowledge support from the National Science Foundation grant ECCS-1408115, and extend our thanks to the developers of Eigen~\cite{Eigen}, VisIt~\cite{VisIt}, and NumPy/SciPy~\cite{NumPy,SciPy} for the software used in our simulation and analysis.
%Finally, we thank Dr.\ P.\ Schwendimann for a critical reading of the manuscript.
%\vspace{.5 cm}
