\section{\label{section:introduction}Introduction}

\dropcap{S}{emiconductor} structures containing a large number of \qds{} offer ideal environments for exploring collective effects induced by light-matter interactions.
Often, these structures exhibit new phenomena due to geometrical randomness and nonlinearities in the underlying system dynamics.
Additionally, optical excitations (excitons) undergo characteristic Rabi oscillations~\cite{Stievater2001,Kamada2001,Htoon2002} in \qds{} analogous to those observed in atomic systems; because \qds{} have stronger dipolar transitions than atoms, these light-induced oscillations generate secondary fields that couple the system more strongly than equivalent atomic species.
We can therefore expect---at least in some regions of the sample---these local secondary fields will produce modified collective behavior in the exciton dynamics.
Phenomena induced by these secondary fields have received considerable theoretical/computational~\cite{Slepyan2002,Slepyan2004} and experimental~\cite{Asakura2013} attention as they may provide new insight on the coherent dynamics of excitons in \qd{} systems.

In the realm of theoretical/computational investigation, researchers in atomic and solid-state optics have developed numerous variations of the Maxwell-Bloch equations~\cite{Gross1982} to describe features such as ringing in pulse propagation~\cite{Burnham1969,MacGillivray1976} or emission fluctuations~\cite{Haake1979}.
Early solution strategies for these equations fell to continuum models~\cite{Rehler1971,MacGillivray1976} that recover effects arising from far-field interactions or describe near-field phenomena assuming spatial homogeneity~\cite{Stroud1972}.
More recently, mesh-based PDE solvers~\cite{Vanneste2001,Fratalocchi2008,Bachelard2015} added a large degree of fidelity to these models, though the finite size of the mesh means they still strugle to resolve short-range effects without unduly increasing the computational cost.
Additionally, the nature of these meshes makes them prohibitively expensive to extend into higher dimensional geometries for optically-large systems.
In this work we develop a computational framework to discover signatures of collective effects in strongly-driven \qds{} within a microscopic formalism.
By constructing the Maxwell-Bloch equations with an integral kernel to describe radiation, we recover near- and far-electric fields with full fidelity across the simulation while allowing for dynamics at the level of individual \qds{}.
Our methodology---based on successful models of other electromagnetic~\cite{Shanker2000,Pray2012,Pray2014} and acoustic~\cite{Ergin1999a,Ergin1999b,Glosser2016} systems---accommodates $10^4$ particles distributed over optically-large regions in three dimensions.

As we explicitly track the evolution of each \qd{} in the system, we will numerically demonstrate that the collective Rabi oscillation can induce significant coupling in sufficiently close \qds{}.
This laser-induced inter-dot coupling manifests itself in different forms:
\begin{inparaenum}[(i)]
  \item The polarization generated in isolated \qd{} pairs dynamically suppresses the Rabi rotation.
  We interpret this as the consequence of a time-dependent energy shift that brings the pair temporarily out of resonance with the external driving field.
  \item In addition to this screening, we observe oscillations in the free-induction decay for larger multiplets of \qds{}.
  \item Optical pulses of integer $\pi$ area, for which we expect no polarization in uncoupled systems following the pulse, produce patterns of residual localized polarization that remain in the system.
  \item The long-range interactions in optically-large systems produce wavelength-scale regions of enhanced and suppressed polarization.
\end{inparaenum}
These effects could, for instance, help identify multiplets of dots that dynamically couple during Rabi oscillations, or help understand nonlinear pulse propagation effects in these media.

We structure the remainder of this chapter as follows: \cref{section:problem statement} motivates the physical model of an ensemble of two-level systems that interact through a classical electric field.
\Cref{section:computational approach} presents the details of our methodology in the context of a global rotating-wave approximation and we offer an implementation of this algorithm at~\cite{quest_release}.
\Cref{section:results} contains the results of our investigation where we observe polarization features not present in noninteracting systems at both sub- and super-wavelength scales.
Finally, \cref{section:conclusion} contains concluding remarks where we hypothesize on the mechanisms underpinning the observed polarization features as well as comment on our future work in this area.
