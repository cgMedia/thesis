\section{\label{section:computational approach}Computational Approach}

\begin{figure}
  \centering
  \begin{tikzpicture}[
  declare function = {
    basis(\x) = 
      and(-1 <= \x, \x < 0)*(1+x)*(2+x)*(3+x)/6 +
      and(0  <= \x, \x < 1)*(1-x)*(1+x)*(2+x)/2 +
      and(1  <= \x, \x < 2)*(1-x)*(2-x)*(1+x)/2 + 
      and(2  <= \x, \x < 3)*(1-x)*(2-x)*(3-x)/6 + 
      or(\x <-1, \x > 3)*0;
  }
  ]

  \begin{axis}[
      %xtick={-4, -3, -2, -1},
      minor x tick num={4},
      minor y tick num={4},
      xlabel = Timestep (dimensionless),
      ylabel = {$T(t/\Delta t)$}
      %xticklabels={$-4\Delta t$, $-3\Delta t$, $-2\Delta t$, $-\Delta t$}
  ]
  
  \addplot[cbblue, ultra thick, domain=-1:3, smooth, samples=256] {basis(x)};
  \end{axis}
\end{tikzpicture}

  \caption{\label{fig:interpolation basis} Nonorthogonal and $C^0$-continuous temporal basis function $T(t/\Delta t)$ constructed from intervals of third-order Lagrange polynomials.}
\end{figure}
To solve \cref{eq:rotating liouville,eq:radiated envelope} for each of $N_s$ \qds{} at $N_t$ equally-spaced timesteps, we begin with a suitable representation of $\tilde{\vb{P}}(\vb{r}, t)$ in terms of spatial and temporal basis functions, i.e.~
\begin{equation}
  \tilde{\vb{P}}(\vb{r}, t) = \sum_{\ell=0}^{N_s - 1} \sum_{m = 0}^{N_t - 1} \mathcal{A}_{\ell}^{(m)} \vb{S}_\ell(\vb{r}) T(t - m \, \Delta t).
  \label{eq:basis function representation}
\end{equation}
As the wavelength of any radiation in the system far exceeds the dimensions of the \qds{} under consideration, we take $\vb{S}_\ell(\vb{r}) \equiv \vb{d}_\ell \delta(\vb{r} - \vb{r}_\ell)$ where $\vb{d}_\ell$ and $\vb{r}_\ell$ denote the dipole moment and position of dot $\ell$.
Furthermore, we require the $T(t)$ to have finite support as well as causal and interpolatory properties so as to recover $\tilde{\vb{P}}$, $\partial_t \tilde{\vb{P}}$, and $\partial^2_t \tilde{\vb{P}}$ at every timestep.
Accordingly, we have elected to use
\begin{equation}
  T(t) = \sum_{j = 0}^p \lambda_j(t)
  \label{eq:basis sum}
\end{equation}
where
\begin{equation}
  \lambda_j(t) =
  \begin{cases}
    \frac{(1 - \tau)_j}{j!} \frac{(1 + \tau)_{p-j}}{(p-j)!} & j-1 \le \tau < j \\
    0 & \text{otherwise,}
  \end{cases}
  \label{eq:basis piece}
\end{equation}
$(a)_k \equiv \Gamma(a + k)/\Gamma(a)$ denotes the Pochhammer rising factorial, and $\tau \equiv t/\Delta t$.
Such a $T(t)$ consists of shifted, backwards-looking Lagrange polynomials of order $p$ (we require $p \ge 3$ to recover a twice-differentiable function), forming a temporal basis set with functions similar to the one shown in \cref{fig:interpolation basis}.
These functions  reliably interpolate smooth functions with controllable error and have a long history of use in studies of radiative systems~\cite{Manara1997,Bluck1997}.

Combining \cref{eq:radiated envelope,eq:basis function representation} and projecting the resulting field onto the spatiotemporal basis functions produces a marching-on-in-time system of the form
\begin{equation}
  \mathcal{L}^{(m)} + \sum_{k = 0}^{m} \mathcal{Z}^{(k)} \mathcal{A}^{(m - k)} = \mathcal{F}^{(m)}.
  \label{eq:zmatrix}
\end{equation}
In this (block $N_s \times N_s$) matrix equation
\begin{subequations}
  \begin{align}
    \mathcal{L}^{(m)}_{\ell} &= \big\langle \mathbf{S}_\ell(\vb{r}), \tilde{\vb{E}}_L(\vb{r}, m \, \Delta t) \big\rangle \\
    \mathcal{Z}^{(k)}_{\ell \ell'} &= \big\langle \mathbf{S}_\ell(\vb{r}), \tilde{\mathfrak{F}}\{ \mathbf{S}_{\ell'}(\mathbf{r}) T(k \, \Delta t) \} \big\rangle \\
    \mathcal{F}^{(m)}_\ell &= \big\langle \mathbf{S}_\ell(\vb{r}), \tilde{\mathbf{E}}(\vb{r}, m \, \Delta t) \big\rangle
  \end{align}
  \label{eq:projections}
\end{subequations}
where $\langle \cdot, \cdot \rangle$ denotes the scalar product of two functions.
As we have elected to use a uniform $\Delta t$, the summation in \cref{eq:zmatrix} represents a discrete convolution that will produce $\mathcal{F}^{(m)}$ given only $\mathcal{A}^{(m' \le m)}$.
To link $\mathcal{A}^{(m)}$ with the density matrix in \cref{eq:rotating liouville}, we take
\begin{equation}
  \mathcal{A}^{(m)}_\ell \equiv \tilde{\rho}_{\ell, 01}(t = m \, \Delta t)
  \label{eq:polarization definition}
\end{equation}
as the off-diagonal matrix elements (coherences) of $\tilde{\rho}_{\ell}$ directly characterize the dipole radiating at $\vb{r}_\ell$ under the rotating-wave approximation.
Consequently, determining $\mathcal{A}^{(m + 1)}$ amounts to integrating \cref{eq:rotating liouville} from $t_i = m \, \Delta t$ to $t_f = \qty(m + 1) \, \Delta t$ for every \qd{} in the system.
For this, we make use of the predictor/corrector scheme detailed in~\cite{Glaser2009}.
Defining $t_m \equiv m \, \Delta t$ and approximating $\tilde{\rho}(t)$ as a weighted sum of complex exponentials, the predictor/corrector scheme proceeds with an extrapolation predictor step,
\begin{equation}
  \tilde{\rho}_\ell(t_{m + 1}) \leftarrow \sum_{w = 0}^{W-1} \mathcal{P}_w^{\qty(0)} \tilde{\rho}_\ell(t_{m-w}) + \mathcal{P}_w^{\qty(1)} \, \partial_t \tilde{\rho}_\ell(t_{m-w}),
  \label{eq:predictor}
\end{equation}
and iterated corrector steps,
\begin{equation}
    \tilde{\rho}_\ell(t_{m + 1}) \leftarrow \sum_{w = -1}^{W - 1} \mathcal{C}_w^{\qty(0)} \tilde{\rho}_\ell(t_{m-w}) + \mathcal{C}_w^{\qty(1)} \, \partial_t \tilde{\rho}_\ell(t_{m-w})
  \label{eq:corrector}
\end{equation}
($\mathcal{C}_{-1}^{(0)} \equiv 0$ by construction) to advance the system by $\Delta t$.
Such an integrator has significantly better convergence properties than Runge-Kutta integrators for equations of the type seen in \cref{eq:rotating liouville}.

To summarize, one timestep of our solution strategy proceeds as follows:
\begin{enumerate}
  \item At timestep $m$, use \cref{eq:predictor,eq:polarization definition} to predict $\mathcal{A}^{(m + 1)}$.
    This prediction depends only on the known history of the system and does not require the calculation of any electromagnetic interactions.
  \item Use \cref{eq:zmatrix} to calculate $\mathcal{F}^{(m + 1)}$.
    Having extrapolated $\mathcal{A}^{(m + 1)}$ in step 1, $\mathcal{F}^{(m + 1)}$ will contain information from \qds{} within $c \, \Delta t$ of each other.
  \item Produce $\partial_t \tilde{\rho}(t_{m + 1})$ (and thus $\partial_t \mathcal{A}^{(m + 1)}$) by evaluating \cref{eq:rotating liouville} with the $\mathcal{F}^{(m+1)}$ found in step 2.
  \item Correct $\mathcal{A}^{(m + 1)}$ with \cref{eq:corrector} and $\partial_t \mathcal{A}^{(m + 1)}$ found in step 3.
    Repeat steps 2 through 4 until $\mathcal{A}^{(m + 1)}$ has sufficiently converged, then set $m \leftarrow m + 1$.
\end{enumerate}
