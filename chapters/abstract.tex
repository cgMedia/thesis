\begin{abstract}
  \dropcap{T}{his} work offers the interested reader a tour of multiphysics simulations in the context of radiative systems at the level of individual particles.
  As such, these models afford an extremely high degree of fidelity in the underlying physics without recourse to continuum or spatially-averaged approximations.

  First, I explore the dynamics of \bubbles{} set into motion by ambient acoustic radiation in a fluid described by potential flow in the long-wavelength limit.
  Variations in the local surface pressure caused by scattering events from each \bubble{} sets each \bubble{} into motion in accordance with Newton's second law.
  By expanding this pressure in terms of spherical harmonics---natural eigenfunctions of the unretarded radiation kernel---we recover an analytic description of the force on each \bubble{} due to an incident waveform with high-order numerical integrations relating the radiation incident on one \bubble{} to the surface pressure on another.
  These simulations predict a dominant translational effect along the direction of propagation of the incident waveform, though they also reveal largely dipolar interactions between \bubbles{} that produce secondary expansions and contractions of the collective \bubble{} system.

  Extending the idea of radiative effects coupling to a system driven by an ordinary differential equation, the optical Bloch equations give the evolution of a quantum dot under the influence of an external electromagnetic field.
  This evolution then produces secondary radiation that couples a collection of \qds{} together.
  In a semiclassical model, radiated fields become a pairwise calculation between \qds{} (without need for a radiation grid or mesh) that subsequently drive the evolution of a set of (implicitly-coupled) Bloch equations.
  In ensembles of up to $10^4$ \qds{}, this model predicts synchronized multiplets of particles that exchange energy, \qds{} that dynamically couple to screen the effect of incident external radiation, localization of the polarization due to randomness and interactions, as well as wavelength-scale regions of enhanced and suppressed polarization.



  \Cref{ch:aim} inherits the same physical system from \cref{ch:quantum dots} with a vastly accelerated computational scheme as the quadratic cost (in both time and space) of evaluating radiated fields between particles severely hampers calculations involving more than $\num{10000}$ \qds{}.
  To remedy this, we consider the interactions between distant particles as occuring on a superimposed spatial grid.
  The translational symmetries of this grid then facillitate rapid field evaluations by exploiting the circulant structure of the resulting operator.

\end{abstract}
