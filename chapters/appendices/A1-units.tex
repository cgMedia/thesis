\chapter{Conventions}

\section{Fourier Transforms\label{sec:general transform}}

The definition of the Fourier transform leaves some room to adopt conventions.
I write the fully-generalized Fourier transform $f(t) \leftrightarrow F(\omega)$ pair as
\begin{subequations}
  \begin{align}
    F(\omega) &= \sqrt{\frac{\abs{s_2}}{\qty(2\pi)^{1-s_1}}} \int_{-\infty}^{\infty} f(t) e^{i s_2 \omega t} \dd{t} \label{eq:forward transform} \\
    f(t) &= \sqrt{\frac{\abs{s_2}}{\qty(2\pi)^{1+s_1}}} \int_{-\infty}^{\infty} F(\omega) e^{- i s_2 \omega t} \dd{\omega} \label{eq:backward transform}
  \end{align}
\end{subequations}
where the $s_1, s_2$ parameters allow for different normalization and frequency conventions. 
Consequently, \cref{eq:backward transform} implies
\begin{equation}
  \partial_t f(t) \leftrightarrow -i \omega s_2 F(\omega).
\end{equation}

\section{Maxwell's Equations}

\subsection{Time-domain}
An unfortunate ambiguity in the prescription of electromagnetic units allows for unit systems that differ by more than simple scaling factors.
In a unit-independant format, the time-domain Maxwell's equations become
\begin{subequations}
  \begin{align}
    \div{\vb{E}} &= 4 \pi k_1 \rho \label{eq:gauss} \\
    \curl{\vb{B}} &= 4 \pi k_2 \alpha \vb{J} + \frac{k_2 \alpha}{k_1} \pdv{\vb{E}}{t} \label{eq:ampere}\\
    \div{\vb{B}} &= 0 \label{eq:no monopoles} \\
    \curl{\vb{E}} &= - k_3 \pdv{\vb{B}}{t} \label{eq:faraday}
  \end{align}
  \label{eq:time-domain maxwell equations}
\end{subequations}
with $k_1$, $k_2$, $\alpha$, and $k_3$ as system-dependent constants. 
One need only specify two of the four constants to pin down the unit system, however; in particular 
\begin{equation}
  \frac{k_1}{k_2 k_3 \alpha} = c^2 \qquad \text{and} \qquad  k_3 = \frac{1}{\alpha}.
\end{equation}

From \cref{eq:no monopoles}, $\vb{B} = \curl{\vb{A}}$. 
Then, using \cref{eq:faraday}, it becomes apparent that
\begin{equation*}
  \curl{\qty(\vb{E} + k_3 \pdv{\vb{A}}{t})} = 0
\end{equation*}
so I may take
\begin{equation*}
  \vb{E} = -\grad{V} - k_3 \pdv{\vb{A}}{t}.
\end{equation*}
Thus, in electorstatic systems, $\vb{E} = -\grad{V}$ regardless of units.
By \cref{eq:gauss}, then,
\begin{equation*}
  \laplacian V = -4 \pi k_1 \rho
\end{equation*}
and thus
\begin{equation*}
  V(\vb{r}) = \int \frac{1}{\abs{\vb{r} - \vb{r}'}} \frac{\rho(\vb{r}')}{k_1} \dd[3]{\vb{r}'}
\end{equation*}
as
\begin{equation*}
  \laplacian \qty(\frac{1}{\abs{\vb{r} - \vb{r}'}}) = -4 \pi \delta(\vb{r} - \vb{r}').
\end{equation*}
This defines the \emph{scalar Green's function},
\begin{equation}
  g(\vb{r}, \vb{r}') \equiv \frac{1}{\abs{\vb{r} - \vb{r}'}},
  \label{eq:scalar green}
\end{equation}
that relates charge to electric potential.

\subsection{Frequency-domain}
Suitably, the generalized Fourier transform of \cref{sec:general transform} means the frequency-domain analogues of \cref{eq:time-domain maxwell equations}
become
\begin{subequations}
  \begin{align}
    \div{\vb{E}} &= 4 \pi k_1 \rho \\
    \curl{\vb{B}} &= 4 \pi k_2 \alpha \vb{J} - i \omega s_2 \frac{k_2 \alpha}{k_1} \vb{E} \\
    \div{\vb{B}} &= 0 \\
    \curl{\vb{E}} &= i \omega s_2 k_3 \vb{B}
  \end{align}
  \label{eq:frequency-domain maxwell equations}
\end{subequations}
Combining the two curl equations gives
\begin{equation}
  \curl{\curl{\vb{E}}} - k^2 s_2^2 \vb{E} = 4 \pi i \omega s_2 k_2 \vb{J}
\end{equation}
where $k^2 = \omega^2/c^2 = \omega^2 k_2/k_1$.
