\chapter{Conventions}

\section{Fourier Transforms\label{sec:general transform}}

The definition of the Fourier transform leaves some room to adopt conventions.
I write the fully-generalized Fourier transform $f(t) \leftrightarrow F(\omega)$ pair as
\begin{subequations}
  \begin{align}
    F(\omega) &= \sqrt{\frac{\abs{s_2}}{\qty(2\pi)^{1-s_1}}} \int_{-\infty}^{\infty} f(t) e^{i s_2 \omega t} \dd{t} \label{eq:forward transform} \\
    f(t) &= \sqrt{\frac{\abs{s_2}}{\qty(2\pi)^{1+s_1}}} \int_{-\infty}^{\infty} F(\omega) e^{- i s_2 \omega t} \dd{\omega} \label{eq:backward transform}
  \end{align}
\end{subequations}
where the $s_1, s_2$ parameters allow for different normalization and frequency conventions. 
Of principal importance, \cref{eq:backward transform} implies
\begin{equation}
  \partial_t f(t) \leftrightarrow -i \omega s_2 F(\omega).
\end{equation}

\section{Maxwell's Equations}

An unfortunate ambiguity in the prescription of electromagnetic units allows for unit systems that differ by more than simple scaling factors.
In a unit-independant format, Maxwell's equations become~\cite{jackson2007classical}
\begin{subequations}
  \begin{align}
    \div{\vb{E}} &= 4 \pi k_1 \rho \label{eq:gauss} \\
    \curl{\vb{B}} &= 4 \pi k_2 \alpha \vb{J} + \frac{k_2 \alpha}{k_1} \pdv{\vb{E}}{t} \label{eq:ampere}\\
    \div{\vb{B}} &= 0 \label{eq:no monopoles} \\
    \curl{\vb{E}} &= - k_3 \pdv{\vb{B}}{t} \label{eq:faraday}
  \end{align}
  \label{eq:time-domain maxwell equations}
\end{subequations}
with $k_1$, $k_2$, $\alpha$, and $k_3$ as system-dependent constants. 
One need only specify two of the four constants to pin down the unit system, however; in particular 
\begin{equation}
  \frac{k_1}{k_2 k_3 \alpha} = c^2 \qq{and} k_3 = \frac{1}{\alpha}.
\end{equation}

From \cref{eq:no monopoles}, $\vb{B} = \curl{\vb{A}}$. 
Then, using \cref{eq:faraday}, it becomes apparent that
\begin{equation*}
  \curl{\qty(\vb{E} + k_3 \pdv{\vb{A}}{t})} = 0
\end{equation*}
so I may take
\begin{equation*}
  \vb{E} = -\grad{\varphi} - k_3 \pdv{\vb{A}}{t}.
\end{equation*}
Thus, in electorstatic systems, $\vb{E} = -\grad{\varphi}$ regardless of units.
By \cref{eq:gauss}, then,
\begin{equation*}
  \laplacian \varphi = -4 \pi k_1 \rho
\end{equation*}
and thus
\begin{equation*}
  \varphi(\vb{r}) = \int \frac{1}{\abs{\vb{r} - \vb{r}'}} \frac{\rho(\vb{r}')}{k_1} \dd[3]{\vb{r}'}
\end{equation*}
as
\begin{equation*}
  \laplacian \qty(\frac{1}{\abs{\vb{r} - \vb{r}'}}) = -4 \pi \delta(\vb{r} - \vb{r}').
\end{equation*}
This defines the \emph{Laplace-kernel Green's function} as
\begin{equation}
  g_L(\vb{r}) \equiv \frac{1}{\abs{\vb{r}}}.
  \label{eq:scalar green}
\end{equation}
Analogously, the \emph{Helmholtz-kernel} and \emph{Wave-kernel} Green's functions become
\begin{equation}
  g_H(\vb{r}, k) \equiv \frac{e^{-i k \abs{\vb{r}}}}{\abs{\vb{r}}} \qquad
  g_W(\vb{r}, t) \equiv \frac{\delta(t - \abs{\vb{r}}/c)}{\abs{\vb{r}}}
\end{equation}

\subsection{Vector wave equation}

From \cref{eq:faraday,eq:ampere},
\begin{equation}
  \curl{\curl{\vb{E}}} + \frac{1}{c^2} \pdv[2]{\vb{E}}{t} = -4\pi k_2 \pdv{\vb{J}}{t}.
  \label{eq:vector wave equation}
\end{equation}
Taking the divergence of both sides and removing one time derivative,
\begin{equation*}
  \frac{1}{c^2} \div{\pdv{\vb{E}}{t}} = -4 \pi k_2 \div{\vb{J}}.
\end{equation*}
As $\curl{\curl{\vb{A}}} = \grad{\qty(\div{\vb{A}})} - \grad^2{\vb{A}}$, the time derivative of \cref{eq:vector wave equation} becomes
\begin{equation*}
  \grad{\qty(\div{\pdv{\vb{E}}{t}})} - \grad^2 \pdv{\vb{E}}{t} + \frac{1}{c^2} \pdv[3]{\vb{E}}{t} = -4\pi k_2 \pdv[2]{\vb{J}}{t}
\end{equation*}
therefore
\begin{equation*}
  \grad^2 \pdv{\vb{E}}{t} - \frac{1}{c^2} \pdv[3]{\vb{E}}{t} = 4\pi k_2 \qty(\pdv[2]{\vb{J}}{t} - c^2 \grad \div{\vb{J}}).
\end{equation*}
Immediately, the left-hand side gives the ``wave operator'' acting on $\partial_t \vb{E}$ and thus
\begin{equation*}
  \pdv{t} \vb{E}(\vb{r}, t) = g_W(\vb{r}, t) \ast -k_2\qty(\pdv[2]{t} - c^2 \grad \div) \vb{J}(\vb{r}, t)
\end{equation*}
from which it follows
\begin{equation}
  \vb{E}(\vb{r}, t) = -k_2 \iint \frac{\delta\qty(t - t'_R )}{\abs{\vb{r} - \vb{r}'}} \qty(\pdv[2]{{t'}} - c^2 \grad' \grad'\boldsymbol{\cdot}) \, \vb{P}(\vb{r}', t') \dd[3]{\vb{r}'} \dd{t'}
\end{equation}
where $t'_R \equiv t' - \abs{\vb{r} - \vb{r}'}/c$ and $\partial_t \vb{P} \equiv \vb{J}$.

