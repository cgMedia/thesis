\chapter{Chebyshev polynomials}

\begin{figure}[h]
  \centering
  \begin{tikzpicture}
    \begin{axis}[cycle list/Blues-7, every axis plot/.append style={smooth}, domain=-1:1]
      \addplot {-1 + 18*x^2 - 48*x^4 + 32*x^6};
      \addplot {5*x - 20*x^3 + 16*x^5};
      \addplot {1 - 8*x^2 + 8*x^4};
      \addplot {-3*x + 4*x^3};
      \addplot {-1 + 2*x^2};
      \addplot {x};
      \addplot {1};
    \end{axis}
  \end{tikzpicture}
  \caption{\label{fig:chebyshev polynomials} Chebyshev polynomials $T_0(x)$ through $T_6(x)$.
    The roots of $T_{m + 1}(x)$ give the sample points for an interpolation scheme of order $m$ on the interval $\qty[-1, 1]$.
  }
\end{figure}

\begin{figure}
  \centering
  \begin{tikzpicture}
    \begin{axis}[samples=128, every axis plot/.append style={smooth}, domain=-1:1, legend entries={$\qty(1 + 16x^2)^{-1/2}$, Lagrange, Chebyshev}]
      \addlegendimage{no markers, black}
      \addlegendimage{cbred, mark=*}
      \addlegendimage{cbblue, mark=diamond*, mark size=2}

      \addplot[black] {(1 + 16 * x^2)^-0.5};

      \pgfplotsset{cycle list shift=-1}
      \addplot {1. + x^2 * (-7.37712779012503 + x^2 * (53.87684354813709 + x^2 * (-228.0754143182082 + x^2 * (496.66170721466034 + x^2 * (-512.1411767023495 + 196.29770367292164 * x^2)))))};

      \addplot {1. + x^2 * (-6.528457748805862 + x^2 * (34.661982001110324 + x^2 * (-101.30878106137038 + x^2 * (156.3571608116639 + x^2 * (-119.85545125891753 + 35.924166950838185 * x^2)))))};

      \pgfplotsset{cycle list shift=-3}
      \addplot+[only marks] coordinates {
        (-1., 0.242536)
        (-0.833333, 0.287348)
        (-0.666667, 0.351123)
        (-0.5, 0.447214)
        (-0.333333, 0.6)
        (-0.166667, 0.83205)
        (0., 1.)
        (0.166667, 0.83205)
        (0.333333, 0.6)
        (0.5, 0.447214)
        (0.666667, 0.351123)
        (0.833333, 0.287348)
        (1., 0.242536)
      };
      \addplot+[only marks, mark=diamond*, mark size=2] coordinates {
        (0.992709, 0.244211)
        (0.935016, 0.258301)
        (0.822984,  0.290658)
        (0.663123, 0.352767)
        (0.464723, 0.473754)
        (0.239316,   0.722374)
        (0., 1.)
        (-0.239316, 0.722374)
        (-0.464723,   0.473754)
        (-0.663123, 0.352767)
        (-0.822984, 0.290658)
        (-0.935016,   0.258301)
        (-0.992709, 0.244211)
      };
    \end{axis}
  \end{tikzpicture}
  \caption{\label{fig:pathological interpolation} Interpolation of a pathological function, $f(x) = \qty(1 + 16x^2)^{-1/2}$, using Chebyshev and equally-spaced Lagrange interpolation schemes.
  Because the Chebyshev scheme clusters sample points in the tails of the interpolation window, it does not suffer from high-order ringing effects (Runge's phenomenon).}
\end{figure}

The Chebyshev polynomials of the first kind, defined by
\begin{equation}
  T_n(x) = \cos(n \arccos(x)),
  \label{eq:chebyshev}
\end{equation}
form an orthogonal (though not orthonormal) sequence well-suited to interpolation.
with respect to a weight function of $w(x) = \sqrt{1 - x^2}$.
As a result, the following two orthogonality relations hold:
\begin{subequations}
  \begin{align}
    \int_{-1}^1 \frac{T_i(x) T_j(x)}{\sqrt{1 - x^2}} \dd{x} &= 
    \begin{cases}
      0 & i \not = j \\
      \pi / 2 & i = j \not = 0 \\
      \pi & i = j = 0
    \end{cases} \label{eq:continuous orthogonality} \\
    \sum_{k = 0}^{m} T_i(x_k) T_j(x_k) &= 
    \begin{cases}
      0 & i \not = j \\
      (m + 1)/2 & i = j \not = 0 \\
      m + 1 & i = j = 0
    \end{cases}
    \label{eq:discrete orthogonality}
  \end{align}
\end{subequations}
where the $\qty{x_k}$ in \cref{eq:discrete orthogonality} represent the Chebyshev nodes\footnote{Many resources detailing Chebyshev polynomials use $0 \leqslant k < m$ instead of $0 \leqslant k < m + 1$. The latter, used here, matches the order of the interpolation to the order of the resulting polynomial.},
\begin{equation}
  x_k = -\cos(\frac{\pi(k + 1/2)}{m +1}) \qquad k = 0, 1, \ldots, m
\end{equation}

\section{Interpolation}

Given a set of nonuniform sample points, $x_\lambda$, on the interval $\qty[-1, 1]$ where
\begin{equation}
  x_\lambda = -\cos(\frac{\pi (\lambda + 1/2)}{M + 1}) \qquad \lambda = 0, 1, 2, \ldots, M
\end{equation}

\begin{equation}
  c_i = \frac{\alpha_{i}}{M + 1}\sum_{\lambda = 0}^M T_{i}(x_\lambda) f(x_\lambda) \qquad \alpha_i = 2 - \delta_{i0}
\end{equation}

\section{Derivatives}

In general,
\begin{equation}
  \qty(1 - x^2) T_n'(x) = n \qty[T_{n-1}(x) - x T_n(x)]
  \label{eq:chebyshev derivatives}
\end{equation}
relates Chebyshev polynomials to their derivatives~\cite{Gil2007}. 
This relationship, however, becomes problematic with finite-precision arithmetic.


\subsection{Three dimensions}
\begin{equation}
  f(\vb{r}) \approx c_{ijk}T_i(x) T_j(y) T_k(z)
\end{equation}
with an implied summation on the bound indices $i$, $j$, and $k$.
Analogously,
\begin{equation}
  c_{ijk} = \frac{\alpha_{ii'} \alpha_{jj'} \alpha_{kk'}}{\qty(M+1)^3} f(x_\lambda, y_\mu, z_\nu) T_{i'}(x_\lambda) T_{j'}(y_\mu) T_{k'}(z_\nu)
\end{equation}
(where $\alpha_{ab} \equiv 2\delta_{ab} - \delta_{a0}\delta_{b0}$ has become a diagonal tensor to ``balance'' the implied-sum indices) and thus
\begin{equation}
  f(\vb{r}) \approx \qty(\frac{\alpha_{ii'} \alpha_{jj'} \alpha_{kk'}}{\qty(M+1)^3} \qty(f(x_\lambda, y_\mu, z_\nu) T_{i'}(x_\lambda) T_{j'}(y_\mu) T_{k'}(z_\nu))) T_i(x) T_j(y) T_k(z)
\end{equation}
as one, \emph{mondo}, tensor expression.
