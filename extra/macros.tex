% --- Universal
%\newcommand{\dropcap}[2]{\lettrine[lines=3, depth=1]{\color{BrickRed}#1}{#2}} % Use with Rothenburg drop-caps
%\newcommand{\dropcap}[2]{\lettrine[lines=4, depth=0]{\color{BrickRed}#1}{#2}} % Use with Goudy Initialen drop-caps
\newcommand{\dropcap}[3][\color{BrickRed}]{%
\lettrine[lines=4, depth=0]{#1#2}{#3}%
}

\newcommand{\naive}{na\"{\i}ve}
\newcommand{\Naive}{Na\"{\i}ve}
\newcommand{\naively}{na\"{\i}vely}
\newcommand{\Naively}{Na\"{\i}vely}
\newcommand{\cafe}{caf\'e}
\newcommand{\etal}{\emph{et~al.}} % and others
\newcommand{\ie}{\emph{i.e.}}     % that is
\newcommand{\eg}{\emph{e.g.}}     % for example

% So cref can appropriately word appendices
\crefname{appsec}{appendix}{appendices}

% Some of the \int_{} symbols get really tight when set immediately before a fraction.
% These macros ('spaced int') give a little bit of extra space when necessary.
\newcommand{\integralspace}{\kern0.1em}
\newcommand{\sint}{\int_{} \integralspace}
\newcommand{\siint}{\iint_{} \integralspace}
\newcommand{\siiint}{\iiint_{} \integralspace}

% Requires mathtools
\DeclarePairedDelimiter\ceil{\lceil}{\rceil}
\DeclarePairedDelimiter\floor{\lfloor}{\rfloor}

\newcommand{\attrib}{} % dummy macro
\newenvironment{frontquote}[1]
{
  \renewcommand{\attrib}{\hspace*{\fill}---#1}
  \begin{quote}\small
}{
  \\ \attrib
  \end{quote}
}

\newcommand{\QuEST}{QuEST}

% --- Common colors
\definecolor{cbred}{HTML}{e41a1c}
\definecolor{cbblue}{HTML}{377eb8}
\definecolor{cbgreen}{HTML}{4daf4a}

\newif\ifmakeplots
\makeplotstrue
\newcommand{\conditionalFigureInput}[1]{\ifmakeplots%
  \input{#1}%
\fi%
}

% --- Chapter 02
\newcommand{\bubble}{microsphere}
\newcommand{\Bubble}{Microsphere}
\newcommand{\bubbles}{\bubble s}
\newcommand{\Bubbles}{\Bubble s}

\newcommand{\GF}{\ensuremath{g_r(\vb{r}, t; \, \vb{r}', t')}}
\DeclareRobustCommand{\redTriangle}{\textcolor{red!75!black}{\tikz{\pgfuseplotmark{triangle*}}}}
\DeclareRobustCommand{\blueCircle}{\textcolor{blue!75!black}{\tikz{\pgfuseplotmark{*}}}}

% --- Chapter 03
\newcommand{\qd}{quantum dot}
\newcommand{\qds}{quantum dots}
\newcommand{\Qd}{Quantum dot}
\newcommand{\Qds}{Quantum dots}

\newcommand{\outerprod}[2]{#1 \! \otimes \! #2}
\newcommand{\tensor}[1]{#1}
